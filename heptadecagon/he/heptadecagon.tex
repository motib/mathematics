\documentclass[11pt,a4paper]{article}

\usepackage[utf8x]{inputenc}
\usepackage[english,hebrew]{babel}

\usepackage{verbatim}
\usepackage{url}

\usepackage{tikz}
\usetikzlibrary{intersections,calc,through,arrows.meta}
\tikzset {>=Stealth}

\textwidth=15cm
\textheight=23cm
\topmargin=0pt
\headheight=0pt
\oddsidemargin=2em
\headsep=0pt
\parindent=0pt
\renewcommand{\baselinestretch}{1.15}
\setlength{\parskip}{0.3\baselineskip plus 1pt minus 1pt}

\newenvironment{form}[1]{%
\begin{displaymath}%
\renewcommand{\arraystretch}{#1}%
\begin{array}{lcl}}%
{\end{array}%
\end{displaymath}%
}

\newcommand*{\disfrac}[2]{\displaystyle\frac{#1}{#2}}
\newcommand*{\sm}[1]{$\scriptstyle #1$}

\renewcommand*{\floatpagefraction}{.9}
\setlength{\textfloatsep}{10pt plus 2pt minus 4pt}
\setlength{\floatsep}{6pt plus 2pt minus 4pt}

\newcommand*{\occ}[2]{%
  \stackrel{%
    \textstyle r^{#1}}%
    {\!\!\!\scriptscriptstyle #2}}

\begin{document}
\thispagestyle{empty}
\selectlanguage{hebrew}
\begin{center}
\textbf{\LARGE בניית מצולע משוכלל עם 
$17$
צלעות}

\bigskip

\textbf{\Large מוטי בן-ארי\\\bigskip\url{http://www.weizmann.ac.il/sci-tea/benari/}}

\bigskip
\end{center}


\begin{footnotesize}
\begin{center}
\L{\copyright{}\  2020} מוטי בן-ארי. 
\end{center}
\selectlanguage{english}
This work is licensed under the Creative Commons Attribution-ShareAlike 3.0 Unported License. To view a copy of this license, visit \url{http://creativecommons.org/licenses/by-sa/3.0/} or send a letter to Creative Commons, 444 Castro Street, Suite 900, Mountain View, California, 94041, USA.

\end{footnotesize}
\selectlanguage{hebrew}
מסמך זה מציג את התובנה של 
\L{Gauss}
שניתן לבנות 
\L{\textbf{heptadecagon}},
מצולע משוכלל עם 
$17$
צלעות באמצעות סרגל ומחוגה.
ההצגה מבוססת על
\L{\cite{jorg}}
אבל מכיל חישובים מפורטים של הפיתוח של הנוסחה של
\L{Gauss}.
מוצגת גם בנייה של ממש לפי
\L{\cite{callagy}},
שוב עם חישובים מפורטים.

\section{בנייה של מצולעים משוכללים}

\paragraph{היסטוריה}
היוונים ידעו איך לבנות  עם סרגל ומחוגה מצולעים משוכללים מסויים: משולש, ריבוע, מחומש ומספר מצולעים שמספר הצלעות שלהם הוא מכפלה שלהם, כגון מצולע משוכלל עם $15$
צלעות.
כמובן, בהינתן מצולע משוכלל עם
$n$
צלעות, קל לבנות מצולע עם 
$2n$
צלעות על ידי בניית חוצי הצלעות.

לא היית התקדמות במשך אלפיים שנה עד שבשנת
$1796$,
קצת לפני יום הולדתו ה-$19$,
\L{Carl Friedrich Gauss}
התעורר בוקר אחד ולאחר "מחשבה מרוכזת" מצא דרך לבנות מצולע משוכלל עם 
$17$
צלעות. הישג זה עודד אותו להיות מתמטיקאי.

הבנייה של מצולע משוכלל עם 
$17$
צלעות היתה אבן דרך למשפט
\L{Gauss-Wantzel}:
מצולע משוכלל עם 
$n$
צלעות ניתן לבנייה עם סרגל ומחוגה אם ורק אם 
$n$
הוא מכפלה של חזקה של
$2$
ואפס או יותר מספרי 
\L{Fermat}
ראשון
\textbf{שונים}
$2^{2^k}+1$. 
מספרי 
\L{Fermat}
ראשונים ידועים הם
$F_0=3, F_1=5, F_2=17, F_3=257, F_4=65537$.
היוונים ידעו לבנות מצולעים משוכללים עם
$3$
ו-%
$5$
צלעות.
\L{Gauss}
הראה שניתן לבנות מצולע עם
$17$
צלעות.
מצולע משוכלל עם
$257$
צלעות נבנה לראשונה על ידי
\L{Magnus Georg Paucker}
ב-%
$1822$
ועל ידי
\L{Friedrich Julius Richelot}
ב-%
$1832$.
ב-%
$1894$
\L{Johann Gustav Hermes}
טען שבנה מצולע משוכלל עם
$65537$
צלעות.
כתב היד שלו נשמר באוניברסיטת 
\L{G\"{o}ttigen},
במקרה שתרצו לבדוק אותו.

\paragraph{הקוסינוס של הזווית המרכזית}
כדי לבנות מצולע משוכלל מספיק לבנות קטע קו באורך 
$\cos \theta$,
כאשר
$\theta$ 
הוא הזווית המרכזית במעגל היחידה 
subtended by
מיתר שהוא צלע של המצולע
)איור
~\ref{fig.cosine}(.
נתון קטע הקו
$\overline{OB}=\cos\theta$,
בנו אנך ב-%
$B$
וסמן את החיתוך שלו עם מעגל היחידה ב-%
$C$.
אזי
$\overline{OC}=1$
ו-%
$\cos \theta=\disfrac{\overline{OB}}{\overline{OC}}=\cos \theta$,
כך שי-%
$\theta = \cos^{-1} (\cos \theta)$.
המיתר 
$\overline{AC}$
הוא צלע של המצולע.
\begin{figure}
\begin{center}
\selectlanguage{english}
\begin{tikzpicture}[scale=1.5]
\coordinate (O) at (0,0) node[left] {$O$} node[above right,xshift=32pt] {$\theta$};
% {$\cos^{-1} (\cos 21.12^\circ)$};
\coordinate (A) at (4,0);
\node[right] at (A) {$A$};
\draw (O) -- (A);
\draw (A) arc(0:21.12:4);
\coordinate (C) at (21.12:4cm);
\draw (O) -- node[above] {$1$} (C);
\node[right] at (C) {$C$};
\draw (C) -- (C |- A) coordinate (B);
\node[below] at (B) {$B$};
\draw[rotate=90] (B) rectangle +(6pt,6pt);
\draw[thick,dashed] (A) -- (C);
\path (O) -- node[below] {$\cos \theta$} (B); % {$\cos 21.12^\circ$} (B);
\end{tikzpicture}
\selectlanguage{hebrew}
\caption{%
בניית צלע מהקוסינו של הזווית מרכזית שהוא שכולא אותו}
\label{fig.cosine}
\end{center}
\end{figure}

\paragraph{פעולות חשבוניות שניתנות ליישם באמצעות בנייה}

נתון קטע קו שאורכו מוגדר כ-%
$1$,
האורכים שניתנים לבנייה הם אלה שניתן לקבל מאורכים קיימים תוך שימוש בפעולות חיבור, חיסור, כפל, חילוק והוצאת שורש ריבועי.

הזווית המרכזית של משולש שווה-צלעות היא
$360^\circ/3=120^\circ$ )איור ~\ref{fig.triangle-pentagon},
שמאל(
וניתן לחשב את הקוסינוס מהנוסחה לקוסינוס של שתי זוויות:
\[
\cos 120^\circ = \cos(90^\circ+30^\circ)=\cos 90^\circ \cos 30^\circ  -\sin 90^\circ \sin 30^\circ = 0\cdot \frac{\sqrt{3}}{2} - 1\cdot \frac{1}{2}=-\disfrac{1}{2}\,.
\]
ברור ש-%
$-\frac{1}{2}$,
מספר רציונלי,
ניתן לבנייה.
\begin{figure}[htb]
\begin{center}
\selectlanguage{english}
\begin{tikzpicture}[scale=.9]
\coordinate (O) at (0,0);
\foreach \x/\name in {0/a,1/b,2/c} {
  \coordinate (\name) at ($(O)+(\x*120:3cm)$);
  \draw (O) -- (\name);
  \fill (\name) circle (1pt);
}
\draw (a) -- (b) -- (c) -- cycle;
\node[above right] at (O) {\sm{120^\circ}};
\draw (O) circle (3cm);
\begin{scope}[xshift=7cm]
\coordinate (O) at (0,0);
\foreach \x/\name in {0/a,1/b,2/c,3/d,4/e} {
  \coordinate (\name) at ($(O)+(\x*72:3cm)$);
  \draw (O) -- (\name);
  \fill (\name) circle (1pt);
}
\draw (a) -- (b) -- (c) -- (d) -- (e) -- cycle;
\node[below right,xshift=4pt] at (O) {\sm{72^\circ}};
\node[above right,xshift=16pt] at (O) {\sm{36^\circ}};
\draw[thick,dashed,name path=P1] (O) -- (36:3cm) coordinate (ten);
\path[name path=P2] (a) -- (b);
\path[name intersections={of=P1 and P2, by={right}}];
\draw[rotate=-144] (right) rectangle +(6pt,6pt);
\fill (ten) circle(1pt);
\draw (O) circle (3cm);
\end{scope}
\end{tikzpicture}
\selectlanguage{hebrew}
\caption{%
שולש שווה-צלעות )שמאל(, מחומש משוכלל )ימין(.}%
\label{fig.triangle-pentagon}
\end{center}
\end{figure}
הזווית המרכזית של מחומש משוכלל היא
$360^\circ/5=72^\circ$ )איור~\ref{fig.triangle-pentagon},
ימין(.
ניתן לחשב את
$\cos 36^\circ$
מ-%
$\cos 90^\circ = \cos(72^\circ+ 18^\circ)$,
אבל החישובים מעט מייגעים
)ראו
~\cite{wiki:pentagon}(.
ניתן לחשב את
$\cos 36^\circ=\disfrac{1+\sqrt{5}}{4}$
על ידי שימוש בפעולות
$\{+,-,\times,\div,\surd\}$
כך שהערך ניתן לבייה.
לאחר שבונים את הזווית
$36^\circ$
קל לבנות מחומש משוכלל על ידי הורדת אנך לרדיוס ב-%
$36^\circ$.
האנך יחתוך את המעגל ב-%
$-36^\circ$
וקטע הקו שנבנה הוא צלע של המחומש.

איור%
~\ref{fig.hept}
מראה מצולע משוכלל עם 
$17$
צלעות
החסום על ידי מעגל היחידה.
הזווית המרכזית היא
$\disfrac{2\pi}{17}$
רדיאנים או
$\disfrac{360^\circ}{17}\approx 21.12^\circ$. 
\begin{figure}
\begin{center}
\selectlanguage{english}
\begin{tikzpicture}[scale=1.1]
\coordinate (O) at (0,0);
\foreach \x/\name in {0/a,1/b,2/c,3/d,4/e,5/f,6/g,7/h,8/i,9/j,10/k,11/l,12/m,13/n,14/o,15/p,16/q} {
  \coordinate (\name) at ($(O)+(\x*21.12:3cm)$);
  \draw (O) -- (\name);
  \fill (\name) circle (1pt);
}
\draw (a) -- (b) -- (c) -- (d) -- (e) -- (f) -- (g) -- (h) -- (i) -- (j) -- (k) -- (l) -- (m) -- (n) -- (o) -- (p) -- (q) -- cycle;
\node[above right,xshift=32pt] at (O) {\sm{2\pi/17=21.12^\circ}};
\draw (O) circle (3cm);
\end{tikzpicture}
\selectlanguage{hebrew}
\caption{מצולע משוכלל עם $17$ צלעות חסום על ידי מעגל היחידה}\label{fig.hept}
\end{center}
\end{figure}
\L{Gauss}
הראה ש:
\begin{form}{3}
\cos\left(\disfrac{2\pi}{17}\right) &=& 
-\disfrac{1}{16}+\disfrac{1}{16}\sqrt{17} + 
     \disfrac{1}{16}\sqrt{34-2\sqrt{17}}
    + \\
    &&
     \disfrac{1}{8}\sqrt{
     17+3\sqrt{17} - 
     \sqrt{34-2\sqrt{17}}
   -2
     \sqrt{34+2\sqrt{17}}
   }\,.
\end{form}
ערך זה ניתן לחשב תוך שימוש בפעולות
$\{+,-,\times,\div,\surd\}$
ולכן הוא ניתן לבנייה. 

סעיפים%
~\ref{s.roots}, \ref{s.gauss}, \ref{s.derivation}
מביא את הרעיונות המתמטיים של
\L{Gauss},
בתוספת החישובים המפורטים.
ההוכחה מחייבת שימוש במספרים מרוכבים, אבל את רובה ניתן להבין ללא ידע במספרים מרוכבים אם אתם מוכנים לקבל מספר עובדות. העובדות הללו רשומות בקטעים מסומנים בטקסט.
סעיף%
~\ref{s.construction}
מראה בנייה יעילה של
$\cos\disfrac{2\pi}{17}$.


\section{השורשים של אחד}\label{s.roots}

\paragraph{המשפט הבסיס של אלגברה}
לכל פולינום במעלה 
$n$
)עם מקדמים מרוכבים(
בדיוק
$n$
שורשים
)מרוכבים(.

\paragraph{%
השורשים של אחד ומצולעים משוכללים%
}
נתבונן במשוואה
$x^{n}-1=0$.
שורש אחד הוא
$x=1$.
לפי המשפט הבסיסי של אלגברה קיימים
$n-1$
שורשים נוספים. נסמן שורש אחד ב-%
$r$
כך ש-%
$r^{n}=1$.
$r$
נקרא
\textbf{שורש של אחד}.

\begin{quote}
\textbf{מספרים מרוכבים}\\
השורש
$r$
הוא
$\cos \left(\disfrac{2\pi}{n}\right) + i\sin  \left(\disfrac{2\pi}{n}\right)$.
לפי נוסחאת
\L{de Moivre}:
\[
\left[\cos \left(\disfrac{2\pi}{n}\right) + i\sin  \left(\disfrac{2\pi}{n}\right)\right]^{n}=
\cos \left(\disfrac{2\cdot n\pi}{n}\right) + i\sin  \left(\disfrac{2\cdot n\pi}{n}\right)= 1\,,
\]
והוכחנו ש-%
$r$
הוא שורש
$n$
של אחד.
\end{quote}
נתבונן כעת ב-%
$r^2$.
אנו רואים ש:
\[
r^{2\cdot n}=(r^{n})^2=1^2=1\,.
\]
כך שהשורשים של
 $x^{n}-1$,
 שורשי
 $n$
 של אחד, הם:
\[
1, r, r^2, \ldots, r^{n-2}, r^{n-1}\,.
\]
אם
$n$
ראשוני 
)כגון
$17$(
החזקות הללו שונות זו מזו,
כך שהם כל שורשי ה-%
$n$
של אחד.

\textbf{הוכחה:}
שורש
$n$
של אחד נקרא פרימיטבי אם הוא אינו שורש
$m$
של אחד עבור
$m<n$.
אם
$n$
ראשוני, כל השורשים פרט ל-%
$1$
הם פרימיטיביים, וכל השורשים של אחד שונים זה מזה.
אם לא,
$r^i=r^j$
עבור
$1\leq i<j\leq n$
כלשהם, כך ש-%
$r^j/r^i=r^{j-i}=1$
ו-%
$r$
אינו פרימיטיבי.

\begin{quote}
\textbf{מספרים מרוכבים}\\
השורשים הם הקואורדינטות הפולאריות של הקודקודים של מצולע משוכלל, כאשר החלק הממשי הוא  קואורדינטת ה-%
$x$
והחלק הדמיוני הוא קואורדינטת ה-%
$y$.

עבור משולש שווה-צלעות השורשים הם
$1+i\cdot 0,\disfrac{-1}{2}\pm i\disfrac{\sqrt{3}}{2}$.

עבור מחומש משוכלל הם:
\[
1+i\cdot 0,\frac{\sqrt{5}-1}{4}\pm i \frac{\sqrt{10+2\sqrt{5}}}{4},\frac{-\sqrt{5}-1}{4}\pm i \frac{\sqrt{10-2\sqrt{5}}}{4}\,.
\]
\end{quote}

\paragraph{הפולינום שמתקבל ממכפלת הגורמים הלינאריים}
נתבונן בפולינום המתקבל על ידי הכפלת כל הגורמים הלינאריים המתקבלים מהשורשים של אחד:
\[
(x-1) (x-r)(x-r^2) \cdots (x-r^{n-1})=x^n-1\,.
\]
נכפיל ונמצא שהמקדם של
$x^2$
הוא:
\[
1+r+r^2+\cdots + r^{n-2}+r^{n-1}\,,
\]
אבל, ברור שהמקדם שווה לאפס
המקדם של
$x^2$
ב-%
$x^{n}-1$.
נשמתמש בעובדה זו בעתיד בצורה:
\[
r+r^2+\cdots + r^{n-2}+r^{n-1}=-1\,.
\]
עבור מצולע משוכלל עם 
$17$
צלעות המשוואה היא:
\[
r+r^2+r^3+r^4+r^5+r^6+r^7+r^8+r^9+r^{10}+r^{11}+r^{12}+r^{13}+r^{14} + r^{15}+r^{16}=-1\,,
\]


\section{ההוכחה של 
\L{Gauss}
שניתן לבנות מצולע משוכלל עם 
$17$
צלעות
}\label{s.gauss}
\L{Gauss}
ראה שאין חובה לעבוד עם השורשים בסדר הטבעי שלהם
$r,r^2,\ldots,r^{16}$. 
במקום זה, נשים לב שהחזקות של
$r^3$
נותנות אם כל השורשים, אבל בסדר שונה:
\begin{form}{1.2}
r^1, \;r^{1\cdot 3 =3},\; r^{3\cdot 3=9},\; r^{9\cdot 3=27=10},\; r^{10\cdot 3=30=13},\; r^{13\cdot 3=39=5},\; r^{5\cdot 3=15},\; r^{15\cdot 3=45=11},&&\\
r^{11\cdot 3 =33=16}, \;r^{16\cdot 3=48=14},\; r^{14\cdot 3=42=8},\; r^{8\cdot 3=24=7},\;r^{7\cdot 3=21=4},\; r^{4\cdot 3=12},\; r^{12\cdot 3=36=2},\; r^{2\cdot 3=6}\,.&&
\end{form}
עבור
$k<17$, $r^{17m+k}=(r^{17})^m\cdot r^k=1^m\cdot r^k=r^k$,
ולכן רשמנו את החזקות כשאריות לאחר חלוקה ב-%
$17$.

חשוב שתבדקו שהרשימה כוללת את כל 
$16$
השורשים בדיוק פעם אחת.

\paragraph{%
פסק זמן על משוואות ריבועיות%
}
נתבונן במשוואה הריבועית עם מקדם אחד ל-%
$x^2$:
\[
y^2+py+q=0\,,
\]
ונניח שהשורשים שלה הם:
$a,b$.
אזי:
\[
(y-a)(y-b)=y^2 - (a+b)y + ab\,.
\]
לכן
$p=-(a+b)$
ו-%
$q=ab$,
כך שאם
\textbf{נתונים}
$a+b$
ו-%
$ab$,
נוכל לרשום את המשוואה הריבועית עבורה הם השורשים.%
\footnote{%
\L{Po-Shen Lo}
השתמש בעובדה זו כדי לפתח שיטה מהירה למצוא את השורשים אל משוואה ריבועית. ראו
\cite{lo}
והמסמך על האתר שלי
)הכתובת בכותרת לעיל, והקליקו על הקישור
\L{\texttt{mathematics}}(.}

כעת נשתמש בעובדה זו על משוואות ריבועיות כדי להראות
שניתן לחשב את הקוסינוס של הזווית המרכזית של מצולע משוכלל עם
$17$ 
צלעות באמצעות שורשים ריבועיים בלבד.

יהי
$a_0$
החיבור של השורשים במקומות האי-זוגיים ברשימה לעיל:
\[
a_0=r + r^9 + r^{13} +r^{15} +r^{16} + r^8+r^4+r^2\,,
\]
ויהי
$a_1$
הסכום של השורשים במקומות הזוגיים ברשימה:
\[
a_1=r^3 + r^{10} + r^{5} +r^{11} +r^{14} + r^7+r^{12}+r^6\,.
\]
לפי תוצאה שכבר מצאנו:
\[
a_0+a_1=r + r^2 + \cdots +r^{16}=-1\,.
\]
כעת עלינו לעבוד קשה מאוד כדי לחשב את
$a_0a_1$!
באיור%
~\ref{fig.a0a1}
מופיע החישוב כאשר הערכים של
$r^ir^j$
רשומים לאחר חישוב
$r^{(i+j) \bmod 17}$.
מתחת לכל שורש נמצא מספר המופעים שלו עד כה;
בדקו שכל שורש מופיע בדיוק ארבע פעמיים כך שערכו של המכפלה הוא 
$-4$.
\begin{figure}[tb]
\begin{form}{1.5}
a_0a_1&=&(r + r^9 + r^{13} +r^{15} +r^{16} + r^8+r^4+r^2)\;\cdot\\
&&(r^3 + r^{10} + r^{5} +r^{11} +r^{14} + r^7+r^{12}+r^6)\\
&=&\occ{4}{1} + \occ{11}{1} + \occ{6}{1} + \occ{12}{1} + \occ{15}{1} + \occ{8}{1} + \occ{13}{1} + \occ{7}{1} +\\

&&\occ{12}{2} + \occ{2}{1} + \occ{14}{1} + \occ{3}{1} + \occ{6}{2} + \occ{16}{1} + \occ{4}{2} + \occ{15}{2} +\\

&&\occ{16}{2} + \occ{6}{3} + \occ{1}{1} + \occ{7}{2} + \occ{10}{1} + \occ{3}{2} + \occ{8}{2} + \occ{2}{2}\;\;\: +\\

&&\occ{1}{2} + \occ{8}{3} + \occ{3}{3} + \occ{9}{1} + \occ{12}{3} + \occ{5}{1} + \occ{10}{2} + \occ{4}{3}\;\;\: +\\

&&\occ{2}{3} + \occ{9}{2} + \occ{4}{4} + \occ{10}{3} + \occ{13}{2} + \occ{10}{4} + \occ{11}{2} + \occ{5}{2} \:+\\

&&\occ{11}{3} + \occ{1}{3} + \occ{13}{3} + \occ{2}{4} + \occ{5}{2} + \occ{15}{3} + \occ{3}{4} + \occ{14}{2} \;+\\

&&\occ{7}{3} + \occ{14}{3} + \occ{9}{3} + \occ{15}{4} + \occ{1}{4} + \occ{11}{4} + \occ{16}{3} + \occ{10}{4} +\\

&&\occ{5}{4} + \occ{12}{4} + \occ{7}{4} + \occ{13}{4} + \occ{16}{4} + \occ{9}{4} + \occ{14}{4} + \occ{8}{4}\\
&=&-4\,.
\end{form}\vspace{-2em}
\caption{החישוב של $a_0a_1$}\label{fig.a0a1}
\end{figure}
מ-%
$a_0+a_1=-1$
ו-%
$a_0,a_1=-4$,
אנו יודעים ש-%
$a_0,a_1$
הם השורשים של המשוואה הריבועית:
\[
y^2+y-4=0\,.
\]
מהנוסחה לפתרון של משוואות ריבועיות מתקבל:
\[
a_{0,1} = \frac{-1\pm\sqrt{17}}{2}\,.
\]
יהי
$b_0,b_1,b_2,b_3$
הסכום של כל שורש רביעי החל מ-%
$r^1,r^3,r^9,r^{10}$,
בהתאמה:
\begin{form}{1.3}
b_0&=& r^1+ r^{13} + r^{16} + r^4\\
b_1&=& r^3+ r^{5} + r^{14} + r^{12}\\
b_2&=& r^9+ r^{15} + r^{8} + r^2\\
b_3&=& r^{10}+ r^{11} + r^{7} + r^6\,.\\
\end{form}
בדקו ש-%
$b_0+b_2=a_0, b_1+b_3=a_1$.
נחשב את המכפלות
)איורים~\ref{fig.b0b2}, \ref{fig.b1b}).
\begin{figure}[tb]
\begin{form}{1.3}
b_0b_2&=&(r + r^{13} + r^{16} +r^4)\;\cdot\\
&&(r^9 + r^{15} + r^{8} +r^{2})\\
&=& r^{10}+r^{16}+r^9+r^3+\\
&& r^{5}+r^{11}+r^4+r^{15}+\\
&& r^{8}+r^{14}+r^7+r^1\;\:+\\
&& r^{13}+r^{2}+r^{12}+r^6\\
&=&-1\,.
\end{form}\vspace{-2em}
\caption{החישוב של $b_0b_2$}\label{fig.b0b2}
\end{figure}
\begin{figure}[tb]
\begin{form}{1.3}
b_1b_3&=&(r^3 + r^{5} + r^{14} +r^{12})\times\\
&&(r^{10} + r^{11} + r^{7} +r^{6})\\
&=& r^{13}+r^{14}+r^{10}+r^9\;+\\
&& r^{15}+r^{16}+r^{12}+r^{11}+\\
&& r^{7}+r^{8}+r^4+r^3\quad\;\;+\\
&& r^{5}+r^{6}+r^{2}+r^1\\
&=&-1\,.
\end{form}\vspace{-2em}
\caption{החישוב של $b_1b_3$}\label{fig.b1b3}
\end{figure}

נסכם את החישובים:
To summarize these computations:
\begin{form}{1.2}
b_0+b_2&=&a_0\\
b_0b_2&=&-1\\
b_1+b_3&=&a_1\\
b_1b_3&=&-1\,,
\end{form}
ולכן
$b_0,b_2$ 
הם השורשים של:
\[
y^2-a_0y-1= 0\,.\\
\]
ו-%
$b_1,b_3$
הם השורשים של:
\[
y^2-a_1y-1 =0\,.
\]
מהנוסחה לפתרון משוואות ריבועיות ומהערכים שחישבנו קודם עבור 
$a_0,a_1$,
מתקבלים השורשים
$b_0,b_1$ )איור~\ref{fig.b0}, \ref{fig.b1}).
\begin{figure}[tb]
\begin{form}{2.6}
b_0&=&\disfrac{a_0+\sqrt{a_0^2+4}}{2}\\
&=&\disfrac{
     \disfrac{(-1+\sqrt{17})}{2} + 
     \sqrt{\left[\disfrac{(-1+\sqrt{17})}{2}\right]^2+4}
   }{2}\\
&=&\disfrac{
     (-1+\sqrt{17}) + 
     \sqrt{\left[-1+\sqrt{17}\right]^2+16}
   }{4}\\
&=&\disfrac{
     (-1+\sqrt{17}) + 
     \sqrt{34-2\sqrt{17}}
   }{4}\,,
\end{form}\vspace{-2em}
\caption{החישוב של $b_0$}\label{fig.b0}
\end{figure}
\begin{figure}
\begin{form}{2.8}
b_1&=&\disfrac{a_1+\sqrt{a_1^2+4}}{2}\\
&=&\disfrac{
     \disfrac{(-1-\sqrt{17})}{2} + 
     \sqrt{\left[\disfrac{(-1-\sqrt{17})}{2}\right]^2+4}
   }{2}\\
&=&\disfrac{
     (-1-\sqrt{17}) + 
     \sqrt{\left[-1-\sqrt{17}\right]^2+16}
   }{4}\\
&=&\disfrac{
     (-1-\sqrt{17}) + 
     \sqrt{34+2\sqrt{17}}
   }{4}\,.
\end{form}\vspace{-2em}
\caption{החישוב של $b_1$}\label{fig.b1}
\end{figure}
%\begin{form}{1.4}
%b_1b_3&=&(r^3 + r^{5} + r^{14} +r^{12})\times\\
%&&(r^{10} + r^{11} + r^{7} +r^{6})\\
%&=& r^{13}+r^{14}+r^{10}+r^9\;+\\
%&& r^{15}+r^{16}+r^{12}+r^{11}+\\
%&& r^{7}+r^{8}+r^4+r^3\quad\;\;+\\
%&& r^{5}+r^{6}+r^{2}+r^1\\
%&=&-1\,.
%\end{form}

לבסוף יהי,
$c_0,c_4$ 
הסכום של כל שורש שמיני החל מ-%
$r^1,r^{13}$,
בהתאמה:
\footnote{יש סכומים נוספים אבל שני אלה יספיקו}
\begin{form}{1.4}
c_0&=&r^1+r^{16}\\
c_4&=&r^{13}+r^4\\
c_0+c_4&=&r^1+r^{16}+r^{13}+r^4=b_0\\
c_0c_4&=&(r^1+r^{16})\cdot(r^{13}+r^4)\\
&=&r^{14}+r^5+r^{12}+r^3=b_1\,,
\end{form}
כך ש-%
$c_0,c_4$
הם השורשים של:
\[
y^2-b_0y+b_1=0
\]
נראה שמספיק לחשב את השורש
$c_0=r^1+r^{16}$
)איור~\ref{fig.c0}).
\begin{figure}
\begin{center}
\selectlanguage{english}
\begin{tikzpicture}[scale=1.5]
\coordinate (O) at (0,0) node[left] {$O$} node[above right,xshift=32pt] {$\theta$} node[below right,xshift=32pt] {$\theta$};
\coordinate (A) at (4,0);
\node[right] at (A) {$1$};
\draw (O) -- (A);
\coordinate (C) at (21.12:4cm);
\coordinate (D) at (-21.12:4cm);
\draw (D) arc(-21.12:21.12:4);
\draw (D) -- (O) -- (C);
\node[right] at (C) {$r^1$};
\node[right] at (D) {$r^{16}$};
\draw (C) -- (C |- A) coordinate (B);
\draw (D) -- (D |- A);
%\node[below left] at (B) {$B$};
\draw[rotate=90] (B) rectangle +(6pt,6pt);
\draw (D) -- node[left,xshift=-6pt] {$-\sin\theta$} (A) -- node[left,xshift=-6pt] {$\sin \theta$} (C);
\path (O) -- node[below] {$\cos \theta$} (B);
\end{tikzpicture}
\selectlanguage{hebrew}
\caption{בניית צלע מהזווית המרכזית שהוא כולא}\label{fig.two-cosine}
\end{center}
\end{figure}


\begin{figure}
\begin{form}{2.8}
c_0&=&\disfrac{b_0+\sqrt{b_0^2-4b_1}}{2}\\
&=&\disfrac{
     \disfrac{
     (-1+\sqrt{17}) + 
     \sqrt{34-2\sqrt{17}}
   }{4}}{2} + \\
&& 
    \disfrac{
       \sqrt{\left[\disfrac{
     (-1+\sqrt{17}) + 
     \sqrt{34-2\sqrt{17}}
   }{4}\right]^2-4\left[\disfrac{
     (-1-\sqrt{17}) + 
     \sqrt{34+2\sqrt{17}}
   }{4}\right]}
   }{2}\\
&=&-\disfrac{1}{8}+\disfrac{1}{8}\sqrt{17} + 
     \disfrac{1}{8}\sqrt{34-2\sqrt{17}}
    + \\
   &&
     \disfrac{1}{8}\sqrt{
     \left[
     (-1+\sqrt{17}) + 
     \sqrt{34-2\sqrt{17}}
   \right]^2-16\left[
     (-1-\sqrt{17}) + 
     \sqrt{34+2\sqrt{17}}
   \right]}
\\
&=&-\disfrac{1}{8}+\disfrac{1}{8}\sqrt{17} + 
     \disfrac{1}{8}\sqrt{34-2\sqrt{17}}
    + \\
   &&
     \disfrac{1}{8}\sqrt{
     (-1+\sqrt{17})^2 + 
     2(-1+\sqrt{17})\sqrt{34-2\sqrt{17}}+
     (34-2\sqrt{17})
   -}\\
   &&\overline{
     \left[(-16-16\sqrt{17}) + 
     16\sqrt{34+2\sqrt{17}}\right]
   }
\\
&=&-\disfrac{1}{8}+\disfrac{1}{8}\sqrt{17} + 
     \disfrac{1}{8}\sqrt{34-2\sqrt{17}}
    + \\
   &&
     \disfrac{1}{8}\sqrt{
     68+12\sqrt{17} + 
     2(-1+\sqrt{17})\sqrt{34-2\sqrt{17}}
   -16
     \sqrt{34+2\sqrt{17}}
   }

\end{form}
\caption{החישוב של $c_0$}\label{fig.c0}
\end{figure}

סיימנו כי:
\[
c_0=r_1+r_{16}=2\cos\left(\frac{2\pi}{17}\right)\,,
\]
כפי שניתן לראות באיור%
~\ref{fig.two-cosine}.
קואורדינטות ה-%
$y$
שוות עם סימנים הפוכים ולכן הסכום שלהם אפס. קואורדינטת ה-%
-coordinates are equal but with opposite signs and cancel $x$
נספר פעמיים.

הוכחנו שהקוסינוס שת הזווית המרכזית של מצולע משוכלל עם 
$17$
צלעות עם סרגל ומחוגה, כי הוא מורכב רק ממספרים רציונליים והפעולות 
$\{+,-,\times,\div,\surd\}$:
\begin{form}{2.2}
\cos\left(\disfrac{2\pi}{17}\right) &=& 
-\disfrac{1}{16}+\disfrac{1}{16}\sqrt{17} + 
     \disfrac{1}{16}\sqrt{34-2\sqrt{17}}
    + \\
    &&
     \disfrac{1}{16}\sqrt{
     68+12\sqrt{17} + 
     2(-1+\sqrt{17})\sqrt{34-2\sqrt{17}}
   -16
     \sqrt{34+2\sqrt{17}}
   }
\end{form}

\begin{quote}
\textbf{מספרים מרוכבים}

הקשר בין השורשים והקוסינוס ברור מהייצוג של השורשים כמספרים מרוכבים:
\begin{form}{1.8}
c_0\!\!\!&=&\!\!\!r_1+r_{16}\\
\!\!\!&=&\!\!\!\cos\left(\disfrac{2\pi}{17}\right)\!+i\sin\left(\disfrac{2\pi}{17}\right)+\cos\left(\disfrac{2\cdot 16\pi}{17}\right)\!+i\sin\left(\disfrac{2\cdot 16\pi}{17}\right)\\
\!\!\!&=&\!\!\!\cos\left(\disfrac{2\pi}{17}\right)\!+i\sin\left(\disfrac{2\pi}{17}\right)+\cos\left(\disfrac{-2\pi}{17}\right)\!+i\sin\left(\disfrac{-2\pi}{17}\right)=2\cos\left(\disfrac{2\pi}{17}\right)\,.
\end{form}
\end{quote}



\section{פיתוח הנוסחה של %
\L{Gauss}%
}\label{s.derivation}

נפשט את הביטוי
$2(-1+\sqrt{17})\sqrt{34-2\sqrt{17}}$:
\begin{form}{1.7}
2(-1+\sqrt{17})\sqrt{34-2\sqrt{17}} &=&
-2\sqrt{34-2\sqrt{17}} +2\sqrt{17}\sqrt{34-2\sqrt{17}}+\\
&&4\sqrt{34-2\sqrt{17}}-4\sqrt{34-2\sqrt{17}}\\
&=&
2\sqrt{34-2\sqrt{17}} +2\sqrt{17}\sqrt{34-2\sqrt{17}}+\\
&&-4\sqrt{34-2\sqrt{17}}\\
%&=&\left(-2\sqrt{34-2\sqrt{17}} +2\sqrt{17}\sqrt{34-2\sqrt{17}}
%+4\sqrt{34-2\sqrt{17}}\right)\\
%&&-4\sqrt{34-2\sqrt{17}}\\
&=&2(1+\sqrt{17})\sqrt{34-2\sqrt{17}}-4\sqrt{34-2\sqrt{17}}\,.\\
\end{form}
נזכור את הביטוי
$-4\sqrt{34-2\sqrt{17}}$
ונפשט את הביטוי הראשון, נרבע אותו ואז ניקח את השורש הריבועי:
\begin{form}{1.8}
2(1+\sqrt{17})\sqrt{34-2\sqrt{17}}&=&
2\sqrt{\left[(1+\sqrt{17})\sqrt{34-2\sqrt{17}}\right]^2}\\
&=&2\sqrt{(18+2\sqrt{17})(34-2\sqrt{17})}\\
&=&2\sqrt{(18\cdot 34-4\cdot17)+\sqrt{17}(2\cdot 34 - 2\cdot 18)}\\
&=&2\cdot 4\sqrt{34+2\sqrt{17}}\,.
\end{form}
נציב את הביוטיים ונקבל את הנוסחה של \L{Gauss}:
\begin{form}{2.2}
\cos\left(\disfrac{2\pi}{17}\right) &=& 
-\disfrac{1}{16}+\disfrac{1}{16}\sqrt{17} + 
     \disfrac{1}{16}\sqrt{34-2\sqrt{17}}
    + \\
    &&
     \disfrac{1}{16}\sqrt{
     68+12\sqrt{17} + 
     2\cdot 4\sqrt{34+2\sqrt{17}}-4\sqrt{34-2\sqrt{17}}
   -16
     \sqrt{34+2\sqrt{17}}
   }\\
&=&-\disfrac{1}{16}+\frac{1}{16}\sqrt{17} + 
     \disfrac{1}{16}\sqrt{34-2\sqrt{17}}
    + \\
    &&
     \disfrac{1}{8}\sqrt{
     17+3\sqrt{17} - 
     \sqrt{34-2\sqrt{17}}
   -2
     \sqrt{34+2\sqrt{17}}
   }
\end{form}
נחשב את הביטוי עם מחשבון ונקבל
$\cos\left(\disfrac{2\pi}{17}\right)\approx 0.298$ 
ו-%
$\cos^{-1}0.298\approx 0.3699$ 
רדיאנים או
$21.2926^\circ$.
הערך הנכון הוא
$\disfrac{2\pi}{17}\approx .3696$ 
רדיאנים או
$\disfrac{360^\circ}{17}\approx 21.176^\circ$.

\section{בנייה עם סרגל ומחוגה}\label{s.construction}

מספר בניות נמצאות ב-%
\cite{wiki:heptadecagon}.
כאן אביא את הבנייה מ-%
\cite{callagy}
כי הבנייה היא של
$\cos \disfrac{2\pi}{17}$
שחישבנו.
הבנייה משתמשת רק במשפט פיתגורס ובמשפט חוצי הזוויות
\cite{wiki:bisector}.

נבנה מעגל יחידה שמרכזו
$O$,
המרכז של מערכת צירים, ויהי החיתוכים שלו עם הצירים
$P,Q,R,S$.
נבנה
$A$
כך ש-%
$\overline{OA}=\disfrac{1}{4}\overline{OR}$.
לפי משפט פיתגורס,
$\overline{AP}=\sqrt{(1/4)^2+1^2}=\sqrt{17}/4$.

\begin{center}
\selectlanguage{english}
\begin{tikzpicture}[scale=1.2]
\coordinate (O) at (0,0);
\draw (O) circle (4cm);
\coordinate (P) at (4,0);
\coordinate (R) at (0,4);
\coordinate (Q) at (-4,0);
\coordinate (S) at (0,-4);
\coordinate (A) at (0,1);
\coordinate (B) at (.78,0);
\coordinate (C) at (-1.28,0);
\draw (P) -- (Q);
\draw (R) -- (S);
\path (O) -- node[left,xshift=2pt,yshift=-4pt] {\sm{\frac{1}{4}}} (A);
\draw[dashed] (P) -- ($(P)!1.5!(A)$);
\draw (A) -- node[above] {$\frac{\sqrt{17}}{4}$} (P);
\draw (C) -- (A) -- (B);
\foreach \c/\where in {O/below left, P/right, Q/left, R/above, S/below, A/above left, B/below, C/below} {
  \fill (\c) circle(1.5pt) node[\where] {$\c$};
}
\draw (O) rectangle(+6pt,+6pt);
\node[below right,xshift=8pt,yshift=-4pt] at (A) {\sm{\alpha}};
\node[below right,xshift=-3pt,yshift=-4pt] at (A) {\sm{\alpha}};
\node[below left,xshift=1pt,yshift=-2pt] at (A) {\sm{\beta}};
\node[below left,xshift=-4pt,yshift=4pt] at (A) {\sm{\beta}};
\draw[<->] ($(C)+(0,-16pt)$) -- node[below] {\sm{\frac{1+\sqrt{17}}{16}}} ($(O)+(0,-16pt)$);
\draw[<->] ($(O)+(0,-16pt)$) -- node[below,xshift=4pt] {
\sm{
\frac{\sqrt{34+2\sqrt{17}}}{16}
}
} ($(B)+(0,-16pt)$);
\end{tikzpicture}
\end{center}

יהי
$B$
החיתוך של חוצה הזווית הפנימי של
$\angle OAP$
וציר ה-%
$x$,
ויהי 
$C$
החיתוך החיצוני של 
חוצה הזווית החיצוני של
$\angle OAP$ 
וציר ה-%
$x$.
לפי משפט חוצה הזווית הפנימית:
\begin{form}{2}
\disfrac{OB}{BP}&=&\disfrac{AO}{AP}\\
\disfrac{OB}{1-OB}&=&\disfrac{1/4}{\sqrt{17}/{4}}\\
OB&=&\disfrac{1}{1+\sqrt{17}}\\
&=&\disfrac{1}{1+\sqrt{17}}\cdot \disfrac{1-\sqrt{17}}{1-\sqrt{17}}\\
&=&\disfrac{-1+\sqrt{17}}{16}\,.
\end{form}
ולפי משפט חוצה הזווית החיצוני:
\begin{form}{2}
\disfrac{OC}{CP}&=&\disfrac{AO}{AP}\\
\disfrac{OC}{1-OC}&=&\disfrac{1/4}{\sqrt{17}/{4}}\\
OC&=&\disfrac{1}{-1+\sqrt{17}}\\
&=&\disfrac{1}{-1+\sqrt{17}}\cdot \disfrac{1+\sqrt{17}}{1+\sqrt{17}}\\
&=&\disfrac{1+\sqrt{17}}{16}\,.
\end{form}

בנו
$D$
על
$OP$
כך ש-%
$CD=CA$:
\begin{center}
\selectlanguage{english}
\begin{tikzpicture}[scale=1.5]
\clip (-5,-1.5) rectangle (5,1.5);
\coordinate (O) at (0,0);
\draw (O) circle (4cm);
\coordinate (P) at (4,0);
\coordinate (R) at (0,4);
\coordinate (Q) at (-4,0);
\coordinate (S) at (0,-4);
\coordinate (A) at (0,1);
\coordinate (B) at (.78,0);
\coordinate (C) at (-1.28,0);

\coordinate (D) at (.344,0);
\coordinate (E) at (2.05,0);
\coordinate (M) at (-1.8275,0);
\coordinate (F) at (0,-1);

\draw (P) -- (Q);
\draw (R) -- (S);
\draw (A) -- (P);
\draw (C) -- node[above] {$a$} (A) -- node[above] {$b$} (B);
\path (Q) -- node[below] {$f$}  (M);
\path (B) -- node[below] {$b$} (E);

\path[name path=circ] (M) circle (2.1725cm);
\path[name path=yaxis] (R) -- (S);
\path[name intersections={of=circ and yaxis,by={F1,F}}];
\draw (M) -- node[below] {$f$} (F);

\draw[<->] ($(C)+(0,-8pt)$) -- node[fill=white] {$a$} ($(D)+(0,-8pt)$);

\foreach \c/\where in {O/below left, P/right, Q/left, R/above, S/below, A/above left, B/below, C/below, D/below, E/below, M/below left, F/below left} {
  \fill (\c) circle(1pt) node[\where] {$\c$};
}
\draw (O) rectangle(+6pt,+6pt);
\end{tikzpicture}
\end{center}

\begin{form}{2}
CD=CA&=&\sqrt{OA^2+OC^2}\\
&=&\sqrt{\left(\disfrac{1}{4}\right)^2+\left(\disfrac{1+\sqrt{17}}{16}\right)^2}\\
&=&\disfrac{1}{16}\sqrt{34+2\sqrt{17}}\,.
\end{form}

\selectlanguage{hebrew}

נבנה
$E$
על
$OP$
כך ש-%
$BE=BA$:
\begin{form}{2}
BE=BA&=&\sqrt{OA^2+OB^2}\\
&=&\sqrt{\left(\disfrac{1}{4}\right)^2+\left(\disfrac{1-\sqrt{17}}{16}\right)^2}\\
&=&\disfrac{1}{16}\sqrt{34-2\sqrt{17}}\,.
\end{form}

בנו
$M$,
as the midpoint of $QD$
ובנו
$F$
על
$OS$
כך ש-%
$MF=MQ$:
\begin{form}{2}
MF=MQ&=&\disfrac{1}{2}QD\\
&=&\disfrac{1}{2}(QC+CD)=\disfrac{1}{2}((1-OC)+CD)\\
&=&\disfrac{1}{2}\left[1-\left(\disfrac{1+\sqrt{17}}{16}\right)+\disfrac{\sqrt{34+2\sqrt{17}}}{16}\right]\\
&=&\disfrac{1}{32}\left(15-\sqrt{17}+\sqrt{34+2\sqrt{17}}\right)\,.
\end{form}

בנו מעגל שקטרו 
$OE$.
בנו מיתר
$OG=OF$.
שימו לב ש-%
$MO=1-MQ=1-MF$:
\begin{center}
\selectlanguage{english}
\begin{tikzpicture}[scale=1.5]
\clip (-5,-1.5) rectangle (5,1.5);
\coordinate (O) at (0,0);
\draw (O) circle (4cm);
\coordinate (P) at (4,0);
\coordinate (R) at (0,4);
\coordinate (Q) at (-4,0);
\coordinate (S) at (0,-4);
\coordinate (A) at (0,1);
\coordinate (B) at (.78,0);
\coordinate (C) at (-1.28,0);

\coordinate (D) at (.5,0);
\coordinate (E) at (2.045,0);
\coordinate (M) at (-1.8275,0);

\path[name path=circ] (M) circle (2.1725cm);
\path[name path=yaxis] (R) -- (S);
\path[name intersections={of=circ and yaxis,by={F1,F}}];
\draw (M) -- node[below] {$f$} (F);

\draw[name path=OEarc,thick,dashed] (E) arc(0:-180:1.025cm);
\path[name path=OFcircle] (O)
  let
    \p1 = ($(O)-(F)$)
  in
    circle({veclen(\x1,\y1)});
\path[name intersections={of=OEarc and OFcircle, by={G}}];
\draw (O) -- node[above,near end,xshift=2pt] {$g$} (G) -- node[below] {$e$} (E);
\path (O) -- node[left] {$g$} (F);

\path[name path=EGcircle] (E) 
  let
  \p1 = ($(E)-(G)$)
  in
    circle({veclen(\x1,\y1)});
\draw[name path=PQ] (P) -- (Q);
\path[name intersections={of=PQ and EGcircle, by={H,H1}}];
\path (E) -- node[below] {$e$} (H);

\draw (R) -- (S);
\draw (A) -- (P);
\draw (C) -- (A) -- (B);
\draw (M) -- (F);

\foreach \c/\where in {O/below left, P/right, Q/left, R/above, S/below, A/above left, B/below, C/below, D/below, E/below right, M/below left, F/below left, G/below, H/below} {
  \fill (\c) circle(1pt) node[\where] {$\c$};
}
\draw (O) rectangle +(+6pt,+6pt);
\draw[rotate=30] (G) rectangle +(+6pt,+6pt);

\end{tikzpicture}
\end{center}
\selectlanguage{hebrew}

\begin{form}{2}
OG=OF&=&\sqrt{MF^2-MO^2}=\sqrt{MF^2-(1-MF)^2}\\
&=&\sqrt{2MF-1}\\
&=&\sqrt{\disfrac{1}{16}\left(15-\sqrt{17}+\sqrt{34+2\sqrt{17}}\right)-1}\\
&=&\disfrac{1}{4}\sqrt{-1-\sqrt{17}+\sqrt{34+2\sqrt{17}}}\,.
\end{form}

יהי
$E$
החיתוך של המעגל עם
$OP$;
לפי ההגדרה,
$OE$
הוא קוטר של המעגל כך ש-%
$\angle OGE$
היא זווית ישרה.
בנו
$H$
על
$OP$
כך ש-%
$EH=EG$:
\begin{form}{2}
EH=EG&=&\sqrt{OE^2-OG^2}=\sqrt{(OB+BE)^2-OG^2}\\
&=&\sqrt{\left(\disfrac{-1+\sqrt{17}}{16}+\disfrac{\sqrt{34-2\sqrt{17}}}{16}\right)^2-
\disfrac{1}{16}\left(-1-\sqrt{17}+\sqrt{34+2\sqrt{17}}\right)}
\\
&=&\disfrac{1}{16}\sqrt{\left(
(18-2\sqrt{17})+ 2(-1+\sqrt{17})\sqrt{34-2\sqrt{17}}+
(34-2\sqrt{17})\right)+}\\
&&\quad\quad\quad\overline{
\left(16+16\sqrt{17}-16\sqrt{34+2\sqrt{17}}\right)}\\
&=&\disfrac{1}{16}\sqrt{
68+12\sqrt{17}-16\sqrt{34+2\sqrt{17}}-2(1-\sqrt{17})\sqrt{34-2\sqrt{17}}
}\,.
\end{form}

נחשב את
$OE$:
\begin{form}{2}
OE=OB+BE&=&\disfrac{-1+\sqrt{17}}{16}+\disfrac{1}{16}\sqrt{34-2\sqrt{17}}\\
&=&\disfrac{1}{16}\left(-1+\sqrt{17}+\sqrt{34-2\sqrt{17}}\right)\,.
\end{form}

לבסוף,
$OH=OE+EH$
שהוא
$\cos \disfrac{2\pi}{17}$
כפי שמופיע באיור%
~\ref{fig.c0}.


\selectlanguage{english}
\bibliographystyle{plain}
\bibliography{heptadecagon}

\end{document}