\documentclass[11pt,a4paper]{report}

\usepackage{mathpazo}
\usepackage{microtype}
\usepackage{titlesec}
\usepackage{url}
\usepackage{graphicx}
\usepackage{verbatim}
\usepackage{epic}

\usepackage{tikz}
\usetikzlibrary{external}
\tikzexternalize[prefix=tikz/]

% Use stealth arrows
\tikzset {
  >=stealth
}

% Space out array environment
\newcommand*{\spacearray}{
\renewcommand{\arraystretch}{1.2}
\setlength{\arraycolsep}{4pt}
}

\textwidth=15cm
\textheight=23cm
\topmargin=12pt
\headheight=0pt
\oddsidemargin=2em
\headsep=0pt
\renewcommand{\baselinestretch}{1.1}
\setlength{\parskip}{0.2\baselineskip plus 1pt minus 1pt}
\parindent=0pt


\begin{document}
\thispagestyle{empty}

The rotation matrix for a rotation of $45^\circ$ around the $y$-axis is:
\[
\spacearray
\left[\begin{array}{ccc}\cos\theta&0&\sin\theta\\0&1&0\\-\sin\theta&0&\cos\theta\\\end{array}\right]
=
\left[\begin{array}{ccc}.707&0&.707\\0&1&0\\-.707&0&.707\\\end{array}\right]\,.
\]
Rotating the unit vector along the $x$-axis by $45^\circ$ around the $y$axis gives:
\[
\spacearray
\left[\begin{array}{ccc}.707&0&.707\\0&1&0\\-.707&0&.707\\\end{array}\right]
\left[
\begin{array}{c}1\\0\\0\\\end{array}
\right]=
\left[
\begin{array}{c}.707\\0\\-.707\\\end{array}
\right]\,.
\]

The quaternion for a rotation of $45^\circ$ around the $y$-axis is:
\[
\cos\frac{\theta}{2} + j\sin\frac{\theta}{2}=\cos 22.5^\circ + j\sin 22.5^\circ= .924+.383j\,.
\]
The norm is one as required for a quaternion representing a rotation:
$\sqrt{.924^2+.383^2}=1$.

Rotating the unit vector along the $x$-axis by $45^\circ$ around the $y$axis gives:
\begin{eqnarray*}
q_R v q_R^*&=&(.924 + 0i+.383j+0k)(0 + 1i + 0j + 0k)(.924-0i-.383j-0k)\\
&=&0 + (.853-.147)i + 0j + (-.354-.354)k\\
&\approx& (.707i - .707k)\,,
\end{eqnarray*}
which is what we got from the Euler angle rotation matrix.

This can also be computed by transforming the rotation quaternion $q_0+q_1i+q_2j+q_3k$ into a rotation matrix:
\begin{displaymath}
\spacearray
M=
2\cdot\left[
\begin{array}{lll}
q_0^2+q_1^2-0.5 & q_1q_2-q_0q_3 & q_0q_2+q_1q_3\\
q_0q_3+q_1q_2 & q_0^2+q_2^2-0.5 & q_2q_3-q_0q_1\\
q_1q_3-q_0q_2 & q_0q_1+q_2q_3 & q_0^2+q_3^2-0.5\\
\end{array}
\right].
\end{displaymath}
For $.924+0i+.383j+0k$:
\begin{displaymath}
\spacearray
M=
2\cdot\left[
\begin{array}{ccc}
.924^2-.5 & 0 & .924\cdot .383\\
0& .924^2+.383^2-.5 & 0\\
-.924\cdot .383 & 0 & .924^2-.5\\
\end{array}
\right]=
2\cdot\left[
\begin{array}{ccc}
.354 & 0 & .354\\
0& .5 & 0\\
-.354 & 0 & .354\\
\end{array}
\right]\,.
\end{displaymath}
Rotating the unit vector along the $x$-axis:
\[
2\cdot\left[
\begin{array}{ccc}
.354 & 0 & .354\\
0& .5 & 0\\
-.354 & 0 & .354\\
\end{array}
\right]
\left[
\begin{array}{c}1\\0\\0\\\end{array}
\right]=
\left[
\begin{array}{c}.707\\0\\.-707\\\end{array}
\right]
\]
gives the same result.
\end{document}
