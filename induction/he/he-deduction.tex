% !TeX root = induction-he.tex

\chapter{%
אינדוקציה ודדוקציה%
}\label{s.ind-ded}

עורבים הם ציפורים שחורות. נהוג לומר ש-%
\textbf{כל העורבים שחורים}.
מה ההצדקה לטיעון זה? ברור שיש מספר סופי של עורבים בעולם, ובאופן תיאורטי ניתן לבדוק את כולם ולוודא שכולם שחורים. ברור שאף אחד לא עשה כך. לאחר בחינת צבעם של מספר מסויים של עורבים, אפשר
\textbf{להכליל}
ממספר קטן של תצפיות ולטעון שכל העורבים שחורים.

התהליך של הכללה ממספר קטן של מקרים נקרא
\textbf{אינדוקציה}.
מדענים ופילוסופים מכירים בעובדה שאינדוקציה יכולה להטעות. תמיד יש אפשרות שקיים עורב ירוק, ואם אכן יתגלה עורב ירוק, הטיעון הכללי לא יהיה תקף יותר.%
\footnote{%
מי שרואה עורב ירוק עלול להתפתות ולטעון שהציפור איננה באמת עורב! טיעון זה הוא מקרה של כשל לוגי הנקרא "לא סקוטי אמיתי". ההסקה אינה תקפה כי חייב להיות אוסף קריטריונים קבוע המגדיר מתי ציפור נחשבת כעורב. ציפור התואמת את הקריטריונים היא עורב גם אם צבעה ירוק.%
}
למרות שלא ניתן לסמוך לגמרי על אינדוקציה, נראה שאין ברירה אלא להשתמש בה במדע.

סוד כמוס הוא שמתמטיקאים משתמשים בצורה זו של אינדוקציה. המתמטיקאי בודק מקרים פרטיים רבים, ורק אחר כך מכליל אותם למשפט שמוכיחים אותו בתהליך הנקרא
\textbf{דדוקציה}.
לפעמים עוברות שנים רבות בין ניסוחו של משפט לבין ההוכחה שלו. כאשר מציגים את המשפט במאמר או בספר לימוד, משמיטים את תהליך האינדוקציה ומציגים רק את המשפט וההוכחה הדדוקטיבית. כך נראה כאילו שהמתמטיקאי שלף את המשפט מהאוויר!

בלוגיקה מתמטית חוקרים מערכות דדוקטיביות )אקסיומות וכללי היסק(. בפועל, מתמטיקאים משתמשים בגרסאות לא-פורמליות של מערכות דדוקטיביות, אבל קיימת הסכמה על מה נחשב כשיטת הוכחה תקפה.%
\footnote{%
קיימות מערכות דדוקטיביות כגון
\L{intuitionism}
השונות מהמערכות המקובלות. מערכות אלו נחקרות בלוגיקה מתמטית.%
}
אינדוקציה מתמטית היא כלל היסק שמתמטיקאים מקבלים כתקף ומשמתמשים בה בצורה שגרתית. אין לה קשר עם המושג המדעי-פילוסופי של אינדוקציה.

%%%%%%%%%%%%%%%%%%%%%%%%%%%%%%%%%%%%%%%%%%%%%%%%%%%%%%%%%%%%%%%%%%%
