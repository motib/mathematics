% !TeX root = induction-he.tex

\chapter{%
אינדוקציה מתמטית: אקסיומה נחוצה%
}\label{s.axiom}

מטרת פרק זה היא לנמק את הצורך באינדוקציה מתמטית, לתת הגדרה פורמלית עבורה ולהציג את אופיה כאקסיומה.

\section{%
למה נחוצה אינדוקציה?
}\label{s.why}

נקפוץ ישר להוכחת מספר משפטים:

\begin{theorem}\label{t.odd}
כל מספר שלם אי-זוגי
$n\geq 3$
הוא מספר ראשוני.
\end{theorem}

\textbf{הוכחה}
המספרים
$3,5,7$
הם מספרים ראשוניים. לכן, כל מספר אי-זוגי
$n\geq 3$
הוא ראשוני.
\qed

\begin{theorem}\label{t.odd-prime}
עבור כל מספר ראשוני
$p$, $2^p-1$
הוא מספר ראשוני.
\end{theorem}

\textbf{הוכחה} 
ברור שהמשפט נכון:
\begin{center}
\selectlanguage{english}
\begin{tabular}{|c|c|c|c|c|}
\hline
$p$ & $2$ & $3$ & $5$ & $7$ \\\hline
$2^p-1$ & $3$ & $7$ & $31$ & $127$ \\\hline
\end{tabular}
\end{center}

\qedd{3}

\begin{theorem}\label{t.fermat}
עבור כל מספר שלם
$n\geq 0$, $2^{2^{n}}+1$
הוא מספר ראשוני.\\
מספרים אלה נקראים מספרי
\L{Fermat}.
\end{theorem}
\vspace*{-2ex}
\textbf{הוכחה} 
ברור שהמשפט נכון:
\begin{center}
\selectlanguage{english}
\renewcommand{\arraystretch}{1.3}
\begin{tabular}{|c|c|c|c|c|c|}
\hline
$n$ & $0$ & $1$ & $2$ & $3$ & $4$ \\\hline
$2^{2^{n}}+1$ & $3$ & $5$ & $17$ & $257$ & $65537$ \\\hline
\end{tabular}
\end{center}

\qedd{5}

$9$
הוא לא מספר ראשוני כך שמשפט~%
\ref{t.odd}
אינו נכון. ניתן להראות ש-%
$2^{11}-1=2047=23\times 89$,
לכן גם משפט~
\ref{t.odd-prime}
לא נכון. המתמטיקאי
\L{Pierre de Fermat}
טען במאה ה-71 שמשפט~%
\ref{t.fermat}
נכון, ועברו כמעט מאה שנים עד ש
\L{Leonhard Euler}
הראה שהמשפט לא נכון כי:
\[
2^{2^5}+1 = 2^{32}+1 = 4294967297 = 641 \times 6700417\,.
\]
מספרי
\L{Fermat}
גדלים מאוד ככל ש-%
$n$
גדל. ידוע שהם אינם ראשוניים עבור
$5\leq n \leq 32$,
אבל הפירוק לגורמים של חלק מהמספרים האלה עדיין לא ידוע.

מה לא נכון ב-"הוכחות" שלנו?

המשפטים האלה מביעים תכונות של
\textbf{קבוצה אינסופית}
של מספרים )לכל מספר אי-זוגי
$n\geq 3$,
לכל מספר ראשוני
$p$,
לכל מספר שלם
$n\geq 0$%
(
אבל בדקנו את התכונות רק עבור מספרים ספורים. אנו זקוקים לשיטת הוכחה שתאפשר לנו להוכיח שתכונה מתקיימת
\textbf{לכל}
האיברים בקבוצה אינסופית של מספרים, למרות שברור שההוכחה עצמה צריכה להיות סופית אם ברצוננו לסיים לכתוב אותה לפני קץ הימים.


\section{%
מאיפה מגיעה אינדוקציה?%
}\label{s.where}

\textbf{אינדוקציה מתמטית}
היא שיטה להוכחת תכונות של קבוצות אינסופיות.

לפני שננסח את כלל ההיסק של אינדוקציה נתחיל עם דוגמה:
\[
1+2+3+4+5\qeq{}\frac{5\cdot 6}{2}\,.
\]
ברור ששני צדי המשוואה שווים ל-%
$15$.
מה עם:
\[
1+2+3+4+5+6+7+8+9+10\qeq{}\frac{10\cdot 11}{2}\,.
\]
במעט יותר מאמץ נמצא ששני הצדדים שווים ל-%
$55$.

כעת נניח שהתבקשת לחשב את הסכום:
\[
1+2+3+\cdots+1528+1529\,.
\]
סביר שאתה עצל מדי לחשב את הסכום אפילו עם מחשבון. מפתה להכליל את הדוגמאות הקודמות ולטעון ש:
\[
1+2+3+\cdots+1528+1529=\frac{1529\cdot 1530}{2}\,.
\]
רק כמה שניות דרושות כדי לחשב במחשבון את הצד הימני ולקבל את התוצאה
$1169685$.
בכל זאת, כפי שראינו בסעיף~%
\ref{s.why},
מאוד מסוכן לטעון לנכונות של טענה על קבוצה אינסופית של מספרים לאחר בדיקת מספרים ספורים בלבד.

אפילו אם היתה לנו הוכחה ש:
\[
1+2+3+\cdots+1528+1529=\frac{1529\cdot 1530}{2}\,,
\]
ההוכחה תקפה רק עבור אותה סדרה ולא עבור סדרות אחרות כגון:
\[
1+2+3+\cdots+2997+2998\,.
\]
אנו זקוקים להוכחה שעבור
\textbf{כל}
המספרים השלמים
$n\geq 1$:
\begin{equation}
\sum_{i=1}^n i = \frac{n(n+1)}{2}\,.\label{eq.sum}
\end{equation}
איך בכלל אפשר להוכיח שהמשוואה נכונה עבור כל אינסוף המספרים החיוביים? ברור שאין אפשרות להוכיח מספר אינסופי של משוואות, אבל נוכל לעשות משהו דומה. נניח שאליס טוענת שהיא יכולה להוכיח את המשוואה~%
\ref{eq.sum}
\textbf{לכל מספר שלם חיובי שרירותי}
גדול ככל שיהיה. אם הטענה של אליס נכונה, מתקבל על הדעת שהמשוואה~%
\ref{eq.sum}
נכונה עבור
\textbf{כל}
$n\geq 1$.
כמובן שלא מקובל המתמטיקה להסתמך על "מקבל על הדעת". עלינו למצוא ניסוח פורמלי של הטיעון.


\section{%
אינדוקציה כשיטת הוכחה
}\label{s.rule}

נראה איך אליס יכולה להוכיח את משוואה~%
\ref{eq.sum}
עבור כל מספר שלם שרירותי
$n\geq 1$.

חברה בוב שואל אותה שאלה קלה: האם המשוואה נכונה עבור
$n=1$?
אליס משיבה שלא צריך להיות גאון כדי להסכים לטענה כי:
\begin{equation}
\sum_{i=1}^1 i = 1 = \frac{1(1+1)}{2}\,.\label{eq.one}
\end{equation}

בוב מציג שאלה קשה יותר. האם המשוואה נכונה עבור
$n=2$?
אליס יכולה להוכיח את הטענה בצורה ישירה:
\[
\sum_{i=1}^2 i = 1 + 2 = 3 = \frac{2(2+1)}{2}=3\,,
\]
אבל ככל מתמטיקאי טוב, היא עצלנית מאוד ומעדיפה להשתמש במשפטים שהיא הוכיחה כבר במקום להתחיל מאפס. אליס שמה לב ש:
\[
\sum_{i=1}^2 i = \sum_{i=1}^1 i + 2\,,
\]
ובנוסף ש:
\[
\sum_{i=1}^1 i
\]
הוא הצד השמאלי של המשוואה~%
\ref{eq.one}.
אליס מציבה במקום
$\sum_{i=1}^1 i$
את הצד הימני של משוואה~%
\ref{eq.one}
ומקבלת:
\[
\sum_{i=1}^2 i = \sum_{i=1}^1 i +2 = \frac{1(1+1)}{2} + 2 = \frac{2 + 4}{2} = \frac{2(2+1)}{2}\,.
\]
אליס מסיקה שמשוואה~%
\ref{eq.sum}
נכונה עבור
$n=2$.

מה עם
$n=3$?
אליס משתמשת באותה שיטה כדי להוכיח את המשוואה עבור
$n=3$:
\[
\sum_{i=1}^3 i = \sum_{i=1}^2 i + 3 = \frac{2(2+1)}{2} + 3 = \frac{6+6}{2} = \frac{3(3+1)}{2}\,.
\]

\begin{exercise}
עזור לאליס להוכיח את המשוואה~%
\ref{eq.sum}
עבור
$n=4$.
\end{exercise}

האם אליס יכולה להוכיח את המשוואה~%
\ref{eq.sum}
עבור
$n=1529$?
בוודאי. כל שעליה לעשות הוא לכתוב
$1528$
הוכחות ולהשתמש בשיטה זו כדי להוכיח את הנוסחה עבור
$1529$.
ברור שאליס עצלנית מדי. במקום זה היא טוענת שאין צורך ממש לכתוב את כל ההוכחות האלו, כי "ברור מאליו" שהשיטה עובדת עבור
\textbf{כל} $n$.
בניסוח פורמלי, אליס מציעה להשתמש בעיקרון של אינדוקציה מתמטית.

\section{%
האקסיומה של אינדוקציה מתמטית%
}\label{s.axiom1}

\begin{axiom}[%
אינדוקציה מתמטית%
]\label{ax.induction}
תהי
$P(n)$
תכונה )כגון משוואה, נוסחה או משפט(, כאשר
$n$
מספר שלם חיובי. נניח שניתן:
\begin{itemize}
\item \textbf{טענת בסיס}:
להוכיח ש-%
$P(1)$
נכונה.
\item \textbf{צעד אינדוקטיבי}:
עבור
$m$
שרירותי, להוכיח ש-%
$P(m+1)$
נכונה, בהנחה ש-%
$P(m)$
נכונה.
\end{itemize}
אזי
$P(n)$
נכונה עבור כל
$n\geq 1$.\\
ההנחה ש-%
$P(m)$
נכונה עבור
$m$
נקראת
\textbf{הנחת האינדוקציה}.
\end{axiom}

נוכיח עכשו את המשוואה~%
\ref{eq.sum}
תוך שימוש באינדוקציה מתמטית.
\begin{theorem}\label{t.sum}
עבור
$n\geq 1$:
\[
\sum_{i=1}^n i = \frac{n(n+1)}{2}\,.
\]
\end{theorem}
\textbf{הוכחה} 
הוכחת טענת הבסיס פשוטה:
\[
\sum_{i=1}^1 i = 1 =\frac{1(1+1)}{2}\,.
\]
הנחת האינדוקציה היא שהמשוואה נכונה עבור
$m$:
\[
\sum_{i=1}^{m} i = \frac{m(m+1)}{2}\,.
\]
הצעד האינדוקטיבי הוא להוכיח את המשוואה עבור
$m+1$:
\begin{eqnarray}
\sum_{i=1}^{m+1} i &=& \sum_{i=1}^m i + (m+1)\label{l.sum1}\\
&\ih{}&\frac{m(m+1)}{2} + (m+1)\label{l.sum2}\\
&=&\frac{m(m+1) + 2(m+1)}{2}\label{l.sum3}\\
&=&\frac{(m+1)(m+2)}{2}\,.\label{l.sum4}
\end{eqnarray}
לפי האקסיומה של אינדוקציה מתמטית )אקסיומה~%
\ref{ax.induction}(:
\[
\sum_{i=1}^n i = \frac{n(n+1)}{2}
\]
נכונה עבור כל
$n\geq 1$.\qed

ננמק כעת את השלבים של הצעד האינדוקטיבי. ב-%
)\ref{l.sum1}(
הסכום הוא של שני גורמים: הראשון הוא סכום המספרים מ-%
$1$
ל-%
$m$
והשני הוא המספר
$(m+1)$.
ב~%
)\ref{l.sum2}(,
הסימן
$\ih{}$
מציין שאנו משתמשים
\textbf{בהנחת האינדוקציה}
כדי להציב
$\frac{m(m+1)}{2}$
עבור
$\sum_{i=1}^m i$.
שאר ההוכחה
)\ref{l.sum3}--\ref{l.sum4}(
משתמש באלגרבה פשוטה.

\begin{exercise}
הוכח:
\[
\sum_{i=1}^n i^2 = \frac{n}{6}(n+1)(2n+1).
\]
\end{exercise}

\begin{center}
\fbox{
\parbox{.75\textwidth}{כאשר משתמשים באינדוקציה מתמטית, כדאי תמיד לכתוב באופן מפורש את טענת הבסיס והנחת האינדוקציה.}
}
\end{center}

\begin{center}
\fbox{
\parbox{.75\textwidth}{%
השימוש בהנחת האינדוקציה יכול לבלבל, כי נראה שאנחנו מניחים את מה שאנחנו מנסים להוכיח. אבל ההוכחה היא
\emph{לא}
מעגלית משום שאנחנו מניחים את התכונה עבור משהו
\emph{קטן}
ומשתמשים בהנחה כדי להוכיח משהו
\emph{גדול}
ממנו.}
}
\end{center}

\section{%
לבני דומינו נופלות%
}

אינדוקציה מתמטית דומה לשורה של לבני דומינו הנופלות כולן כאשר מפילים את הלבנה הראשונה. הנפילה של הלבנה הראשונה מפילה את הלבנה השנייה, המפילה את הלבנה השלישית, וכו'. אינדוקציה מתמטית היא הטענה ש: )1( אם מפילים את הלבנה הראשונה, ו-)2( אם כל לבנה נופלת גורמת ללבנה השכנה ליפול, אזי כל הלבנים ייפלו, ללא תלות בכמות הלבנים. אם קשה לכם להאמין, חפשו ביוטיוב קטעי ווידיאו המציגים עשרות ואפילו מאות אלפי לבנים נופלות כאשר מפילים את הראשונה! מומלץ במיוחד היצירות של תלמידה בתיכון\\
\L{Lily Hevesh (\url{http://www.hevesh5.com/})}.

\section{%
אינדוקציה היא אקסיומה}

היסק לוגי במתמטיקה מבוסס על המושג
\textbf{מערכת אקסיומטית}:
מתחילים עם מושגים בסיסיים שאינם מוגדרים. ניתנת רשימת אקסיומות וכללי ההיסק מפרטים איך לבנות הוכחה. המערכת האקסיומתית של
\L{Euclid}
לגיאומטריה במישור מוכרת היטב. היא כוללות מושגים בסיסיים כגון נקודה וקו, ואקסיומות כגון:
\begin{itemize}
\item
בין שתי נקודות ניתן למתוח קו אחד.
\item
בהינתן קו ונקודה שלא נמצאת על הקו, קיים קו אחד שעובר דרך הנקודה והוא מקביל לקו הנתון.
\end{itemize}
כללי היסק לוגיים משמשים להוכחת משפטים. כלל ההיסק השכיח שביותר הוא
\L{\emph{modus ponens (MP)}}:
\begin{itemize}
\item
אם הטענה P גוררת את הטענה Q.
\item
אם הטענה P נכונה.
\item
אזי הטענה Q נכונה.
\end{itemize}



בהוכחה שלהלן
\L{MP}
מצדיק את ההיסק מהטענות בשורות~2 ו-3 למסקנה בשורה 4:

\begin{enumerate}
\item לכל
$a,b,c$,
אם
$a>b$
ו-%
$c>0$
אזי
$ac>bc$
\item 
אם
$9>3$
ו-%
$5>0$
אזי
$9\cdot 5> 3\cdot 5$
\item $9>3$
ו-%
$5>0$
\item $9\cdot 5> 3\cdot 5$
\item $45>15$
\end{enumerate}
במסגרת החקירה והלמידה של לוגיקה מתמטית, כללי ההיסק מוצגים בצורה פורמלית. בעיסוק היומיומי במתמטיקה, משתמשים בכללים בצורה לא פורמלית, כי הם מוכרים ומקובלים. אקסיומות אמורות להיות טענות שנכונותן ברורה, אבל האמיתות של האקסיומות
\textbf{אינה}
רלוונטית לתקפות של הוכחה! חשוב רק שהמערכת הדדוקטיבית תהיה
\textbf{נאותה},
שמשמעותה היא:
\begin{quote}
\textbf{אם}
האקסיומות נכונות,
\textbf{אז}
המשפטים המוכחים מהאקסיומות גם הם נכונים.
\end{quote}
ההוכחה הבאה היא נאותה למרות שהמסקנה אינה נכונה, כי הטענה בשורה 3 אינה נכונה:
\begin{enumerate}
\item לכל
$a,b,c$,
אם
$a>b$
ו-%
$c>0$
אזי
$ac>bc$
\item 
אם
$9>3$
ו-%
$-5>0$
אזי
$9\cdot -5> 3\cdot 5$
\item $9>3$
ו-%
$-5>0$
\item $9\cdot -5> 3\cdot -5$
\item $-45>-15$
\end{enumerate}

במאה ה-91 מתמטיקאים התחילו לחקור מה יקרה אם במערכת האקסיומות של
\L{Euclid}
יחליפו את אקסיומת הקווים המקבילים באחת מהאקסיומות הבאות:
\begin{itemize}
\item
בהינתן קו ונקודה שלא נמצאת על הקו, לא קיים קו שעובר דרך הנקודה והוא מקביל לקו הנתון.
\item
בהינתן קו ונקודה שלא נמצאת על הקו, קיימים אינסוף קווים שעוברים דרך הנקודה והוא מקביל לקו הנתון.
\end{itemize}
החלפת האקסיומה יצרה גיאומטריות אחרות, השונות מגיאומטריית המישור המוכרת. עם זאת, התיאוריות הגיאומטריות החדשות הן
\textbf{עקביות},
 שמשמעותה שלא תיווצר סתירה בפיתוח התיאוריות. הגיאומטריות החדשות התגלו כשימושיות, למשל, בתיאוריית היחסות של איינשטיין.

מערכת אקסיומטית עובר המספרים הטבעים לא פותח עד תחילת המאה ה-02. היא כוללת אקסיומות ברורות מאליהן כגון:

\begin{itemize}
\item $x+0=x$.
\item $x_1=x_2$
ו-%
$x_1=x_3$
גוררות
$x_2=x_3$.
\end{itemize}
אינדוקציה מתמטית היא כלל היסק במערכות, לכן אין כלל שאלה האם אפשר
\textbf{להוכיח}
שהיא נכונה. עליך לקבל אותה כמו שאתה מקבל כל אנקסיומה אחרת, כמו
$x+0=x$.
כמובן שאתה רשאי לדחות את האקסיומה של אינדוקציה מתמטית, אבל אז עליך לדחות כמעט את כל המתמטיקה המודרנית.

אפשר להחליף את אקסיומת האינדוקציה באקסיומה אחרת הנקראת
\textbf{עיקרון הסדר הטוב}.
שתי האקסיומות שקולות במובן שכל אחת נובעת מהשנייה. עיקרון הסדר הטוב הוא יותר אינטואיטיבי מאינדוקציה, אך קל יותר להשתמש באינדוקציה. פרק~%
\ref{s.well}
מביא את עיקרון הסדר הטוב והוכחה ששתי האקסיומות שקולות.

כמובן שאינדוקציה אינה השיטה היחידה להוכחת משפטים במתמטיקה. אולם, כמעט תמיד משתמשים באינדוקציה להוכחת משפטים על תכונות של קבוצות אינסופיות כגון כל המספרים השלמים או כל הפוליגונים. אינדוקציה מתאימה כאשר מבנה בנוי ממבנים קטנים יותר עד לרמה שהוכחה עבור המבנה הקטן ביותר היא ממש פשוטה. נתחיל עם משפטים על המספרים החיוביים או לא-שליליים, כאשר המספר הקטן ביותר הוא 1 או 0, והמספרים הגדולים יותר בנויים, למשל, מחיבור של מספרים קטנים. בהמשך נכליל את שיטת האינדוקציה למבנים אחרים.

%%%%%%%%%%%%%%%%%%%%%%%%%%%%%%%%%%%%%%%%%%%%%%%%%%%%%%%%%%%%%%%%%%%
