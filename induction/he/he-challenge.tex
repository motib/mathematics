% !TeX root = induction-he.tex

\appendix

\chapter[תרגילי אתגר]{%
\vspace*{-2ex}
תרגילי אתגר%
}\label{a.challenge}

\vspace*{-4ex}

בתרגילים אלה נדרשת רק אלגברה ברמה של בית ספר תיכון, אבל התרגילים לא פשוטים.

נסה לפתור אותם בלי להסתכל ברמזים בעמוד הבא.

\section{%
התרגילים%
}

\begin{exercise}\label{e.coins}
הנח שיש לך מספר בלתי מוגבל של מטבעות בערכים של
$4$
ש"ח ו-%
$7$
ש"ח. מה המספר הקטן ביותר
$n$
המקיים את התכונה הבאה: ניתן לשלם
\textbf{כל}
סכום גדול או שווה ל
$n$
ש"ח עם המטבעות האלה?
\end{exercise}

\textbf{דוגמה}
לא ניתן לשלם את הסכום
$10$
ש"ח, כי לא ניתן לחלק אותו ב-%
$4$
או
$7$
בלבד, וכל צירוף של ערכים אלה יהיה גדול מ-%
$10$.

\begin{exercise}\label{e.proper}
\textbf{שבר אמיתי}
הוא שבר שהמונה שלו קטן מהמכנה שלו. 
\textbf{שבר יסודי}
הוא שבר אמיתי שהמונה שלו הוא
$1$.
הוכח שכל שבר אמיתי שווה לסכום של שברים יסודיים
\textbf{שונים}.
\end{exercise}


\textbf{דוגמה}
קל מאוד לבטא שבר אמיתי כסכום של שברים יסודיים:
\[
\frac{4}{5} = \frac{1}{5} + \frac{1}{5} + \frac{1}{5} + \frac{1}{5}\,,
\]
אבל קשה למצוא סכום של שברים יסודיים שונים:
\[
\frac{4}{5} = \frac{16}{20} = \frac{1}{2} + \frac{1}{4} + \frac{1}{20}\,.
\]

\vspace*{-1ex}

\begin{exercise}\label{e.binet}
הוכח את הנוסחה של
\L{\emph{Binet}}
למספרי פיבונצ'י:
\begin{displaymath}
f_n = \frac{\phi^n - \bar{\phi}^n}{\sqrt{5}}, \;\;\;\;\;
\phi = \frac{1+\sqrt{5}}{2},\;\bar{\phi} = \frac{1-\sqrt{5}}{2}\,.
\end{displaymath}
\end{exercise}

\vspace*{-3ex}

\begin{exercise}\label{e.pascal}
הוכח:
\[
f_n = {n \choose 0} + {n-1 \choose 1} + {n-2 \choose 2} + \cdots.
\]
\end{exercise}

\section{%
רמזים%
}

\textbf{תרגיל
~\ref{e.coins}}
\begin{itemize}
\item
הנח שיכולת לבנות כל המספרים הזוגיים )מעל למספר מסויים(. הראה שניתן לבנות את כל המספרים )מעל למספר מסויים(.
\item 
אם
$n$
מתחלק ב-%
$4$,
האם ניתן לבנות את
$n$?
\item 
מצא את המספר הגדול ביותר 
$k$
שכנראה לא ניתן לבנות.
\item
הראה באמצעות אינדוקציה שניתן לבנות
$n$,
עבור כל
$n\geq k+1$, $n$
זוגי ו-%
$n$
לא ניתן לחלוקה ב-%
$4$.
\end{itemize}

\medskip

\textbf{תרגיל
~\ref{e.proper}}
\begin{itemize}
\item 
מהי טענת הבסיס הפשוטה?
\item 
עבור
$\frac{a}{b}$,
כאשר
$a>1$,
יהי
\[
\frac{1}{q} < \frac{a}{b}\,.
\]
אזי:
\[
\frac{a}{b} = \frac{1}{q} + \left( \frac{a}{b} - \frac{1}{q} \right).
\]
עכשיו השתמש באינדוקציה.
\item 
הראה שאם
$\frac{1}{q}$
הוא השבר היסודי הגודל ביותר שהוא פחות מ-%
$\frac{a}{b}$
כך ש:
\[
\frac{a}{b} < \frac{1}{q-1}\,,
\]
אזי כל השברים היסודיים שונים.
\end{itemize}

\textbf{תרגיל
~\ref{e.binet}} 
הוכח
$\phi^2=\phi+1$
ו-%
$\bar{\phi}^2=\bar{\phi}+1$.

\medskip

\textbf{תרגיל
~\ref{e.pascal}}
הוכח את החוק של
\L{Pascal}:
\[
{n \choose k} + {n \choose k+1} = {n+1 \choose k+1}.
\]

%%%%%%%%%%%%%%%%%%%%%%%%%%%%%%%%%%%%%%%%%%%%%%%%%%%%%%%%%%%%%%%%%%%
