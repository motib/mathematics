% !TeX root = induction-he.tex

\chapter{%
מוקשים שיש להיזהר מהם
}\label{s.watch}

\section{%
אינדוקציה היא לא דרך ההוכחה היחידה%
}

אינדוקציה נמצאת בשימוש נרחב בהוכחות במתמטיקה ולפעמים היא השיטה הטובה ביותר להוכחת משפט, אבל היא לא שיטת ההוכחה היחידה. הנה הוכחה של משפט~%
\ref{t.div2}
שלא משתמשת באינדוקציה.
\begin{theorem}
לכל
$n\geq 1$, $n(n+1)$
מתחלק ב-%
$2$.
\end{theorem}

\textbf{הוכחה} 
אם
$n$
מתחלק ב-%
$2$
המשפט נכון. אחרת,
$n=2k+1$
עבור
$k$
כלשהו. לכן,
$n+1=2k+2=2(k+1)$
מתחלק ב-%
$2$.\qed

זו בעצם אותה הוכחה כמו הצעד האינדוקטיבי בהוכחה באמצעות אינדוקציה.
\begin{exercise}
הוכח ללא שימוש באינדוקציה: לכל
$n\geq 1$, $n(n+1)(n+2)$
מתחלק ב-%
$3$.
\end{exercise}


\L{Carl Friedrich Gauss}
היה מגדולים המתמטיקאים בכל הזמנים. לפי האגדה, כאשר התפרע מעט בבית הספר היסודי, המורה הטיל עליו לסכם את המספרים החיוביים מ-
$1$
ל-%
$100$,
בתקווה שזה יעסיק אותו זמן ממושך.
\L{Gauss}
פתר את הבעיה מיד כאשר שם לב שניתן להציג את המספרים כך:
\[
\begin{array}{crrrrrrr}
& 1 & 2 & 3 & \cdots & 98 & 99 & 100\\
+& 100 & 99 & 98 & \cdots & 3 & 2 & 1\\
\hline
& 101 & 101 & 101 & \cdots & 101 & 101 & 101\\
\end{array}
\]
ומיד מקבלים את הנוסחה
$\frac{100\cdot 101}{2}$
שהוכחנו במשפט~%
\ref{t.sum}. 

ניתן גם להוכיח את המשפט עבור
$n$
שרירותי כך:
\begin{eqnarray*}
\sum_{i=1}^{n} i &=& \sum_{i=1}^{n} (n-i+1)\\
&=& \frac{1}{2}\left[\sum_{i=1}^{n} i + \sum_{i=1}^{n} (n-i+1)\right]\\
&=& \frac{1}{2}\left[\sum_{i=1}^{n} (n+1)\right]\\
&=& \frac{n(n+1)}{2}\,.
\end{eqnarray*}
\vspace*{-3ex}
\begin{exercise}
הוכח עם ובלי אינדוקציה ש-%
$x-1$
מחלק את
$x^n-1$.
איזו הוכחה עדיפה בעיניך?
\end{exercise}
\textbf{רמז}
להוכחה ללא אינדוקציה, חשב:
\[
(x-1)\sum_{i=0}^{i=n}x^i\,.
\]
להוכחה באינדוקציה, מצא פולינומים
$p(x),q(x)$
ששניהם מתחלקים ב-%
$x-1$,
כך ש-%
$x^{n+1} - 1 =p(x)+q(x)$.

\section{%
לפעמים אי אפשר להשתמש באינדוקציה%
}

ניתן להשמתמש באינדוקציה רק אם המבנים הגדולים נוצרים צעד אחר צעד ממבנים קטנים יותר, ואם יש מבנה קטן ביותר. לא ניתן להרחיב את המשפטים שהוכחנו למספרים רציונליים כי אין מספר רציונלי "ראשון".%
\footnote{%
\L{Georg Cantor}
מצא דרך לסדר את המספרים הרציונלים בסדרה: 
$\{\frac{1}{1},\frac{1}{2},\frac{2}{1},\frac{1}{3},\frac{3}{1},\frac{1}{4},\frac{2}{3},\frac{3}{2},\frac{4}{1},\frac{1}{5},\frac{5}{1},\ldots\}$,
אבל סדרה זו היא לא אינטואיטיבית ולא מתאימה להוכחות באינדוקציה.%
}
נניח שאנו רוצים להוכיח שלכל זוג מספרים רציונלים
\textbf{חיובים}
$a,b$,
כך ש-%
$a<b$
מתקיים
$a\leq \frac{a+b}{2} \leq b$.
\begin{itemize}
\item
מהי טענת הבסיס? נניח ש-%
$a$
הוא המספר הרציונלי החיובי הקטן ביותר. אבל המספר
$\frac{a}{2}$
הוא רציונלי, חיובי וקטן מ-%
$a$.
\item
מהי הנחת האינדוקציה? מהו הצעד האינדוקטיבי? נניח שהנחת האינדוקציה היא שהנוסחה נכונה עבור 
$a,b$.
בצעד האינדוקטיבי יש להוכיח את נכונות הנוסחה עבור
\textbf{זוג הערכים הבא},
שניתן לכתוב אותו כ-%
$a+\frac{p_a}{q_a}$, $b+\frac{p_b}{q_b}$.
אבל
$a+\frac{p_a}{q_a+1}$
גדול מ-%
$a$
וקטן מ-%
$a+\frac{p_a}{q_a}$,
כך שאין "זוג הערכים הבא".
\end{itemize}

\section{%
היזהרו מהוכחות לא נכונות%
}

\begin{theorem}
לכל
$n\geq 0$
ו-%
$a\geq 1$, $a^n=1$.
\end{theorem}

\textbf{הוכחה}
טענת בסיס:
$a^0=1$.
הנחת האינדוקציה:
$a^k=1$
עבור
$0\leq k\leq n$.
הצעד האינדוקטיבי הוא:
\[
a^{n+1}=a^n\cdot a \ih{} 1\cdot a = a = \frac{a^{n-1}}{a^{n-2}} \ih{} \frac{a^{n-1}}{1} \ih{} \frac{1}{1} = 1.
\]

\qedd{6}

כמובן שזוהי שטות כי 
$2^3=8 \neq 1$.

\begin{exercise}
איפה השגיאה בהוכחה?
\end{exercise}

\begin{theorem}
לכל התלמידים בכיתה מסויימת יש אותו צבע שיער.
\end{theorem}
\textbf{הוכחה}
טענת בסיס: בחר תלמיד כלשהו
$s_1$.
לתלמיד צבע שיער
$c$.
הנחת האינדוקציה: בכל קבוצה של 
$n$
תלמידים, לכולם צבע שיער
$c$.
צעד אינדוקטיבי:
יהיו
$s_1,s_2,\ldots,s_n,s_{n+1}$
התלמידים בקבוצה שבה
$n+1$
תלמידים. נבדוק את התת-קבוצה 
$s_1,s_2,\ldots,s_n$.
לפי הנחת האינדוקציה צבע השיער שלהם הוא
$c$.
באופן דומה, בתת-קבוצה
$s_1,s_2,\ldots,s_{n-1},s_{n+1}$,
לכולם צבע שיער
$c$.
לכן, צבע השיער של תלמיד
$s_{n+1}$
הוא
$c$,
בדיוק כמו צבע השיער של
$s_1$,
שהוא אותו צבע שיש לתלמידים
$s_2,\ldots,s_n$.
לכן, לכל התלמידים
$s_1,s_2,\ldots,s_n,s_{n+1}$
צבע שיער
$c$.\qed

ברור שלתלמידים שונים יש צבע שיער שונה )שלא לדבר על אלה שצובעים בכל גווני הקשת( כך שההוכחה שגויה.

\begin{exercise}
איפה השגיאה בהוכחה?
\end{exercise}

%%%%%%%%%%%%%%%%%%%%%%%%%%%%%%%%%%%%%%%%%%%%%%%%%%%%%%%%%%%%%%%%%%%
