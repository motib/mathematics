% !TeX root = induction-he.tex

\chapter{%
אינדוקציה מעל המספרים השלמים%
}\label{s.integers}

הפרק זה מציג דוגמאות לתכונות של מספרים שלמים הניתנות להוכחה באמצעות אינדוקציה מתמטית.

\section{%
לא רק שוויונות%
}

התכונה הראשונה שהוכחנו באמצעות אינדוקציה היתה השוויון:
\[
\sum_{i=1}^n i = \frac{n(n+1)}{2}\,.
\]
ניתן גם להוכיח אי-שוויונות באמצעות אינדוקציה:
\begin{theorem}
עבור כל
$n\geq 1$, $2^n \geq n+1$.
\end{theorem}

\textbf{הוכחה}
טענת הבסיס פשוטה מאוד להוכחה:
$2^1=2 \geq 1+1 = 2$.
הנחת האינדוקציה היא 
$2^n \geq n+1$.%
\footnote{%
בהצגה של האקסיומה בסעיף~%
\ref{s.axiom1},
השתמשנו בשמות שונים עבור המשתנה בהנחת האינדוקציה
$m$
והמשתנה בצעד האינדוקטיבי
$n$.
מכאן והלאה, נשתמש בשם 
$n$
גם בהנחה וגם בצעד. זה לא אמור לבלבל.%
}
הצעד האינדוקטיבי הוא להוכיח ש-%
$2^{n+1} \geq (n+1)+1$:
\[
2^{n+1}= 2^n\cdot 2 \ihge{} (n+1)\cdot 2 = 2n + 2 \geq n+2 = (n+1)+1\,.
\]
ההנחה ש-
$n$
חיובי מצדיקה את המסקנה ש-%
$2n+2\geq n+2$.
על פי אקסיומת האינדוקציה, האי-שיוון
$2^n \geq n+1$
מתקיים לכל
$n\geq 1$.\qed

\begin{exercise}
לכל
$n\geq 1$, $2n! \,\geq\, 2^n$.
\end{exercise}

\section{%
לא רק משוואות%
}

\begin{theorem}\label{t.div2}
עבור כל
$n\geq 1$, $n(n+1)$
מתחלק ב-%
$2$.
\end{theorem}

\textbf{הוכחה}
טענת הבסיס פשוטה מאוד להוכחה:
$1\cdot (1+1) = 2$
מתחלק ב-%
$2$.
הנחת האינדוקציה היא ש-%
$n(n+1)$
מתחלק ב-%
$2$,
והצעד האינדוקטיבי הוא להוכיח ש-
\[
(n+1)((n+1)+1)=(n+1)(n+2)
\]
מתחלק ב-%
$2$.
לפי ההנחה,
$n(n+1)$
מתחלק ב-%
$2$,
כך שיש שתי אפשרויות:

אפשרות
$1$: $n+1$
מתחלק ב-%
$2$.
אם כן, ברור ש-%
$(n+1)(n+2)$
גם מתחלק ב-%
$2$.

אפשרות
$2$: $n$
מתחלק ב-%
$2$.
אם כך, קיים
$k\geq 1$
כך ש-%
$n=2k$.
אבל:
\[
n+2 = 2k+2 = 2(k+1)\,,
\]
ולכן
$n+2$
וגם
$(n+1)(n+2)$
מתחלקים ב-%
$2$.\qed

\begin{exercise}\label{e.div3}
עבור כל
$n\geq 1$, $n(n+1)(n+2)$
מתחלק ב-%
$3$.
\end{exercise}

\begin{exercise}\label{e.div6}
עבור כל
$n\geq 1$, $n^3-n$
מתחלק ב-%
$6$.
\end{exercise}

\begin{exercise}
עיקרון שובך היונים: אי-אפשר להניח
$n+1$
יונים ב-%
$n$
תאים כך שיש לכל היותר יונה אחת בכל תא.%
\footnote{%
זה אולי נשמע כבדיחה אבל עיקרון שובך היונים הוא שימושי מאוד בהוכחות שונות במתמטיקה.%
}
\end{exercise}

\section{%
לא רק טענת בסיס מ-%
$1$%
}

\begin{theorem}
עבור כל
$n\geq 1$, $n^2\geq 2n+1$.
\end{theorem}
\textbf{הוכחה}
טענת הבסיס קלה להוכחה:
\[
1^2 = 1 \qge 2\cdot 1+1 = 3\,.
\]
משהו לא בסדר! נבדוק אם המשפט נכון עבור
$n=2$:
\[
2^2 = 4 \qge 2\cdot 2+1 = 5\,.
\]
עדיין לא נכון. נבדוק
$n=3$:
\[
3^2 = 9 \qge 2\cdot 3+1 = 7\,.
\]
נראה בסדר. האם אפשר להוכיח את צעד האינדוקציה? הנחת האינדוקציה היא
$n^2 \geq 2n+1$.
לכן, עבור כל
$n+1$:
\[
(n+1)^2 = n^2 + 2n + 1 \ihge{}
(2n+1)+(2n+1) = 2(n+1) + 2n \geq 2(n+1)+1\,,
\]
כי
$n$
חיובי. מכאן שלכל 
$n\geq 3$, $n^2\geq 2n+1$.\qed

המשפט כפי שניסחנו לא נכון, אבל אם נשנה את המשפט כך שלא נטען שהוא נכון עבור
$1,2$,
המשפט נכון וניתן להוכיח אותו.

נניח שיש לנו שורה של לבני דומינו ונסלק את שני הראשונים: עדיין שאר הלבנים תיפולנה כאשר נפיל את הלבנה השלישית. העיקרון של אינדוקציה מתמטית הוא ללא שינוי: טענת הבסיס יכולה לטעון עבור מספר כלשהו והתכונה נכונה עבור כל מספר גדול או שווה לו.

\begin{exercise}
עבור איזה מספרים מתקיים
$2^n\geq n^2$?
הוכח באינדוקציה.
\end{exercise}

\section{%
לא רק צעד אינדוקטיבי של
$n+1$}

\begin{theorem}
הסכום של
$n$
המספרים האי-זוגיים הראשונים הוא
$n^2$:
\[
\overbrace{1 + 3 + \cdots + (2n-1)}^{n} = n^2\,.
\]
\end{theorem}
\vspace{-5ex}
\textbf{הוכחה}
ההוכחה של טענת הבסיס פשוטה:
$2\cdot 1-1=1=1^2$.
הנחת האינדוקציה היא:
\[
\overbrace{1 + 3 + \cdots + (2n-1)}^{n} = n^2\,.
\]
הצעד האינדוקטיבי הוא:
\begin{eqnarray*}
\overbrace{1 + 3 + \cdots + (2n-1) + (2n+1)}^{n+1} &=& \overbrace{1 + 3 + \cdots + (2n-1)}^{n} + (2n+1)\\
&\ih{}& n^2 + (2n+1)\\
&=&(n+1)^2\,.
\end{eqnarray*}

\qedd{7}

כל צעד באינדוקציה מוסיף
$2$
לאיבר בסכום מ-%
$2n-1$
ל-%
$2n+1$,
שלא כמו בדוגמאות הקודמות בהן הצעד היה
$1$.
אולם, עם סימון טוב יותר, ניתן לכתוב את הצעד האינדוקטיבי כך שהצעד הוא מ-%
$n$
ל-%
$n+1$.
\begin{theorem}
הסכום של
$n$
המספרים האי-זוגיים הוא
$n^2$:
\[
\sum_{i=1}^n (2i-1) = n^2\,.
\]
\end{theorem}
\textbf{הוכחה}
הוכחת טענת הבסיס פשוטה ביותר:
$2\cdot 1-1=1=1^2$.
הנחת האינדוקציה היא
$\sum_{i=1}^n (2i-1) = n^2$
והצעד האינדוקטיבי הוא:
\[
\sum_{i=1}^{n+1} (2i-1) \ih{} n^2 + (2(n+1)-1) = n^2 + 2n + 1 = (n+1)^2\,.
\]

\qedd{7}

\begin{exercise}
הסכום של
$n$
המספרים הזוגיים הוא
$n(n+1)$.
\end{exercise}

\section{%
לא רק טענת בסיס אחת%
}

\begin{theorem}\label{th.three}
לכל
$n\geq 1$,
כל מספר עם
$3n$
ספרות זהות מתחלק ב-%
$3$.
\end{theorem}

\textbf{הוכחה}
טענת הבסיס: כל מספר עם
$3$
ספרות זהות מתחלק ב-%
$3$.
קיימות עשר )!( טענות בסיס, ואין לנו ברירה אלא להוכיח כל אחת בנפרד:
\[
\begin{array}{lllll}
000=3\cdot 0 & \;111 = 3\cdot 37 & \;222 = 3\cdot 74 &
\;333=3\cdot 111 & \;444=3\cdot 148\\
555=3\cdot 185 & \;666=3\cdot 222 & \;777=3\cdot 259 &
\;888=3\cdot 296 & \;999=3\cdot 333.
\end{array}
\]
הצעד האינדוקטיבי פשוט יחסית. לפי הנחת האינדוקציה, מספר עם
$3n$
ספרות זהות
$k$
מתחלק ב-%
$3$.
לכן:
\[
\overbrace{kkk}^{n+1} = \overbrace{kkk}^{n}\cdot 1000 + kkk \ih{}
3i\cdot 1000 + kkk.
\]
ברור שהגורם הראשון מתחלק ב-%
$3$
וניתן להראות שהגורם השני מתחלק ב-%
$3$
לפי החישובים של טענות הבסיס.
\qed

משפט זה הוא מקרה קיצוני של עשר טענות בסיס, אבל לא נדיר להיתקל בהוכחות המחייבות יותר מטענת בסיס אחת.

\begin{exercise}
לכל
$n\geq 1$,
כל מספר עם
$3^n$
ספרות זהות מתחלק ב-%
$3^n$.
\end{exercise}
\textbf{רמז}
לכל
$k\geq 1$,
מה השארית של
$10^k$
לאחר חלוקה ב-%
$3$?

\section{%
לא רק הנחת אינדוקציה אחת%
}

\begin{theorem}\label{t.prime}
יהי
$n>1$.
אזי ניתן לפרק את
$n$
למכפלה של מספרים ראשוניים.
\end{theorem}

\textbf{הוכחה} 
טענת הבסיס היא עבור מספר ראשוני
$n$
ואין מה להוכיח.
אם
$n$
אינו ראשוני,
$n=n_1 n_2$
עבור
$2 < n_1, n_2 < n$.
הנחת האינדוקציה היא ש-%
\textbf{כל}
מספר
$1<m<n$
ניתן לפרק למכפלה של מספרים ראשוניים. הצעד האינדוקטיבי: לפי הנחת האינדוקציה,
$n_1=p_1' \cdots p_{k_1}'$
ו-%
$n_2=p_1'' \cdots p_{k_2}''$
ולכן:
\[
n = n_1 n_2 = p_1' \cdots p_{k_1}' p_1'' \cdots p_{k_2}'',
\]
שהוא פירוק של
$n$
למכפלה של מספרים ראשוניים.
\qed

אינדוקציה זו שונה ממקרים קודמים כי הצעד לא מוכיח 
$P(n+1)$
מ-%
$P(n)$.
כאן מוכיחים את
$P(n)$
מ-%
$P(n_1)$
ו-%
$P(n_2)$,
כאשר 
$n_1,n_2$
שניהם קטנים מ-%
$n$
ואפילו קטנים מאוד. למשל,
$n=945$
מתפרק לשני גורמים
$21$
ו-%
$45$.
לפי הנחת האינדוקציה,
$m$
מתפרק לגורמים ראשוניים עבור
\emph{כל}
$m<945$,
כאן,
$21=3\cdot 7$
ו-%
$45=3\cdot 3\cdot 5$,
כך ש-%
$945=3\cdot 3\cdot 3\cdot 5\cdot 7$.

\begin{axiom}[%
אינדוקציה מתמטית שלמה%
]
תהי
$P(n)$
תכונה כאשר
$n$
הוא מספר שלם. נניח שניתן:
\begin{itemize}
\item \textbf{טענת בסיס}:
להוכיח ש-%
$P(n_0)$
נכונה עבור מספר כלשהו
$n_0$.\footnote{%
אנו כבר לא מגבילים את עצמנו לטענת בסיס עבור
$n=1$.} 
\item \textbf{צעד אינדוקטיבי}:
עבור מספר שרירותי
$m$,
להוכיח ש-%
$P(m+1)$
נכונה בהנחה ש-%
$P(k)$
נכונה
\textbf{לכל}
$n_0\leq k\leq m$.
\end{itemize}
אזי
$P(n)$
נכונה
\emph{לכל}
$n\geq n_0$.
\end{axiom}

אינדוקציה שלמה דומה למקרה שבו לבנת דומינו נופלת רק אם היא מקבלת מכה מיותר מאשר לבנה אחת.

אינדוקציה שלמה היא לא ממש הרחבה של המושג אינדוקציה. באינדוקציה מתמטית רגילה, כדי להוכיח
$P(n+1)$
השתמשנו בעובדה שניתן להוכיח
$P(n)$
וכדי להוכיח
$P(n)$
השתמשנו בעובדה שניתן להוכיח
$P(n-1)$, \ldots,
עד
$P(1)$.
אם כן, למה לא להניח בצורה מפורשת את כל הטענות
$P(1),\ldots,P(n)$?
למעשה, ניתן לקבוע אינדוקציה שלמה כאקסיומה כי שתי הצורות ניתנות להוכחה אחת מהשנייה.

\begin{exercise}
יהי
$a_1=5, a_2=7, a_n=3a_{n-1} - 2a_{n-2}$.
אזי
$a_n=3+2^n$.
\end{exercise}

\section{%
אינדוקציה לא-מפורשת%
}\label{s.unique}

\begin{theorem}
יהי
$n>1$.
אזי ניתן לפרק את
$n$
כמכפלה של מספרים ראשוניים
\emph{בדרך אחת בלבד}.
)שינוי סדר המספרים לא נחשב כפירוק שונה.(
\end{theorem}

במשפט~%
\ref{t.prime}
הראינו שקיים פירוק של כל מספר שלם לגורמים ראשוניים. בהוכחה שקיים פירוק יחיד, נגלה שלושה מקרים נוספים של שימוש באינדוקציה, לפעמים בצורה לא מפורשת.

ההוכחה משתמשת בשתי למות. נשאיר את ההוכחות שלהן כתרגילים.

\begin{lemma}[%
הלמה של
\L{Euclid}]
יהיו
$n_1,n_2$
מספרים שלמים
ו-%
$p$
מספר ראשוני. אם
$p \mid n_1 n_2$,
אזי
$p\mid n_1$ 
או
$p\mid n_2$.
משמעות הסימן
$|$
היא "מחלק".
\end{lemma}

\begin{lemma}[%
הזהות של
\L{Bezout}]
יהי
$n_1, n_2$
מספרים שלמים שלפחות אחד מהם לא אפס. אזי קיימים מספרים שלמים
$a,b$
כך ש-%
$\gcd(n_1,n_2)=an_1+bn_2$. $\gcd(a,b)$
הוא המחלק המשותף הגודל ביותר של
$a$
ו-%
$b$.
\end{lemma}

משתמשים בזהות של
\L{Bezout}
כדי להוכיח את הלמה של
\L{Euclid}.
מוכיחים את הזהות של
\L{Bezout}
על ידי הגדרת קבוצה של מספרים
\emph{חיוביים}
וטיעון שלקבוצה מוכרח להיות איבר קטן ביותר. מעט ספרי לימוד יטרחו להוכיח את הטיעון שהוא כל כך ברור באמצעות עיקרון הסדר הטוב השקול לאינדוקציה )פרק~%
\ref{s.well}(.
קיים כאן שימוש באינדוקציה אבל הוא לא מפורש.

הלמה של
\L{Euclid}
טוענת טענה על מכפלה של שני מספרים, אבל אנו זקוקים לטענה על מכפלה כללית.
\begin{lemma}[%
הלמה הכללית של
\L{Euclid}]
יהי
$n_1,\ldots,n_k$
מספרים שלמים ו-%
$p$
מספר ראשוני. אם
$p \mid n_1 \cdots n_k$,
אזי
$p\mid n_i$
עבור
$1\leq i \leq k$.
\end{lemma}
את הלמה ניתן להוכיח באינדוקציה.

קיים שימוש שלישי של אינדוקציה בהוכחה אבל הוא מוסתר עמוק יותר. ההוכחה שקיים פירוק יחיד מתחילה כך:

נניח שקיימים שני פירוקים למספרים ראשוניים:
\[
n=p_1 \cdots p_k = q_1 \cdots q_m\,.
\]
ברור ש-%
$p_1\mid p_1 \cdots p_k$
ולכן גם
$p_1 \mid q_1 \cdots q_m$.
לפי הלמה הכללית של
\L{Euclid},
$p_1 \mid q_i$,
עבור
$1\leq i \leq m$,
ו-%
\textbf{ללא הגבלת הכלליות},
$p_1 \mid q_1$.
מה המשמעות של ביטוי זה השגור בהוכחות מתמטיות? המשמעות היא שלכל
$i$,
ניתן להחליף את המיקום במכפלה של
$q_1$
ו-%
$q_i$:
\[
q_1 \cdots q_i \cdots q_m = q_i \cdots q_1 \cdots q_m.
\]
זה ברור לגמרי, אבל זהו טיעון שחייבים להוכיח אותו וההוכחה היא על ידי אינדוקציה על
$i$
תוך שימוש בכללי הקומטטיביות והאסוציאטיביות של מספרים שלמים.

הוכחת משפט הפירוק למכפלה של מספרים ראשוניים,
\textbf{%
המשפט היסודי של האריתמטיקה%
},
מדגימה שאינדוקציה נמצאת בכל מקום, גם אם בצורה לא מפורשת.

\begin{exercise}
)מאתגר( הוכח את הזהות של
\L{Bezout}.
\end{exercise}
\textbf{רמז}
תהי
$S=\{x = an_1+bn_2: x>0\}$.
כדי לפשט את ההוכחה, נניח ש-%
$n_1,n_2>0$,
כך ש-%
$S$
היא קבוצה לא-ריקה המכילה
$1\cdot n_1+1\cdot n_2$.
לפי עיקרון הסדר הטוב )סעיף~%
\ref{s.well}(,
קיים איבר קטן ביותר
$d\in S$.
השתמש באלגוריתם לחילוק מספרים שלמים והעובדה ש-%
$d$
הוא האיבר הקטן ביותר ב-%
$S$
כדי להראות ש-%
$d\mid n_1$,
ש-%
$d\mid n_2$,
ו-%
$d=\gcd(n_1,n_2)$.

\begin{exercise}
הוכח את הלמה של
\L{Euclid}.
\end{exercise}
\textbf{רמז}
אם
$p\mid n_1n_2$
ו-%
$p$
לא מחלק את
$n_1$,
אז
$\gcd(p,n_1)=1$.
עכשיו ניתן להשתמש בזהות של
\L{Bezout}.

\section{%
הגדרה רקורסיבית%
}

\textbf{רקורסיה}
היא מושג מרכזי במתמטיקה ומדעי המחשב הקשור קשר הדוק עם אינדוקציה. בהוכחות באמצעות אינדוקציה, אנו מוכיחים טענת בסיס על מבנה קטן ומרחיבים את ההוכחה למבנים גדולים יותר. ברקורסיה, אנו
\emph{מגדירים}
מבנה כמורכב ממבנים קטנים יותר עד שמגיעים למבנה בסיס הקטן ביותר.

\subsection*{%
דוגמה להגדרה רקוסיבית%
}

הנה דוגמה פשוטה של רקורסיה:
\begin{eqnarray*}
a_1 &=& 2\\
a_{n+1} &=& a_n + 2, \;\;\textrm{for}\;\; n\geq 1\,.
\end{eqnarray*}
מה הערך של
$a_5$?
ובכן:
\begin{eqnarray*}
a_5 &=& a_4 + 2\\
&=& a_3+2+2\\
&=& a_2+2+2+2\\
&=& a_1+2+2+2+2\\
&=& 2+2+2+2+2\\
&=&10\,.
\end{eqnarray*}

בגלל הקשר בין רקורסיה לאינדוקציה, לא מפתיע שתכונות של מבנים המוגדרים על ידי רקורסיה ניתנות להוכחה באמצעות אינדוקציה. המשפט הבא פשוט מאוד אבל מדגים את השיטה.
\begin{theorem}\label{t.recursive}
לכל
$n\geq 1$, $a_n = 2n$.
\end{theorem}

\textbf{הוכחה}
הוכחת טענת הבסיס פשוטה מאוד:
$a_1=2=2\cdot 1$.
הנחת האינדוקציה היא
$a_n = 2n$.
הצעד האינדוקטיבי הוא:
$a_{n+1} = a_n + 2 \ih{} 2n + 2 = 2(n+1)$.\qed

\begin{exercise}
לכל
$n\geq 1$, $\sum_{i=1}^n a_i = n(n+1)$.
\end{exercise}

\subsection*{%
מספרי פיבונצ'י}

מספרי פיבונצ'י מהווים דוגמה קלסית להגדרה רקורסיבית:
\begin{eqnarray*}
f_1 &=& 1\\
f_2 &=& 1\\
f_n &=& f_{n-1} + f_{n-2}, \;\;  n \geq 3 \;\; \textrm{\R{עבור}}\,.
\end{eqnarray*}
שנים עשר מספרי פיבונצ'י הראשונים הם:
\[
1, 1, 2, 3, 5, 8, 13, 21, 34, 55, 89, 144\,.
\]
\begin{theorem}
כל מספר פיבונצ'י רביעי מתחלק ב-%
$3$.
\end{theorem}
\textbf{דוגמאות}
$f_4=3=3\cdot 1,\; f_8=21=3\cdot 7,\; f_{12}=144=3\cdot 48$.

\textbf{הוכחה}
טענת הבסיס מתקבלת באופן מיידי כי
$f_4=3$
מתחלק ב-%
$3$.
הנחת האינדוקציה היא ש-%
$f_{4n}$
מתחלק ב-%
$3$.
הצעד האינדוקטיבי הוא:
\begin{eqnarray*}
f_{4(n+1)} &=& f_{4n+4}\\
&=& f_{4n+3}+f_{4n+2}\\
&=& (f_{4n+2}+f_{4n+1})+f_{4n+2}\\
&=& ((f_{4n+1}+f_{4n})+f_{4n+1})+f_{4n+2}\\
&=& ((f_{4n+1}+f_{4n})+f_{4n+1})+(f_{4n+1}+f_{4n})\\
&=& 3f_{4n+1}+2f_{4n}\,.
\end{eqnarray*}
ברור ש-%
$3f_{4n+1}$
מתחלק ב-%
$3$
ולפי הנחת האינדוקציה
$f_{4n}$
מתחלק ב-%
$3$,
ולכן,
$f_{4(n+1)}$
מתחלק ב-%
$3$.\qed

\begin{exercise}
כל מספר פיבונצ'י חמישי מתחלק ב-%
$5$.
\end{exercise}

\begin{exercise}
$f_n < (\frac{7}{4})^n$.
\end{exercise}

%%%%%%%%%%%%%%%%%%%%%%%%%%%%%%%%%%%%%%%%%%%%%%%%%%%%%%%%%%%%%%%%%%%
