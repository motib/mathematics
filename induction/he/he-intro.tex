% !TeX root = induction-he.tex

\thispagestyle{empty}

\begin{center}
\textbf{\LARGE התחפושות הרבות של אינדוקציה}

\bigskip
\bigskip

\textbf{\Large מוטי בן-ארי}

\bigskip

\textbf{\large המחלקה להוראת המדעים}

\smallskip

\textbf{\large מכון ויצמן למדע}

\bigskip

\url{http://www.weizmann.ac.il/sci-tea/benari/}

\bigskip

\large{גרסה
$1.6.1$}
\end{center}

\bigskip

\selectlanguage{english}
\begin{center}
\copyright{}\  2016--19 by Moti Ben-Ari.
\end{center}

This work is licensed under the Creative Commons Attribution-ShareAlike 3.0 Unported License. To view a copy of this license, visit \url{http://creativecommons.org/licenses/by-sa/3.0/} or send a letter to Creative Commons, 444 Castro Street, Suite 900, Mountain View, California, 94041, USA.

\begin{center}
\includegraphics[width=.2\textwidth]{../../by-sa.png}
\end{center}

\selectlanguage{hebrew}

\setcounter{tocdepth}{0}
\tableofcontents


\chapter[מבוא]{\vspace*{-2ex}מבוא}\label{s.intro}

\vspace*{-4ex}

אינדוקציה היא אחת משיטות ההוכחה השכיחות ביותר במתמטיקה, עד כדי כך שמשתמשים בה בצורה שיגרתית ואף בצורה לא-מפורשת. תופעת לוואי של השימוש השיגרתי היא שאינדוקציה נלמדת כמתכון של צעדים לביצוע ולא כמושג יסודי. מסמך זה מביא קורס קצר על אינדוקציה, המראה את העושר של המושג והדרכים השונות שמשתמשים בה במתמטיקה ובמדעי המחשב.

ההצגה היא רחבה במקום עמוקה: המטרה היא להציג אינדוקציה בכמה שיותר צורות, ולכן כל נושא יוצג על ידי דוגמה אחת ותרגיל אחד או שניים, שלא כמו בספר לימוד עם דוגמאות רבות ושפע תרגילים. המסמך כתוב ברמות שונות: חלקו מתאים לתלמידי תיכון מתקדמים וחלקו לסטודנטים ולמורים למתמטיקה ומדעי המחשב. אפשר לדלג על נושאים לא מוכרים.

פרקים~%
\ref{s.axiom}--\ref{s.watch}
אמורים להיות נגישים לתלמידי תיכון. פרק~%
\ref{s.axiom}
מכיל דיון די ארוך שמטרתו לנמק את הצורך באינדוקציה מתמטית ולהציג אותה כאקסיומה. פרק~%
\ref{s.integers}
מציג אינדוקציה המתמטית בצורתה הקלסית מעל למספרים השלמים, תוך דיון בצורות שונות של יישום האקסיומה. פרק~%
\ref{s.notjust}
חשוב במיוחד כי הוא מראה שאינדוקציה היא שיטת הוכחה הפועלת על מבנים מתמטיים בנוסף למספרים השלמים. פרק~%
\ref{s.watch}
מזהיר ממוקשים: אינדוקציה היא לא שיטת ההוכחה היחידה והיא לא בהכרח השיטה המתאימה ביותר לכל משפט.

פרקים~%
\ref{s.logic}--\ref{s.verif}
מציגים את השימוש באינדוקציה במתמטיקה ברמת האוניברסיטה ובמדעי המחשב. פרק~%
\ref{s.logic}
דן באינדוקציה מעל נוסחאות של לוגיקה מתמטית ופרק~%
\ref{s.models}
מדגים את השימוש באינדוקציה בהקשר של אוטומטים ושפות פורמליות. פרק~%
\ref{s.verif}
מדגים את השימוש באינדוקציה בהוכחת נכונות של תכניות מחשב.

פרקים~%
~\ref{s.ind-ded}--\ref{s.well}
דנים בהיבטים פילוסופיים ותיאורטיים של אינדוקציה. פרק~%
\ref{s.ind-ded}
מסביר איך המשמעות של המונח אינדוקציה במתמטיקה מתנגשת עם משמעותו במדע. פרק~%
\ref{s.well}
מראה כיצד ניתן להוכיח את כלל ההיסק של אינדוקציה מתמטית, אם מניחים את עיקרון הסדר הטוב כאקסיומה. פרק~%
\ref{s.conclusion}
מסכם את הקורס.

נספח~%
\ref{a.challenge}
מציג תרגילים מאתגרים שאינם מחייבים ידע מתקדם במתמטיקה.

התשובות לכל התרגילים במסמך נמצאות בפרק~%
\ref{a.solutions}.

\bigskip

\textbf{הבעת תודה}
אני מודה למיכל ארמוני וליוני עמיר על הערותיהן המועילות.

%%%%%%%%%%%%%%%%%%%%%%%%%%%%%%%%%%%%%%%%%%%%%%%%%%%%%%%%%%%%%%%%%%%
