\documentclass[12pt,a4paper]{article}
\usepackage[utf8x]{inputenc}
\usepackage[english,hebrew]{babel}
\usepackage{graphicx}
\usepackage{verbatim}
\usepackage{url}

\usepackage{tikz}
\usetikzlibrary{external,positioning,through,calc,intersections,arrows.meta}
\tikzexternalize[prefix=tikz/]

\textwidth=15.5cm
\textheight=23cm
\topmargin=0pt
\headheight=0pt
\oddsidemargin=2em
\headsep=0pt
\parindent=0pt
\renewcommand{\baselinestretch}{1.1}
\setlength{\parskip}{0.3\baselineskip plus 1pt minus 1pt}

\begin{document}
\thispagestyle{empty}

\selectlanguage{hebrew}
\begin{center}
\textbf{%
תרגיל
$16$
בעמוד 
$228$
בספר בני גורן, מתמטיקה
$5$
יח"ל - חלק ב%
$2$
- שאלון
$806$.}

\textbf{מוטי בן-ארי}
\end{center}
עבור הפונקציה
$y=3\cos x -  \cos 3x$
בתחום
$0\leq x \leq \displaystyle \frac{3\pi}{2}$,
מצא )א( נקודות קיצון, )ב( תחומי עלייה וירידה, )ג( נקודות חיתוך עם הצירים, )ד( שרטט את גרף הפונקציה.

הנגזרת: 
\[
y'=-3\sin x + 3 \sin 3x
\]
מתאפסת כאשר 
$\sin x = \sin 3x$.
מתי שני סינוסים שווים? כאשר הקואורדינטות בציר ה-%
$y$
שוות, וזה קורה כאשר קואורדינטות ה-%
$x$
שוות או כאשר אחת מהן היא
$\pi$
פחות השנייה:
\begin{center}
\selectlanguage{english}
% The unit circle with main axes
\begin{tikzpicture}[scale=1.5]
\coordinate (origin) at (0,0);
% Draw circle
\draw (origin) circle [radius=1];
% Draw axes
\draw (-1,0) -- (1,0);
\draw (0,-1) -- (0,1);
\draw[thick,dashed] (-1.2,.3) -- (1.2,.3);
\draw[thick,dashed] (-1.2,-.5) -- (1.2,-.5);
\fill (18:1) node[above right] {$x_1$} circle [radius=1pt];
\fill (162:1) node[above left] {$\pi-x_1$} circle [radius=1pt];
\fill (210:1) node[below left] {$\pi-x_2$} circle [radius=1pt];
\fill (-30:1) node[below right] {$x_2$} circle [radius=1pt];
\end{tikzpicture}
\end{center}
הוספת כפולה של
$2\pi$
לקואורדינטת ה-%
$x$
לא משנה את הערך של פונקציה טריגונומטרית. הפתרונות של
$\sin x = \sin 3x$
בתחום 
$0\leq x \leq \displaystyle \frac{3\pi}{2}$
הם:
\begin{eqnarray*}
3x &=& x + 2\pi k\\
x &=& \pi k\\
x&=& 0, \; \pi\,,
\end{eqnarray*}
\vspace{-2ex}
ו:
\vspace{-2ex}
\begin{eqnarray*}
3x &=& (\pi - x) + 2\pi k\\
x &=& \frac{\pi}{4} + \frac{\pi}{2} k\\
x&=& \frac{\pi}{4},\; \frac{3\pi}{4}, \; \frac{5\pi}{4}\,.
\end{eqnarray*}

\vspace{-2ex}

$\displaystyle x=\frac{3\pi}{2}$
היא נקודת קיצון )מקומית( נוספת. נסמן את נקודות הקיצון על מעגל היחידה:
\begin{center}
\selectlanguage{english}
% The unit circle with main axes
\begin{tikzpicture}[scale=1.5]
\coordinate (origin) at (0,0);
% Draw circle
\draw (origin) circle [radius=1];
% Draw axes
\draw (-1,0) node [left] {$\pi$} -- (1,0) node [right] {$0$};
\draw (0,-1) node [below] {$3\pi/2$} -- (0,1) node [above] {$\pi/2$};
\fill (1,0) circle [radius=1pt];
\fill (-1,0) circle [radius=1pt];
\fill (45:1) node[above right] {$\pi/4$} circle [radius=1pt];
\fill (135:1) node[above left] {$3\pi/4$} circle [radius=1pt];
\fill (225:1) node[below left] {$5\pi/4$} circle [radius=1pt];
\fill (270:1) node[below] {$3\pi/2$} circle [radius=1pt];
\draw (315:1) node[below right] {$(7\pi/4)$} circle [radius=0pt];
%\node at (315:1) {$(7\pi/4$)};
\end{tikzpicture}
\end{center}
כדאי לבדוק שאם מכפילים ב-%
$3$
את הזוויות המסומנת בנקודה מקבלים זווית עם
\textbf{אותו סינוס}
)לא אותה נקודה(. למשל:
\[
3\cdot \frac{5\pi}{4} = \frac{15\pi}{4} = \frac{8\pi}{4}+\frac{7\pi}{4}= 2\pi + \frac{7\pi}{4}\,,
\]

והסינוס של
$\displaystyle \frac{7\pi}{4}$
שווה לסינוס של
$\displaystyle \frac{5\pi}{4}$.

נקודות הקיצון הן:

\[
(0,2),\quad \left(\frac{\pi}{4},2\sqrt{2}\right),\quad \left(\frac{3\pi}{4},-2\sqrt{2}\right),\quad (\pi,-2),\quad \left(\frac{5\pi}{4},-2\sqrt{2}\right),\quad \left(\frac{3\pi}{2},0\right)\,.
\]
אין אסימטוטות אנכיות, ולכן תחומי העלייה והירידה ואיפיון נקודות הקיצון כמינימום או מקסימום מוגדרים על ידי ההפרשים בקואורדינטות ה-%
$y$
של נקודות הקיצון:

\[
(0,2) \;\nearrow \; \left(\frac{\pi}{4},2\sqrt{2}\right) \;\searrow \;  \left(\frac{3\pi}{4},-2\sqrt{2}\right) \;\nearrow \;  (\pi,-2) \;\searrow \;  \left(\frac{5\pi}{4},-2\sqrt{2}\right) \;\searrow \;  \left(\frac{3\pi}{2},0\right)\,.
\]

\vspace{-4ex}
$\quad$
מקסימום
$\quad\quad$
מינימום
$\quad\quad\quad$
מקסימום
$\quad\quad$
מינימום 
$\quad\quad\quad$
מקסימום
$\quad\quad$
מינימום 

עבור נקודות הקיצון בפנימיות, אפשר לחשב מינימום או מקסימום לפי הנגזרת השנייה:
\[
y'' = -3\cos x + 9\cos 3x=3(3\cos 3x-\cos x)\,.
\]
כדי לבדוק את הסימן אפשר להשמיט את הגורם הקבוע ולחשב
$3\cos 3x-\cos x$:
\[
\begin{array}{|c|c|c|c|c|}
\hline
x& \pi/4 & 3\pi/4 & \pi & 5\pi/4 \\\hline
y''& -2\sqrt{2}<0&\sqrt{2}>0 & -2<0&2\sqrt{2}>0 \\\hline
& \textrm{max}& \textrm{min}& \textrm{max}&\textrm{min}\\\hline
\end{array}
\]


נקודות החיתוך עם הצירים הן הנקודות בקצה התחום: 
$(0,2)$
עם ציר ה-%
$y$,\ \ 
$(\frac{3}{2},0)$
עם ציר ה-%
$x$,\ \ 
ועוד חיתוך עם ציר ה-% 
$x$
בין
$\frac{\pi}{4}$
ו-%
$\frac{3\pi}{4}$.
אנו יודעים ש-%
$\cos \frac{\pi}{2} =\cos \frac{3\pi}{2}=0$,
ולכן:
\[
y= 3\cos\frac{\pi}{2} - \cos \frac{3\pi}{2} = 0-0=0\,,
\]
ונקודת החיתוך היא 
$(\frac{3\pi}{2},0)$.

הגרף של הפונקציה הוא:
\begin{center}
\selectlanguage{english}
\begin{tikzpicture}[scale=.8]
\draw (0,3)--(0,-3);
\draw (0,0)--(5,0);
\draw[samples=100,domain=0:3*pi/2] plot (\x,{3*cos(\x r) - cos(3*\x r)});
\fill (0,2) node[left] {$(0,2)$} circle [radius=2pt];
\fill (pi/4,2*sqrt 2) node[above] {$(\frac{\pi}{4},2\sqrt{2})$} circle [radius=2pt];
\fill (3*pi/4,-2*sqrt 2) node[below left,xshift=12pt] {$(\frac{3\pi}{4},-2\sqrt{2})$} circle [radius=2pt];
\fill (pi,-2) node[above] {$(\pi,-2)$} circle [radius=2pt];
\fill (5*pi/4,-2*sqrt 2) node[below right,xshift=-12pt] {$(\frac{5\pi}{4},-2\sqrt{2})$} circle [radius=2pt];
\fill (3*pi/2,0) node[above] {$(\frac{3\pi}{2},0)$} circle [radius=2pt];
\end{tikzpicture}
\end{center}
\end{document}
