% !TeX root = trigonometric-functions.tex

\hypersetup{pageanchor=false}
\thispagestyle{empty}

\vspace*{2ex}

\begin{center}


\textbf{\LARGE A Functional Approach to Teaching Trigonometry}

\bigskip
\bigskip
\bigskip
\bigskip

\textbf{\Large Avital Elbaum-Cohen and Moti Ben-Ari}

\bigskip
\bigskip

\url{http://www.weizmann.ac.il/sci-tea/benari/}

\bigskip

Version 0.1

\end{center}


\vfill

\begin{footnotesize}
\begin{center}
\copyright{}\ 2019 by Avital Elbaum-Cohen and Moti Ben-Ari.
\end{center}

This work is licensed under the Creative Commons Attribution-ShareAlike 3.0 Unported License. To view a copy of this license, visit \url{http://creativecommons.org/licenses/by-sa/3.0/} or send a letter to Creative Commons, 444 Castro Street, Suite 900, Mountain View, California, 94041, USA.
\end{footnotesize}

\bigskip

\begin{center}
\includegraphics[width=.15\textwidth]{../../by-sa.png}
\end{center}


\newpage
\thispagestyle{empty}

\tableofcontents
\thispagestyle{empty}


\newpage
\hypersetup{pageanchor=true}

\setcounter{page}{1}

\chapter{The Triangular and the Functional Approaches}

\section{Introduction}

This document presents an approach to teaching secondary-school trigonometry based upon functions instead of right triangles.
We offer pedagogical guidance including Geogebra projects.
The document is intended for teachers and those engaged in mathematical education who wish to become familiar with this pedagogical approach and to use the learning materials that were developed.

Secondary-school mathematics textbooks typically present trigonometry in two contexts: (1) functions defined as ratios of the length of the sides of right triangles, and (2) functions of a real variable defined as the coordinates of points obtained by rotating a radius vector from the origin around the circumference of the unit circle.
The first context is closely connected with Euclidean geometry and, while the second context is closely connected with functions of a real variable, in particular, as they are studied in calculus.
The study of trigonometry provides a rare opportunity to use functions to solve problems having an applied aspect.
It is important to note that trigonometry can be studied independently in both contexts, so there seems to be no impediment to teaching and studying these chapters in either order.

In practice, secondary-school students begin the study of trigonometry only after studying the following topics:
\begin{itemize}
\item Euclidean geometry including circles and similar triangles.
\item Polynomial functions including tangents and derivatives.
\item Analytical geometry including the unit circle.
\end{itemize}

When using dynamic graphical software such as Geogebra in teaching, we must consider the best representation of the main feature we seek to demonstrate.
Technology allows us to create continuous and rapid change, which can leave a strong impression on the viewer, but the teacher must consider if the dynamic display enables the students to participate in a coherent mathematical discussion.

Appendix~\ref{a.geogebra} contains a table of links to the Geogebra projects. Within the text, the projects are numbered.
