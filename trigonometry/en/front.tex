% !TeX root = trigonometric-functions.tex

\thispagestyle{empty}

\begin{center}
\textbf{\LARGE A Functional Approach to Teaching Trigonometry}

\bigskip
\bigskip
\bigskip
\bigskip

\textbf{\Large Avital Elbaum-Cohen and Moti Ben-Ari}

\bigskip

\url{http://www.weizmann.ac.il/sci-tea/benari/}

\bigskip

Version 0.1

\end{center}


\vfill

\begin{footnotesize}
\begin{center}
\copyright{}\ 2019 by Avital Elbaum-Cohen and Moti Ben-Ari.
\end{center}

This work is licensed under the Creative Commons Attribution-ShareAlike 3.0 Unported License. To view a copy of this license, visit \url{http://creativecommons.org/licenses/by-sa/3.0/} or send a letter to Creative Commons, 444 Castro Street, Suite 900, Mountain View, California, 94041, USA.
\end{footnotesize}

\bigskip

\begin{center}
\includegraphics[width=.15\textwidth]{../../by-sa.png}
\end{center}

\newpage
\thispagestyle{empty}
\mbox{}
\newpage
\thispagestyle{empty}

\tableofcontents
\newpage
\mbox{}
\newpage

\chapter[Teaching Trigonometry: The Triangular and the Functional Approaches]{Teaching Trigonometry:\\The Triangular and the Functional Approaches}

This document examines the teaching of the secondary-school trigonometry and presents an approach to teaching trigonometry based upon functions instead of triangle.
We offer pedagogical guidance including Geogeraba constructions.
The document is not a textbook; it is intended for teachers and those engaged in mathematical education who wish to become familiar with the pedagogical approach and to make use of the learning materials that were developed.

Secondary-school mathematics textbooks typically present trigonometry in two contexts: (1) functions defined as ratios in right triangles and (2) functions on a real variable defined using the unit circle.
The first context is closely connected with Euclidean geometry and, while the second context is closely connected with general functions of a real variable, in particular, as they are studied in calculus.
In fact, the study of trigonometry provides an opportunity, which is not common in the curriculum, to use functions to solve problems with an applied aspect.
It is important to note that trigonometry can be studied independently in both contexts, so there seems to be no impediment to teaching and studying these chapters in either order.

In the light of the above, in this chapter we explore both approaches for teaching trigonometry and examine the advantages and disadvantages of each one.

The first approach, which is the most widely accepted approach, is to define the trigonometric functions as ratios of lengths of sides in right angles.
The second approach is to define the trigonometric functions as the coordinates of points obtained by rotating a radius vector from the origin around the circumference of a circle.

In practice, secondary-school students begin the study of trigonometry only after studying the following topics:
\begin{itemize}
\item Euclidean geometry (including circles and similarity in triangles).
\item Polynomial functions including tangents and derivatives.
\item Analytical geometry, in particular, the unit circle: a circle of radius $1$ whose center is at the origin of the Cartesian axes.
\end{itemize}


When using dynamic graphical software such as Geogebra in teaching, we must consider the best representation of the main feature whose variability we seek to demonstrate.
Technology allows us to create continuous and rapid change, which can leave a strong impression on the viewer, but the teacher must consider if the dynamic display enables the students to participate in coherent mathematical discussion.

The project at \url{https://www.geogebra.org/m/pff9bvvt} shows a dynamic display of the change of the sine function when its argument is multiplied by a constant.
Nevertheless, it seems preferable, certainly in the early stages of teaching the subject, to start with a static approach where students investigage functions that are transformations of known trigonometric functions.
The students use Geogebra projects to test their hypotheses.

