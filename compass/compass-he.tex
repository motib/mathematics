\documentclass[12pt,a4paper]{article}
\usepackage[utf8x]{inputenc}
\usepackage[english,hebrew]{babel}
\usepackage{graphicx}
\usepackage{verbatim}
\usepackage{url}
\usepackage{titlesec}
\usepackage{float}

\usepackage{tikz}
\usetikzlibrary{external,positioning,through,calc,intersections}
\tikzexternalize[prefix=tikz/]

\textwidth=15.5cm
\textheight=23cm
\topmargin=0pt
\headheight=0pt
\oddsidemargin=2em
\headsep=0pt
\renewcommand{\baselinestretch}{1.1}
\setlength{\parskip}{0.3\baselineskip plus 1pt minus 1pt}
\parindent=0pt

\titlespacing*{\section}{0pt}{3ex plus 1ex minus.2ex}{2ex plus .2ex}

\begin{document}
\thispagestyle{empty}

\selectlanguage{hebrew}

\begin{center}
\textbf{\Huge%
אמא'לה, המחוגה שלי התמוטטה!
}

\bigskip

\bigskip

\textbf{\Large מוטי בן-ארי}

\bigskip

\textbf{\Large המחלקה להוראת המדעים}

\bigskip

\textbf{\Large מכון ויצמן למדע}

\bigskip

\selectlanguage{english}
\url{http://www.weizmann.ac.il/sci-tea/benari/}
\end{center}

\bigskip
\bigskip

\begin{center}
\selectlanguage{english}
\copyright{}\  2018 by Moti Ben-Ari.

\end{center}

\selectlanguage{english}

{\small This work is licensed under the Creative Commons Attribution-ShareAlike 3.0 Unported License. To view a copy of this license, visit \url{http://creativecommons.org/licenses/by-sa/3.0/} or send a letter to Creative Commons, 444 Castro Street, Suite 900, Mountain View, California, 94041, USA.}

\bigskip

\begin{center}
\includegraphics[width=.2\textwidth]{../by-sa.png}
\end{center}

\selectlanguage{hebrew}

\newpage

%%%%%%%%%%%%%%%%%%%%%%%%%%%%%%%%%%%%%%%%%%%%%%%%%%%%%%%%%%%%%%%

\section{
\R{מחוגה קבועה ומחוגה מתמוטטת}
}

במחוגה מודרנית ניתן לקבע את המרחק בין הרגליים, וכך להעתיק קטע קו או מעגל ממקום למקום. ראיתי בספרי לימוד גיאומטריה המציגים בניית אנך אמצעי של קטע קו על ידי בניית שני מעגלים עם רדיוסים שווים שמרכזם נקודות הקצה של הקו, ובלבד שהרדיוסים
\textbf{גדלים ממחצית אורך הקטע}
 )איור~%
\ref{fig.bisector}
שמאל(. נקרא למחוגה זו: "מחוגה קבועה".

אוקלידס השתמש במחוגה "מתמוטטת" )%
\L{collapsing}%
(,
שרגליה מתקפלות כאשר מרימים אותה מהנייר. מחוגה המורכבת מגיר הקשור לחוט היא מחוגה מתמוטטת, כי אי-אפשר לשמור את הרדיוס כאשר מרימים אותה מהלוח. איור~%
\ref{fig.bisector}
)ימין( מראה בנייה של אנך אמצעי באמצעות מחוגה מתמוטטת: האורך של
$AB$
שווה כמובן לאורך של
$BA$,
ולכן הרדיוסים של המעגלים שווים ללא צורך להעביר אורך קטע ממקום למקום.

\begin{figure}[H]
\begin{center}
\selectlanguage{english}
\begin{tikzpicture}[scale=0.5]
\begin{scope}
\coordinate (A) at (0,0);
\coordinate (B) at (4,0);
\draw (A) node[below left] {$A$} -- (B) node[below right] {$B$};
\fill (A) circle[radius=3pt];
\fill (B) circle[radius=3pt];
\draw[name path=larc] (A) ++(-60:3cm) arc (-60:60:3cm);
\draw[name path=rarc] (B) ++(-120:3cm) arc (-120:-240:3cm);
\path [name intersections={of=larc and rarc,by={b,t}}];
\fill (t) node[above right,xshift=-2pt,yshift=5pt] {$C$} circle[radius=3pt];
\fill (b) node[below left,xshift=2pt,yshift=-5pt] {$D$} circle[radius=3pt];
\draw ($ (b) ! 1.2 ! (t)$) -- ($ (t) ! 1.2 ! (b)$);
\end{scope}
\begin{scope}[xshift=12cm]
\coordinate (A) at (0,0);
\coordinate (B) at (4,0);
\draw (A) node[below left] {$A$} -- (B) node[below right] {$B$};
\fill (A) circle[radius=3pt];
\fill (B) circle[radius=3pt];
\draw[name path=larc] (A) ++(-80:4cm) arc (-80:80:4cm);
\draw[name path=rarc] (B) ++(-100:4cm) arc (-100:-260:4cm);
\path [name intersections={of=larc and rarc,by={b,t}}];
\fill (t) node[above right,xshift=-2pt,yshift=3pt] {$C$} circle[radius=3pt];
\fill (b) node[below left,xshift=2pt,yshift=-3pt] {$D$}circle[radius=3pt];
\draw ($ (b) ! 1.2 ! (t)$) -- ($ (t) ! 1.2 ! (b)$);
\end{scope}
\end{tikzpicture}
\selectlanguage{hebrew}
\caption{אנך אמצעי}\label{fig.bisector}
\end{center}
\end{figure}

\vspace*{-5ex}

ההוכחה שהקו שנבנה הוא האנך האמצעי היא לא מאוד פשוטה, כי צריך להשתמש במושגים יחסית מתקדמים כגון משולשים חופפים. לעומתה, ההוכחה שבאותה הבנייה מתקבל משולש שווה צלעות פשוטה )איור~%
\ref{fig.equilateral}
ימין(. 
\begin{figure}[H]
\begin{center}
\selectlanguage{english}
\begin{tikzpicture}[scale=0.5]
\begin{scope}
\coordinate (A) at (0,0);
\coordinate (B) at (4,0);
\draw (A) node[below left] {$A$} -- (B) node[below right] {$B$};
\fill (A) circle[radius=3pt];
\fill (B) circle[radius=3pt];
\draw[name path=larc] (A) ++(-60:3cm) arc (-60:60:3cm);
\draw[name path=rarc] (B) ++(-120:3cm) arc (-120:-240:3cm);
\path [name intersections={of=larc and rarc,by={b,t}}];
\fill (t) node[above right,xshift=-2pt,yshift=5pt] {$C$} circle[radius=3pt];
\fill (b) node[below left,xshift=2pt,yshift=-5pt] {$D$} circle[radius=3pt];
\draw (A) -- (t);
\draw (B) -- (t);
\end{scope}
\begin{scope}[xshift=12cm]
\coordinate (A) at (0,0);
\coordinate (B) at (4,0);
\draw (A) node[below left] {$A$} -- (B) node[below right] {$B$};
\fill (A) circle[radius=3pt];
\fill (B) circle[radius=3pt];
\draw[name path=larc] (A) ++(-80:4cm) arc (-80:80:4cm);
\draw[name path=rarc] (B) ++(-100:4cm) arc (-100:-260:4cm);
\path [name intersections={of=larc and rarc,by={b,t}}];
\fill (t) node[above right,xshift=-2pt,yshift=3pt] {$C$} circle[radius=3pt];
\fill (b) node[below left,xshift=2pt,yshift=-3pt] {$D$}circle[radius=3pt];
\draw (A) -- (t);
\draw (B) -- (t);
\end{scope}
\end{tikzpicture}
\selectlanguage{hebrew}
\caption{משולש שווה צלעות?}\label{fig.equilateral}
\end{center}
\end{figure}
אורך הקטע
$AC$
שווה לאורך הקטע
$AB$
כי שניהם רדיוסים של אותו מעגל, ומאותה סיבה האורך של
$BC$
שווה לאורך של
$BA$.
מכאן:
\[
AC = AB = BA = BC\,.
\]
איור~%
\ref{fig.equilateral}
)שמאל( מראה שאם משתמשים במחוגה קבועה עם רדיוסים שרירותיים, מתקבל משולש שווה שוקיים שהוא לא בהכרח משולש שווה צלעות.

בנייה זו של משולש שווה צלעות היא המשפט הראשון בספר "יסודות" של אוקלידס. המשפט השני בספר מראה שאפשר להעתיק קטע קו נתון
$AB$
לקטע קו שאחת מנקודות הקצה שלה היא נקודה נתונה
$C$.
אם בונים מעגל שמרכזו
$C$
והרדיוס שלו הוא העותק של
$AB$,
מתקבל עותק של מעגל נתון שמרכזו 
$A$
והרדיוס שלו
$AB$.
המסקנה היא שלכל בנייה באמצעות מחוגה קבועה, קיימת בנייה שקולה באמצעות מחוגה מתמוטטת.

במאמר מרתק 
\L{\cite{toussaint}}, 
\L{Godfried Toussaint}
מראה שפורסמו הוכחות שגויות רבות של המשפט, ודווקא אוקלידס הוא זה שנתן הוכחה נכונה! במסמך זה, אביא את הבנייה של אוקלידס בשלבים, ביחד עם הוכחת הנכונות, ואחר כך בנייה שגויה שניתן למצוא אפילו בספרים שפרוסמו לאחרונה. אחר כך, אסקור כלים מורכבים יותר לבנייה גיאומטרית. לבסוף, אביא הוכחה
\textbf{שכל}
משולש הוא שווה שוקיים כדי לראות שאי-אפשר לסמוך על תרשים.

%%%%%%%%%%%%%%%%%%%%%%%%%%%%%%%%%%%%%%%%%%%%%%%%%%%%%%%%%%%%%%%

\section{%
העתקת קטע קו לפי אוקלידס%
}

\textbf{
משפט
\L{(Compasss Equivalency Theorem)}:%
}
נתון קטע קו
$AB$
ונקודה
$C$
)איור~%
\ref{fig.ce-1}
שמאל(,
ניתן לבנות )באמצעות מחוגה מתמוטטת( בנקודה
$C$
קטע קו שאורכו שווה לאורכו של 
$AB$.

\begin{figure}[H]
\begin{center}
\selectlanguage{english}
\begin{tikzpicture}[scale=0.5]
\begin{scope}
\coordinate (C) at (0,0);
\coordinate (A) at (2.5,0);
\coordinate (B) at (5.5,2);
\draw (A) node[below,xshift=-2pt,yshift=-2pt] {$A$} -- (B) node[right] {$B$};
\fill (A) circle[radius=3pt];
\fill (B) circle[radius=3pt];
\fill (C) node[below,xshift=2pt,yshift=-2pt] {$C$} circle[radius=3pt];
\end{scope}
\begin{scope}[xshift=12cm]
\coordinate (C) at (0,0);
\coordinate (A) at (2.5,0);
\coordinate (B) at (5.5,2);
\draw (A) node[below,xshift=-2pt,yshift=-2pt] {$A$} -- (B) node[right] {$B$};
\fill (A) circle[radius=3pt];
\fill (B) circle[radius=3pt];
\fill (C) node[below,xshift=2pt,yshift=-2pt] {$C$} circle[radius=3pt];
\draw (A) -- (C);
\path[name path=larc] (C) ++(-70:2.5cm) arc (-70:70:2.5cm);
\path[name path=rarc] (A) ++(-110:2.5cm) arc (-110:-250:2.5cm);
\path [name intersections={of=larc and rarc,by={d,D}}];
\fill (D) node[above] {$D$} circle[radius=3pt];
\draw (A) -- (D);
\draw (C) -- (D);
\end{scope}
\end{tikzpicture}
\selectlanguage{hebrew}
\caption{%
העתק
$AB$
לנקודה
$C$.
בנה משולש שווה צלעות על
$AC$.%
}\label{fig.ce-1}
\end{center}
\end{figure}

\vspace*{-7ex}
\textbf{%
הבנייה:%
}
\vspace*{-2ex}
\begin{itemize}
\item
חבר בקו את הנקודות
$A$
ו-%
$C$.
\item
בנה משולש שווה צלעות שבסיסו
$AC$.
סמן את הקודקוד של המשולש ב-%
$D$
)איור~%
\ref{fig.ce-1}
ימין(. לפי המשפט הראשון של אוקלידס, ניתן לבנות את המשולש באמצעות מחוגה מתמוטטת.
\item
בנה קרן בהמשך של
$DA$
וקרן בהמשך של 
$DC$
)איור~%
\ref{fig.ce-2}
שמאל(.
\item
בנה מעגל שמרכזו 
$A$
עם רדיוס
$AB$.
סמן את החיתוך של המעגל עם הקרן
$DE$
ב-%
$E$
)איור~%
\ref{fig.ce-2}
ימין(.
\item
בנה מעגל שמרכזו 
$D$
עם רדיוס 
$DC$.
סמן את החיתוך של הקרן
$DC$
עם המעגל ב-%
$F$
)איור~%
\ref{fig.ce-3}(.
\item
אורכו של קטע הקו
$CF$
שווה לאורך קטע הקו
$AB$.
\end{itemize}
\begin{figure}[H]
\begin{center}
\selectlanguage{english}
\begin{tikzpicture}[scale=0.5]
\begin{scope}
\coordinate (C) at (0,0);
\coordinate (A) at (2.5,0);
\coordinate (B) at (5.5,2);
\draw (A) node[below,xshift=-2pt,yshift=-2pt] {$A$} -- (B) node[right] {$B$};
\fill (A) circle[radius=3pt];
\fill (B) circle[radius=3pt];
\fill (C) node[below,xshift=2pt,yshift=-2pt] {$C$} circle[radius=3pt];
\draw (A) -- (C);
\path[name path=larc] (C) ++(-70:2.5cm) arc (-70:70:2.5cm);
\path[name path=rarc] (A) ++(-110:2.5cm) arc (-110:-250:2.5cm);
\path [name intersections={of=larc and rarc,by={d,D}}];
\fill (D) node[above] {$D$} circle[radius=3pt];
\draw (A) -- (D);
\draw (C) -- (D);
\draw[name path=ray2] (D) -- ($ (D) ! 3 ! (C) $);
\draw[name path=ray1] (D) -- ($ (D) ! 3 ! (A) $);
\end{scope}
\begin{scope}[xshift=12cm]
\coordinate (C) at (0,0);
\coordinate (A) at (2.5,0);
\coordinate (B) at (5.5,2);
\draw (A) node[below,xshift=-2pt,yshift=-2pt] {$A$} -- (B) node[right] {$B$};
\fill (A) circle[radius=3pt];
\fill (B) circle[radius=3pt];
\fill (C) node[below,xshift=2pt,yshift=-2pt] {$C$} circle[radius=3pt];
\draw (A) -- (C);
\path[name path=larc] (C) ++(-70:2.5cm) arc (-70:70:2.5cm);
\path[name path=rarc] (A) ++(-110:2.5cm) arc (-110:-250:2.5cm);
\path [name intersections={of=larc and rarc,by={d,D}}];
\fill (D) node[above] {$D$} circle[radius=3pt];
\draw (A) -- (D);
\draw (C) -- (D);
\draw[name path=ray2] (D) -- ($ (D) ! 3 ! (C) $);
\draw[name path=ray1] (D) -- ($ (D) ! 3 ! (A) $);
\node[draw,circle through=(B),name path=c1] at (A) {};
\path [name intersections={of=c1 and ray1,by={E,e}}];
\fill (E) node[right,xshift=2pt,yshift=-2pt] {$E$} circle[radius=3pt];
\end{scope}
\end{tikzpicture}
\selectlanguage{hebrew}
\caption{%
קרנות מ-%
$D$
דרך
$A,C$.
מעגל שמרכזו
$A$
עם רדיוס
$AB$.%
}\label{fig.ce-2}
\end{center}
\end{figure}
\vspace*{-7ex}
\textbf{הוכחה:}
$DC=DA$
כי הם צלעות של משולש שווה צלעות.
$AE=AB$
כי שניהם רדיוסים של אותו מעגל שמרכזו 
$A$.
$DF=DE$
כי שניהם רדיוסים של אותו מעגל שמרכזו
$D$.
לכן, אורך קטע הקו 
$CF$ 
הוא:
\[
CF = DF - DC = DE - DC = DE - DA = AE = AB\,.
\].

\begin{figure}[H]
\begin{center}
\selectlanguage{english}
\begin{tikzpicture}[scale=0.5]
\coordinate (C) at (0,0);
\coordinate (A) at (2.5,0);
\coordinate (B) at (5.5,2);
\draw (A) node[below,xshift=-2pt,yshift=-2pt] {$A$} -- node[above] {$x$} (B) node[right] {$B$};
\fill (A) circle[radius=3pt];
\fill (B) circle[radius=3pt];
\fill (C) node[below,xshift=2pt,yshift=-2pt] {$C$} circle[radius=3pt];
\draw (A) -- (C);
\path[name path=larc] (C) ++(-70:2.5cm) arc (-70:70:2.5cm);
\path[name path=rarc] (A) ++(-110:2.5cm) arc (-110:-250:2.5cm);
\path [name intersections={of=larc and rarc,by={d,D}}];
\fill (D) node[above] {$D$} circle[radius=3pt];
\draw (A) -- node[right] {$y$} (D);
\draw (C) -- node[left] {$y$} (D);
\draw[name path=ray2] (D) -- ($ (D) ! 3 ! (C) $);
\draw[name path=ray1] (D) -- ($ (D) ! 3 ! (A) $);
\node[draw,circle through=(B),name path=c1] at (A) {};
\path [name intersections={of=c1 and ray1,by={E,e}}];
\fill (E) node[right,xshift=2pt,yshift=-2pt] {$E$} circle[radius=3pt];
\node[draw,circle through=(E),name path=c2] at (D) {};
\path [name intersections={of=c2 and ray2,by={F,f}}];
\fill (F) node[left,xshift=-2pt,yshift=-2pt] {$F$} circle[radius=3pt];
\path (A) -- node[right] {$x$} (E);
\path (C) -- node[left] {$x$} (F);
\end{tikzpicture}
\selectlanguage{hebrew}
\caption{%
מעגל שמרכזו
$D$
עם רדיוס
$DE$.
קטעים שווים:
$CF=AE=AB$.%
}\label{fig.ce-3}
\end{center}
\end{figure}

%%%%%%%%%%%%%%%%%%%%%%%%%%%%%%%%%%%%%%%%%%%%%%%%%%%%%%%%%%%%%%%
\newpage

\section{%
העתקה שגויה של קטע קו
}\label{sec.error}

הבנייה לקוחה מ-%
\L{\cite{rusty}}.\\
\textbf{
משפט
\L{(Compasss Equivalency Theorem)}:
}
נתון קטע קו
$AB$
ונקודה
$C$
)איור~%
\ref{fig.rusty-1}
שמאל(,
ניתן לבנות )באמצעות מחוגה מתמוטטת( בנקודה
$C$
קטע קו שאורכו שווה לאורכו של 
$AB$.

\textbf{בנייה:}
\vspace*{-2ex}
\begin{itemize}
\item
בנה מעגל שמרכזו 
$A$
עם רדיוס
$AB$
)איור~
\ref{fig.rusty-1}(.

\begin{figure}[H]
\begin{center}
\selectlanguage{english}
\begin{tikzpicture}[scale=0.5]
\begin{scope}
\coordinate (C) at (-2,0);
\coordinate (A) at (2.5,0);
\coordinate (B) at (4.5,1.5);
\draw (A) node[below,xshift=-2pt,yshift=-2pt] {$A$} -- (B) node[right] {$B$};
\fill (A) circle[radius=3pt];
\fill (B) circle[radius=3pt];
\fill (C) node[below,xshift=2pt,yshift=-2pt] {$C$} circle[radius=3pt];
\end{scope}
\begin{scope}[xshift=12cm]
\coordinate (C) at (-2,0);
\coordinate (A) at (2.5,0);
\coordinate (B) at (4.5,1.5);
\draw (A) node[below,xshift=-2pt,yshift=-2pt] {$A$} -- (B) node[right] {$B$};
\fill (A) circle[radius=3pt];
\fill (B) circle[radius=3pt];
\fill (C) node[below,xshift=2pt,yshift=-2pt] {$C$} circle[radius=3pt];
\node[draw,circle through=(B),name path=c1] at (A) {};
\end{scope}
\end{tikzpicture}
\selectlanguage{hebrew}
\caption{%
העתק
$AB$
לנקודה
$C$.
בנה מעגל שמרכזו
$A$
עם רדיוס
$AB$.%
}\label{fig.rusty-1}
\end{center}
\end{figure}
\vspace*{-8ex}
\item
בנה מעגל שמרכזו
$A$
עם רדיוס
$AC$
ומעגל שמרכזו
$C$
עם רדיוס
$AC=CA$.
סמן את שתי נקודות החיתוך של המעגלים ב-%
$E,F$.
סמן את החיתוך של המעגל שמרכזו
$C$
עם המעגל שמרכזו
$A$
ב-%
$D$
)איור~%
\ref{fig.rusty-2}(.
\item
בנה מעגל שמרכזו 
$E$
עם רדיוס 
$ED$.
סמן את החיתוך של מעגל זה עם המעגל שמרכזו
$C$
ב-%
$G$
)איור~%
\ref{fig.rusty-3}(.
\item
ארכו של קטע הקו 
$GC$
שווה לאורכו של
$AB$.
\end{itemize}
חשוב להשתכנע שהבנייה אפשרית עם מחוגה מתמוטטת.

את ההוכחה ש-%
$AB=GC$
ניתן למצוא ב-%
\L{\cite{rusty}}.
בהוכחה יש להראות ששני המשולשים המסומנים בקווים במקוקווים חופפים כך ש-%
$GC=DA=AB$.
ההוכחה ארוכה הרבה יותר מההוכחה של אוקלידס ומשתמשת במושגים מתקדמים יחסית להוכחה של אוקלידס המשתמשת רק במושגים: כל הרדיוסים של מעגל שווים וכל הצלעות של משולש שווה צלעות שווים.

\begin{figure}[H]
\begin{center}
\selectlanguage{english}
\begin{tikzpicture}[scale=0.5]
\coordinate (C) at (-2,0);
\coordinate (A) at (2.5,0);
\coordinate (B) at (4.5,1.5);
\draw (A) node[below right] {$A$} -- (B) node[right] {$B$};
\fill (A) circle[radius=3pt];
\fill (B) circle[radius=3pt];
\fill (C) node[left,xshift=-2pt] {$C$} circle[radius=3pt];
\node[draw,circle through=(B),name path=c1] at (A) {};
\node[draw,circle through=(C),name path=c2] at (A) {};
\node[draw,circle through=(A),name path=c3] at (C) {};
\path [name intersections={of=c1 and c3,by={D,f}}];
\path [name intersections={of=c2 and c3,by={E,F}}];
\fill (D) node[below right,xshift=4pt] {$D$} circle[radius=3pt];
\fill (E) node[above,yshift=2pt] {$E$} circle[radius=3pt];
\fill (F) node[below,yshift=-2pt] {$F$} circle[radius=3pt];
\end{tikzpicture}
\selectlanguage{hebrew}
\caption{%
מעגל שמרכזו
$A$
עם רדיוס
$AC$
ומעגל שמרכזו
$C$
עם רדיוס
$AC=CA$.%
}\label{fig.rusty-2}
\end{center}
\end{figure}

\begin{figure}[H]
\begin{center}
\selectlanguage{english}
\begin{tikzpicture}[scale=0.5]
\coordinate (C) at (-2,0);
\coordinate (A) at (2.5,0);
\coordinate (B) at (4.5,1.5);
\draw (A) node[below right] {$A$} -- (B) node[right] {$B$};
\fill (A) circle[radius=3pt];
\fill (B) circle[radius=3pt];
\fill (C) node[below left] {$C$} circle[radius=3pt];
\node[draw,circle through=(B),name path=c1] at (A) {};
\node[draw,circle through=(C),name path=c2] at (A) {};
\node[draw,circle through=(A),name path=c3] at (C) {};
\path [name intersections={of=c1 and c3,by={D,f}}];
\path [name intersections={of=c2 and c3,by={E,F}}];
\fill (D) node[below right,xshift=4pt] {$D$} circle[radius=3pt];
\fill (E) node[above,yshift=2pt] {$E$} circle[radius=3pt];
\fill (F) node[below,yshift=-2pt] {$F$} circle[radius=3pt];
\node[draw,circle through=(D),name path=c4] at (E) {};
\path [name intersections={of=c2 and c4,by={g,G}}];
\fill (G) node[below left,xshift=-4pt] {$G$} circle[radius=3pt];
\draw (C) -- (G);
\draw[dashed] (G) -- (E) -- (C);
\draw[dashed] (A) -- (D) -- (E) -- cycle;
\end{tikzpicture}
\selectlanguage{hebrew}
\caption{%
מעגל שמרכזו
$E$
עם רדיוס
$ED$.
קטעים שווים:
$AB=DA=GC$.%
}\label{fig.rusty-3}
\end{center}
\end{figure}
\vspace*{-8ex}
עברו בעיון על ההוכחה ב-%
\cite{rusty}
ש-%
$AB=GC$
וחפש את השגיאה. התשובה: אין שום שגיאה בהוכחה! השגיאה נובעת ממקור אחר: השווין
$AB=GC$
מתקיים רק כאשר אורכו של 
$AB$
קטן מאורכו של
$AC$.
הבנייה של אוקלידס נכונה ללא קשר לאורך היחסי של הקווים ולמיקום של הנקודה
$C$
ביחס לקטע הקו
$AB$
)\cite{toussaint}(.

%%%%%%%%%%%%%%%%%%%%%%%%%%%%%%%%%%%%%%%%%%%%%%%%%%%%%%%%%%%%%%%

\section{%
חקר הבניות עם גיאוגברה%
}
הכנתי קבצי גיאוגברה עבור שתי הבניות:
\begin{center}
\L{\texttt{compass-equivalency.ggb, rusty-compass.ggb}}.
\end{center}
ניתן להזיז את הנקודות
$A,B,C$
כדי לראות איך הציור משתנה, ולמדוד את שני אורכים כדי לבדוק אם הם שווים.

\textbf{%
שימו לב:%
}
בבנייה של אוקלידס, אם מתחילים עם 
$AB<AC$,
מזיזים את
$C$
כך ש-%
$AB>AC$
וחוזרים, התצוגה מתקלקלת. הסיבה היא שכאשר
$AB<AB$,
לקרן
$DA$
יש שתי נקודות חיתוך עם המעגל שמרכזו
$A$.
כאשר חוזרים למצב ש-%
$AB<AC$
נאבד נקודת החיתוך. כדי להתגבר על הבעייה הכנתי שני קבצים עבור שני מצבים.

%%%%%%%%%%%%%%%%%%%%%%%%%%%%%%%%%%%%%%%%%%%%%%%%%%%%%%%%%%%%%%%
\newpage

\section{
דרך "פשוטה יותר" להעתקת מעגל
}

נתון קטע קו 
$AB$
ונקודה
$C$,
אם נוכל לבנות מקבילית כאשר שלושת הנקודות הן קודקודים, נקבל קטע קו עם 
$C$
בקצה אחד שאורכו שווה לאורכו של
$AB$
)איור~%
\ref{fig.parallel-1}
שמאל(.
ראו
\L{\cite[%
207--208
\R{עמ'}%
]{roads}}.


\begin{figure}[H]
\begin{center}
\selectlanguage{english}
\begin{tikzpicture}[scale=0.5]
\coordinate (A) at (0,0);
\coordinate (B) at (4,0);
\coordinate (C) at (5,2);
\draw (A) -- (B);
\path (A) -- node[above] {$x$} (B);
\fill (A) node[below] {$A$} circle[radius=3pt];
\fill (B) node[below] {$B$} circle[radius=3pt];
\fill (C) node[above] {$C$} circle[radius=3pt];
\draw (B) -- node[right] {$y$} (C);
\coordinate (D) at ($(C)+(-40mm,0cm)$);
\draw (D) -- node[above] {$x$} (C);
\draw (A) -- node[left] {$y$} (D);
\fill (D) node[above] {$D$} circle[radius=3pt];
\begin{scope}[xshift=12cm]
\coordinate (A) at (0,0);
\coordinate (B) at (4,0);
\coordinate (C) at (5,2);
\draw ($ (B) ! 1.2 ! (A) $) -- ($ (A) ! 1.5 ! (B) $);
\path (A) -- (B);
\fill (A) node[below right] {$A$} circle[radius=3pt];
\fill (B) node[below] {$B$} circle[radius=3pt];
\fill (C) node[above] {$C$} circle[radius=3pt];
\draw (B) -- (C);
\draw[name path=ray1] ($(C)+(-5cm,0cm)$) -- ($(C)+(1cm,0cm)$);
\draw[name path=ray2] ($(A)+(-.25,-.65)$) -- ($(A)+(1,2.6)$);
\path [name intersections={of=ray1 and ray2,by={E}}];
\fill (E) node[above left] {$E$} circle[radius=3pt];
\coordinate (D) at (C |- B);
\draw (C) -- (D);
\fill (D) node[below] {$D$} circle[radius=3pt];
\end{scope}
\end{tikzpicture}
\selectlanguage{hebrew}
\caption{%
מקבילית: 
$DC=AB$.%
}\label{fig.parallel-1}
\end{center}
\end{figure}
\vspace*{-8ex}
\textbf{בנייה:} 
)איור~%
\ref{fig.parallel-1}
ימין(

\begin{itemize}
\item
חבר את
$B$
ו-%
$C$.
\item
בנה אנך מ-%
$C$
לקו המכיל את הקטע
$AB$.
נסמן את נקודת החיתוך ב-%
$D$.
\item
בנה אנך לקטע
$CD$
מהנקודה
$C$.
הקו המכיל את האנך מקביל ל-%
$AB$.
\item
באותה דרך בנה קו המקביל ל-%
$BC$
דרך 
$A$. 
נסמן את נקודת החיתוך של שני הקווים ב-%
$E$.
\item
אורכו של קטע הקו
$EC$
שווה לאורכו של
$AB$
ו-%
$C$
היא נקודת קצה שלו.
\end{itemize}

יש לוודא שאפשר לבנות את המקבילית עם מחוגה מתמוטטת. למעשה, הבנייה יחידה הנחוצה היא של אנך מנקודה שרירותית נתונה לקו המכיל קטע קו נתון )איור~%
\ref{fig.perp-1}(.

\begin{figure}[H]
\begin{center}
\selectlanguage{english}
\begin{tikzpicture}[scale=0.5]
\coordinate (A) at (0,0);
\coordinate (B) at (4,0);
\coordinate (C) at (5,2);
\draw ($ (B) ! 1.5 ! (A) $) -- ($ (A) ! 1.5 ! (B) $);
\fill (A) node[below] {$A$} circle[radius=3pt];
\fill (B) node[below] {$B$} circle[radius=3pt];
\fill (C) node[right] {$C$} circle[radius=3pt];
\end{tikzpicture}
\selectlanguage{hebrew}
\caption{%
אנך מ-%
$C$
לקו המכיל את הקטע
$AB$.%
}\label{fig.perp-1}
\end{center}
\end{figure}

\vspace*{-8ex}
נבנה מעגל שמרכזו
$C$
עם רדיוס הגדול מהמרחק של
$C$
מהקו )איור~%
\ref{fig.perp-2}(.

\begin{figure}[H]
\begin{center}
\selectlanguage{english}
\begin{tikzpicture}[scale=0.5]
\coordinate (A) at (0,0);
\coordinate (B) at (4,0);
\coordinate (C) at (5,2);
\draw[name path=ray] ($ (B) ! 1.5 ! (A) $) -- ($ (A) ! 2.5 ! (B) $);
\fill (A) node[below] {$A$} circle[radius=3pt];
\fill (B) node[below] {$B$} circle[radius=3pt];
\fill (C) node[right] {$C$} circle[radius=3pt];
\draw[name path=arc] (C) ++(-160:3.5cm) arc (-160:-20:3.5cm);
\path [name intersections={of=arc and ray,by={D,E}}];
\fill (D) node[below left] {$D$} circle[radius=3pt];
\fill (E) node[below right] {$E$} circle[radius=3pt];
\end{tikzpicture}
\selectlanguage{hebrew}
\caption{%
$D$
ו-%
$E$
במרחק שווה מ-%
$C$.
}\label{fig.perp-2}
\end{center}
\end{figure}
\vspace*{-6ex}


בנה אנך אמצעי ל-%
$DE$
דרך 
$C$
)איור~%
\ref{fig.perp-3}(.
$CD=CE$
כי הם שני רדיוסים שווים של אותו מעגל וניתן לבנות את האנך באמצעות מחוגה מתמוטטת.
\begin{figure}[H]
\begin{center}
\selectlanguage{english}
\begin{tikzpicture}[scale=0.5]
\coordinate (A) at (0,0);
\coordinate (B) at (4,0);
\coordinate (C) at (5,2);
\draw[name path=ray] ($ (B) ! 1.5 ! (A) $) -- ($ (A) ! 2.5 ! (B) $);
\fill (A) node[below] {$A$} circle[radius=3pt];
\fill (B) node[below] {$B$} circle[radius=3pt];
\fill (C) node[right] {$C$} circle[radius=3pt];
\path[name path=arc] (C) ++(-160:3.5cm) arc (-160:-20:3.5cm);
\path [name intersections={of=arc and ray,by={D,E}}];
\fill (D) node[below left] {$D$} circle[radius=3pt];
\fill (E) node[below right] {$E$} circle[radius=3pt];
\draw[name path=larc] (D) ++(-60:3.5cm) arc (-60:60:3.5cm);
\draw[name path=rarc] (E) ++(-120:3.5cm) arc (-120:-240:3.5cm);
\path [name intersections={of=larc and rarc,by={b,t}}];
\fill (b) circle[radius=3pt];
\draw ($ (b) ! 1.2 ! (t)$) -- ($ (t) ! 1.2 ! (b)$);
\end{tikzpicture}
\selectlanguage{hebrew}
\caption{%
אנך לקו
$AB$
העובר דרך
$C$.
}\label{fig.perp-3}
\end{center}
\end{figure}
\vspace*{-6ex}

למה אוקלידס לא הביא בנייה זו? כפי שהזכרנו לעיל, הוכחת הנכונות של הבנייה של אנך אמצעי כלל לא פשוטה, בעוד הוכחת הנכונות של הבנייה של אוקלידס מסתמכת על משפט פשוט אחד.


%%%%%%%%%%%%%%%%%%%%%%%%%%%%%%%%%%%%%%%%%%%%%%%%%%%%%%%%%%%%%%%

\section{
הגבלות והרחבות של בנייה באמצעות סרגל ומחוגה
}

ראינו שמה שניתן לבנות עם סרגל ומחוגה קבועה ניתן לבנות עם סרגל ומחוגה מתקפלת. מתמטיקאים חקרו אפשרות מוגבלות יותר:

\begin{itemize}
\item
כל בנייה עם סרגל ומחוגה ניתנת לבנייה עם מחוגה בלבד! כמובן, אם אין סרגל לא נראה קווים, אבל שתי נקודות במישור מגדירות קו ואין צורך ממש לראות אותו. למשל, אם ניתנות שתי נקודות
$A,B$
אפשר לבנות עם מחוגה בלבד נקודה
$C$
שהמרחק שלה מ-%
$A$
ומ-%
$B$
שווה למרחק
$AB$
)איור~%
\ref{fig.mm}(.
בנינו משולש שווה צלעות, אמנם ללא צלעות. המשפט הוכח בשנת
\L{1672}
על ידי
\L{Georg Mohr}
ובאופן עצמאי בשנת
\L{1797}
על ידי
\L{Lorenzo Mascheroni}.
\begin{figure}[H]
\begin{center}
\selectlanguage{english}
\begin{tikzpicture}[scale=0.5]
\coordinate (A) at (0,0);
\coordinate (B) at (4,0);
\path (A) node[below left] {$A$} -- (B) node[below right] {$B$};
\fill (A) circle[radius=3pt];
\fill (B) circle[radius=3pt];
\draw[name path=larc] (A) ++(-10:4cm) arc (-10:80:4cm);
\draw[name path=rarc] (B) ++(-170:4cm) arc (-170:-260:4cm);
\path [name intersections={of=larc and rarc,by={t}}];
\fill (t) node[above] {$C$} circle[radius=3pt];
\end{tikzpicture}
\selectlanguage{hebrew}
\caption{%
בניית משלוש שווה צלעות עם מחוגה בלבד.%
}\label{fig.mm}
\end{center}
\end{figure}
\vspace*{-8ex}
\item
אי-אפשר להסתפק בסרגל בלבד, אבל אם קיים במישור מעגל אחד בלבד )לא משנה איפה מרכז המעגל או הרדיוס שלו(, ניתן לבנות את כל מה שאפשר לבנות עם סרגל ומחוגה. המשפט הוכח ב-%
\L{1833}
על ידי 
\L{Jacob Steiner}.
\end{itemize}
ההוכחות של שני המשפטים מעט ארוכות אבל לא מסובכות במיוחד. אפשר למצוא אותן בספרו של
\L{Heinrich D\"{o}rrie}
\L{\cite{dorrie1}}.
ספר זה נגיש יותר במהדורה חדשה 
\L{\cite{dorrie2}}.

בכיוון השני, היוונים חקרו מה אפשר לבנות אם משתמשים בכלים מורכבים יותר מסרגל )ללא סימנים( ומחוגה. במאה ה-%
\L{19}
הוכח שלא ניתן לחלק זווית לשלושה חלקים שווים באמצעות סרגל ומחוגה. ניתן לבצע את הבנייה עם סרגל בעל שני סימנים הנקרא
\L{neusis}
או עם מכשיר הנקרא
\L{quadratrix},
המורכב משני סרגלים המחוברים כך שהם יכולים לגלוש אחד ליד השני ולהסתובב. כתבתי מסמך המתאר את הבניות
\textbf{איך לחלק זווית לשלושה )אם אתם מוכנים לרמות(}
שניתן למצוא באתר שלי:\\
\selectlanguage{english}
\url{http://www.weizmann.ac.il/sciÎtea/benari/mathematics}.


%%%%%%%%%%%%%%%%%%%%%%%%%%%%%%%%%%%%%%%%%%%%%%%%%%%%%%%%%%%%%%%

\selectlanguage{hebrew}

\section{
אין לסמוך על ציור
}
בסעיף~%
\ref{sec.error},
ראינו שאין לסמוך על ציור. כדי להדגים את המלכודת הממתינה למי שמסתמך על ציור, אני מביא הוכחה 
\textbf{\R{שכל}}
משולש הוא משולש שווי שוקיים!

באיור~%
\ref{fig.isosceles},
$\triangle ABC$
הוא משלוש שרירותי. $P$ היא נקודת החיתוך בין חוצה הזווית של
$\angle A$
לבין האנך האמצעי 
$DP$
ל-% 
$BC$.
$E,F$
הן נקודות החיתוך של האנכים מ-%
$P$
לצלעות
$AB,AC$.
\begin{figure}[H] 
\begin{center}
\selectlanguage{english}
\begin{tikzpicture}[scale=1.3]
\coordinate (P) at (0,0);
\node[xshift=4mm,yshift=1mm] at (P) {$P$};
\coordinate [label=left:$B$] (B)  at (-2,-2);
\coordinate [label=right:$C$] (C)  at (4,-2);
\coordinate [label=above:$A$] (A)  at (-1,2);
\node[below,yshift=-12pt,xshift=2pt] at (A) {$\alpha$};
\node[below,yshift=-12pt,xshift=15pt] at (A) {$\alpha$};
\draw (A) -- (B);
\draw (A) -- (C);
\draw (B) -- (C);
\draw (A) -- (P);
\draw (B) -- (P);
\draw (C) -- (P);
\coordinate[label=left:$E$] (E) at ($ (A) ! .44 ! (B) $);
\draw[rotate=-100] (E) rectangle +(4pt,4pt);
\draw (P) -- (E);
\coordinate (F) at ($ (A) ! .33 ! (C) $);
\node[right,xshift=2pt,yshift=2pt] at (F) {$F$};
\draw[rotate=-132] (F) rectangle +(4pt,4pt);
\draw (P) -- (F);
\coordinate[label=below:$D$] (D) at ($ (B) ! .33 ! (C) $);
\draw (D) rectangle +(4pt,4pt);
\draw (P) -- (D);
\node[left] at ($ (A) ! .5 ! (E) $) {};
\node[left] at ($ (B) ! .5 ! (E) $) {};
\node[below] at ($ (B) ! .5 ! (D) $) {$a$};
\node[below] at ($ (C) ! .5 ! (D) $) {$a$};
\node[right,xshift=2pt] at ($ (A) ! .5 ! (F) $) {};
\node[right,xshift=2pt] at ($ (C) ! .5 ! (F) $) {};
\foreach \n in {A,B,C,D,E,F,P} {
  \fill (\n) circle[radius=1pt];
}
\end{tikzpicture}
\selectlanguage{hebrew}
\caption{%
משולש שווה שוקיים%
}\label{fig.isosceles}
\end{center}
\end{figure}
\vspace*{-8ex}
המשולשים
$\triangle APE, \triangle APF$
הם משולשים ישר זווית עם זוויות שוות
$\alpha$
וצלע
$AP$
משותף, ולכן המשולשים חופפים לפי זווית-זווית-צלע, ו-%
$AE=AF$.

$PD$
הוא אנך אמצעי כך ש-% 
$BD=DC$.
הצלע 
$PD$
משותף, ולכן המשולשים
$\triangle DPB, \triangle DPC$
חופפים לפי צלע-זווית-צלע, וניתן להסיק ש-%
$PB=PC$.

כבר הראינו ש-%
$\triangle APE, \triangle APE$
חופפים ולכן
$EP=FP$.
מכאן שהמשלושים
$\triangle EPB, \triangle FPC$
חופפים כי הם משלושים ישר זווית עם שני צלעות שווים. )ניתן לחשב את הצלע השלישי לפי משפט פיתגורס כך שהם חופפים לפי צלע-צלע-צלע.( במשולשים החופפים,
$EB=FC$.

ניתן להסיק ש-%
$AE+EB=AF+FC$
והמשולש
$\triangle ABC$
שווה שוקיים.


ההוכחה
\textbf{%
נכונה%
}
אבל הציור אינו נכון. הכנתי קובץ גיאוגברה
\L{\texttt{isosceles.ggb}}
המראה משולש עבורו הנקודה
$P$
נמצאת 
\textbf{\R{מחוץ}}
למשולש.




%%%%%%%%%%%%%%%%%%%%%%%%%%%%%%%%%%%%%%%%%%%%%%%%%%%%%%%%%%%%%%%

\newpage

\selectlanguage{english}
\bibliographystyle{plain}
\renewcommand{\refname}{\raggedleft\Large\R{מקורות}}
\bibliography{compass}


\end{document}

