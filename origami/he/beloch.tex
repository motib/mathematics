% !TeX root = origami-math-he.tex

%%%%%%%%%%%%%%%%%%%%%%%%%%%%%%%%%%%%%%%%%%%%%%%%%%%%%%%%%%%%%%%%
\chapter{%
הקיפול של
\L{Beloch}
והריבוע של
\L{Beloch}}\label{c.beloch}

\section{%
הקיפול של
\L{Beloch}}\label{s.beloch-fold}

\L{Margharita P. Beloch}
גילתה קשר מרתק בין אוריגמי והשיטה של
\L{Lill}
למציאת שורשים של פולינומים ממעלה שלוש. היא מצאה ששהפעלה אחת בלבד של אקסיומה%
~$6$
)פרק%
~\L{\ref{s.ax6}}(
מאפשרת מציאת שורש ממשי של כל פולינום ממעלה שלוש. לכבודה, לעתים מכנים את הפעולה של האקסיומה "הקיפול של
\L{Beloch}".

נדגים את השיטה על הפולינום
$p(x)=x^3+6x^2+11x+6$
מסעיף%
~\L{\ref{s.magic}}.
באיור שלהלן הדגשנו את המסלול השני ושנינו את הסימנים של מספר נקודות. כדי לפתור את המשוואה, כל שעלינו לעשות הוא להפעיל קיפול של 
\L{Beloch}
כדי להניח בבת אחת את הנקודות 
$P',Q'$
על הקווים
$a_2,a_1$,
בהתאמה. לכאורה זה נראה כשימוש פשוט של הקיפול של
\L{Beloch}.
אולם, עם מפעילים את האקסיומה, המסלול מוצא פתרון למשוואה: 
$Q'$
נמצא רחוק ימינה, כך שהזווית ב-%
$P'$
ו-%
$Q'$
אינן זוויות ישרות.
\begin{center}
\selectlanguage{english}
\begin{tikzpicture}[scale=.8]
% Draw help lines and axes
\draw[step=10mm,white!60!black] (-11,-1) grid (2,7);
\draw[thick] (-11,0) -- (2,0);
\draw[thick] (0,-1) -- (0,7);
\foreach \x in {-10,...,2}
  \node at (\x-.3,-.2) {\sm{\x}};
\foreach \y in {1,...,7}
  \node at (-.2,\y-.3) {\sm{\y}};
  
%Draw first path with five points
\coordinate (A) at (0,0);
\coordinate (B) at (1,0);
\coordinate (C) at (1,6);
\coordinate (D) at (-10,6);
\coordinate (E) at (-10,0);
\foreach \x in {A,B,C,D,E}
  \fill (\x) circle(2pt);
\node[below right,yshift=-6pt]  at (A) {$P$};
\node[below left,yshift=-6pt] at (E) {$Q$};

\draw[thick] (A) -- (B);
\draw[thick,name path=bc] (B) --
  node[right,near end] {$a_2$} (C);
\draw[thick,name path=cd] (C) -- 
  node[above] {$a_1$} (D);
\draw[thick,name path=de] (D) -- (E);

% Draw first segment of second path
\path[name path=a2] (A) -- +(63.4:4);
\path [name intersections = {of = a2 and bc, by = {A2}}];
\fill (A2) circle(2pt) node[above right] {$P'$};
\draw[ultra thick,dotted] (A) -- (A2);
\draw[rotate=153.4] (A2) rectangle +(10pt,10pt);

% Draw second segment of second path
\path[name path=b2] (A2) -- +(153.4:10);
\path [name intersections = {of = b2 and cd, by = {B2}}];
\fill (B2) circle(2pt) node[above left]  {$Q'$};
\draw[ultra thick,dotted] (A2) -- (B2);
\draw[rotate=243.4] (B2) rectangle +(10pt,10pt);

% Draw third segment of second path
\draw[ultra thick,dotted] (B2) -- (E);
\end{tikzpicture}
\end{center}

\newpage

נזכור שקיפול הוא האנך האמצעי של קטע הקו בין נקודה ושיקופה מסביב לקיפול. אנו רוצים שהקיפול יהיה
$\overline{P'Q'}$
כך שהוא יהיה ניצב גם ל-%
$\overline{QQ'}$
וגם ל-%
$\overline{PP'}$.
אם 
$\overline{P'Q'}$
הוא האנך האמצעי של
$\overline{QQ'}$
ו-%
$\overline{PP'}$,
אזי
$P',Q'$,
השיקופים של
$P,Q$,
חייבים להיות באותו מרחק מהקיפול כמו
$P$
ו-%
$Q$,
בהתאמה. עם שינוי קל בסימונים, מתקבל האיור שלהלן:
%\newpage
\begin{center}
\selectlanguage{english}
\begin{tikzpicture}[scale=.8]
% Draw help lines and axes
\draw[step=10mm,white!60!black] (-11,-1) grid (3,13);
\draw[thick] (-11,0) -- (3,0);
\draw[thick] (0,-1) -- (0,13);
\foreach \x in {-10,...,3}
  \node at (\x-.3,-.2) {\sm{\x}};
\foreach \y in {1,...,13}
  \node at (-.2,\y-.3) {\sm{\y}};
  
% Draw first path with five points
\coordinate (A) at (0,0);
\coordinate (B) at (1,0);
\coordinate (C) at (1,6);
\coordinate (D) at (-10,6);
\coordinate (E) at (-10,0);
\foreach \x in {A,B,C,D,E}
  \fill (\x) circle(2pt);
\node[below right,yshift=-6pt] at (A) {$P$};
\node[below left,yshift=-6pt] at (E) {$Q$};

\draw[thick] (A) -- (B);
\draw[thick,name path=bc] (B) -- node[right,near end] {$a_2$} (C);
\draw[thick,name path=cd] (C) -- node[above] {$a_1$} (D);
\draw[thick,name path=de] (D) -- (E);

% Draw parallel lines
\draw[ultra thick,dotted,name path=bpcp] ($(B)+(1,-1)$) --
  node[above right] {$a_2'$}
  ($(C)+(1,7)$);
\draw[ultra thick,dotted,name path=cpdp] ($(C)+(2,6)$) -- 
  node[above left,xshift=-24pt] {$a_1'$} 
  ($(D)+(-1,6)$);

% Draw first segment of second path
\path[name path=a2] (A) -- +(63.4:4);
\path [name intersections = {of = a2 and bc, by = {A2}}];
\draw[ultra thick,dotted] (A) -- (A2);
\fill (A2) circle(2pt) node[above right,xshift=4pt] {$R$};
\draw[rotate=153.4] (A2) rectangle +(10pt,10pt);

% Draw second segment of second path
\path[name path=b2] (A2) -- +(153.4:10);
\path [name intersections = {of = b2 and cd, by = {B2}}];
\fill (B2) circle(2pt) node[above left]  {$S$};
\draw[very thick,dashed] (A2) -- (B2);
\draw[rotate=243.4] (B2) rectangle +(10pt,10pt);

% Draw third segment of second path
\draw[ultra thick,dotted] (B2) -- (E);

% Locate reflections on parallel lines and draw lines
\coordinate (PP) at ($(A2)+(1,2)$);
\fill (PP) circle(2pt) node[above right] {$P'$};
\draw[ultra thick,dotted] (A2) -- (PP);

\coordinate (QP) at ($(B2)+(3,6)$);
\fill (QP) circle(2pt) node[above right] {$Q'$};
\draw[ultra thick,dotted] (B2) -- (QP);
\end{tikzpicture}
\end{center}
נבנה את הקו
$a_2'$
מקביל ל-%
$a_2$
ובאותו מרחק מ-%
$a_2$
כמו המרחק של
$a_2$
מ-%
$P$.
באופן דומה נבנה את הקו
$a_1'$
מקביל ל-%
$a_1$
ובאותו מרחק מ-%
$a_1$
כמו המרחק של
$a_1$
מ-%
$Q$.
נפעיל את אקסיומה%
~$6$
כדי להניח בו-זמנית את 
$P$
ב-%
$P'$
על 
$a_2'$
ולהניח את 
$Q$
ב-%
$Q'$
על
$a_1'$.
הקיפול 
$\overline{RS}$
הוא האנך האמצעי של הקווים
$\overline{PP'}$
ו-%
$\overline{QQ'}$.
מכאן שהזוויות ב-%
$R$
ו-%
$S$
הן ישרות כפי שמתחייב.

\newpage

ננסה את הקיפול של 
\L{Beloch}
על הפולינום
$x^3-3x^2-3x+1$
מסעיף%
~\L{\ref{s.negative}}. $a_2$
הוא קטע הקו האנכי באורך
$3$
שהמשוואה לו הוא
$x=1$,
והקו המקביל לו הוא
$a_2'$
שהמשוואה שלו היא
$x=2$,
כי 
$P$
נמצאת במרחק 
$1$ 
מ-%
$a_2$.
$a_1$
הוא קטע הקו האופקי באורך 
$3$
שהמשוואה שלו היא
$y=-3$,
והקו המקביל לו הוא
$a_1'$
שהמשוואה שלו הוא
$y=-2$
כי 
$Q$
נמצאת במרחק
$1$
מ-%
$a_1$.
הקיפול 
$RS$
הוא האנך האמצעי של
$\overline{PP'}$
וגם של
$\overline{QQ'}$.
המסלול
$\overline{PRSQ}$
זהה למסלול בסעיף%
~\L{\ref{s.negative}}.

\begin{center}
\selectlanguage{english}
\begin{tikzpicture}[scale=1]
% Draw help lines and axes
\draw[step=10mm,white!50!black] (-1,-5) grid (6,2);
\foreach \x in {0,...,6}
  \node at (\x-.3,-.2) {\sm{\x}};
\foreach \y in {-4,...,-1}
  \node at (-.3,\y-.3) {\sm{\y}};
\foreach \y in {1,...,2}
  \node at (-.3,\y-.3) {\sm{\y}};

% Draw first path
\coordinate (A) at (0,0);
\coordinate (B) at (1,0);
\coordinate (C) at (1,-3);
\coordinate (D) at (4,-3);
\coordinate (E) at (4,-4);
\node[above left] at (A) {$P$};
\node[below right] at (E) {$Q$};
\foreach \x in {A,B,C,D,E}
  \fill (\x) circle(2pt);

\draw[very thick,{Stealth[scale=1.4,inset=2pt,reversed]}-] (A) --
  (B);
\draw[very thick,{Stealth[scale=1.4,inset=2pt]}-,name path=bc] (B) -- 
  node[left] {$a_2$} (C);
\draw[very thick,{Stealth[scale=1.4,inset=2pt]}-,name path=cd] (C) --
  node[above] {$a_1$}(D);
\draw[very thick,{Stealth[scale=1.4,inset=2pt,reversed]}-,name path=de] (D) --
 (E);

% Draw extensions of first path
\draw[very thick,loosely dotted,name path=b] (1,-4) -- (1,2);
\draw[very thick,loosely dotted,name path=c] (-1,-3) -- (6,-3);

% Draw reflected points
\coordinate (PP) at (2,2);
\coordinate (QP) at (6,-2);
\fill (PP) circle(2pt) node[above left] {$P'$};
\fill (QP) circle(2pt) node[below right] {$Q'$};

% Midpoints of bisected lines
\coordinate (R) at (1,1);
\coordinate (S) at (5,-3);
\fill (R) circle(2pt) node[above left] {$R$};
\fill (S) circle(2pt) node[below right] {$S$};

% Draw reflected lines
\draw[ultra thick,dotted] ($(B)+(1,2)$) --
  node[right,very near end,yshift=-8pt] {$a_2'$} ($(C)+(1,-2)$);
\draw[ultra thick,dotted] ($(C)+(-2,1)$) --
  node[above,very near start,xshift=-8pt,yshift=-1pt] {$a_1'$} ($(D)+(2,1)$);
\draw[ultra thick,dotted] (A) -- (PP);
\draw[ultra thick,dotted] (E) -- (QP);

% Draw fold
\draw[very thick,dashed] (R) -- (S);
\draw[rotate=-45] (R) rectangle +(8pt,8pt);
\draw[rotate=45] (S) rectangle +(8pt,8pt);
\end{tikzpicture}
\end{center}

\newpage

\section{הריבוע של
\L{Beloch}}\label{s.beloch-square}

ניתן להציג את הבנייה בסעיף הקודם לפי הריבוע של
\L{Beloch}:
נתונות שתי נקודות
$P,Q$
ושני קווים
$a_2,a_1$,
בנה ריבוע
$\overline{ARSB}$
כך ש:
\begin{itemize}
\item 
צלע אחד הוא
$\overline{RS}$
כאשר
$R$
נמצאת על
$a_2$
ו-%
$S$
נמצאת על
$a_1$;
\item $P$
נמצאת על
$\overline{RA}$
ו-%
$Q$
נמצאת על
$\overline{SB}$.
\end{itemize}
האיור שלהלן מדגים את הריבוע של
\L{Beloch}
עבור הפולינום
$x^3+6x^2+11x+6$.
האורך של
$RS$
הוא
$\sqrt{80}=4\sqrt{5}\approx 8.94$. 
ניתן לבנות את הריבוע על ידי הוספת שלושה צלעות של אורך זה.
\begin{center}
\selectlanguage{english}
\begin{tikzpicture}[scale=.8]
% Draw help lines and axes
\draw[step=10mm,white!60!black] (-12,-7) grid (2,7);
%\draw[thick] (-12,0) -- (2,0);
%\draw[thick] (0,-7) -- (0,7);
\foreach \x in {-11,...,3}
  \node at (\x-.3,-.2) {\sm{\x}};
\foreach \y in {1,...,7}
  \node at (-.4,\y-.3) {\sm{\y}};
\foreach \y in {-6,...,-1}
  \node at (-.4,\y-.3) {\sm{\y}};

% Draw first path
\coordinate (A) at (0,0);
\coordinate (B) at (1,0);
\coordinate (C) at (1,6);
\coordinate (D) at (-10,6);
\coordinate (E) at (-10,0);
\foreach \x in {A,B,C,D,E}
  \fill (\x) circle(2pt);
\draw[thick] (A) -- (B);
\draw[thick,name path=bc] (B) -- node[right,near end] {$a_2$} (C);
\draw[thick,name path=cd] (C) -- node[above] {$a_1$} (D);
\draw[thick,name path=de] (D) -- (E);

% Find first segment of second path
\path[name path=a2] (A) -- +(63.4:4);
\path [name intersections = {of = a2 and bc, by = {A2}}];
\draw[rotate=153.4] (A2) rectangle +(10pt,10pt);

% Draw second segment of second path
\path[name path=b2] (A2) -- +(153.4:10);
\path [name intersections = {of = b2 and cd, by = {B2}}];
\draw[very thick,dashed] (A2) -- (B2);
\draw[rotate=243.4] (B2) rectangle +(10pt,10pt);

% Draw square
\draw[very thick,dashed] (B2) -- +(243.4:8.94) coordinate (AB);
\draw[very thick,dashed] (A2) -- +(243.4:8.94) coordinate (BB);
\fill (AB) circle (2pt) node[below left] {$B$};
\fill (BB) circle (2pt) node[below right] {$A$};
\draw[very thick,dashed] (AB) -- (BB);

\path[fill=white!85!black,rotate=-26.6] (AB) rectangle +(8.94,8.94);

% Draw labels of points
\fill (A2) circle(2pt) node[above right,xshift=4pt] {$R$};
\fill (B2) circle(2pt) node[above left]  {$S$};
\fill (A)  circle(2pt) node[below right,yshift=-6pt]  {$P$};
\fill (E)  circle(2pt) node[above left]  {$Q$};
\end{tikzpicture}
\end{center}
