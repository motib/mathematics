% !TeX root = origami-math-he.tex

\selectlanguage{hebrew}
\appendix


\chapter{קישוריות לגיאוגברה}\label{a.geo}

\begin{center}
\begin{tabular}{|l|l|}
\hline
$1$ \R{אקסיומה}& \url{https://www.geogebra.org/m/fq9d5hms}\\\hline
$2$ \R{אקסיומה}& \url{https://www.geogebra.org/m/fgmfss27}\\\hline
$3$ \R{אקסיומה}& \url{https://www.geogebra.org/m/ek3mqupw}\\\hline
$4$ \R{אקסיומה}& \url{https://www.geogebra.org/m/renzzbdg}\\\hline
$5$ \R{אקסיומה}& \url{https://www.geogebra.org/m/aszn9ywu}\\\hline
$6$ \R{אקסיומה}& \url{https://www.geogebra.org/m/bxe5e5ku}\\\hline
$7$ \R{אקסיומה}& \url{https://www.geogebra.org/m/yeq5gmeg}\\\hline
Abe \R{חקלוקת זווית לשלושה של} & \url{https://www.geogebra.org/m/dxrcvjam}\\\hline
Martin \R{חקלוקת זווית לשלושה של}& \url{https://www.geogebra.org/m/caky7edd}\\\hline
Messer \R{הכפלת קוביה של} & \url{https://www.geogebra.org/m/mrcwjqh8}\\\hline
Beloch \R{הכפלת קוביה של}  & \url{https://www.geogebra.org/m/enzmmwua}\\\hline
\end{tabular}
\end{center}
לאור תקלה בגיאוגברה, בפרוייקטים המשתמשים באקסיומה $6$, נקודות המוגדרות על ידי שיקוף מסביב למשיק המשותף אינן נשמרות או נשמרות בצורה שגוייה.

\chapter{פיתוח הזהיות הטריגונומטריות}\label{a.tangent}

ניתן לפתח את הזהויות הטריגונומטריות עבור טנגנס שהשתמשנו בהוכחה של אקסיומה~$3$ מזהויות עבור סינוס וקסינוס:

\begin{form}{2.2}
\tan (\theta_1+\theta_2) &=& \disfrac{\sin(\theta_1+\theta_2)}{\cos(\theta_1+\theta_2)}\\
&=&\disfrac{\sin\theta_1\cos\theta_2+\cos\theta_1\sin\theta_2}{\cos\theta_1\cos\theta_2-\sin\theta_1\sin\theta_2}\\
&=&\disfrac{\sin\theta_1+\cos\theta_1\tan\theta_2}{\cos\theta_1-\sin\theta_1\tan\theta_2}\\
&=&\disfrac{\tan\theta_1+\tan\theta_2}{1-\tan\theta_1\tan\theta_2}\,.
\end{form}
נשמתש משוואה זו עם
$\theta=(\theta/2)+(\theta/2)$
ונקבל משוואה ריבועית במשתנה
$\tan(\theta/2)$:
\begin{form}{2.2}
\tan \theta=\disfrac{\tan(\theta/2)+\tan(\theta/2)}{1-\tan^2(\theta/2)}\\
\tan\theta \,(\tan(\theta/2))^2 \;+\; 2\,(\tan (\theta/2)) \;-\;\tan \theta = 0\,.
\end{form}
הפתרונות שלה הם:
\[
\tan(\theta/2) = \disfrac{-1\pm\sqrt{1+\tan^2\theta}}{\tan\theta}\,.
\]

\chapter{פרבולות}\label{a.parabola}

לראשנוה, תלמידים מכירים פרבולות כגרפים של פולינומים ריבועיים:
\[
y=ax^2+bx+c\,.
\]
אולם, ניתן הגדיר פרבולות באמצעות גיאומטריה: נתונה נקודוה, 
\textbf{המוקד} \L{(focus)},
ונתון קו,
\textbf{המדריך} \L{(directrix)},
המקום הגיאומטרי של הנקודות הנמצאות במרחק שווה מהמוקד ומהמדריך מגדיר פרבולה.

האיור שלהלן מראה את המוקד -- הנקודה הגדולה ב-%
$p=(0,f)$, 
והמדריך -- הקו העבה עם המשוואה
$y=-f$. 
הפרבולה המתקבלת מוצגת בקו מנוקד. 
\textbf{הקודקוד} \L{(vertex)}
שלה 
$p_2$
נמצא במרכז הצירים.

%\begin{figure}[H]
\begin{center}
\selectlanguage{english}
\begin{tikzpicture}[scale=1.1]
\draw (-6,0) -- node[very near start,below,xshift=-32pt] {$x$-\textsf{axis}} (6,0);
\draw (0,-3) -- node[very near start,right,yshift=-20pt] {$y$-\textsf{axis}} (0,4.5);
\draw[ultra thick] (-6,-2) -- node[near start,below] {\textsf{directrix} $\quad y=-f$} (6,-2);
\draw[domain=-6:6,samples=50,very thick,dotted] plot (\x,{\x*\x/8});
\coordinate (F) at (0,2);
\fill (F) circle (3pt) node[above left,xshift=-2pt,yshift=15pt] {$(0,f)$} node[above left,xshift=-2pt,yshift=30pt] {\textsf{focus}} node[above right] {$p$};
\fill (0,0) circle (1.5pt) node[below right] {$p_2$};
\fill (0,-2) circle (1.5pt);
\fill (2,-2) circle (1.5pt);
\fill (3,-2) circle (1.5pt);
\fill (5,-2) circle (1.5pt);
\coordinate (FP) at (-5,-2);
\fill (FP) circle (1.5pt) node[below] {$p'$};
\coordinate (F1) at (2,.5);
\fill (F1) circle (1.5pt) node[below right] {$p_3$};
\coordinate (F2) at (3,1.125);
\fill (F2) circle (1.5pt) node[below right] {$p_4$};
\coordinate (F3) at (5,3.125);
\fill (F3) circle (1.5pt) node[below right] {$p_5$};
\coordinate (F4) at (-5,3.125);
\fill (F4) circle (1.5pt) node[above right] {$p_1$};
\draw (F) -- node[left] {$a_2$} (0,0) -- node[left] {$a_2$} (0,-2);
\draw (F) -- node[near end,left] {$a_3$} (F1) -- node[left] {$a_3$} (2,-2);
\draw (F) -- node[near end,above] {$a_4$} (F2) -- node[left] {$a_4$} (3,-2);
\draw (F) -- node[above] {$a_5$} (F3) -- node[left] {$a_5$} (5,-2);
\draw (F) -- node[above] {$a_1$} (F4) -- node[left] {$a_1$} (FP);
\draw[thick,dashed] ($(F4)!-.4!(-2.5,0)$) -- node[very near end,right,xshift=2pt] {$l$} ($(F4)!1.8!(-2.5,0)$);
\draw[very thick,dotted] (F) -- (FP);
\coordinate (H) at (-2.5,0);
\node[above,xshift=1pt] at (H) {$?$};
\draw[rotate=36] (H) rectangle +(10pt,10pt);
\path (FP) -- node[above,yshift=2pt] {$b?$} (H) -- node[above,yshift=2pt] {$b?$} (F);
\draw (0,-2) rectangle +(8pt,8pt);
\draw (2,-2) rectangle +(8pt,8pt);
\draw (3,-2) rectangle +(8pt,8pt);
\draw (5,-2) rectangle +(8pt,8pt);
\draw (-5,-2) rectangle +(8pt,8pt);
\end{tikzpicture}
\end{center}
%\end{figure}
בחרנו חמש נקודות
$p_i$, $i=1,\ldots,5$
על הפרבולה. כל נקודה
$p_i$
הוא במרחק
$a_i$
גם מהמוקד וגם מהמדריך. נוריד ניצב למדריך מ-%
$p_1$,
ונמסן ב-%
$p'$
את נקודת החיתוך של הניצב עם המדריך.
$p_1$
נמצאת על הפרבולה, ולכן
$\overline{p'p_1}=\overline{p_1p}=a_1$.
אנו טוענים שהמשיק 
$l$
לפרבולה בנקודה 
$p_1$
)הקו המקווקוו( הוא קיפול המשקף את 
$p$
על
$p'$.

\newpage

נוכיח ש-%
$l$
הוא האנך האמצעי של
$\overline{pp'}$. 
נחלץ איור פשוט מהאיור הקודם:
\begin{center}
\selectlanguage{english}
\begin{tikzpicture}[scale=1.1]
\draw[thick] (-6.4,0) -- node[very near start,below,xshift=-20pt] {$x$-\textsf{axis}} (1,0);
\draw[thick] (0,-3) -- node[very near start,right,yshift=-20pt] {$y$-\textsf{axis}} (0,3.5);
\draw[thick] (-6.4,-2) -- (1,-2);
\coordinate (F) at (0,2);
\fill (F) circle (1.5pt) node[right] {$p\;\;$ \textsf{focus}};
\fill (0,0) circle (1.5pt) node[below right] {$p_2$};
\coordinate (FP) at (-5,-2);
\fill (FP) circle (1.5pt) node[below] {$p'$};
\path (FP) -- node[below,yshift=-2pt] {\textsf{directrix}} (0,-2);
\fill (0,-2) circle (1.5pt) node[below right] {$s$};
\coordinate (F4) at (-5,3.125);
\fill (F4) circle (1.5pt) node[above right] {$p_1$};
\draw (F) -- node[right] {$a_2$} (0,0) -- node[right] {$a_2$} (0,-2);
\draw (F) -- node[above] {$a_1$} (F4) -- (FP);
\draw[<->] ($(F4)+(-.5,-.1)$) -- node[fill=white] {$a_1$}($(FP)+(-.5,.1)$);
\draw[thick,dashed] ($(F4)!-.15!(-2.5,0)$) -- node[very near end,right,xshift=2pt] {$l$} ($(F4)!1.8!(-2.5,0)$);
\draw[very thick,dotted] (F) -- (FP);
\coordinate (H) at (-2.5,0);
\node[above,xshift=1pt] at (H) {\textsf{\bfseries ?}};
\draw[rotate=36] (H) rectangle +(10pt,10pt);
\path (FP) -- node[above,yshift=2pt] {$b$ \textsf{\bfseries ?}} (H) -- node[above,yshift=2pt] {$b$ \textsf{\bfseries ?}} (F);
\draw (0,0) rectangle +(8pt,8pt);
\draw (0,-2) rectangle +(8pt,8pt);
\draw (-5,0) rectangle +(8pt,8pt);
\draw (-5,-2) rectangle +(8pt,8pt);
\fill (-5,0) circle (1.5pt) node[below right] {$q$};
\fill (-2.5,0) circle (1.5pt) node[below,yshift=-4pt] {$r$};
\path (-5,0) -- node[right] {$a_2$} (-5,-2);
\path (F4) -- node[left,xshift=-2pt] {$a$} (-2.5,0);
\end{tikzpicture}
\end{center}
\begin{itemize}
\item 
המדריך מקביל לציר ה-%
$x$,
המוקד
$p$
נמצא על ציר ה-
$y$, 
ו-%
$\overline{p_1p'}$ 
ניצב למדריך. מכאן ש-%
$\angle p'qr$
ו-%
$\angle pp_2r$ 
הן זוויות ישרות.
\item 
$\overline{qp'}$ 
ו-%
 $\overline{p_2s}$
הם צלעות נגדיים של מלבן, כך ש-%
$\overline{qp'}=\overline{p_2s}$,
ו-%
$\overline{p_2s}$
בעצמו שווה ל-%
$\overline{pp_2}$
כי
$p_2$
נמצאת על הפרבולה ובמרחק שווה מ-%
$p$
ו-%
$s$.
\item $\angle qrp'$
ו-%
$\angle p_2rp$ 
הן זוויות קודקודיות.
\item 
למשולשים ישר-זווית
$\triangle qrp'$
ו-%
$\triangle p_2rp$ 
זווית חדה זהה וצלע שווה, לכן הם חופפים. מכאן ש-%
$\overline{p'r}=\overline{rp}$ 
 ו-%
$\overline{p_1r}$
הוא התיכון של
$\triangle pp_1p'$.
\item $p_1$
נמצאת על הפרבולה כך ש-%
$\overline{pp_1}=\overline{p_1p'}$.
לכן
$\triangle pp_1p'$
הוא משולש שווה שוקיים.
\item 
במשולש שווה-השוקיים
$\triangle pp_1p'$,
התיכון
$\overline{p_1r}$
הוא גם האנך האמצעי של
$\overline{pp'}$.
\item
$l$
הוא קיפול המכיל את קטע הקו
$\overline{p_1r}$.
\end{itemize}

