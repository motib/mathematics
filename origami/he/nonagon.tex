% !TeX root = origami-math-he.tex

%%%%%%%%%%%%%%%%%%%%%%%%%%%%%%%%%%%%%%%%%%%%%%%%%%%%%%%%%%%%%%%%
\chapter{בניית מתושע}\label{c.nonagon}

\section{בניית מצולעים משוכללים}

לפי משפט
\L{Gauss-Wantzel}
ניתן לבנות באמצעות סרגל ומחוגה מצולע משוכלל שמספר צלעותיו:
\[
n=2^k\cdot F_1 \cdot \:\cdots\: \cdot F_m\,,
\]
כאשר ה-%
$F_i$
)אם הם קיימים( הם מספרי 
\L{Fermat}
ראשוניים שונים 
$F_m=2^{2^m}+1$.
ידוע שקיימים חמישה מספרי 
\L{Fermat}
ראשוניים:
$F_0=3, F_1=5, F_2=17, F_3=257, F_4=65537$.%
\footnote{%
בגיל
$19$ \L{Gauss}
בנה מצולע משוכלל עם 
$17$
צלעות והישג זה שיכנע אותו )לשמחתנו( להיות מתמטיקאי. מצולע משוכלל עם 
$257$
צלעות נבנה על ידי
\L{Magnus Georg Paucker}
ב-%
$1822$
ועל ידי
\L{Friedrich Julius Richelot}
ב-%
$1832$.
\L{Johann Gustav Hermes}
טען בשנת 
$1894$
שהוא בנה מצולע משוכלל עם 
$65537$.
צלעות. כתב היד שלו שמור באוניברסיטת
\L{G\"{o}ttigen}
במקרה שתרצו לבדוק את נכונות הבנייה.%
} 
לכן לא ניתן לבנות
\textbf{מתושע},
מצולע משוכלל עם תשעה צלעות.


מצולעים משוכללים עם:
\[
n=2^i\cdot 3^j \cdot p_1 \cdot \: \cdots\: \cdot p_m
\]
צלעות ניתנים לבנייה באמצעות אוריגמי, אם ה-%
$p_i$
)אם הם קיימים(
הם מספרים ראשוניים שונים מהצורה
$2^k\cdot 3^l+1$
\L{\cite{alperin}}.
כאן נבנה מתושע תוך שימוש בשיטה של
\L{Lill}
והקיפול של
\L{Beloch}.

\section{המשוואה ממעלה שלוש עבור מתושע}

ניתן לבנות מצולע משוכלל עם 
$n$
צלעות על ידי בניית הזווית המרכזית
$360^\circ/n$.
עבור מתושע הזווית המרכזית היא
$\theta=360^\circ/9=40^\circ$:
\begin{center}
\selectlanguage{english}
\begin{tikzpicture}
\coordinate (O) at (0,0);
\fill (O) circle (1.5pt);
\foreach \x/\name in {0/a,40/b,80/c,120/d,160/e,200/f,240/g,280/h,320/i} {
  \coordinate (\name) at ($(O)+(\x:3cm)$);
  \draw (O) -- (\name);
  \fill (\name) circle (1.5pt);
}
\draw (a) -- (b) -- (c) -- (d) -- (e) -- (f) -- (g) -- (h) -- (i) -- cycle;
\node[above right,xshift=12pt] at (O) {$40^\circ$};
\end{tikzpicture}
\end{center}

הזווית
$40^\circ$
היא שליש מ-%
$120^\circ$
שניתנת לבנייה על ידי הצמדת זווית של 
$30^\circ$ 
)שניתנת לבנייה על ידי חציית זווית של משולש שווה צלעות( לזווית של
$90^\circ$
)שניתנת לבנייה על ידי בניית אנך(. ניתן לחלק זווית לשלושה חלקים שווים באוריגמי, ולכן ניתן לבנות 
$40^\circ$
באוריגמי. כאן נביא בנייה שונה המבוססת על מציאת פיתרון למשוואה ממעלה שלוש.

חישוב פשוט המשתמש בנוסחה עבור
$\cos(\alpha+\beta)$
מראה ש:
\[
\cos 3\theta=4\cos^3 \theta -3\cos\theta\,.
\]
נציב
$x=\cos \theta$ 
ו-%
$a=\cos 3\theta$.
אם נצליח לפתור את המשוואה ממעלה שלוש
$4x^3-3x-a=0$
עבור
$x$,
ניתן לבנות את הזווית על ידי בניית משולש ישר זווית עם צלע באורך 
$x$
ויתר באורך 
$1$.
הבנייה מתוארת בהמשך.


עבור מתושע המשוואה היא
$4x^3-3x+\frac{1}{2}=0$
כי
$\cos 120^\circ=-\frac{1}{2}$.

נבנה מסלול עבור המשוואה כפי שנדרש בשיטה של 
\L{Lill}:
\begin{center}
\selectlanguage{english}
\begin{tikzpicture}[scale=1.1]
% Draw help lines and axes
\draw[step=10mm,white!60!black] (-1,-4) grid (9,1);
\draw[thick] (-1,0) -- (9,0);
\draw[thick] (0,-4) -- (0,1);
\foreach \x in {1,...,9}
  \node at (\x-.3,.3) {\sm{\x}};
\foreach \y in {-3,...,1}
  \node at (-.3,\y-.3) {\sm{\y}};
  
% Points of first path
\coordinate (A) at (0,0);
\coordinate (B) at (4,0);
\coordinate (C) at (7,0);
\coordinate (D) at (7,-.5);
\foreach \x in {A,B,C,D}
  \fill (\x) circle(2pt);
\node[above left] at (A) {$P$};
\node[below right,xshift=12pt] at (A) {\sm{-37.4537^\circ}};
\node[below right] at (D) {$Q$};

% Draw first path
\draw[very thick,-{Stealth[scale=1.4,inset=2pt]}] 
  (A) -- node[below] {$a_3$} (B);
\draw[{Stealth[scale=1.4,inset=2pt,reversed]}-,very thick]
  (B) -- ($(B)+(0,.1)$);
\draw[name path=c,very thick,{Stealth[scale=1.4,inset=2pt]}-]
  (B) -- node[below] {$a_1$} (C);
\draw[very thick,-{Stealth[scale=1.4,inset=2pt]}]
  (C) -- node[right,yshift=-2pt] {$a_0$} (D);

% Draw extension of second segment of first path
\draw[very thick,loosely dotted,name path=b] 
  ($(B)+(0,-4)$) -- ($(B)+(0,1)$);

% Draw second path
\path[name path=one] (A) -- +(-37.4537:6cm);
\path [name intersections = {of = b and one, by = {R}}];
\fill (R) circle (2pt) node[below left] {$R$};
\draw[thick,dashed] (A) -- (R);

\path[name path=two] (R) -- +(52.5463:6cm);
\path [name intersections = {of = c and two, by = {S}}];
\fill (S) circle (2pt) node[above] {$S$};
\draw[thick,dashed] (R) -- (S);

\draw[thick,dashed] (S) -- (D);

% Draw right angle rectangles
\draw[thick,rotate=52.5463] (R) rectangle +(8pt,8pt);
\draw[thick,rotate=-127.4537] (S) rectangle +(8pt,8pt);
\end{tikzpicture}
\end{center}
המקדם 
$a_2$
הוא אפס ולכן מבצעים סיבוב של
$90^\circ$
אבל לא בונים קטע קו באותו כיוון.

המסלול השני מתחיל מ-%
$P$
בזווית 
$-37.4537^\circ$
ומשתקף ב-%
$\pm 90^\circ$
כנדרש על ידי השיטה של
\L{Lill}.
הקטע האחרון חותך את
$Q$,
ולכן
$x=-\tan -37.4537^\circ=0.766044$
הוא שורש של
$4x^3-3x+\frac{1}{2}$.



\section{פתרון המשוואה על ידי הקיפול של
\L{Beloch}}

בתיאור לעיל רימיתי כי ידעתי ששורש המשוואה הוא
$x=-\tan -37.4537^\circ=0.766044$:
\[
4(0.766044)^3-3(0.766044)+0.5\approx 0\,.
\]
השתמשתי בערך זה כאשר ציירתי את המסלול השני. 

ניתן למצוא את השורש באמצעות הקיפול של 
\L{Beloch}
כפי שתיארנו בפרק~%
\L{\ref{c.beloch}}.

נתחיל בלמתוח קו מקביל ל-%
$a_2$\footnote{
בגלל ש-%
$a_2=0$
הקו נמתח מקביל לקו שהיינו מציירים אם 
$a_2$
היה שונה מאפס.}
באותו מרחק מ-%
$a_2$
כמו המרחק מ-%
$a_2$
ל-%
$P$.
באופן דומה, נמתח קו מקביל ל-%
$a_1$
באותו מרחק מ-%
$a_1$
כמו המרחק מ-%
$a_1$
ל-%
$Q$.
הקיפול של
\L{Beloch},
הקו
$\overline{RS}$,
מניח בו-זמנית את
$P$
ב-%
$P'$
על הקו המקביל ל-%
$a_2$,
ואת
$Q$
ב-%
$Q'$
על הקו המקביל ל-%
$a_1$.
כך בונים את הזווית
$\angle SPR=-37.4537^\circ$.
המסלול השני של השיטה של 
\L{Lill}
הוא
$\overline{PRSQ}$.
\begin{center}
\selectlanguage{english}
\begin{tikzpicture}[scale=1.1]
% Draw help lines and axes
\draw[step=10mm,white!60!black] (-1,-7) grid (9,1);
\draw[thick] (-1,0) -- (9,0);
\draw[thick] (0,-7) -- (0,1);
\foreach \x in {1,...,9}
  \node at (\x-.3,.3) {\sm{\x}};
\foreach \y in {-6,...,1}
  \node at (-.3,\y-.3) {\sm{\y}};
  
% Points of first path
\coordinate (A) at (0,0);
\coordinate (B) at (4,0);
\coordinate (C) at (7,0);
\coordinate (D) at (7,-.5);
\foreach \x in {A,B,C,D}
  \fill (\x) circle(2pt);
\node[above right] at (A) {$P$};
\node[below right] at (D) {$Q$};

% Draw first path
\draw[very thick,-{Stealth[scale=1.4,inset=2pt]}] 
  (A) -- node[below] {$a_3$} (B);
\draw[{Stealth[scale=1.4,inset=2pt,reversed]}-,very thick]
  (B) -- ($(B)+(0,.1)$);
\draw[name path=c,very thick,{Stealth[scale=1.4,inset=2pt]}-]
  (B) -- node[below] {$a_1$} (C);
\draw[very thick,-{Stealth[scale=1.4,inset=2pt]}]
  (C) -- node[right,yshift=-2pt] {$a_0$} (D);

% Draw extension of second segment of first path
\draw[very thick,loosely dotted,name path=b] 
  ($(B)+(0,-7)$) -- ($(B)+(0,1)$);

% Draw second path
\path[name path=one] (A) -- +(-37.4537:6cm);
\path [name intersections = {of = b and one, by = {R}}];
\fill (R) circle (2pt) node[below left] {$R$};
%\draw[thick] (A) -- (R);

\path[name path=two] (R) -- +(52.5463:6cm);
\path [name intersections = {of = c and two, by = {S}}];
\fill (S) circle (2pt) node[above] {$S$};
\draw[thick,dashed] (R) -- (S);

% Draw parallel lines
\draw[very thick,loosely dotted,name path=para-2] 
  (8,1) -- (8,-7);
\draw[very thick,loosely dotted,name path=para-1] 
  (-1,.5) -- (9,.5);

% Draw second segments of the folds
\path[name path=p-two] (A) -- +(-37.4537:11cm);
\path [name intersections = {of = para-2 and p-two, by = {PP}}];
\fill (PP) circle (2pt) node[below left] {$P'$};
\draw[ultra thick,dotted] (A) -- (PP);

\path[name path=p-one] (D) -- +(142.5463:2cm);
\path [name intersections = {of = para-1 and p-one, by = {QP}}];
\fill (QP) circle (2pt) node[above] {$Q'$};
\draw[ultra thick,dotted] (D) -- (QP);

% Draw right angle indications
\draw[thick,rotate=-37.4537] (R) rectangle +(8pt,8pt);
\draw[thick,rotate=-127.4537] (S) rectangle +(8pt,8pt);
\end{tikzpicture}
\end{center}

\section{בניית הזווית המרכזית של המתושע}

אנו יודעים ש-%
$\cos\theta=0.766044$
הוא שורש של המשוואה. נוכל לבנות את המתושע אם נבנה 
$\cos^{-1} 0.766044=40^\circ$.
 נשמתש בבנייה לעיל של הזווית
$37.4537^\circ$
שהטנגנס שלה הוא
$0.766044$.

נתונים קטע קו באורך
$1$
והזווית
$37.4537^\circ$,
במשולש ישר זווית עם זווית זו הצלע ליד הזווית הוא 
$1$
והצלע מול הזווית הוא
$0.766044$,
לפי ההגדרה של טנגנס:
\begin{center}
\selectlanguage{english}
\begin{tikzpicture}[scale=1.25]
\draw (0,0) coordinate (A) -- node[below] {$1$} (4,0) coordinate (B);
\draw (B) -- node[right] {$0.766044$} 
  ($(B)+(0,0.766044*4)$) coordinate (C);
\draw (A) -- (C);
\draw[rotate=90] (B) rectangle +(8pt,8pt);
\fill (A) circle (1pt) node[above left] {$A$}
  node[above right,xshift=14pt] {$37.4537^\circ$};
\fill (B) circle (1pt) node[above right] {$B$};
\fill (C) circle (1pt) node[right] {$C$};
\end{tikzpicture}
\end{center}

\newpage

נקפל את הצלע
$\overline{CB}$
מעל 
$\overline{AB}$
ונסמן את המקום עליו מונח הנקודה
$C$
ב-%
$D$.
נעתיק את קטע הקו
$\overline{AB}$ 
ימינה )ללא סיבוב( כך שהנקודה
$A$
מונחת על
$D$,
ונסמן את המקום עליו מונחת הנקודה 
$B$
ב-% 
$E$.
\L{\cite{alperin}}
מסביר איך להעתיק קטע קו.
\begin{center}
\selectlanguage{english}
\begin{tikzpicture}[scale=1.25]
\draw (B) -- node[right] {$0.766044$} 
  ($(B)+(0,0.766044*4)$) coordinate (C);
\draw (A) -- (C);
\draw[rotate=90] (B) rectangle +(8pt,8pt);
\fill (A) circle (1pt) node[above left] {$A$};
\fill (B) circle (1pt) node[above right] {$B$};
\fill (C) circle (1pt) node[right] {$C$};
\coordinate (D) at (0.233956*4,0);
\fill (D) circle (1pt) node[below left] {$D$};
\coordinate (E) at ($(D)+(4,0)$);
\draw (D) -- node[fill=white] {$0.766044$} (B);
\draw (B) -- (E);
\fill (E) circle(1pt) node[above right] {$E$};
\draw[very thick,dotted,->,bend right=50] ($(C)+(-.2,0)$) to ($(A)+(.94,.2)$);
\draw[<->] ($(D)+(0,-.6)$) -- node[fill=white] {$1$} ($(E)+(0,-.6)$);
\end{tikzpicture}
\end{center}

כעת,נקפל את 
$\overline{DE}$
מעל להמשך של 
$\overline{CB}$
כך שהנקודה
$E$
מונחת על הקו בנקודה
$F$.
נקבל:
\[
\angle EDF=\cos^{-1} \disfrac{0.766044}{1}=40^\circ\,.
\]
\begin{center}
\selectlanguage{english}
\begin{tikzpicture}[scale=1.25]
\draw (B) -- ($(B)+(0,0.766044*4)$) coordinate (C);
\draw[rotate=90] (B) rectangle +(8pt,8pt);
\fill (B) circle (1pt) node[above right] {$B$};
\fill (C) circle (1pt) node[right] {$C$};
\coordinate (D) at (0.233956*4,0);
\fill (D) circle (1pt) node[above left] {$D$}
  node[above right,xshift=10pt,yshift=2pt] {$40^\circ$};
\coordinate (E) at ($(D)+(4,0)$);
\draw (D) -- node[fill=white] {$0.766044$} (B);
\draw (B) -- (E);
\fill (E) circle(1pt) node[right] {$E$};
\coordinate (F) at ($(B)+(0,4)$);
\draw[very thick,dotted,->,bend right=50] ($(E)+(.1,.2)$) to ($(F)+(.2,0)$);
\draw (B) -- ($(B)+(0,5)$);
\fill (F) circle (1pt) node[left] {$F$};
\draw (D) -- node[fill=white] {$1$} (F);
\draw[<->] ($(D)+(0,-.6)$) -- node[fill=white] {$1$} ($(E)+(0,-.6)$);
\end{tikzpicture}
\end{center}

