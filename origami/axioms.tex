\documentclass[11pt,a4paper]{article}

%\usepackage[utf8x]{inputenc}

\usepackage{mathpazo}
\usepackage{microtype}
\usepackage{graphicx}
\usepackage{verbatim}
\usepackage{url}
%\usepackage{bm}
%\usepackage{wrapfig}
%\usepackage{float}
%\usepackage{hyperref}

\usepackage{tikz}
\usetikzlibrary{intersections,calc}
\tikzset {>=stealth}

\textwidth=155mm
\textheight=230mm
\topmargin=0pt
\headheight=0pt
\oddsidemargin=0mm
\evensidemargin=0mm
\headsep=0pt
\parindent=0pt
\renewcommand{\baselinestretch}{1.15}
\setlength{\parskip}{0.3\baselineskip plus 1pt minus 1pt}
\addtolength{\jot}{3pt}

\newcommand*{\disfrac}[2]{\displaystyle\frac{#1}{#2}}

\newenvironment{form}[1]{%
\begin{displaymath}%
\renewcommand{\arraystretch}{#1}%
\begin{array}{lcl}}%
{\end{array}%
\end{displaymath}%
}

\begin{document}

%\hypersetup{pageanchor=false}
\thispagestyle{empty}

\vspace*{2ex}

\begin{center}


\textbf{\LARGE The Mathematics of the Origami Axioms}

\bigskip
\bigskip

\textbf{\Large Moti Ben-Ari}

\bigskip
\bigskip

\url{http://www.weizmann.ac.il/sci-tea/benari/}

\end{center}


%\vfill

\begin{footnotesize}
\begin{center}
\copyright{}\ 2020 Moti Ben-Ari.
\end{center}

This work is licensed under the Creative Commons Attribution-ShareAlike 3.0 Unported License. To view a copy of this license, visit \url{http://creativecommons.org/licenses/by-sa/3.0/} or send a letter to Creative Commons, 444 Castro Street, Suite 900, Mountain View, California, 94041, USA.
\end{footnotesize}

%%%%%%%%%%%%%%%%%%%%%%%%%%%%%%%%%%%%%%%%%%%%%%%%%%%%%%%%%%%%%%%%

This document develops the mathematical formulas of the seven origami axioms and gives examples of each one. The examples were constructed using GeoGebra so that the results can be checked.

The terminology uses the word ``fold'' which is a line.  The equations of lines are given in the familiar slope-intercept form, rather than the linear or parametric form.

\section{Axiom 1}


\textbf{Axiom 1} 
Given two distinct points $p_1=(x_1,y_1)$, $p_2=(x_2,y_2)$, there is a unique fold that passes through both of them.

This is the line passing through both points; the slope and intercept can be computed from their coordinates:
\begin{equation}
y - y_1 = \disfrac{y_2-y_1}{x_2-x_1}(x-x_1)\,.
\end{equation}

\textbf{Example}

For the points $(2,2), (6,4)$:
\begin{form}{1.5}
y-2&=&\disfrac{4-2}{6-2}(x-2)\\
y&=&\disfrac{1}{2}(x-2)+2=\disfrac{1}{2}x+1\,.
\end{form}

%%%%%%%%%%%%%%%%%%%%%%%%%%%%%%%%%%%%%%%%%%%%%%%%%%%%%%%%%%%%%%%%


\section{Axiom 2}


\textbf{Axiom 2} 
Given two distinct points $p_1=(x_1,y_1)$, $p_2=(x_2,y_2)$, there is a unique fold that places $p_1$ onto $p_2$.

This line is the perpendicular bisector of $p_1$ and $p_2$. Its slope is the negative inverse of the slope of the line connecting $p_1$ and $p_2$. The line passes through the midpoint between the points.
\begin{equation}
y - \disfrac{y_1+y_2}{2} = -\disfrac{x_2-x_1}{y_2-y_1}\left(x-\disfrac{x_1+x_2}{2}\right)\,.\label{eq.midpoint}
\end{equation}

\textbf{Example}

For the points $(2,2), (6,4)$:
\begin{form}{1.5}
y-3&=&-\disfrac{6-2}{4-2}(x-4)\\
y&=&-2(x-4)+3=-2x+11\,.
\end{form}

%%%%%%%%%%%%%%%%%%%%%%%%%%%%%%%%%%%%%%%%%%%%%%%%%%%%%%%%%%%%%%%%


\section{Axiom 3}


\textbf{Axiom 3} 
Given two lines $l_1$ and $l_2$, there is a fold that places $l_1$ onto $l_2$.

If the lines are parallel, $l_1$ is $y=mx+b_1$, $l_2$ is $y=mx+b_2$, the fold is the line parallel to $l_1,l_2$ and halfway between them $y=mx+\disfrac{b_1+b_2}{2}$.

If the lines intersect, compute the point of intersection:
\begin{form}{1.5}
y&=&m_1x+b_1\\
y&=&m_2x+b_2\\
m_1x+b_2&=&m_2x+b_2\\
x &=& \disfrac{b_2-b_1}{m_1-m_2}\\
y &=&m_1\disfrac{b_2-b_1}{m_1-m_2}+b_1\,.
\end{form}

\textbf{Example}
\begin{form}{1.5}
y&=&2x-2\\
y &=& -x+8\\
x_m&=&\disfrac{8-(-2)}{2-(-1)}=\disfrac{10}{3}=3.33\\
y_m &=& 2\cdot\disfrac{10}{3}-2=\disfrac{14}{3}=4.67\,.
\end{form}

At the point of intersection of the two lines, they form an angle. Actually, two pairs of vertical angles. The folds are the angle bisectors of these angles. If the angle of lines $l_1$ relative to the $x$-axis is $\theta_1$ and the angle of lines $l_2$ relative to the $x$-axis is $\theta_2$, then the fold is the line which makes an angle of $\theta_b=\disfrac{\theta_1+\theta_2}{2}$ with the $x$-axis. The values $m_1,m_2$ are given, but $m_1=\tan\theta_1$ and $m_2=\tan\theta_2$, so the slope of the angle bisector is:
\[
m_b=\tan\theta_b=\tan\disfrac{\theta_1+\theta_2}{2}\,.
\]
The computation requires the use of the following trigonometric identities which you can look up:
\begin{form}{1.5}
\tan \disfrac{\alpha}{2}&=& \disfrac{-1\pm\sqrt{1+\tan^2\alpha}}{\tan \alpha}\\
\tan(\alpha_1+\alpha_2)&=& \disfrac{\tan\alpha_1+\tan\alpha_2}{1-\tan\alpha_1\tan\alpha_2}\,.
\end{form}
First compute:
\[
m'=\tan(\theta_1+\theta_2)= \disfrac{m_1+m_2}{1-m_1m_2}\,,
\]
and then:
\begin{form}{1.5}
m_b&=& \tan\disfrac{\theta_1+\theta_2}{2}\\
&=&\disfrac{-1\pm\sqrt{1+\tan^2(\theta_1+\theta_2)}}{\tan (\theta_1+\theta_2)}\\
&=&\disfrac{-1\pm\sqrt{1+m'^2}}{m'}\,.
\end{form}

\textbf{Example}
\begin{form}{1.5}
y&=&2x-2\\
y &=& -x+8\\
m'&=&\disfrac{1}{1-(-1\cdot 2)}=\disfrac{1}{3}\,,
\end{form}
and:
\[
m_b=\disfrac{-1\pm\sqrt{1+(1/3)^2}}{1/3}=-3\pm \sqrt{10}=-6.16,\; 0.162\,.
\]
Let us compute the line of the fold with the positive slope, where we recall that the point of intersection is $\left(\disfrac{10}{3},\disfrac{14}{3}\right)$:
\begin{form}{1.5}
\disfrac{14}{3} &=& (-3+\sqrt{10}) \cdot \disfrac{10}{3} + b\\ b&=&\disfrac{44-10\sqrt{10}}{3}\\
y_b &=& (-3+\sqrt{10})x_b + \disfrac{44-10\sqrt{10}}{3}\\
y_b&\approx&0.162x_b+4.13\,.
\end{form}


%%%%%%%%%%%%%%%%%%%%%%%%%%%%%%%%%%%%%%%%%%%%%%%%%%%%%%%%%%%%%%%%


\section{Axiom 4}


\textbf{Axiom 4} 
Given a point $p_1$ and a line $l_1$, there is a unique fold perpendicular to $l_1$ that passes through point $p_1$.

\[
y = mx + b,  y_1=-\disfrac{1}{m} x_1 + b', b'= \disfrac{(my_1+x_1)}{m}
\]

\[
y'=-\disfrac{1}{m} x' +\disfrac{(my_1+x_1)}{m}
\]

\textbf{Example}
\[
y = \disfrac{1}{5}x+\disfrac{8}{5}
\]

\[
y'=-5x' + \disfrac{\disfrac{1}{5}\cdot 7 + 5}{\disfrac{1}{5}}=-5x' + 32
\]


%%%%%%%%%%%%%%%%%%%%%%%%%%%%%%%%%%%%%%%%%%%%%%%%%%%%%%%%%%%%%%%%

\section{Axiom 5}


\textbf{Axiom 5} 
Given two points $p_1$ and $p_2$ and a line $l_1$, there is a fold that places $p_1$ onto $l_1$ and passes through $p_2$. For a given pair of points and a line, there may be zero, one or two solutions.

Let $l$ be a fold through $p_2$ and $p_1'$ be the reflection of $p_1$ around $l$. Then the length of the line segment $p_1p_2$ equals the length of the line segment $p_1'p_2$. The locus of points equidistant from $p_2$ is the circle centered at $p_2$ whose radius is the length of $p_1p_2$. The intersections of this circle with the line $l_1$ give the possible points $p_1'$. For any circle and any line, there are zero, one or two points of intersection.

If $l_1$ is $y=mx + b$ and $p_1=(x_1,y_1)$,  $p_2=(x_2,y_2)$ the equation of the circle is:
\[
(x-x_2)^2 + (y-y_2)^2 = r^2\,,
\]
where:
\[
r^2= (x_2-x_1)^2 + (y_2-y_1)^2\,.
\]
Substituting the equation of the line into the equation for the circle:
\[
(x-x_2)^2+(mx+b-y_2)^2=r^2\,,
\]
and after a bit of messy algebra we obtain a quadratic equation for $x$:
\begin{equation}
x^2(1+m^2) \,+\, 2(-x_2+mb-my_2)x \,+\, (x_2^2 +b^2 - 2by_2+y_2^2-r^2)=0\,.\label{eq.intersections}
\end{equation}
There will be at most two solutions for $x$ and the associated $y$ coordinate can be obtained from $y=mx+b$.

The fold lines will be the perpendicular bisectors of $p_1p'$ for any intersection of $p_1=(x_1,y_1)$ with a point $p'=(x',y')$ that is a solution of Equation~\ref{eq.intersections}. The equation of a perpendicular bisector is given by Equation~\ref{eq.midpoint}:
\begin{equation}
y - \disfrac{y_1+y'}{2} = -\disfrac{x'-x_1}{y'-y_1}\left(x-\disfrac{x_1+x'}{2}\right)\,.\label{eq.midpoint}
\end{equation}

\textbf{Example}

Let $p_1=(2,8)$, $p_2=(4,4)$ and $l_1$ is $y=-\disfrac{1}{2}x +3$. The equation of the circle is:
\[
(x-4)^2 + (y-4)^2 = (4-2)^2+(4-8)^2=20\,.
\]
Substitute the equation of the line into the equation of the circle and simplify to obtain a quadratic equation for the $x$-coordinates of the intersections:
\begin{form}{1.5}
(x-4)^2 + \left((-\disfrac{1}{2}x+3)-4\right)^2&=&20\\
%x^2-8x+16 + \disfrac{1}{4}x^2 +x+1 &=& 20\\
\disfrac{5}{4}x^2-7x-3 &=&0\\
5x^2 -28x -12&=&0, 
\end{form}
The quadratic equation factors into $(5x+2)$ and $(x-6)$, giving two points of intersection:
\[
p_1'=\left(-\disfrac{2}{5},\disfrac{16}{5}\right) = (-0.4,3.2)\,,\quad p_1''=(6,0)\,.
\]
For $p_1=(2,8)$ and $p_1'=\left(-\disfrac{2}{5},\disfrac{16}{5}\right)$, the fold line is:
\begin{form}{1.5}
y-\disfrac{8+\frac{16}{5}}{2}&=&-\disfrac{-\frac{2}{5}-2}{\frac{16}{5}-8}\left(x-\disfrac{2-\frac{2}{5}}{2}\right)\\
y&=&-\disfrac{1}{2}x+6\,.
\end{form}

For $p_1=(2,8)$ and $p_1''=(6,0)$, the fold line is:
\begin{form}{1.5}
y-\disfrac{8+0}{2}&=&-\disfrac{6-2}{0-8}\left(x-\disfrac{2+6}{2}\right)\\
y&=&\disfrac{1}{2}x+2\,.
\end{form}

%%%%%%%%%%%%%%%%%%%%%%%%%%%%%%%%%%%%%%%%%%%%%%%%%%%%%%%%%%%%%%%%
\newpage
\section{Axiom 6}


\textbf{Axiom 6} 
Given two points $p_1$ and $p_2$ and two lines $l_1$ and $l_2$, there is a fold that places $p_1$ onto $l_1$ and $p_2$ onto $l_2$. For a given pair of points and pair of lines, there may be zero, one, two or three solutions.

A fold that places $p_i$ onto $l_i$ is a line such that the distance from $p_i$ to the fold line is equal to the distance from $l_i$ to the fold line. The locus of points that are equidistant from $p_i$ and $l_i$ is a parabola with focus $p_i$ and directrix $l_i$. A fold is any line tangent to that parabola. For a fold to simultaneously place $p_1$ onto $l_1$ and $p_2$ onto $l_2$, it must be a tangent common to the two parabolas. As we shall see, there may be zero, one, two or three common tangent.

The formula for an arbitrary parabola is quite complex,\footnote{The general equation of the parabola with focus $(f_1,f_2)$ and directrix $ax+by+c$ is:
\[
\disfrac{(ax+by+c)^2}{a^2+b+2} = (x-f_1)^2+(y-f_2)^2.
\]
}
so we limit the presentation to parabolas with the the $y$-axis as the axis of symmetry. An example will also be given where one of the parabolas has the $x$-axis as its axis of symmetry.

Let $(0,f)$ be the focus of a parabola with directrix $y=d$. Define $p=f-d$, the signed length of the line segment between the focus and the directrix. If $d=-f$ so that the vertex of the parabola is on the $x$-axis, the equation of the parabola is $y=\disfrac{x^2}{2p}$. To move the parabola up or down the $y$-axis, add a value $h$, and define $a=2ph$. Then the equation of the parabola is:
\begin{form}{1.5}
y=\disfrac{x^2}{2p}+\disfrac{a}{2p}\\
x^2-2py+a=0\,.
\end{form}
To obtain a tangent, substitute the equation of a line $y=mx+b$ into the equation of the parabola:
\begin{form}{1.5}
x^2-2p(mx+b)+a=0\\
x^2+(-2mp)x+(-2pb+a)=0\,.
\end{form}
A line in the plane can have zero, one or two intersections with a parabola. The line will be tangent to the parabola iff this quadratic equation $x^2+bx+c$ has exactly one solution iff the discriminant is $b^2-4c$ zero:
\[
(-2mp)^2-4(-2pb+a)=0\,,
\]
which simplifies to:
\begin{equation}
m^2p^2+2pb-a=0\,.\label{eq.disc}
\end{equation}
Compute the equations for the two parabolas and solve for $m$ and $b$.


\textbf{Example}

Parabola 1: focus $(0,4)$ and directrix $y=2$, so $p=2$ and $a=12$. The equation of the parabola is:
\[
\begin{array}{l}
x^2-2\cdot 2y +12=0\,.
\end{array}
\]
Substituting into Equation~\ref{eq.disc} and simplifying:
\[
m^2+b-3=0\,.
\]
Parabola 2: focus $(0,-4)$ and directrix $y=-2$, so $p=-2$ and $a=12$. The equation of the parabola is:
\[
x^2-2\cdot (-2)y+12=0\,.
\]
Substituting into y=mx+b\ref{eq.disc} and simplifying:
\[
m^2-b-3=0\,.
\]
Now solve the two equations:
\begin{form}{1.5}
m^2+b-3=0\\
m^2-b-3=0\,.
\end{form}
The solutions are $m=\pm\sqrt{3}\approx \pm 1.73$ and $b=0$. There are two common tangents that are fold lines:
\[
y=\sqrt{3}x\,,\quad y=-\sqrt{3}x\,.
\]

\textbf{Example}

Parabola 1 is unchanged. Parabola 2: focus $(0,-6)$ and directrix $y=-2$, so $p=-4$ and $a=32$. The equation of the parabola is:
\[
x^2-2\cdot (-4)y +32=0\,.
\]
Substituting into \ref{eq.disc} and simplifying:
\[
2m^2-b-4=0\,.
\]
Let us solve the two equations:
\begin{form}{1.5}
m^2+b-3=0\\
2m^2-b-4=0\,.
\end{form}
The solutions are $m=\pm\sqrt{\disfrac{7}{3}}\approx \pm 1.53$ and $b=\disfrac{2}{3}$. There are two common tangents that are fold lines:
\[
y=\sqrt{\disfrac{7}{3}}x+\disfrac{2}{3}\,,\quad y=-\sqrt{\disfrac{7}{3}}x+\disfrac{2}{3}\,.
\]

%%%%%%%%%%%%%%%%%%%%%%%%%%%%%%%%%%%%%%%%%%%%%%%%%%%%%%%%%%%%%%%%

\textbf{Example}

Let us now try a parabola with the $x$-axis as the axis of symmetry, focus $(4,0)$ and directrix $x=2$, so $p=2$ and $a=12$. The equation of the parabola is:
\[
y^2-4x+12 = 0\,.
\]
Substituting the equation for a line:
\begin{form}{1.5}
(mx+b)^2-4x+12=0\\
m^2x^2+(2mb-4)x+(b^2+12)=0\,.
\end{form}
Set the discriminant equal to zero and simplify:
\begin{form}{1.5}
(2mb-4)^2-4m^2(b^2+12)=0\\
%(4b^2-4b^2-48)m^2 - 16mb+16=0\\
-3m^2-mb+1=0
\end{form}
Let us solve the two equations:
\begin{form}{1.5}
m^2+b-3=0\\
-3m^2-mb+1=0\,.
\end{form}
The result is a cubic equation:
\[
m^3-3m^2-3m+1=0\,.
\]
The formula for solving cubic equations is quite complicated so I used a calculator on the internet and obtained three solutions: $3.73, -1, 0.27$. Choosing $m=0.27$ which is the slope of the tangent I chose on Geogebra, we get:
\[
b=3-m^2=2.93\,,
\]
so the equation of the fold line is:
\[
y=0.27x+2.93\,.
\]
Actually, we could have guessed that $-1$ is a root. If so, we can divide the cubic equation by $m+1$ to obtain the quadratic equation $m^2-4m+1$ whose roots are $2\pm\sqrt{3}\approx 3.73, 0.27$.


%%%%%%%%%%%%%%%%%%%%%%%%%%%%%%%%%%%%%%%%%%%%%%%%%%%%%%%%%%%%%%%%

\textbf{Points on the lines}

We are not quite done yet. Having chosen a tangent line for the fold line, we want to compute the position of $p_i$ on line $l_i$. In general, to reflect $p'=(x',y')$ around $l'$ whose equation is $y=m'x+b'$, we first find the line through $p'$ that is perpendicular to $l'$:
\begin{form}{1.5}
y=-\disfrac{1}{m}x+b'\\
b'=y'+\disfrac{x'}{m}\\
y=\disfrac{-x}{m}+y_1+\disfrac{x_1}{m}\,.
\end{form}
Next we find the intersection $p_t=(x_t,y_mt$ of $l'$ with the tangent line whose equation is $y=mx+b$:
\begin{form}{1.5}
mx_t+b=\disfrac{-x_t}{m}+y'+\disfrac{x'}{m}\\
x_t=\disfrac{\left(y'+\disfrac{x'}{m}-b\right)}{\left(m+\disfrac{1}{m}\right)}\,.
\end{form}
The $y$-coordinate can be found using the equation of either line:
\[
y_t=mx_t+b\,.
\]
This intersection is the midpoint between $p_i$ and its reflection $p_i'$ around the fold line, so it is easy to compute the coordinates of $p_i'$ from:
\begin{form}{1.5}
x_t=\disfrac{x_i+x_i'}{2}\\
y_t=\disfrac{y_i+y_i'}{2}\,.
\end{form}


\textbf{Example}

Point $p_1$ is $(x_1,y_1)=(0,4)$, line $l_1$ is $y=\sqrt{3}x$:

\begin{form}{1.5}
x_t=\disfrac{\left(4+\disfrac{0}{\sqrt{3}}-0\right)}{\left(\sqrt{3}+\disfrac{1}{\sqrt{3}}\right)}=\disfrac{4\sqrt{3}}{4}=\sqrt{3}\\
y_t=\sqrt{3}\sqrt{3}+0=3\\
x_1'=2x_t-x_1=2\sqrt{3}-0=2\sqrt{3}\approx 3.46\\
y_1'=2y_t-y_1=2\cdot 3 - 4 = 2\,.
\end{form}

%%%%%%%%%%%%%%%%%%%%%%%%%%%%%%%%%%%%%%%%%%%%%%%%%%%%%%%%%%%%%%%%

\section{Axiom 7}

\textbf{Axiom 7} 
Given one point $p_1$ and two lines $l_1$ and $l_2$, there is a fold that places $p_1$ onto $l_1$ and is perpendicular to $l_2$.

Let $p_1=(x_1,y_1)$, $l_1$ be $y = m_1x + b_1$ and $l_2$ be $y=m_2+b_2$.

If the fold line is perpendicular to $l_2$ then the line segment connecting $p_1$ and $p_1'$ is parallel to $l_2$ and passes through $p_1$ so:
\begin{form}{1.5}
y_1=m_2x_1+b_s\\
b_s=y_1-m_2x_1\,.
\end{form}
The equation of the line is:
\[
y=m_2x+(y_1-m_2x_1)\,.
\]
The intersection of this line with $l_1$ is the reflection $p_1'=(x_1',y_1')$ between this line segment and $l_1$. Let us compute its coordinates:
\begin{form}{1.5}
m_1x_1+b_1=m_2x_1+(y_1-m_2x_1)\\
x_1'=\disfrac{y_1-m_2x_1-b_1}{m_1-m_2}\\
y_1'=m_1x_i+b_1\,.
\end{form}
The midpoint $p_m=(x_m,y_m)$ is:
\[
(x_m,y_m)=\left(\disfrac{x_1+x_1'}{2},\disfrac{y_1+y_1'}{2}\right)\,,
\]
so the fold line which is the perpendicular bisector can be computed:
\begin{form}{2}
y_m=-\disfrac{1}{m_2}x_m+b_m\\
b_m=y_m+\disfrac{m_2}x_m\\
y=-\disfrac{1}{m_2}x+\left(y_m+\disfrac{1}{m_2}x_m\right)\,.
\end{form}
\textbf{Example}

\begin{form}{2}
l_1:y=2x-2\\
l_2:y=-x+10\\
p_1:(2,8)\\
x_1'=\disfrac{8-(-1)\cdot 2-(-2)}{2-(-1)}=4\\
y_1'=6\\
p_m=(3,7)\\
y=-\disfrac{1}{-1}\cdot x+\left(7+\disfrac{1}{-1}\cdot 3\right)=x+4\,.
\end{form}



\section*{References}

The presentation is based upon \url{https://en.wikipedia.org/wiki/Huzita-Hatori_axioms}, which gives parametric equations for the first five axioms. Questions and answers on \url{https://math.stackexchange.com} helped me write the section on Axiom 6. For a rigorous presentation of the mathematics of origami, see Chapter~10 ``Paperfolding'' of George E. Martin. \textit{Geometric Constructions}, Springer, 1998.





\end{document}
