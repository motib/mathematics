% !TeX root = origami-activities-en.tex

%%%%%%%%%%%%%%%%%%%%%%%%%%%%%%%%%%%%%%%%%%%%%%%%%%%%%%%%%%%%%%%%%%
%%%%%%%%%%%%%%%%%%%%%%%%%%%%%%%%%%%%%%%%%%%%%%%%%%%%%%%%%%%%%%%%%%
%%%%%%%%%%%%%%%%%%%%%%%%%%%%%%%%%%%%%%%%%%%%%%%%%%%%%%%%%%%%%%%%%%

\section{Activities for learning geometric loci with origami}

\subsection{Introduction}

The activities are based on the origami axioms through which the students will experiment with constructing geometric loci. Paper folding can contribute to learning this concept because it provides a concrete visualization of the meaning of the concept.

\textbf{How to present the activities}

\begin{enumerate}

\item Each activity has two parts: in the first the students will get to know the axioms and the how the activities can be adapted for use in a classroom. The second part is a worksheet with the activity itself. Some students can be given the worksheets for independent study, while others may need to be guided through the steps of the activity.

\item The activities are arranged in an order that we believe is pedagogical optimal, not in the standard order of the axioms. Of course, the teacher is free to present the activities in a different order. Activity $1$ is essential to what follows and should be presented first, even if the teacher chooses a different order for the subsequent activities.

\item The activities include Geogebra applications intended to deepen the students' understanding of the geometric loci constructed by the axioms.
\end{enumerate}

\textbf{Paper for folding}

Several types of paper are appropriate for folding: (a) origami paper available in specialty shops; (b) any square or rectangular sheets of paper; (c) rectangular or square baking sheets which facilitate marking points and lines.

Each student should have at least six sheets of paper on hand (seven for the opening activity).

\subsection{The structure of an activity for leaning geometric loci}

\textbf{Topic} Geometric loci through paper folding.

\bigskip

\textbf{Goals} (a) Learning the concept and meaning of \emph{geometric locus} through paper folding; (b) learning the origami axioms; (c) exercising methods of finding geometric loci.

\bigskip

\textbf{Target audience} Secondary-school students studying advanced mathematics.

\bigskip

\textbf{Prerequisites} Analytic geometry with emphasis on circles and parabolas.

\newpage

\textbf{Structure}

\textbf{Part $1$}

\textbf{First phase: Acquaintance with origami}

\begin{enumerate}
\item The teacher will start with a brief presentation of the historical background of origami, describing how simple paper folding became an advanced mathematical theory and pioneering technological tool. At this point the students can be shown part of Robert Lang's lecture \cite{lang-ted}. Show four minutes starting from minute eleven.

\item The teacher will explain the origami axioms.

\item The teacher will acquaint the students with the basic concepts: point, line and fold.
\end{enumerate}

\textbf{Second phase: Familiarization and experimentation with the origami axioms} For the initial encounter with the axioms we suggest doing the activities described in Appendix~\ref{a.experimenting} \textit{Experimenting with folding according to the origami axioms}. 

\textbf{Third phase: Familiarization with the concept of geometric locus} The teacher will define the concept: the geometric locus is the set of points that satisfy a given condition. The locus is expressed as an expression of $y$ in terms of $x$. Often the locus can be expressed in geometric terms such as a point or line with a certain property, or a geometric figure like a circle or a parabola.

\bigskip

\textbf{Part $2$}

\textbf{First phase: Familiarization with geometric loci through the origami axioms} The teacher will distribute worksheets that ask the students to find geometric loci by paper folding (Section~\ref{s.discovery}). The students can work independently or work on each step followed by a group discussion.

\textbf{Second phase: Exercises} Section~\ref{s.exercises} contain mathematical exercises for each axiom. The first and second phases can be combined: at the conclusion of an activity from the first phase, the students can be asked to solve the respective exercise from the second phase.
