% !TeX root = origami-activities-en.tex

\thispagestyle{empty}

\begin{center}
\textbf{\LARGE The Mathematics of Origami for\\\medskip Secondary-School Students}

\bigskip
\bigskip

\textbf{\Large Oriah Ben-Lulu  and Moti Ben-Ari}

\bigskip
\bigskip

\url{http://www.weizmann.ac.il/sci-tea/benari/}

\bigskip
\bigskip

Version $1.0$
\end{center}

\vfill

\begin{small}
\begin{center}
\copyright{}\ 2020 Oriah Ben-Lulu and Moti Ben-Ari
\end{center}

This work is licensed under the Creative Commons Attribution-ShareAlike 3.0 Unported License. To view a copy of this license, visit \url{http://creativecommons.org/licenses/by-sa/3.0/} or send a letter to Creative Commons, 444 Castro Street, Suite 900, Mountain View, California, 94041, USA.
\end{small}

\newpage

\tableofcontents
\newpage


\section{Introduction}

The topic of this document is the mathematics of paper folding---origami. The goal is to expose teachers and students to the set of axioms of this theory and to use the axioms to construct the concept of \emph{geometric locus}. The first part of the document is intended for teachers and includes the historical background, the mathematical content and the pedagogy associated with the theory of origami and its axioms. The second part offers activities for students on the concept of locus; the activities use paper folding and are based upon the axioms of origami.


