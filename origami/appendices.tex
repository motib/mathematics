% !TeX root = origami-math.tex

\appendix

\newpage

\section{GeoGebra links}\label{a.geo}

\begin{center}
\begin{tabular}{|l|l|}
\hline
Axiom 1& \url{https://www.geogebra.org/m/fq9d5hms}\\\hline
Axiom 2& \url{https://www.geogebra.org/m/fgmfss27}\\\hline
Axiom 3& \url{https://www.geogebra.org/m/ek3mqupw}\\\hline
Axiom 4& \url{https://www.geogebra.org/m/renzzbdg}\\\hline
Axiom 5& \url{https://www.geogebra.org/m/aszn9ywu}\\\hline
Axiom 6& \url{https://www.geogebra.org/m/bxe5e5ku}\\\hline
Axiom 7& \url{https://www.geogebra.org/m/yeq5gmeg}\\\hline
Abe's trisection & \url{https://www.geogebra.org/m/dxrcvjam}\\\hline
Martin's trisection & \url{https://www.geogebra.org/m/caky7edd}\\\hline
Messer's doubling of the cube & \url{https://www.geogebra.org/m/mrcwjqh8}\\\hline
Beloch's doubling of the cube & \url{https://www.geogebra.org/m/enzmmwua}\\\hline
\end{tabular}
\end{center}
Due to a bug in Geogebra, in projects that use Axiom~6, points defined by reflection around the common tangent are not saved or are saved incorrectly.

\section{Derivation of the trigonometric identities}\label{a.tangent}

The trigonometric identifies for tangent used in the proof of Axiom 3 can be derived from identifies for the sine and cosine:

\begin{form}{2.2}
\tan (\theta_1+\theta_2) &=& \disfrac{\sin(\theta_1+\theta_2)}{\cos(\theta_1+\theta_2)}\\
&=&\disfrac{\sin\theta_1\cos\theta_2+\cos\theta_1\sin\theta_2}{\cos\theta_1\cos\theta_2-\sin\theta_1\sin\theta_2}\\
&=&\disfrac{\sin\theta_1+\cos\theta_1\tan\theta_2}{\cos\theta_1-\sin\theta_1\tan\theta_2}\\
&=&\disfrac{\tan\theta_1+\tan\theta_2}{1-\tan\theta_1\tan\theta_2}\,.
\end{form}

We use this formula with $\theta=(\theta/2)+(\theta/2)$ to obtain a quadratic equation in $\tan(\theta/2)$:
\begin{form}{2.2}
\tan \theta=\disfrac{\tan(\theta/2)+\tan(\theta/2)}{1-\tan^2(\theta/2)}\\
\tan\theta \,(\tan(\theta/2))^2 \;+\; 2\,(\tan (\theta/2)) \;-\;\tan \theta = 0\,.
\end{form}
Its solutions are:
\[
\tan(\theta/2) = \disfrac{-1\pm\sqrt{1+\tan^2\theta}}{\tan\theta}\,.
\]

\newpage

\section{Parabolas}\label{a.parabola}


Students are usually introduced to parabolas as the graphs of second degree equations:
\[
y=ax^2+bx+c\,.
\]
However, parabolas can be defined geometrically: given a point, the \emph{focus}, and a line, the \emph{directrix}, the locus of points equidistant from the focus and the directrix defines a parabola.


The following diagram shows the focus---the large point at $p=(0,f)$, and the directrix---the thick line whose equation is $y=-f$. The resulting parabola is shown as a dotted curve. Its vertex $p_2$ is at the origin of the axes.

%\begin{figure}[H]
\begin{center}
\begin{tikzpicture}[scale=1.1]
\draw (-6,0) -- node[very near start,below,xshift=-32pt] {$x$-\textsf{axis}} (6,0);
\draw (0,-3) -- node[very near start,right,yshift=-20pt] {$y$-\textsf{axis}} (0,4.5);
\draw[ultra thick] (-6,-2) -- node[near start,below] {\textsf{directrix} $\quad y=-f$} (6,-2);
\draw[domain=-6:6,samples=50,very thick,dotted] plot (\x,{\x*\x/8});
\coordinate (F) at (0,2);
\fill (F) circle (3pt) node[above left,xshift=-2pt,yshift=15pt] {$(0,f)$} node[above left,xshift=-2pt,yshift=30pt] {\textsf{focus}} node[above right] {$p$};
\fill (0,0) circle (1.5pt) node[below right] {$p_2$};
\fill (0,-2) circle (1.5pt);
\fill (2,-2) circle (1.5pt);
\fill (3,-2) circle (1.5pt);
\fill (5,-2) circle (1.5pt);
\coordinate (FP) at (-5,-2);
\fill (FP) circle (1.5pt) node[below] {$p'$};
\coordinate (F1) at (2,.5);
\fill (F1) circle (1.5pt) node[below right] {$p_3$};
\coordinate (F2) at (3,1.125);
\fill (F2) circle (1.5pt) node[below right] {$p_4$};
\coordinate (F3) at (5,3.125);
\fill (F3) circle (1.5pt) node[below right] {$p_5$};
\coordinate (F4) at (-5,3.125);
\fill (F4) circle (1.5pt) node[above right] {$p_1$};
\draw (F) -- node[left] {$a_2$} (0,0) -- node[left] {$a_2$} (0,-2);
\draw (F) -- node[near end,left] {$a_3$} (F1) -- node[left] {$a_3$} (2,-2);
\draw (F) -- node[near end,above] {$a_4$} (F2) -- node[left] {$a_4$} (3,-2);
\draw (F) -- node[above] {$a_5$} (F3) -- node[left] {$a_5$} (5,-2);
\draw (F) -- node[above] {$a_1$} (F4) -- node[left] {$a_1$} (FP);
\draw[thick,dashed] ($(F4)!-.4!(-2.5,0)$) -- node[very near end,right,xshift=2pt] {$l$} ($(F4)!1.8!(-2.5,0)$);
\draw[very thick,dotted] (F) -- (FP);
\coordinate (H) at (-2.5,0);
\node[above,xshift=1pt] at (H) {$?$};
\draw[rotate=36] (H) rectangle +(10pt,10pt);
\path (FP) -- node[above,yshift=2pt] {$b?$} (H) -- node[above,yshift=2pt] {$b?$} (F);
\draw (0,-2) rectangle +(8pt,8pt);
\draw (2,-2) rectangle +(8pt,8pt);
\draw (3,-2) rectangle +(8pt,8pt);
\draw (5,-2) rectangle +(8pt,8pt);
\draw (-5,-2) rectangle +(8pt,8pt);
\end{tikzpicture}
\end{center}
%\end{figure}

We have selected five points $p_i$, $i=1,\ldots,5$ on the parabola. Each point $p_i$ is at a distance of $a_i$ both from the focus and from the directrix.

Consider the point $p'$ that is the intersection of the perpendicular from $p_1$ to the directrix. Since $p_1$ is on the parabola $\overline{p'p_1}=\overline{p_1p}=a_1$. We claim that the tangent $l$ to the parabola at $p_1$ (dashed line) is a fold that reflects $p$ onto $p'$.

\newpage

We have to prove the $l$ is the perpendicular bisector of $\overline{pp'}$. Let us extract a simplified diagram:
\begin{center}
\begin{tikzpicture}[scale=1.1]
\draw[thick] (-6.4,0) -- node[very near start,below,xshift=-20pt] {$x$-\textsf{axis}} (1,0);
\draw[thick] (0,-3) -- node[very near start,right,yshift=-20pt] {$y$-\textsf{axis}} (0,3.5);
\draw[thick] (-6.4,-2) -- (1,-2);
\coordinate (F) at (0,2);
\fill (F) circle (1.5pt) node[right] {$p\;\;$ \textsf{focus}};
\fill (0,0) circle (1.5pt) node[below right] {$p_2$};
\coordinate (FP) at (-5,-2);
\fill (FP) circle (1.5pt) node[below] {$p'$};
\path (FP) -- node[below,yshift=-2pt] {\textsf{directrix}} (0,-2);
\fill (0,-2) circle (1.5pt) node[below right] {$s$};
\coordinate (F4) at (-5,3.125);
\fill (F4) circle (1.5pt) node[above right] {$p_1$};
\draw (F) -- node[right] {$a_2$} (0,0) -- node[right] {$a_2$} (0,-2);
\draw (F) -- node[above] {$a_1$} (F4) -- (FP);
\draw[<->] ($(F4)+(-.5,-.1)$) -- node[fill=white] {$a_1$}($(FP)+(-.5,.1)$);
\draw[thick,dashed] ($(F4)!-.15!(-2.5,0)$) -- node[very near end,right,xshift=2pt] {$l$} ($(F4)!1.8!(-2.5,0)$);
\draw[very thick,dotted] (F) -- (FP);
\coordinate (H) at (-2.5,0);
\node[above,xshift=1pt] at (H) {\textsf{\bfseries ?}};
\draw[rotate=36] (H) rectangle +(10pt,10pt);
\path (FP) -- node[above,yshift=2pt] {$b$ \textsf{\bfseries ?}} (H) -- node[above,yshift=2pt] {$b$ \textsf{\bfseries ?}} (F);
\draw (0,0) rectangle +(8pt,8pt);
\draw (0,-2) rectangle +(8pt,8pt);
\draw (-5,0) rectangle +(8pt,8pt);
\draw (-5,-2) rectangle +(8pt,8pt);
\fill (-5,0) circle (1.5pt) node[below right] {$q$};
\fill (-2.5,0) circle (1.5pt) node[below,yshift=-4pt] {$r$};
\path (-5,0) -- node[right] {$a_2$} (-5,-2);
\path (F4) -- node[left,xshift=-2pt] {$a$} (-2.5,0);
\end{tikzpicture}
\end{center}
\begin{itemize}
\item The directrix is parallel to the $x$-axis, the focus $p$ is on the $y$-axis and $\overline{p_1p'}$ is perpendicular to the directrix. Therefore, $\angle p'qr$ and $\angle pp_2r$ are right angles.
\item $\overline{qp'}$ and $\overline{p_2s}$ are opposite sides of a rectangle, so $\overline{qp'}=\overline{p_2s}$, which in turn is equal to $\overline{pp_2}$ since $p_2$ is on the parabola and thus equidistant from $p$ and $s$.
\item $\angle qrp'$ and $\angle p_2rp$ are equal vertical angles.
\item The right triangles $\triangle qrp'$ and $\triangle p_2rp$ have one acute angle equal and one side equal so they are congruent. Therefore, $\overline{p'r}=\overline{rp}$ and $\overline{p_1r}$ is the median of $\triangle pp_1p'$.
\item $p_1$ is on the parabola so $\overline{pp_1}=\overline{p_1p'}$. Therefore, $\triangle pp_1p'$ is an isoceles triangle.
\item In the isoceles triangle $\triangle pp_1p'$, the median $\overline{p_1r}$ is also the perpendicular bisector of $\overline{pp'}$.
\item Line $l$ contains the line segment $\overline{p_1r}$ and is the perpendicular bisector of $\overline{pp'}$.
\end{itemize}

