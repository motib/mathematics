% !TeX root = origami-math-en.tex

%%%%%%%%%%%%%%%%%%%%%%%%%%%%%%%%%%%%%%%%%%%%%%%%%%%%%%%%%%%%%%%%
\chapter{Constructing a Nonagon}\label{c.nonagon}

\section{Construction of regular polygons}

The Gauss-Wantzel Theorem states that a regular polygon is constructible by straightedge and compass if the number of sides is:
\[
n=2^k\cdot p_1 \cdot \:\cdots\: \cdot p_m\,,
\]
where the $p_i$'s (if any) are distinct Fermat primes of the form $F_m=2^{2^m}+1$. There are five known Fermat primes: $F_0=3, F_1=5, F_2=17, F_3=257, F_4=65537$.\footnote{At age $19$ Gauss constructed the regular $17$-gon and decided (fortunately for us) that he would 	become a mathematician. The regular $257$-gon was constructed by Magnus Georg Paucker in $1822$ and by Friedrich Julius Richelot in $1832$. Johann Gustav Hermes claimed to have constructed the regular $65537$-gon in $1894$. Should you wish to check its correctness, his manuscript is preserved at the University of G\"{o}ttigen.} Therefore, a \emph{nonagon}, a regular polygon with nine sides, is not constructible.

Regular polygons with:
\[
n=2^i\cdot 3^j \cdot p_1 \cdot \: \cdots\: \cdot p_m
\]
sides can be constructed using origami, where the $p_i$'s (if any) are distinct primes of the form $2^k\cdot 3^l+1$ \cite{alperin}. Here we construct a nonagon using Lill's method and Beloch's fold.

\section{The cubic equation for a nonagon}

A regular $n$-gon can be constructed by constructing its central angle $360^\circ/n$. For a nonagon the central angle is $\theta=360^\circ/9=40^\circ$:
\begin{center}
\begin{tikzpicture}
\coordinate (O) at (0,0);
\fill (O) circle (1.5pt);
\foreach \x/\name in {0/a,40/b,80/c,120/d,160/e,200/f,240/g,280/h,320/i} {
  \coordinate (\name) at ($(O)+(\x:3cm)$);
  \draw (O) -- (\name);
  \fill (\name) circle (1.5pt);
}
\draw (a) -- (b) -- (c) -- (d) -- (e) -- (f) -- (g) -- (h) -- (i) -- cycle;
\node[above right,xshift=12pt] at (O) {$40^\circ$};
\end{tikzpicture}
\end{center}

$40^\circ$ is one-third of $120^\circ$, which is constructible by appending a $30^\circ$ angle (obtained by bisecting an angle of an equilateral triangle) to a $90^\circ$ angle (obtained by constructing a perpendicular). Since angles can be trisected using origami, $40^\circ$ can be constructed using origami. Here we present a different construction that solves a cubic equation.

It is a simple calculation using the formula for $\cos(\alpha+\beta)$ to show that:
\[
\cos 3\theta=4\cos^3 \theta -3\cos\theta\,.
\]
Let $x=\cos \theta$ and $a=\cos 3\theta$. If we can solve the cubic equation $4x^3-3x-a=0$ for $x$, the angle itself can be constructed by constructing a right triangle with a side the length of $x$ and a hypotenuse of length $1$ (see the construction below).

For the nonagon the equation is $4x^3-3x+\disfrac{1}{2}=0$ since $\cos 120^\circ=-\disfrac{1}{2}$.

Let us construct a path for the equation as required for Lill's method:

\begin{center}
\begin{tikzpicture}[scale=1]
% Draw help lines and axes
\draw[step=10mm,white!60!black] (-1,-4) grid (9,1);
\draw[thick] (-1,0) -- (9,0);
\draw[thick] (0,-4) -- (0,1);
\foreach \x in {1,...,9}
  \node at (\x-.3,.3) {\sm{\x}};
\foreach \y in {-3,...,1}
  \node at (-.3,\y-.3) {\sm{\y}};
  
% Points of first path
\coordinate (A) at (0,0);
\coordinate (B) at (4,0);
\coordinate (C) at (7,0);
\coordinate (D) at (7,-.5);
\foreach \x in {A,B,C,D}
  \fill (\x) circle(2pt);
\node[above left] at (A) {$P$};
\node[below right,xshift=12pt] at (A) {\sm{37.4537^\circ}};
\node[below right] at (D) {$Q$};

% Draw first path
\draw[very thick,-{Stealth[scale=1.4,inset=2pt]}] 
  (A) -- node[below] {$a_3$} (B);
\draw[{Stealth[scale=1.4,inset=2pt,reversed]}-,very thick]
  (B) -- ($(B)+(0,.1)$);
\draw[name path=c,very thick,{Stealth[scale=1.4,inset=2pt]}-]
  (B) -- node[below] {$a_1$} (C);
\draw[very thick,-{Stealth[scale=1.4,inset=2pt]}]
  (C) -- node[right,yshift=-2pt] {$a_0$} (D);

% Draw extension of second segment of first path
\draw[very thick,loosely dotted,name path=b] 
  ($(B)+(0,-4)$) -- ($(B)+(0,1)$);

% Draw second path
\path[name path=one] (A) -- +(-37.4537:6cm);
\path [name intersections = {of = b and one, by = {R}}];
\fill (R) circle (2pt) node[below left] {$R$};
\draw[thick,dashed] (A) -- (R);

\path[name path=two] (R) -- +(52.5463:6cm);
\path [name intersections = {of = c and two, by = {S}}];
\fill (S) circle (2pt) node[above] {$S$};
\draw[thick,dashed] (R) -- (S);

\draw[thick,dashed] (S) -- (D);

% Draw right angle rectangles
\draw[thick,rotate=52.5463] (R) rectangle +(8pt,8pt);
\draw[thick,rotate=-127.4537] (S) rectangle +(8pt,8pt);
\end{tikzpicture}
\end{center}
The coefficient $a_2$ is zero so the $90^\circ$ turn is performed but no line segment is constructed in that direction.

The second path starts from $P$ at an angle of $-37.4537^\circ$ and is then reflected by $\pm 90^\circ$ as specified in Lill's method. The second path intersects $Q$ so we know that $x=-\tan -37.4537^\circ=0.766044$ is a root of $4x^3-3x+\disfrac{1}{2}$.

\section{Solving the equation with Beloch's fold}

I cheated above because I know that a root of the equation is $x=-\tan -37.4537^\circ=0.766044$ and drew the second path using this angle. You can check that:
\[
4(0.766044)^3-3(0.766044)+0.5\approx 0\,.
\]
The answer can be obtained using Beloch's fold as described in Chapter~\ref{c.beloch}. We start by drawing a line parallel to $a_2$\footnote{Since $a_2=0$ the line is drawn parallel to a line that would be drawn if $a_2$ were not equal to $0$.} at the same distance from $a_2$ as $a_2$ is from $P$. Similarly, we draw a line parallel to $a_1$ at the same distance from $a_1$ as $a_1$ is from $Q$. Beloch's fold (the line $\overline{RS}$) simultaneously places $P$ at $P'$ on the line parallel to $a_2$ and $Q$ at $Q'$ on the line parallel to $a_1$. This constructs the angle $\angle SPQ=37.4537^\circ$.
\begin{center}
\begin{tikzpicture}[scale=1]
% Draw help lines and axes
\draw[step=10mm,white!60!black] (-1,-7) grid (9,1);
\draw[thick] (-1,0) -- (9,0);
\draw[thick] (0,-7) -- (0,1);
\foreach \x in {1,...,9}
  \node at (\x-.3,.3) {\sm{\x}};
\foreach \y in {-6,...,1}
  \node at (-.3,\y-.3) {\sm{\y}};
  
% Points of first path
\coordinate (A) at (0,0);
\coordinate (B) at (4,0);
\coordinate (C) at (7,0);
\coordinate (D) at (7,-.5);
\foreach \x in {A,B,C,D}
  \fill (\x) circle(2pt);
\node[above right] at (A) {$P$};
\node[below right] at (D) {$Q$};

% Draw first path
\draw[very thick,-{Stealth[scale=1.4,inset=2pt]}] 
  (A) -- node[below] {$a_3$} (B);
\draw[{Stealth[scale=1.4,inset=2pt,reversed]}-,very thick]
  (B) -- ($(B)+(0,.1)$);
\draw[name path=c,very thick,{Stealth[scale=1.4,inset=2pt]}-]
  (B) -- node[below] {$a_1$} (C);
\draw[very thick,-{Stealth[scale=1.4,inset=2pt]}]
  (C) -- node[right,yshift=-2pt] {$a_0$} (D);

% Draw extension of second segment of first path
\draw[very thick,loosely dotted,name path=b] 
  ($(B)+(0,-7)$) -- ($(B)+(0,1)$);

% Draw second path
\path[name path=one] (A) -- +(-37.4537:6cm);
\path [name intersections = {of = b and one, by = {R}}];
\fill (R) circle (2pt) node[below left] {$R$};
%\draw[thick] (A) -- (R);

\path[name path=two] (R) -- +(52.5463:6cm);
\path [name intersections = {of = c and two, by = {S}}];
\fill (S) circle (2pt) node[above] {$S$};
\draw[thick,dashed] (R) -- (S);

% Draw parallel lines
\draw[very thick,loosely dotted,name path=para-2] 
  (8,1) -- (8,-7);
\draw[very thick,loosely dotted,name path=para-1] 
  (-1,.5) -- (9,.5);

% Draw second segments of the folds
\path[name path=p-two] (A) -- +(-37.4537:11cm);
\path [name intersections = {of = para-2 and p-two, by = {PP}}];
\fill (PP) circle (2pt) node[below left] {$P'$};
\draw[ultra thick,dotted] (A) -- (PP);

\path[name path=p-one] (D) -- +(142.5463:2cm);
\path [name intersections = {of = para-1 and p-one, by = {QP}}];
\fill (QP) circle (2pt) node[above] {$Q'$};
\draw[ultra thick,dotted] (D) -- (QP);

% Draw right angle indications
\draw[thick,rotate=-37.4537] (R) rectangle +(8pt,8pt);
\draw[thick,rotate=-127.4537] (S) rectangle +(8pt,8pt);
\end{tikzpicture}
\end{center}

\section{Constructing the central angle of the nonagon}

We have that $x=\cos\theta=0.766044$ is a root of the equation, so we need to construct $\cos^{-1} 0.766044=40^\circ$ in order to construct a nonagon. We use the previous construction of the angle $37.4537^\circ$ whose tangent is $0.766044$.

Given a line segment of length $1$ and the angle $37.4537^\circ$, the right triangle with this angle whose adjacent side is $1$ will have an opposite side of $0.766044$ by definition of tangent:
\begin{center}
\begin{tikzpicture}[scale=1.25]
\draw (0,0) coordinate (A) -- node[below] {$1$} (4,0) coordinate (B);
\draw (B) -- node[right] {$0.766044$} 
  ($(B)+(0,0.766044*4)$) coordinate (C);
\draw (A) -- (C);
\draw[rotate=90] (B) rectangle +(8pt,8pt);
\fill (A) circle (1pt) node[above left] {$A$}
  node[above right,xshift=14pt] {$37.4537^\circ$};
\fill (B) circle (1pt) node[above right] {$B$};
\fill (C) circle (1pt) node[right] {$C$};
\end{tikzpicture}
\end{center}

\newpage

Let us fold the side $\overline{CB}$ onto the side $\overline{AB}$. Then  translate $\overline{AB}$ to the right so that its endpoints are $D,E$ (\cite{alperin} explains how to translate a line segment).
\begin{center}
\begin{tikzpicture}[scale=1.25]
\draw (B) -- node[right] {$0.766044$} 
  ($(B)+(0,0.766044*4)$) coordinate (C);
\draw (A) -- (C);
\draw[rotate=90] (B) rectangle +(8pt,8pt);
\fill (A) circle (1pt) node[above left] {$A$};
\fill (B) circle (1pt) node[above right] {$B$};
\fill (C) circle (1pt) node[right] {$C$};
\coordinate (D) at (0.233956*4,0);
\fill (D) circle (1pt) node[below left] {$D$};
\coordinate (E) at ($(D)+(4,0)$);
\draw (D) -- node[fill=white] {$0.766044$} (B);
\draw (B) -- (E);
\fill (E) circle(1pt) node[above right] {$E$};
\draw[very thick,dotted,->,bend right=50] ($(C)+(-.2,0)$) to ($(A)+(.94,.2)$);
\draw[<->] ($(D)+(0,-.6)$) -- node[fill=white] {$1$} ($(E)+(0,-.6)$);
\end{tikzpicture}
\end{center}

Now fold $\overline{DE}$ onto the extension of $\overline{CB}$ so that the point $E$ is placed onto this line at $F$. Then $\angle EDF=\cos^{-1} \disfrac{0.766044}{1}=40^\circ$.
\begin{center}
\begin{tikzpicture}[scale=1.25]
\draw (B) -- ($(B)+(0,0.766044*4)$) coordinate (C);
\draw[rotate=90] (B) rectangle +(8pt,8pt);
\fill (B) circle (1pt) node[above right] {$B$};
\fill (C) circle (1pt) node[right] {$C$};
\coordinate (D) at (0.233956*4,0);
\fill (D) circle (1pt) node[above left] {$D$}
  node[above right,xshift=10pt,yshift=2pt] {$40^\circ$};
\coordinate (E) at ($(D)+(4,0)$);
\draw (D) -- node[fill=white] {$0.766044$} (B);
\draw (B) -- (E);
\fill (E) circle(1pt) node[right] {$E$};
\coordinate (F) at ($(B)+(0,4)$);
\draw[very thick,dotted,->,bend right=50] ($(E)+(.1,.2)$) to ($(F)+(.2,0)$);
\draw (B) -- ($(B)+(0,5)$);
\fill (F) circle (1pt) node[left] {$F$};
\draw (D) -- node[fill=white] {$1$} (F);
\draw[<->] ($(D)+(0,-.6)$) -- node[fill=white] {$1$} ($(E)+(0,-.6)$);
\end{tikzpicture}
\end{center}

