\documentclass[12pt,a4paper]{article}
\usepackage[utf8x]{inputenc}
\usepackage[english,hebrew]{babel}
\usepackage{graphicx}
\usepackage{verbatim}
\usepackage{url}
\usepackage{float}

\usepackage{tikz}
\usetikzlibrary{external,positioning,through,calc,intersections}
\tikzexternalize[prefix=tikz/]

\textwidth=15.5cm
\textheight=23cm
\topmargin=0pt
\headheight=0pt
\oddsidemargin=2em
\headsep=0pt
\renewcommand{\baselinestretch}{1.1}
\setlength{\parskip}{0.3\baselineskip plus 1pt minus 1pt}
\parindent=0pt

\newcommand*{\disfrac}[2]{\displaystyle\frac{#1}{#2}}


\begin{document}
\thispagestyle{empty}

\selectlanguage{hebrew}

\begin{center}
\textbf{\Huge%
אני מסתפק בסרגל )ועוד משהו(
}

\bigskip
\bigskip
\bigskip
\bigskip

\textbf{\Large מוטי בן-ארי}

\bigskip

\textbf{\Large המחלקה להוראת המדעים}

\bigskip

\textbf{\Large מכון ויצמן למדע}

\bigskip

\selectlanguage{english}
\url{http://www.weizmann.ac.il/sci-tea/benari/}
\selectlanguage{hebrew}\end{center}

\bigskip
\bigskip
\bigskip
\bigskip

\begin{center}
\selectlanguage{english}
\copyright{}\  2018 by Moti Ben-Ari.
\end{center}

\selectlanguage{english}

{\small This work is licensed under the Creative Commons Attribution-ShareAlike 3.0 Unported License. To view a copy of this license, visit \url{http://creativecommons.org/licenses/by-sa/3.0/} or send a letter to Creative Commons, 444 Castro Street, Suite 900, Mountain View, California, 94041, USA.}

\bigskip

\begin{center}\selectlanguage{english}
\includegraphics[width=.2\textwidth]{../by-sa.png}
\selectlanguage{hebrew}\end{center}

\selectlanguage{hebrew}

\newpage

%%%%%%%%%%%%%%%%%%%%%%%%%%%%%%%%%%%%%%%%%%%%%%%%%%%%%%%%%%%%%%%

\section{%
מבוא
}\label{s.intro}

כל בנייה בסרגל ומחוגה ניתנת לבנייה עם מחוגה בלבד. משפט זה הוכח בשנת 
$1672$
על ידי
\L{Georg Mohr}
וב-
$1797$
על ידי
\L{Lorenzo Mascheroni}.
נשאלת השאלה: האם כל בנייה בסגל ומחוגה ניתנת לבנייה עם סרגל בלבד? התשובה היא שלילית. ב-
$1822$
המתיטיקאי הצרפתי
\L{Jean-Victor Poncelet}
שיער שכן ניתן להסתפק בסרגל בלבד, בתנאי שקיים במישור מעגל
\textbf{%
אחד%
}
עם מרכזו. המשפט הוכח ב-
$1833$
על ידי המתימטיקאי השוויצרי
\L{Jakob Steiner}.

במסמך זה אביא את הוכחת המשפט המבוססת על הוכחה שמופיעה כבעייה
$34$
בספר:

\selectlanguage{english}
Heinrich D\"{o}rrie: \textit{100 Problems of Elementary Mathematics: Their History and Solution} (Dover, 1965),
\selectlanguage{hebrew}

ועובדה על ידי
\L{Michael Woltermann}.\footnote{\url{http://www2.washjeff.edu/users/mwoltermann/Dorrie/DorrieContents.htm}.
ברצוני להודות לו על הרשות להתשמש בעבודתו.
}.

%%%%%%%%%%%%%%%%%%%%%%%%%%%%%%%%%%%%%%%%%%%%%%%%%%%%%%%%%%%%%%%


במאה ה-
$19$
הוכח שאם נתחיל עם קטע קו עם אורך 
$1$
)המידות לא משנות, פשוט קובעים שאורך הקטע הוא אחד(, ניתן לבנות קטעי קו באורך המתקבלים מהמספרים הרציונליים על ידי פעולות החשבון 
$+,-,\times,\div$
ועוד פעולת שורש ריבועי 
$\sqrt{}$,
\textbf{%
ורק%
}
מספרים אלה.

משפט זה מסביר למה לא ניתן לפתור את הבעיות המפורסמות שהציגו היוונים: חלוקת זוויות לשלושה חלקים שווים, בניית קוביה שהנפח שלו פי שניים מהנפח של קוביה נתונה, ובניית ריבוע שהשטח שלו שווה לשטח של מעגל נתון. שתי הבניות הראשונות מחייבות בניית קטע קו שאורכו שורש שלישי של קו אחר, וריבוע המעגל מחייב בניית קטע באורך
$\pi$
שהוא מספר "טרנסנדנטלי", כלומר, אי אפשר לחשב אותו מהמספרים הרציונליים ועוד פעולה שורש מחזקה כלשהי.

עם סרגל בלבד ניתן לבנות רק קווים הנחתכים ביניהם, שהיא פעולה לשמעשה פותרת משוואות מסדר ראשון, כלומר, אי אפשר לחשב שורש ריבועי. בנייה של
\L{Poncelet-Steiner}
מראה שאם קיים מעגל אחד, ניתן להשתמש בו כדי לחשב שורש ריבועי וכך לבניות כל בנייה על סרגל ומחוגה.

%%%%%%%%%%%%%%%%%%%%%%%%%%%%%%%%%%%%%%%%%%%%%%%%%%%%%%%%%%%%%%%

עיון בבנייה גיאומטרית יגלה שכל צעד הוא אחת משלוש פעולות:
\begin{itemize}
\item
מציאת נקודת החיתוך של שני קווים ישרים.
\item
מציאת נקודות החיתוך בין קו ישר ומעגל.
\item
מציאת נקודות החיתוך בין שני מעגלים.
\end{itemize}
ברור שניתן לבצע את הפעולה הראשונה עם סרגל בלבד. עלינו להראות שעבור שתי הפעולות הראשונות ניתן למוצא בנייה שקולה המשתמשת רק בסרגל עם מעגל אחד.

%%%%%%%%%%%%%%%%%%%%%%%%%%%%%%%%%%%%%%%%%%%%%%%%%%%%%%%%%%%%%%%

סימונים:
\begin{itemize}
\item $c(O,A)$: 
המעגל שמרכזו
$O$
העובר דרך הנקודה
$A$.
\item $c(O,r)$:
המעגל שמרכזו
$O$
עם רדיוס
$r$.
\item $c(O,AB)$:
המעגל שמרכזו
$O$
עם רדיוס שהוא אורך קטע קו נתון
$AB$.
\end{itemize}



%%%%%%%%%%%%%%%%%%%%%%%%%%%%%%%%%%%%%%%%%%%%%%%%%%%%%%%%%%%%%%%
מה המשמעות של בנייה עם סרגל בלבד? מעגל מוגדר על יהי המרכז שלו
$O$
וקטע קו שאחת מהנקודות הקצה שלה היא
$O$,
קטע המגדיר את הרדיוס. אם נצליח לבנות את הנקודות
$X,Y$
בהתרשים להלן, נוכל לטעון שהצלחנו לבנות את נקודות החיתוך של מעגל נתון וקו נתון ושל שני מעגלים. המעגלים מצויירים בקו מקווקוו כי אם לא ממש מפיעים בבנייה. נמשיך לנקוט בסימון זה: המעגל היחיד הנתון יצוייר בקו רגיל, ומעגלים המשמשים רק להדגמת הבנייה והוכחתה יהיו מקווקווים.


\begin{center}
\selectlanguage{english}
\selectlanguage{english}
\begin{tikzpicture}[scale=.9]
\fill (0,0) node[above right] {$O$} circle[radius=2pt];
\draw[thick,dashed,name path=circle] (0,0) circle[radius=2cm];
\draw (0,0) -- node[left] {$r$} ++(-60:2cm);
\fill (0,0) ++(-60:2cm) circle[radius=2pt];
\draw[name path=line] (-3,-.5) -- ++(20:6cm);
\path [name intersections={of=circle and line,by={X,Y}}];
\fill (X) node[above right,xshift=-2pt,yshift=4pt] {$X$} circle[radius=2pt];
\fill (Y) node[above left] {$Y$} circle[radius=2pt];
\begin{scope}[xshift=6cm]
\fill (0,0) node[above right] {$O_1$} circle[radius=2pt];
\fill (3,0) node[above right] {$O_2$} circle[radius=2pt];
\draw[thick,dashed,name path=circle1] (0,0) circle[radius=2cm];
\draw[thick,dashed,name path=circle2] (3,0) circle[radius=2cm];
\draw (0,0) -- node[left] {$r_1$} ++(-60:2cm);
\draw (3,0) -- node[left,below] {$r_2$} ++(-20:2cm);
\fill (3,0) ++(-20:2cm) circle[radius=2pt];
\path [name intersections={of=circle1 and circle2,by={X,Y}}];
\fill (X) node[above,yshift=4pt] {$X$} circle[radius=2pt];
\fill (Y) node[below,yshift=-4pt] {$Y$} circle[radius=2pt];
\end{scope}
\end{tikzpicture}
\selectlanguage{hebrew}
\end{center}


%%%%%%%%%%%%%%%%%%%%%%%%%%%%%%%%%%%%%%%%%%%%%%%%%%%%%%%%%%%%%%%


תחילה נביא חמש בניות עזר נחוצות )סעיפים
\L{\ref{s.parallel}--\ref{s.root}}%
(,
ואחר כך נראה איך למצוא תקודות חיתוך בין קו ומעגל )סעיף
\L{\ref{s.line-circle}}%
( ובין שני מעלגים )סעיף
\L{\ref{s.circle-circle}}%
(.
%%%%%%%%%%%%%%%%%%%%%%%%%%%%%%%%%%%%%%%%%%%%%%%%%%%%%%%%%%%%%%%

\section{%
בניית קו המקביל לקו נתון%
}\label{s.parallel}

\textbf{%
נתון קו
$l$
המוגדר על ידי שתי נקודות
$A,B$,
ונקודה 
$P$
)שאיננה על הקו( ניתן לבנות קו דרך
$P$
המקביל ל-
$AB$.%
}

נפריד את הבנייה לשני מקרים, כאשר הבנייה של המקרה הראשון תעזור לנו בבנייה של המקרה השני.
\begin{itemize}
\item
"קו מכוון": נתון שתי נקודות
$A,B$
על הקו והנקודה 
$M$
החוצה את
$AB$.
\item
כל קו אחר.
\end{itemize}

\textbf{%
קו מכוון:
}
נבנה קו המכיל את
$AP$.
נבחר נקודה
$S$,
נקודה כלשהי על הממשך הקו מעבר ל-%
$AP$.
נבנה את הקווים
$MS$, $BS$.
נסמן ב-%
$O$
את החיתוך של 
$BP$
ו-%
$MS$.
נבנה קו המכיל את
$AO$
ונסמן ב-%
$Q$
את החיתוך של ההמשך של
$AO$
עם
$BS$.
\begin{center}
\selectlanguage{english}
\vspace*{-8pt}
\begin{tikzpicture}
\draw[name path=pq] (-4,0) -- (4,0);
\draw (-2,-2) node[below left] {$A$} coordinate (A) -- (2,-2) node[below right] {$B$} coordinate (B);
\fill (A) circle[radius=2pt];
\fill (B) circle[radius=2pt];
\draw[name path=as] (A) -- ++(50:4cm) node[above] {$S$} coordinate (S);
\fill (S) circle[radius=2pt];
\draw[name path=sb] (S) -- (B);
\path [name intersections={of=pq and as,by={P}}];
\path [name intersections={of=pq and sb,by={Q}}];
\fill (P) node[above left] {$P$} circle[radius=2pt];
\fill (Q) node[above right] {$Q$} circle[radius=2pt];
\draw[name path=pb] (P) -- (B);
\draw[name path=qa] (Q) -- (A);
\path [name intersections={of=pb and qa,by={O}}];
\fill (O) node[right,xshift=2pt] {$O$} circle[radius=2pt];
\fill (0,-2) coordinate (M) node[below right] {$M$} circle[radius=2pt];
\draw (S) -- (M);
\end{tikzpicture}
\vspace*{-6pt}
\selectlanguage{hebrew}
\end{center}
מהתרשים נראה ברור שהקו 
$PQ$
מקביל ל-%
$AB$,
אבל זה בדיוק מה שעלינו להוכיח.

ההוכחה תשתמש במשפט של
\L{Ceva}
שנוכיח בהמשך. לפי המשפט, קיים קשר בין האורכים של קטעים המרכיבים את היקף המשולש:
\[
\frac{AM}{MB}\frac{BQ}{QS}\frac{SP}{PA} = 1
\]
נזכור ש-%
$M$
היא הנקודה החוצה של
$AB$,
כך ש-%
$AM=MB$
ו-%
$\frac{AM}{MB}=1$.
הגורם הראשון של המכלפה יורד ונרשום את המשוואה:
\begin{equation}
\frac{BS}{QS}=\frac{AS}{PS}\,.
\end{equation}
נוכיח שהמשולש
$\triangle ABS$
דומה ל-%
$\triangle PQS$,
ולכן הקו
$PQ$
מקביל לקו
$AB$
כי
$\angle ABS = \angle PQS$.
ההוכחה שהמשולשים דומים היא:
\[
\renewcommand*{\arraystretch}{2}
\begin{array}{rcl}
BS&=&BQ+QS\\
\disfrac{BS}{QS}&=&\disfrac{BQ}{QS}+\disfrac{QS}{QS} = \disfrac{BQ}{QS}+1\\
AS&=&AP+PS\\
\disfrac{AS}{PS} &=& \disfrac{AP}{PS} + \disfrac{PS}{PS} = \disfrac{AP}{PS} + 1\\
\disfrac{BS}{QS}&=&\disfrac{AS}{PS}\,.
\end{array}
\]
כאשר המשוואה האחרונה מתקבלת ממשוואה-)1(.

כדי להוכיח את המשפט של
\L{Ceva},
נתבונן בתרשימים שלהן:
\begin{center}
\selectlanguage{english}
\vspace*{-8pt}
\begin{tikzpicture}
\path[name path=pq] (-4,0) -- (4,0);
\draw (-2,-2) node[below left] {$A$} coordinate (A) -- (2,-2) node[below right] {$B$} coordinate (B);
\coordinate (M) at (0,-2);
\draw[name path=as] (A) -- ++(50:4cm) node[above] {$S$} coordinate (S);
\draw[name path=sb] (S) -- (B);
\path [name intersections={of=pq and as,by={P}}];
\path [name intersections={of=pq and sb,by={Q}}];
\path[name path=pb] (P) -- (B);
\path[name path=qa] (Q) -- (A);
\path [name intersections={of=pb and qa,by={O}}];
\draw[fill=gray!40] (B) -- (O) -- (Q);
\draw[fill=gray!70] (S) -- (O) -- (Q);
\draw (B) -- (O) -- (A);
\draw (S) -- (O) -- (A);
\draw (A) -- (B) -- (S) -- cycle;
\draw (S) -- (O);
\draw (B) -- (O);
\fill (A) circle[radius=2pt];
\fill (B) circle[radius=2pt];
\fill (S) circle[radius=2pt];
\fill (Q) node[above right] {$Q$} circle[radius=2pt];
\fill (O) node[above left] {$O$} circle[radius=2pt];
\path[name path=al1] (O) -- ($(Q)!(O)!(B)$);
\path [name intersections={of=al1 and sb,by={A1}}];
\draw[thick,dashed] (O) -- (A1);
\begin{scope}[xshift=6cm]
\path[name path=pq] (-4,0) -- (4,0);
\draw (-2,-2) node[below left] {$A$} coordinate (A) -- (2,-2) node[below right] {$B$} coordinate (B);
\coordinate (M) at (0,-2);
\draw[name path=as] (A) -- ++(50:4cm) node[above] {$S$} coordinate (S);
\draw[name path=sb] (S) -- (B);
\path [name intersections={of=pq and as,by={P}}];
\path [name intersections={of=pq and sb,by={Q}}];
\draw[name path=pb] (P) -- (B);
\draw[name path=qa] (Q) -- (A);
\path [name intersections={of=pb and qa,by={O}}];
\draw (B) -- (O) -- (Q);
\draw (A) -- (Q) -- (B);
\draw[fill=gray!40] (B) -- (Q) -- (A);
\draw[fill=gray!70] (S) -- (Q) -- (A);
\draw (A) -- (B) -- (S) -- cycle;
\draw (S) -- (O);
\draw (B) -- (O);
\fill (A) circle[radius=2pt];
\fill (B) circle[radius=2pt];
\fill (S) circle[radius=2pt];
\fill (Q) node[above right] {$Q$} circle[radius=2pt];
\fill (O) node[above left] {$O$} circle[radius=2pt];
\path[name path=al2] (A) -- ($(Q)!(A)!(B)$);
\path [name intersections={of=al2 and sb,by={A2}}];
\draw[thick,dashed] (A) -- (A2);
\end{scope}
\end{tikzpicture}
\vspace*{-10pt}
\selectlanguage{hebrew}
\end{center}
אם הגבהים של שני משולשים שוואים, יחס השטחים שווה ליחס הבסיסים:
\[
A_1 = \frac{1}{2}hb_1,\quad A_2 = \frac{1}{2}hb_2, \quad \frac{A_1}{A_2}=\frac{b_1}{b_2}\,.
\]
בכל אחד מהתרשימים, הגבהים של זוג המשולשים המסומנים באפור שווים. לכן:%
\footnote{%
נשתמש בשם המשולש כקיצור לשטחו.%
}
\[\frac{\triangle BQO}{\triangle SQO} = \frac{BQ}{QS}\;,\quad\quad \frac{\triangle BQA}{\triangle SQA} = \frac{BQ}{QS}\;.
\]
על ידי חיסור של המשולשים המסומנים, נקבל יחס בין המשולשים המסומנים באפור בתרשים שלהן:


\begin{center}\selectlanguage{english}
\vspace*{-8pt}
\begin{tikzpicture}
\path[name path=pq] (-4,0) -- (4,0);
\draw (-2,-2) node[below left] {$A$} coordinate (A) -- (2,-2) node[below right] {$B$} coordinate (B);
\coordinate (M) at (0,-2);
\draw[name path=as] (A) -- ++(50:4cm) node[above] {$S$} coordinate (S);
\draw[name path=sb] (S) -- (B);
\path [name intersections={of=pq and as,by={P}}];
\path [name intersections={of=pq and sb,by={Q}}];
\path[name path=pb] (P) -- (B);
\draw[thick,name path=qa] (Q) -- (A);
\path [name intersections={of=pb and qa,by={O}}];
\draw[fill=gray!50] (B) -- (O) -- (A);
\draw[fill=gray!70] (S) -- (O) -- (A);
\draw (B) -- (O) -- (A);
\draw (S) -- (O) -- (A);
\draw (A) -- (B) -- (S) -- cycle;
\draw (S) -- (O);
\draw (B) -- (O);
\fill (A) circle[radius=2pt];
\fill (B) circle[radius=2pt];
\fill (S) circle[radius=2pt];
\fill (Q) node[above right] {$Q$} circle[radius=2pt];
\fill (O) node[right,xshift=2pt] {$O$} circle[radius=2pt];
\end{tikzpicture}
\vspace*{-6pt}
\selectlanguage{hebrew}\end{center}
\[
\frac{BQ}{QS} = \frac{\triangle BQA - \triangle BQO}{\triangle SQA-\triangle SQO} = \frac{\triangle BOA}{\triangle SOA}\,.
\]
החישוב עלול להיראות חשוד. נסביר אותו תוך שימוש בסימונים פשוטים יותר:
\[
\renewcommand*{\arraystretch}{1.6}
\begin{array}{rcl}
 \disfrac{c}{d} &=&\disfrac{a}{b}\\
 \disfrac{e}{f} &=&\disfrac{a}{b}\\
c-e &=& \disfrac{ad}{b} - \disfrac{af}{b}\\
c-e &=& \disfrac{a}{b}(d-f)\\
\disfrac{c-e}{d-f} &=& \disfrac{a}{b}\,.
\end{array}
\]
באופן דומה ניתן להוכיח:
\[
\frac{AM}{MB} = \frac{\triangle AOS}{\triangle BOS}\;,\quad\quad \frac{SP}{PA} =\frac{\triangle SOB}{\triangle AOB}\;,
\]
ומכאן
\[
\frac{AM}{MB}\frac{BQ}{QS}\frac{SP}{PA} = \frac{\triangle AOS}{\triangle BOS}\frac{\triangle BOA}{\triangle SOA}\frac{\triangle SOB}{\triangle AOB}=1\,,
\]
כי השטחים במונה ובמכנה מצטצמים )זכרו שסדר הקודקים במשלוש לא חשוב(.
%%%%%%%%%%%%%%%%%%%%%%%%%%%%%%%%%%%%%%%%%%%%%%%%%%%%%%%%%%%%%%%

\textbf{%
כל קו אחר:
}
נסמן את הקו ב-%
$l$,
את המעגל הקבוע ב-%
$c=c(O,r)$,
והנקודה שלא על הקו ב-%
$P$.
עליך להשתכנע שהבנייה, כאן ובהמשך, לא תלוייה מיקום המעגל במישור או ברדיוס שלו.

נבחר 
$M$,
נקודה כלשהי על הקו 
$l$,
ובנה את הקו מ-%
$M$
ל-%
$O$
)נקודת חיתוך
$U$(
ונמשיך את הקו עד לנקודת החיתוך השני
$V$.
\begin{center}
\selectlanguage{english}
\begin{tikzpicture}[scale=.8]
\coordinate (O) at (0,0);
\fill (O) node[below right] {$O$} circle[radius=2pt];
\draw[name path=circle] (O) circle[radius=2cm];
\draw[name path=l] (-4,-3) -- node[above, near end] {$l$} +(9,0);
\path[name path=mo] (-2,-3) coordinate (M) -- ($(-2,-3)!1.65!(O)$);
\fill (M) node[below] {$M$} circle[radius=2pt];
\path [name intersections={of=circle and mo,by={V,U}}];
\fill (U) node[below,xshift=2pt,yshift=-4pt] {$U$} circle[radius=2pt];
\fill (V) node[right,xshift=4pt] {$V$} circle[radius=2pt];
\draw (M) -- (V);
\node at (-1.6,1.6) {$c$};
\fill (-1,-4) node[right] {$P$} circle[radius=2pt];
\end{tikzpicture}
\vspace*{-16pt}
\selectlanguage{hebrew}
\end{center}
קו זה הוא
\textbf{%
קו מכוון%
}
כי 
$O$,
מרכז המעגל, חוצה את קטע הקו
$UV$.
נבחר נקודה שנייה 
$A$
על הקו 
$l$
ונשתמש בבבנייה המתואר במקרה הראשון כי לבנות קו מקביל ל-%
$UV$
החותך את המעגל בנקודות
$X,Y$.
\begin{center}
\selectlanguage{english}
\begin{tikzpicture}[scale=.8]
\coordinate (O) at (0,0);
\fill (O) node[below right] {$O$} circle[radius=2pt];
\draw[name path=circle] (O) circle[radius=2cm];
\draw[name path=l] (-4,-3) -- node[above,near end,xshift=24pt] {$l$} +(9,0);
\path[name path=mo] (-2,-3) coordinate (M) -- ($(-2,-3)!1.65!(O)$);
\fill (M) node[below] {$M$} circle[radius=2pt];
\path [name intersections={of=circle and mo,by={V,U}}];
\fill (U) node[below,xshift=2pt,yshift=-4pt] {$U$} circle[radius=2pt];
\fill (V) node[right,xshift=4pt] {$V$} circle[radius=2pt];
\draw (M) -- (V);
\path[name path=ax] (-3,-3) coordinate (A) -- ($(-3,-3)!1.8!(-1,0)$);
\fill (A) node[below] {$A$} circle[radius=2pt];
\path [name intersections={of=circle and ax,by={Y,X}}];
\fill (X) node[left] {$X$} circle[radius=2pt];
\fill (Y) node[above] {$Y$} circle[radius=2pt];
\node at (-1.6,1.6) {$c$};
\draw (A) -- (Y);
\fill (-1,-4) node[right] {$P$} circle[radius=2pt];
\end{tikzpicture}
\vspace*{-10pt}
\selectlanguage{hebrew}
\end{center}
נבנה קוטר עם נקודת קצה דרך
$X$
וקוטר עם נקודת קצה דרך
$Y$,
ונגדיר את נקודות החיתוך האחרות שלהם
$X',Y'$.
נבנה את הקו 
$X',Y'$
ונמשיך אותו עד ל-%
$B$,
נקודת החיתוך עם 
$l$.
\begin{center}
\selectlanguage{english}
\begin{tikzpicture}[scale=.8]
\coordinate (O) at (0,0);
\fill (O) node[below right] {$O$} circle[radius=2pt];
\draw[name path=circle] (O) circle[radius=2cm];
\draw[name path=l] (-4,-3) -- node[above,near end,xshift=24pt] {$l$} +(9,0);
\path[name path=mo] (-2,-3) coordinate (M) -- ($(-2,-3)!1.65!(O)$);
\fill (M) node[below] {$M$} circle[radius=2pt];
\path [name intersections={of=circle and mo,by={V,U}}];
\fill (U) node[below,xshift=2pt,yshift=-4pt] {$U$} circle[radius=2pt];
\fill (V) node[right,xshift=4pt] {$V$} circle[radius=2pt];
\draw (M) -- (V);
\path[name path=ax] (-3,-3) coordinate (A) -- ($(-3,-3)!1.8!(-1,0)$);
\fill (A) node[below] {$A$} circle[radius=2pt];
\path [name intersections={of=circle and ax,by={Y,X}}];
\fill (X) node[left] {$X$} circle[radius=2pt];
\fill (Y) node[above] {$Y$} circle[radius=2pt];
\node at (-1.6,1.6) {$c$};
\draw (A) -- (Y);
\fill (-1,-4) node[right] {$P$} circle[radius=2pt];
\path[name path=xo] (X) -- ($(X)!2.2!(O)$);
\path[name intersections={of=circle and xo,by={Xp}}];
\fill (Xp) node[right,xshift=2pt,yshift=-2pt] {$X'$} circle[radius=2pt];
\draw (X) -- (Xp);
\path[name path=yo] (Y) -- ($(Y)!2.2!(O)$);
\path[name intersections={of=circle and yo,by={y,Yp}}];
\fill (Yp) node[below right] {$Y'$} circle[radius=2pt];
\draw (Y) -- (Yp);
\path[name path=xy] (Xp) -- ($(Xp)!1.6!(Yp)$);
\path[name intersections={of=l and xy,by={B}}];
\fill (B) node[below] {$B$} circle[radius=2pt];
\draw (Xp) -- (B);
\draw[thick,dashed,name path=z] (-4,0) -- (4,0) node[above,near end,xshift=40pt] {$l'$};
\path[name intersections={of=ax and z,by={Z}}];
\path[name intersections={of=xy and z,by={Zp}}];
\fill (Z) node[above left] {$Z$} circle[radius=2pt];
\fill (Zp) node[below right] {$Z'$} circle[radius=2pt];
\end{tikzpicture}
\vspace*{-10pt}
\selectlanguage{hebrew}
\end{center}
\textbf{%
טענה:%
}
הקו המכיל את 
$AB$
הוא
\textbf{%
קו מכוון
}
כי
$M$
חוצה אותו.

מהטענה אפשר לבנות קו דרך 
$P$
מקביל ל-%
$AB$
כמתואר במקרה הראשון.

\textbf{%
הוכחה:%
}
$OX,OX',OY,OY'$
הם כולם רדיוסים את המעגל, ו-%
$\angle XOY = \angle X'OY'$
כי הן זוויות נגדיות. לכן,
$\triangle XOY$
חופף ל-%
$\triangle X'OY'$
לפי צלע-זווית-צלע. נגדיר
$l'$,
קו מקביל ל-%
$l$,
החותך את 
$XY$
בנקודה
$Z$
וחותך את 
$X',Y'$
בנקודה
$Z'$.
המשולשים
$\triangle XOZ$,
$\triangle X'OZ'$
חופפים לפי זווית-צלע-זווית, ולכן
$ZO=OZ'$. 
הוכחנו ש-%
$AMOZ$
ו-%
$BMOZ'$
מקביליות, ולכן
$AM=ZO=OZ'=MB$.

\textbf{%
מסקנה:%
}
נתון קטע קו 
$AB$
ונקודה
$P$
שאיננה על הקו. ניתן לבנות קטע קו
$PQ$
עם נקודת קצה 
$P$
המקביל ל-%
$AB$.

\textbf{%
הוכחה:%
}
הוכחנו שנתון 
$AB$
ו-%
$P$
ניתן לבנות קו 
$m$
דרך 
$P$
המקביל ל-%
$AB$,
וקו
$n$
דרך 
$B$
המקביל ל-%
$AP$.
המרובע 
$ABQP$
הוא מקבילית, ולכן הצלעות הנגדיות שוות ו-%
$AB=PQ$
ו-%
$PQ$
מקביל ל-%
$AB$.
\begin{center}
\selectlanguage{english}
\begin{tikzpicture}[scale=.8]
\coordinate (P) at (0,0);
\coordinate (Q) at (3,0);
\coordinate (A) at (-2,2.5);
\coordinate (B) at (1,2.5);
\draw ($(P)!-.6!(Q)$) -- node[above,near end,xshift=36pt] {$m$} ($(P)!1.8!(Q)$);
\fill (P) node[below] {$P$} circle[radius=2pt];
\fill (Q) node[below left] {$Q$} circle[radius=2pt];
\draw ($(A)!-.6!(B)$) -- node[above,near end,xshift=40pt] {$l$} ($(A)!2.5!(B)$);
\fill (A) node[above left] {$A$} circle[radius=2pt];
\fill (B) node[above right] {$B$} circle[radius=2pt];
\draw (A) -- (P);
\draw ($(B)!-.3!(Q)$) -- node[above,near end,xshift=24pt,yshift=-24pt] {$n$} ($(B)!1.4!(Q)$);
\end{tikzpicture}
\selectlanguage{hebrew}
\end{center}

%%%%%%%%%%%%%%%%%%%%%%%%%%%%%%%%%%%%%%%%%%%%%%%%%%%%%%%%%%%%%%%

\section{%
בניית אנח לקו נתון%
}\label{s.perpendicular}

\textbf{%
נתון קו
$l$
המוגדר על ידי שתי נקודות
$A,B$,
ונקודה
$P$
)שאיננה על הקו( ניתן לבנות קו דרך
$P$
אנח ל-
$AB$.%
}

נבנה )כפי שתואר בסעיף
\L{\ref{s.parallel}}(
קו
$l'$
מקביל ל-%
$l$
החותך את המעגל הקבוע ב-%
$U,V$.
נבנה את הקוטר
$UOU'$
והמיתר
$VU'$.
\begin{center}
\selectlanguage{english}
\begin{tikzpicture}[scale=.8]
\coordinate (O) at (0,0);
\coordinate (P) at (3.5,.6);
\node at (-1.6,1.6) {$c$};
\draw[name path=circle] (O) circle[radius=2cm];
\draw[name path=l] (-4,-3) -- node[above,near end,xshift=45pt] {$l$} ++(9,0);
\draw[name path=lp] (-3,-1) -- node[above,near end,xshift=45pt] {$l'$} ++(7,0);
\fill (O) node[left] {$O$} circle[radius=2pt];
\fill (P) node[right] {$P$} circle[radius=2pt];
\path[name intersections={of=circle and lp,by={U,V}}];
\fill (U) node[below left] {$U$} circle[radius=2pt];
\fill (V) node[below right] {$V$} circle[radius=2pt];
\path[name path=d] (U) -- ($(U)!2.3!(O)$);
\path[name intersections={of=circle and d,by={Up}}];
\draw (U) -- (Up);
\fill (Up) node[above right] {$U'$} circle[radius=2pt];
\draw (Up) -- (V);
\path[name path=p] (P) -- ++(0,-4);
\draw[name intersections={of=p and l,by={X}}];
\fill (X) circle[radius=2pt];
\draw[thick,dashed] (P) -- (X);
\draw ($(U)!.9!(V)$) -- ++(0,.3) -| (V);
\end{tikzpicture}
\vspace*{-12pt}
\selectlanguage{hebrew}
\end{center}
$\angle UVU'$
היא זווית הנשענת על מחצית המעגל, לכן היא זווית ישרה. מכאן ש-%
$VU'$
הוא אנח ל-%
$UV$
ו-
$l$.
לבסוף, נבנה קו העובר דרך 
$P$
מקביל ל-%
$VU'$
כפי שתואר בסעיף
\L{\ref{s.parallel}}.



%%%%%%%%%%%%%%%%%%%%%%%%%%%%%%%%%%%%%%%%%%%%%%%%%%%%%%%%%%%%%%%

\section{%
העתקת קטע קו נתון בכיוון נתון%
}\label{s.direction}

\textbf{%
נתון נקודה
$A$,
קטע קו
$PQ$
וכיוון, ניתן לבנות קטע קו
$AS$
בכיוון הנתון.%
}
הכוונה של המושג כיוון היא שנתון קו המוגדר על ידי שתי נקובות
$A',H'$
המגדיר זווית
$\theta$
יחסית לציר מסויים. המשימה היא להעתיק קטע קו
$PQ$
ל-%
$AS$
כך ש-%
$AS$
יהיה באותה זווית
$\theta$.
\begin{center}
\selectlanguage{english}
\vspace*{-4pt}
\begin{tikzpicture}[scale=.8]
\coordinate (A) at (0,0);
\coordinate (P) at (1,-1.5);
\coordinate (Q) at (2.5,-1.5);
\draw (P) -- (Q);
\fill (P) node[left] {$P$} circle[radius=2pt];
\fill (Q) node[right] {$Q$} circle[radius=2pt];
\coordinate (A1) at (-3,1);
\draw (A1) -- ++(60:3cm) coordinate (H1);
\draw[thick,dashed] (A1) -- ++(0:1.5cm);
\fill (A1) node[left] {$A'$} circle[radius=2pt];
\fill (H1) node[left] {$H'$} circle[radius=2pt];
\draw[thick,dashed] (A) -- ++(60:1.5cm);
\draw[thick,dashed] (A) -- ++(1.5,0);
\fill (A) node[left] {$A$} circle[radius=2pt];
\node[above right,xshift=4pt] at (A1) {$\theta$};
\node[above right,xshift=4pt] at (A) {$\theta$};
\end{tikzpicture}
\vspace*{-8pt}
\selectlanguage{hebrew}
\end{center}
נתחיל את הבנייה על ידי העתקת קטע הקו
$A'H'$
אל 
$AH$
כך ש-%
$AH$
יהיה מקביל ל-%
$A'H'$.
הבנייה אפשרית לפי המסקנה בסוף סעיף
\L{\ref{s.parallel}}.
באותו אופן נעתיק את
$PQ$
ל-%
$AK$
כך ש-%
$AK$
מקביל ל-%
$PQ$.
\begin{center}
\selectlanguage{english}
\vspace*{-4pt}
\begin{tikzpicture}[scale=.8]
\coordinate (A) at (0,0);
\coordinate (P) at (1,-1.5);
\coordinate (Q) at (2.5,-1.5);
\draw (P) -- (Q);
\fill (P) node[left] {$P$} circle[radius=2pt];
\fill (Q) node[right] {$Q$} circle[radius=2pt];
\coordinate (A1) at (-3,1);
\draw (A1) -- ++(60:3cm) coordinate (H1);
\draw[thick,dashed] (A1) -- ++(0:1.5cm);
\fill (A1) node[left] {$A'$} circle[radius=2pt];
\fill (H1) node[left] {$H'$} circle[radius=2pt];
\draw (A) -- ++(60:3cm) coordinate (H);
\fill (H) node[left] {$H$} circle[radius=2pt];
\draw (A) -- ++(1.5,0) coordinate (K);
\fill (K) node[below right] {$K$} circle[radius=2pt];
\draw (A) -- (K);
\fill (A) node[left] {$A$} circle[radius=2pt];
\node[above right,xshift=4pt] at (A1) {$\theta$};
\node[above right,xshift=4pt] at (A) {$\theta$};
\end{tikzpicture}
\vspace*{-8pt}
\selectlanguage{hebrew}
\end{center}
הזווית
$\angle HAK$
שווה ל-%
$\theta$,
לכן כל מה שנשאר הוא למצוא נקודה
$S$
על
$AH$
כך ש-%
$AS=AK$.

במעגל הקבוע 
$c$
נבנה שני רדיוסים
$OU$
ו-%
$OV$
מקביליים ל-%
$AH$
ו-%
$AK$,
בהתאמה. לבסוף, נבנה קו דרך
$K$
מקביל ל-%
$UV$.

\textbf{%
טענה:
}
$S$
היא נקודת החיתוך של הקו עם 
$AH$
ו-%
$AS=AK$.
\begin{center}
\selectlanguage{english}
\vspace*{-4pt}
\begin{tikzpicture}[scale=.8]
\coordinate (A) at (0,0);
\coordinate (P) at (1,-1.5);
\coordinate (Q) at (2.5,-1.5);
\draw (P) -- (Q);
\fill (P) node[left] {$P$} circle[radius=2pt];
\fill (Q) node[right] {$Q$} circle[radius=2pt];
\coordinate (A1) at (-3,1);
\draw (A1) -- ++(60:3cm) coordinate (H1);
\fill (A1) node[left] {$A'$} circle[radius=2pt];
\fill (H1) node[left] {$H'$} circle[radius=2pt];
\draw (A) -- ++(60:3cm) coordinate (H);
\fill (A) node[left] {$A$} circle[radius=2pt];
\fill (H) node[left] {$H$} circle[radius=2pt];
\coordinate (O) at (6,1);
\node at (4.8,3.4) {$c$};
\draw[name path=circle] (O) circle[radius=2.5cm];
\fill (O) node[above left] {$O$} circle[radius=2pt];
\draw (A) -- ++(1.5,0) coordinate (K);
\fill (K) node[below right] {$K$} circle[radius=2pt];
\draw (A) -- (K);
\path[name path=u] (O) -- ++(60:2.5cm);
\path[name path=v] (O) -- ++(2.5,0);
\path[name intersections={of=circle and u,by={U}}];
\path[name intersections={of=circle and v,by={V}}];
\fill (U) node[above right] {$U$} circle[radius=2pt];
\fill (V) node[right] {$V$} circle[radius=2pt];
\draw (O) -- (U) -- (V) -- cycle;
\path (A) -- ++(60:1.5cm) coordinate (S);
\fill (S) node[above left] {$S$} circle[radius=2pt];
\draw (K) -- (S);
\draw[very thick] (A) -- (S);
\node[above right,xshift=4pt] at (A) {$\theta$};
\node[above right,xshift=4pt] at (O) {$\theta$};
\node[above right,xshift=4pt] at (A1) {$\theta$};
\draw[thick,dashed] (A1) -- ++(1.5,0);
\end{tikzpicture}
\vspace*{-8pt}
\selectlanguage{hebrew}
\end{center}
\textbf{%
הוכחה:
}
$AH$
מקביל ל-%
$OU$
ו-%
$AK$
מקביל ל-%
$OV$
ולכן
$\angle SAK=\angle HAK=\theta=\angle UOV$.
$SK$
מקביל ל-%
$UV$,
ולכן המשולש
$\triangle SAK$
דומה למשולש
$\triangle UOV$
שהוא משולש שווה שוקיים 
($OU$, $OV$
הם רדיוסים של אותו מעגל.( מכאן,
$\triangle SAK$
הוא משולש שווה שוקיים ו-%
$AS=AK=PQ$.



%%%%%%%%%%%%%%%%%%%%%%%%%%%%%%%%%%%%%%%%%%%%%%%%%%%%%%%%%%%%%%%

\section{%
בניית קטע קו שאורכו מוגדר יחסית לשלושה קטעי קו אחרים%
}\label{s.relative}

\textbf{%
נתון קטעי קו באורכים
$n, m, s$,
ניתן לבנות קטע קו באורך
$x=\disfrac{n}{m}s$.%
}

קטעי הקו הנתונים נמצאים במיקומים כלשהם במישור ובכיוונים כלשהם.
\begin{center}
\selectlanguage{english}
\begin{tikzpicture}[scale=.9]
\draw (0,0) -- node[above] {$s$} ++(30:1.5cm);
\draw (2,1.2) -- node[above] {$m$} ++(-10:2.5cm);
\draw (-2,1.5) -- node[above] {$n$} ++(5:2cm);
\fill (0,0) circle[radius=2pt];
\fill (2,1.2) circle[radius=2pt];
\fill (-2,1.5) circle[radius=2pt];
\fill (0,0) ++(30:1.5cm) circle[radius=2pt];
\fill (2,1.2) ++(-10:2.5cm) circle[radius=2pt];
\fill (-2,1.5) ++(5:2cm) circle[radius=2pt];
\end{tikzpicture}
\vspace*{-8pt}
\selectlanguage{hebrew}
\end{center}
נבחר נקודה כלשהי
$A$
ונבה שתי קרונות 
$AB$ and $AC$.
כפי שמתואר בסעיף
\L{\ref{s.direction}}
ניתן למצוא נקודות
$M,N,S$
כך ש-%
$AM= m$, $AN =n$
ו-%
$AS=s$.
נבנה דרך
$N$
קו המקביל ל-%
$MS$
החותך את
$AC$
ב-%
$X$,
ונסמן את אורכו ב-%
$x$.
המשולש
$\triangle MAS$
דומה למשולש
$\triangle NAX$, 
ולכן:
\[
\frac{m}{n}=\frac{s}{x}, \quad\quad x=\disfrac{n}{m}s\,.
\]
\begin{center}
\selectlanguage{english}
\vspace*{-10pt}
\begin{tikzpicture}
\coordinate (A) at (0,0);
\draw[name path=ac] (A) node[left] {$A$} -- ++(7,0) node[right] {$C$};
\draw (A) -- ++(40:5cm) node[right] {$B$};
\fill (A) circle[radius=2pt];
\fill (A) ++(40:5cm) circle[radius=2pt];
\fill (A) ++(7,0) circle[radius=2pt];
\path (A) -- node[above,xshift=-2pt] {$m$} ++(40:3cm) coordinate (M) node[above left] {$M$};
\path (A) -- ++(40:4cm) coordinate (N) node[above left] {$N$};
\fill (M) circle[radius=2pt];
\fill (N) circle[radius=2pt];
\path[name path=ms] (M) -- ++(-50:3.5cm);
\path[name path=nx] (N) -- ++(-50:4cm);
\path[name intersections={of=ac and ms,by={S}}];
\path[name intersections={of=ac and nx,by={X}}];
\fill (S) circle[radius=2pt] node[below] {$S$};
\fill (X) circle[radius=2pt] node[below] {$X$};
\path (A) -- node[below] {$s$} (S);
\draw (S) -- (M);
\draw (X) -- (N);
\node at (7,2.5) {$AN=n$};
\node at (7,2) {$AX=x$};
\end{tikzpicture}
\selectlanguage{hebrew}
\end{center}



%%%%%%%%%%%%%%%%%%%%%%%%%%%%%%%%%%%%%%%%%%%%%%%%%%%%%%%%%%%%%%%

\section{%
בניית שורש ריבועי של המכפלה של אורכי שני קטעי קו נתונים%
}\label{s.root}

\textbf{%
נתון קטעי קו באורכים
$a,b$,
ניתן לבעות קטע קו שאורכו
$\sqrt{ab}$.}

אפו שואפים להראות שניתן לבטא את
$x=\sqrt{ab}$
בצורה
$x=\frac{n}{m}s$
כדי להשתמש בבנייה מסעיף
\L{\ref{s.relative}}.
עבור
$n$
נשתמש ב-%
$d$
הקוטר של המעגל הקבוע, עבור
$m$
נשתמש ב-%
$t=a+b$
שניתן לבנות מהאורכים הנתונים
$a,b$
לפי הבנייה בסעיף
\L{\ref{s.direction}},
ועבור
$s$
נמצא הגדרה של האורכים
$h,k$
תוך שימוש באורכים
$a,b,t,d$
כך ש-%
$s=\sqrt{hk}$,
ונראה איך ניתן לבנות קטע קו באורך זה.

נגדיר
$h=\frac{d}{t}a$, $k=\frac{d}{t}b$, $s=\sqrt{hk}$.
נחשב:
\[
x=\sqrt{ab}=\sqrt{\frac{th}{d}\frac{tk}{d}}=\frac{t}{d}s\,.
\]
ונחשב גם: 
\[
h+k = \frac{d}{t}a + \frac{d}{t}b = \frac{d(a+b)}{t} = \frac{dt}{t} = d\,.
\]
לפי סעיף
\L{\ref{s.direction}}
ניתן לבנות קטע קו
$HA= h$
על הקוטר
$HK$
של המעגל הקבוע, ו-%
$AK=HK-HA=d-h=k$.
לפי סעיף
\L{\ref{s.perpendicular}}
ניתן לבנות דרך
$A$
אנח ל-%
$HK$.
נסמן ב-%
$S$
את החיתוך של האנח עם המעגל הקבוע.
\begin{center}
\selectlanguage{english}
\vspace*{-8pt}
\begin{tikzpicture}[scale=.8]
\coordinate (O) at (0,0);
\coordinate (H) at (-3,0);
\coordinate (K) at (3,0);
\node at (-2.4,2.4) {$c$};
\draw (H) -- (K);
\draw[name path=circle] (O) circle[radius=3cm];
\fill (O) node[below] {$O$} circle[radius=2pt];
\fill (H) node[left] {$H$} circle[radius=2pt];
\fill (K) node[right] {$K$} circle[radius=2pt];
\path[name path=as] (1,0) coordinate (A) -- ++(0,3.2);
\fill (A) node[below] {$A$} circle[radius=2pt];
\path[name intersections={of=circle and as,by={S}}];
\fill (S) node[above] {$S$} circle[radius=2pt];
\draw (A) -- node[right,yshift=-6pt] {$\sqrt{hk}$} node[right,near end,yshift=-6pt] {$s=$} (S);
\path (H) -- node[above] {$h$} (A);
\path (A) -- node[above] {$k$} (K);
\draw[thick,dashed] (O) -- node[left] {$\frac{d}{2}$} (S);
\node at (.5,-1.5) {$\frac{d}{2}-k$};
\draw[->] (.5, -1.2) -- ++(0,1);
\draw (.8,0) -- ++(0,.2) -- ++(.2,0);
\end{tikzpicture}
\vspace*{-8pt}
\selectlanguage{hebrew}
\end{center}
הטטע הקו
$OS$
הוא רדיוס של המעגל ולכן אורכו שווה ל-%
$\frac{d}{2}$.
לפי הבנייה אורכו של
$OA$
שווה ל-%
$\frac{d}{2}-k$.
לפי משפט פיתגורס:
\[
\renewcommand*{\arraystretch}{1.5}
\begin{array}{rcl}
SA^2 &=& \frac{d}{2}^2 - (\frac{d}{2}-k)^2\\
&=& dk - k^2\\
&=& k(d-k)\\
&=& kh,\quad\quad \textrm{since}\; h+k=d,\\
s&=&SA=\sqrt{hk}\,.
\end{array}
\]
כעת ניתן לבנות
$x=\disfrac{t}{d}s$
לפי סעיף
\L{\ref{s.relative}}.


%%%%%%%%%%%%%%%%%%%%%%%%%%%%%%%%%%%%%%%%%%%%%%%%%%%%%%%%%%%%%%%

\section{%
בניית נקודות חיתוך בין קו למעגל%
}\label{s.line-circle}
\textbf{%
נתון קו
$l$
ומעגל
$c(O,r)$,
ניתן לבנות את נקודות החיתוך בין הקו והמעגל.%
}

לא מדובר על המעגל הקבוע, אלא על מעגל המוגדר על ידי מרכזו וקטע קו שהוא הרדיוס.
\begin{center}
\selectlanguage{english}
\begin{tikzpicture}[scale=.7]
\coordinate (O) at (0,0);
\node at (2.6,-2) {$c$};
\draw[thick,dashed] (O) circle[radius=3cm];
\fill (O) node[below right] {$O$} circle[radius=2pt];
\draw (O) -- node[right] {$r$} ++(-110:3cm) coordinate (R);
\fill (R) circle[radius=2pt] node[above right,xshift=2pt] {$R$};
\draw (O) +(170:4cm) -- node[below, near end,xshift=40pt,yshift=8pt] {$l$} ++(30:4.5cm);
\end{tikzpicture}
\vspace*{-8pt}
\selectlanguage{hebrew}
\end{center}
כפי שתואר בסעיף
\L{\ref{s.perpendicular}}
ניתן לבנות את הנקודה
$M$
שהיא נקודת החיתוך של הקו עם האנח אליו ממרכז המעגל.
$M$
תהיה החוצה של המיתר 
$XY$,
כאשר 
$X,Y$
הן נקודות החיתוך של של הקו עם המעגל. באיור שלהלן
$X$, $Y$, $s$
הם רק סימונים. טרם בנינו את נקודות החיתוך.
\begin{center}
\selectlanguage{english}
\begin{tikzpicture}[scale=.7]
\coordinate (O) at (0,0);
\node at (2.6,-2) {$c$};
\draw[thick,dashed,name path=circle] (O) circle[radius=3cm];
\fill (O) node[below right] {$O$} circle[radius=2pt];
\draw (O) -- node[right] {$r$} ++(-110:3cm) coordinate (R);
\fill (R) node[above right,xshift=2pt] {$R$} circle[radius=2pt];
\draw[name path=l] (O) ++(170:4cm) -- node[below, near end,xshift=40pt,yshift=12pt] {$l$} ++(20:8cm);
\path[name intersections={of=circle and l,by={Y,X}}];
\fill (X) node[above left] {$X$} circle[radius=2pt];
\fill (Y) node[above right] {$Y$} circle[radius=2pt];
\draw[thick,dashed] (O) -- node[below] {$r$} (X);
\path (X) -- ($(X)!.5!(Y)$) coordinate (M);
\fill (M) node[above] {$M$} circle[radius=2pt];
\draw[thick,dashed] (O) -- node[right] {$t$} (M);
\path (X) -- node[above] {$s$} (M);
\path (M) -- node[above] {$s$} (Y);
\draw (O) ++(170:4cm) -- ++(20:3.1cm) -- ++(-70:10pt) -- ++(20:10pt);
\end{tikzpicture}
\vspace*{-6pt}
\selectlanguage{hebrew}
\end{center}
$\triangle OMX$
הוא מעגל ישר זווית. לפי משפט פיתגורס,
$s^2=r^2-t^2=\sqrt{(r+t)(r-t)}$.
קטע קו באורך
$r$
נתון כרדיוס המעגל, ובנינו את הנקודה
$M$
כך שיש לנו את 
$OM$
באורך
$t$.
כפי שמתואר בסעיף
\L{\ref{s.direction}}
ניתן לבנות קטע קו שאורכו 
$t$
מהנקודה 
$O$
בכיוון
$RO$,
וגם קטע קו שאורכו
$t$
מהנקודה
$O$
בכיוון 
$OR$,
כלומר, ניתן לבנות קטעים באורכים
$r+t,r-t$.
כפי שתואר בסעיף
\L{\ref{s.root}}
ניתן לבנות קטע קו באורך
$s=\sqrt{(r+t)(r-t)}$.
שוב לפי סעיף
\L{\ref{s.direction}},
ניתן לבנות קטעי קו באורך 
$s$
על הקו הנתון
$l$
מהנקודה
$M$
בשני הכיוונים. הקצה השני של כל אחד מקטעי הקו הוא נקודת חיתוך של הקו והמעגל.

%%%%%%%%%%%%%%%%%%%%%%%%%%%%%%%%%%%%%%%%%%%%%%%%%%%%%%%%%%%%%%%

\section{%
בניית נקודות חיתוך בין שני מעגלים%
}\label{s.circle-circle}

\textbf{%
נתון שני מעגלים
$c_1(O_1,r),c_2(O_2,r)$,
ניתן לבנות את נקודות החיתוך שלהם
$X,Y$%
}
המעגלים נתונים כנקודות מרכז וקטעי קו שהם הרדיוסים. עם הסרגל ניתן לבנות את קנע הקו
$O_1O_2$
המחבר את המרכזים. נסמן את אורכו ב-%
$t$.
\begin{center}
\selectlanguage{english}
\begin{tikzpicture}[scale=1.1]
\coordinate (O1) at (0,0);
\coordinate (O2) at (2.5,0);
\fill (O1) node[below left] {$O_1$} circle[radius=2pt];
\fill (O2) node[below right] {$O_2$} circle[radius=2pt];
\draw[thick,dashed,name path=circle1] (O1) circle[radius=2cm];
\draw[thick,dashed,name path=circle2] (O2) circle[radius=1.6cm];
\path [name intersections={of=circle1 and circle2,by={X,Y}}];
\draw (O1) -- node[above] {$r_1$} ++(160:2cm);
\draw (O2) -- node[above] {$r_2$} ++(30:1.6cm);
\fill (O1) ++(160:2cm) circle[radius=2pt];
\fill (O2) ++(30:1.6cm) circle[radius=2pt];
\draw (O1) -- (O2);
\node at (-1.7,1.6) {$c_1$};
\node at (3.8,1.4) {$c_2$};
\draw[<->] (0,-1) -- node[fill=white] {$t$} (2.5,-1);
\node at (6,0) {$t=O_1O_2$};
\end{tikzpicture}
\selectlanguage{hebrew}
\end{center}
נסמן ב-%
$A$
את נקודת החיתוך בין
$O_1O_2$
לבין
$XY$,
והאורכים
$q=O_1A,x=XA$.
שימו לב שטרם גנינו את הנקודה
$A$.
\begin{center}
\selectlanguage{english}
\begin{tikzpicture}[scale=1.1]
\coordinate (O1) at (0,0);
\coordinate (O2) at (2.5,0);
\fill (O1) node[below left] {$O_1$} circle[radius=2pt];
\fill (O2) node[below right] {$O_2$} circle[radius=2pt];
\draw[thick,dashed,name path=circle1] (O1) circle[radius=2cm];
\draw[thick,dashed,name path=circle2] (O2) circle[radius=1.6cm];
\path [name intersections={of=circle1 and circle2,by={X,Y}}];
\fill (X) node[above,yshift=4pt] {$X$} circle[radius=2pt];
\fill (Y) node[below,yshift=-4pt] {$Y$} circle[radius=2pt];
\draw[thick,dashed] (O1) -- node[above,xshift=-4pt] {$r_1$} (X);
\draw[thick,dashed] (O2) -- node[above,xshift=4pt] {$r_2$} (X);
\draw[name path=oo] (O1) -- (O2);
\node at (-1.7,1.6) {$c_1$};
\node at (3.8,1.4) {$c_2$};
\draw[name path=xy] (X) -- (Y);
\path[name intersections={of=xy and oo,by={A}}];
\fill (A) node[below left] {$A$} circle[radius=2pt];
\path (O1) -- node[below,xshift=-2pt] {$q$} (A);
\path (X) -- node[left,yshift=-2pt] {$x$} (A);
\draw[<->] (0,-1) -- node[fill=white] {$t$} (2.5,-1);
\node at (6,.5) {$t=O_1O_2$};
\node at (6,0) {$q=O_1A$};
\node at (6,-.5) {$x=XA$};
\end{tikzpicture}
\selectlanguage{hebrew}
\end{center}
אם נצליח לבנות את האורכים
$q,x$,
לפי סעיף
\L{\ref{s.direction}}
נוכל לבנות את 
$A$
באורך
$q$
מהנקודה
$O_1$
לכיוון
$O_1O_2$.
לפי סעיף
\L{\ref{s.perpendicular}}
ניתן לבנות את האנח ל-%
$O_1O_2$
בנקודה
$A$,
ושוב לפי סעיף
\L{\ref{s.direction}}
קטעי קו באורך
$x$
מהנקודה
$A$
בשני הכיוונים לאורך האנח. הקצה השני של כל קטע קו
$X,Y$
הוא נקודת חיתוך של שני המעגלים כי הן נמצאות במרחקים 
$r_1,r_2$
מהמרכזים
$O_1O_2$.

\textbf{%
בניית האורך
$q$:}
נסמן
$d=\sqrt{r_1^2+t^2}$.
$d$
הוא האורך של היתר של משולש ישר זווית, ולפי סעיפים
\L{\ref{s.perpendicular},\ref{s.direction}}
ניתן לבנות אותו מ-%
$r,t$,
האורכים הידועים של שני הצלעות האחרות, לפי חוק הקוסינוסים במשולש
$\triangle O_1O_2X$:
\[
\renewcommand*{\arraystretch}{1.8}
\begin{array}{rcl}
r_2^2 &=& r_1^2 + t^2 - 2r_1t\cos\angle XO_1O_2\\
      &=& r_1^2 + t^2 - 2t(r_1\cos\angle XO_1O_2)\\
&=& r_1^2 + t^2 - 2tq\\
2tq &=& (r_1^2+t^2) - r_2^2\\
q&=&\disfrac{(d+r_2)(d-r_2)}{2t}\,.
\end{array}
\]
נסמן
$n= d+ r_2$, $m= 2t$, $s =d -r_2$.
ניתן לבנות את כל האורכים האלה לפי סעיף
\L{\ref{s.direction}}.
לבסוף, ניתן לבנות את
$q=\frac{n}{m}s$
לפי סעיף
\L{\ref{s.relative}}.

שימו לב שהמשולש לא מופיע האיור. ניתן לבנות אותו בכל מקום במישור כדי לבנות את האורכים
$d,q$
מהארוכים הידועים
$r_1,r_2,t$.

\textbf{%
בניית האורך
$x$:}
$\triangle AO_1X$
הוא משולש ישר זווית, ולכן
$x^2=r_1^2-q^2 =\sqrt{(r_1+q)(r_1-q)}$.
לפי
\L{\ref{s.direction}}
ניתן לבנות את
$h =r_1+ q$
ו-%
$k= r_1 - q$,
ולפי
\L{\ref{s.root}}
ניתן לבנות את
$x= \sqrt{hk}$. 

\end{document}


