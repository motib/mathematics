\documentclass[12pt,a4paper]{article}
\usepackage[utf8x]{inputenc}
\usepackage[english,hebrew]{babel}
\usepackage{graphicx}
\usepackage{verbatim}
\usepackage{url}

\textwidth=15.5cm
\textheight=23cm
\topmargin=0pt
\headheight=0pt
\oddsidemargin=2em
\headsep=0pt
\parindent=0pt
\renewcommand{\baselinestretch}{1.1}
\setlength{\parskip}{0.3\baselineskip plus 1pt minus 1pt}

\begin{document}
\thispagestyle{empty}

\selectlanguage{hebrew}

\begin{center}
\textbf{\Huge הבעיה של
\L{Langford}}

\bigskip
\bigskip

\textbf{\Large מוטי בן-ארי}

\bigskip

\textbf{\Large המחלקה להוראת המדעים}

\bigskip

\textbf{\Large מכון ויצמן למדע}

\bigskip

\url{http://www.weizmann.ac.il/sci-tea/benari/}

\bigskip

\end{center}

\selectlanguage{english}

\begin{center}
\copyright{}\  2017 by Moti Ben-Ari.
\end{center}

\begin{footnotesize}
This work is licensed under the Creative Commons Attribution-ShareAlike 3.0 Unported License. To view a copy of this license, visit \url{http://creativecommons.org/licenses/by-sa/3.0/} or send a letter to Creative Commons, 444 Castro Street, Suite 900, Mountain View, California, 94041, USA.
\end{footnotesize}

\bigskip

\begin{center}
\includegraphics[width=.2\textwidth]{../by-sa.png}
\end{center}

\bigskip

\selectlanguage{hebrew}
\newpage

\begin{center}
\textbf{\Large הגדרת הבעיה של
\L{Langford}
}
\end{center}

המתמטיקאי הסקוטי
\L{C. Dudley Langford}
שם לב שבנו סידר קוביות צבעוניות לפי הסדר:
\begin{center}
\selectlanguage{english}
\includegraphics[width=\textwidth]{blocks.png}
\end{center}
%\begin{center}
%\fbox{\rule[-5pt]{0pt}{16pt}\R{כחול}\ }
%\fbox{\rule[-5pt]{0pt}{16pt}\R{ירוק}\ }
%\fbox{\rule[-5pt]{0pt}{16pt}\ \R{אדום}}
%\fbox{\rule[-5pt]{0pt}{16pt}\R{כחול}\ }
%\fbox{\rule[-5pt]{0pt}{16pt}\ \R{אדום}}
%\fbox{\rule[-5pt]{0pt}{16pt}\R{ירוק}\ }
%\end{center}
קוביה אחת נמצאת בין שתי הקוביות האדומות, שתי קוביות בין הקוביות הכחולות, ושלוש קוביות בין הקוביות הירוקות. ניתן לנסח את הבעיה כך:

\begin{quote}
נתון שק של מספרים
$\{1,1,2,2,3,3\}$,
האם אפשר לסדר אותם בסדרה כך שלכל
$1\leq i \leq 3$,
$i$
מספרים נמצאים בין שני המופעים של
$i$?\footnote{
שק הוא כמו קבוצה רק שאיבר יכול להופיע מספר פעמים.}
\end{quote}
מהסידור של הקוביות הצבעונית, אנו מקבלים את הפתרון
$312132$.

הבעיה הכללית היא:
\begin{quote}
\textbf{
הבעיה של
\L{Langford} $L(n)$}:
נתון שק של מספרים
$\{1,1,2,2,3,3,\ldots,n,n\}$,
האם ניתן לרשום אותם בסדרה כך שלכל
$1\leq i \leq n$, $i$
מספרים נמצאים בין שני המופעים של
$i$?
\end{quote}

\begin{center}
\textbf{\Large
הבעיה של
\L{Langford}
כבעיית כיסוי}
\end{center}

ניתן להציג את הבעיה של
\L{Langford}
באמצעות מערך. עבור
$L(3)$,
יש
$6$
עמודות, אחת לכל 
$2\cdot 3$
המספרים. השורות מוגדרות לפי הגדרת הבעיה: בין שני המופעים של 
$k$
קיימים
$k$
מספרים. קל לראות שיש ארבעה מקומות אפשריים עבור
$1$,
שלושה עבור
$2$
ושניים עבור
$3$:

\selectlanguage{english}
\begin{center}
\begin{tabular}{|c||c|c|c|c|c|c|}
\hline
&1&2&3&4&5&6\\\hline\hline
1&1&&1&&&\\\hline
2&&1&&1&&\\\hline
3&&&1&&1&\\\hline
4&&&&1&&1\\\hline
5&2&&&2&&\\\hline
6&&2&&&2&\\\hline
7&&&2&&&2\\\hline
8&3&&&&3&\\\hline
9&&3&&&&3\\\hline
\end{tabular}
\end{center}
\selectlanguage{hebrew}

כדי לפתור את הבעיה, עלינו לבחור שורה
\textbf{אחת}
עבור המופעים של
$1$
בסדרה, שורה
\textbf{אחת}
עבור המופעים של
$2$
ושורה
\textbf{אחת}
עבור המופעים של
$3$,
כך שאם הנמקם את השורות אחת מעל לשניה, בכל עמודה יש רק מספר אחד:

\selectlanguage{english}
\begin{center}
\begin{tabular}{|c||c|c|c|c|c|c|}
\hline
&1&2&3&4&5&6\\\hline\hline
2&&1&&1&&\\\hline
7&&&2&&&2\\\hline
8&3&&&&3&\\\hline
\end{tabular}
\end{center}
\selectlanguage{hebrew}

תחילה, נשים לב ששורה
$9$
אינה נחוצה בגלל סימטריה: סדרה המתחילה עם השורה
$9$
זהה לסדרה מתקבלת מהפיכת הסדר של סדרה המתקבלת כאשר מתחילים עם שורה
$8$.

שורה 
$8$
היא היחידה המכילה את המספר
$3$
כך שחובה לבחור אותה, והסדרה המתקבלת היא
\L{3\textvisiblespace \textvisiblespace \textvisiblespace 3\textvisiblespace}. 
אי אפשר להשתמש בכל שורה שיש לה מספרים בעמודות
$1$
ו-
$5$,
כי מותר רק מספר אחד בכל מקום בסדרה. נסמן את השורות שניתן לבחור ושלא ניתן לבחור כך:
$\not 1,2,\not 3,4,\not 5, \not 6, 7, 8$.

שורה
$7$
היא השורה היחידה שנשארת עבור המספר
$2$
כך שחובה לבחור בה והתוצאה היא
\L{3\textvisiblespace 2\textvisiblespace 3{}2}.
אם נמחק את השורות שלא ניתן לבחור אותן נקבל:
$\not 1,2,\not 3,\not 4,\not 5, \not 6, 7, 8$.

כעת, ניתן לבחור רק שורה
$2$
ומתקבל הפתרון
$3{}1{}2{}1{}3{}2$.
למעשה הוכחנו שזה הפתרון היחיד.

\begin{center}
\textbf{\Large 
הבעיה של
\L{Langford}
$L(4)$
}
\end{center}

הנה המערך עבור
$L(4)$:
\selectlanguage{english}
\begin{center}
\begin{tabular}{|c||c|c|c|c|c|c|c|c|}
\hline
&1&2&3&4&5&6&7&8\\\hline\hline
1&1&&1&&&&&\\\hline
2&&1&&1&&&&\\\hline
3&&&1&&1&&&\\\hline
4&&&&1&&1&&\\\hline
5&&&&&1&&1&\\\hline
6&&&&&&1&&1\\\hline
7&2&&&2&&&&\\\hline
8&&2&&&2&&&\\\hline
9&&&2&&&2&&\\\hline
10&&&&2&&&2&\\\hline
11&&&&&2&&&2\\\hline
12&3&&&&3&&&\\\hline
13&&3&&&&3&&\\\hline
14&&&3&&&&3&\\\hline
15&&&&3&&&&3\\\hline
16&4&&&&&4&&\\\hline
17&&4&&&&&4&\\\hline
18&&&4&&&&&4\\\hline
\end{tabular}
\end{center}
\selectlanguage{hebrew}

הקורא מוזמן להוכיח שהפרתון היחיד הוא
$41312432$.

\newpage

\begin{center}
\textbf{\Large
עבור איזה ערכים של
$n$
ניתן לפתור את הבעיה של
\L{Langford}?}
\end{center}

\textbf{משפט}
ניתן לפתור את
$L(n)$
אם ורק אם
$n=4k$
או
$n=4k-1$.

נביא שתי הוכחות מבוססות על
\L{Miller (2014)}
שאם יש פתרון אזי $n=4k$
או
$n=4k-1$.
ניתן למצוא את ההוכחה של הטענה ההפוכה ב-
\L{Davies (1959)}.

\bigskip

\textbf{הוכחה
\L{1}}



אם המופע הראשון של המספר
$k$
נמצא במקום
$i_k$,
המופע השני נמצא במקום
$i_k+k+1$.
לכן, סכום המקומות של כל המספרים הוא:
\[
S_n=\sum_{k=1}^{n}i_k+\sum_{k=1}^{n}(i_k+k+1)\,.
\]
אבל
$S_n$,
סכום כל המקומות, הוא פשוט
$1+2+3+\cdots+2n$
ו:
\[
S_n=\sum_{k=1}^{2n}k = \frac{2n(2n+1)}{2}\,,
\]
לפי הנוסחה לסכום של סדרה חשבונית. נפשט:
\[
S_n=\sum_{k=1}^{n}i_k+\sum_{k=1}^{n}(i_k+k+1) = 2\sum_{k=1}^{n}i_k+\sum_{k=1}^{n}(k+1) = 2\sum_{k=1}^{n}i_k+\frac{n(n+3)}{2}\,,
\]
נשווה את שני הביטויים עבור
$S_n$:
\[
2\sum_{k=1}^{n}i_k+\frac{n(n+3)}{2} = \frac{2n(2n+1)}{2}\,,
\]
ונקבל:
\[
\sum_{k=1}^{n}i_k = \frac{1}{2}\left(\frac{2n(2n+1)}{2} - \frac{n(n+3)}{2}\right) = \frac{3n^2-n}{4}\,.
\]
הצד השמאלי חייב להיות מספר שלם כי הוא סכום של מספרים שלמים )המקומות בסדרה(. לכן, הצד הימני חייב גם הוא להיות מספר שלם. מתי 
$3n^3-n$
מתחלק ב-%
$4$?
נפרק לגורמים ונקבל
$3n^2-n=n(3n-1)$,
כך שאם 
$n$
מתחלק ב-%
$4$,
המכפלה גם היא מתחלקת ב-%
$4$.

מתי 
$3n-1$
מתחלק ב-%
$4$?
ניתן להציג כל מספר
$n$
כ-%
$n=4i+j$
עבור
$j=0,1,2,3$.
אם 
$3n-1$
מתחלק ב-%
$4$,
גם
$3(4i+j)-1 = 12i+3j-1$
מתחלק ב-%
$4$.
ברור ש-%
$12i$
מתחלק ב-%
$4$,
וקל לראות ש-%
$3j-1$
מתחלק ב-%
$4$
)עבור 
$j=0,1,2,3$%
( רק אם
$j=3$, 
כלומר,
$n=4i+3=4(i+1)-1$.



\newpage

\textbf{הוכחה
\L{2}}


נעיין בפתרון עבור
$n=4$:

\selectlanguage{english}
\begin{center}
\begin{tabular}{cccccccc}
1&2&3&4&5&6&7&8\\
4&1&3&1&2&4&3&2
\end{tabular}
\end{center}
\selectlanguage{hebrew}

המקומות של המופעים של
$4$
הם
$1$
ו-
$6$,
והמקומות של המופעים של
$2$
הם
$5$
ו-
$8$.
מקום אחד זוגי והשני אי-זוגי. במקרה הכללי, אם
$i$
הוא המקום של המופע הראשון של מספר
\textbf{זוגי}
$k$,
אזי המקום של המופע השני הוא
$i+k+1$.
בגלל ש-
$k$
זוגי,
$k+1$
אי-זוגי. חיבור של אי-זוגי ואי-זוגי נותן זוגי, וחיבור של זוגי ואי-זוגי נותן אי-זוגי. לכן, אחד מ-
$i$, $i+k+1$
יהיה זוגי והשני אי-זוגי.


המקומות של המופעים של
$1$
הם
$2$
ו-
$4$,
והמקומות של המופעים של
$3$
הם
$3$
ו-
$7$.
במקרה הכללי, אם
$k$
הוא מספר
\textbf{אי-זוגי},
$k+1$
הוא מספר זוגי, כך שאם
$i$
זוגי, גם
$i+k+1$
הוא זוגי, ואם
$i$
אי-זוגי, גם
$i+k+1$
הוא אי-זוגי.

ברור שרשימת המקומות של המספרים בסדרה,
$1,2,\ldots,2n-1,2n$,
מכילה מספר שווה של מקומות זוגיים ומקומות אי-זוגיים. כאשר מציבים את שני המופעים של מספר בסדרה, הם "תופסים" שני מקומות. כאשר מסיימים להציב את כל המספרים בפתרון, חייבים להיות מספר שווה של מקומות זוגיים ואי-זוגיים ש-"תפסו" אותם. נשתמש במונח
\textbf{זוגיות}
עבור ההפרש בין מספר המקומות הזוגיים שנתפסו לבין מספר המקומות האי-זוגיים שנתפסו. תחילה, הזוגיות הוא אפס, ואם יש פתרון, לסדרה השלמה יש גם זוגיות אפס.

כאשר ממקמים את שני המופעים של מספר זוגי, הם תופסים מקום אחד זוגי ומקום אחר אי-זוגי, כך שאין שינוי בזוגיות. כאשר ממקמים את שני המופעים של מספר אי-זוגי, הזוגיות משתנה ב-
$+2$
או
$-2$,
כך שחייב להיות שני מופעים של מספר אי-זוגי
\textbf{אחר}
המקזז את השינוי בזוגיות שגרם המופע הראשון. כלומר,
\textbf{חייב להיות מספר זוגי של מספרים אי-זוגיים ב}-%
$\{1,\ldots,n\}$.

המשפט טוען שאם יש פתרון אזי
$n=4k$
או
$n=4k-1$,
ואם אין פתרון אזי
$n=4k-2$
או
$4k-3$.
נוכיח באינדוקציה. עבור טענת הבסיס,
$k=1$,
ברור שלקבוצות
$\{1\}$
ו-
$\{1,2\}$
אין פתרון, ובנוסף, יש מספר אי-זוגי של מספרים אי-זוגיים. לקבוצות
$\{1,2,3\}$
ו-
$\{1,2,3,4\}$
הראינו שיש להן פתרונות, ובנוסף, יש מספר זוגי של מספרים אי-זוגיים.

הטענה האינדוקטיבית היא שיש פתרונות עבור 
$\{1,\ldots,4k-j\}$, $k\ge 1$, $j=0,1$
)ויש להן מספר זוגי של מספרים אי-זוגיים(, ואין פתרונות עבור
$j=2,3$
)ויש להן מספר אי-זוגי של מספרים אי-זוגיים(. בגלל ש-
$4k+1$
אי-זוגי, אם נוסיף אותו לקבוצה, נקבל מספר אי-זוגי של מספרים אי-זוגיים כך אין פתרון. באופן דומה, אם נוסיף
$4k+1,4k+2$,
לא יהיה פתרון כי 
$4k+1$
אי-זוגי ו-
$4k+2$
זוגי. אם נוסיף
$4k+1,4k+2,4k+3$
או
$4k+1,4k+2,4k+3,4k+4$,
נוסיף שני מספרים אי-זוגיים, כך שנישאר עם מספר זוגי של מספרים אי-זוגיים ויכול להיות פתרון. הוכחנו שהטענה נכונה עבור
$4(k+1)-j$,
ולפי אינדוקציה היא נכונה עבור כל
$n\ge 1$.

\newpage

\selectlanguage{hebrew}

\begin{center}
\L{\textbf{\Large SAT solving}}
\end{center}

בתחשיב הפסוקים, השמה של "אמת" ו-"שקר" לפסוקים האטומיים של נוסחה
$A$
מספקת את
$A$
)$A$ \L{satisfies}(
אם מתקבל "אמת" מחישוב הערך של
$A$.
\L{SAT solver}
היא תכנה הבודקת אם נוסחה בצורת
\L{CNF}
היא
\L{satisifable}
או
\L{unsatisfiable}.
)למבוא על
\L{SAT solving}
ראו
\L{Ben-Ari (2012)}
פרק 6.(
\L{Knuth (2015)}
מראה איך ניתן למצוא פתרונות לבעיית
\L{Langford}
באמצעות
\L{SAT solver},
על ידי קידוד המערך כנוסחת
\L{CNF}.

נגדיר שהמשתנה
$x_i$
הוא "אמת" אם בוחרים בשורה
$i$.
עבור
$L(3)$,
הפסקאות שלהלן מקדדות שניתן לבחור רק אחד מהשורת 
$1$--$4$
המכילות את המספר
$1$:

\selectlanguage{english}
\begin{ttfamily}
  [x1,x2,x3,x4],\par
  [\verb+~+x1,\verb+~+x2], [\verb+~+x1,\verb+~+x3], [\verb+~+x1,\verb+~+x4], [\verb+~+x2,\verb+~+x3], [\verb+~+x2,\verb+~+x4], [\verb+~+x3,\verb+~+x4]
\end{ttfamily}
\selectlanguage{hebrew}

הפסקה הראשונה מקדדת את העובדה שיש לבחור לפחות אחת מהשורות. הפסקה הבאה מקדדת שאם בוחרים שורה
$1$,
$x_1=T$,
לא ניתן לבחור בשורה
$2$,
$x_2=F$.
וכנ"ל עבור שאר הזוגות של השורות. פסקאות נוספות מקדדות שניתן לבחור בדיוק אחת מהשורות
$5,6,7$,
וכן חובה לבחור את שורה
$8$.

נחוצים גם פסקאות המביעות את האילוצים על העמודות. למשל, עמודה
$1$
קובעת שניתן לבחור בדיוק אחד מתוך השורות
$1,5,8$:

\selectlanguage{english}
\begin{ttfamily}
  [x1,x5,x8], [\verb+~+x1,\verb+~+x5], [\verb+~+x1,\verb+~+x8], [\verb+~+x5,\verb+~+x8]
\end{ttfamily}
\selectlanguage{hebrew}

הרצת
\L{SAT solver}
מחזיר את הפתרון שיש לבחור את השורות
$2,7,8$:

\selectlanguage{english}
\begin{ttfamily}
Satisfying assignments: [x1=0,x2=1,x3=0,x4=0,x5=0,x6=0,x7=1,x8=1]
\end{ttfamily}
\selectlanguage{hebrew}

\bigskip


נוסחאות ה-
\L{CNF}
עבור
$L(3)$, $L(4)$
ניתן למצוא בארכיון של
\L{LearnSAT}, \L{SAT solver}
שפיתחתי לצורך הוראה. ניתן להוריד מהאתר
\L{\url{http://www.weizmann.ac.il/sci-tea/benari/}}
דרך הקישוריות
\L{Software and Learning Materials}
ואח"כ
\L{LearnSAT}.


\bigskip

\textbf{\Large
 מקורות
}

\selectlanguage{english}
\begin{small}
M. Ben-Ari. \textit{Mathematical Logic for Computer Science (Third Edition)}, Springer, 2012.

R.O. Davies. On Langford's problem (II), \textit{Mathematical Gazette}, 43, 1959, 253-5.

D.E. Knuth. \textit{The Art of Computer Programming, Volume 4, Fascicle 6: Satisfiability}, Pearson, 2015.

J.E. Miller. \textit{Langford's Problem, Remixed}, \url{http://dialectrix.com/langford.html}, 2014.

\end{small}

\end{document}

\begin{comment}

%%%%%%%%%%%% Solution for L(4) %%%%%%%%%%%%%%%%%%

By symmetry, row 18 may be eliminated.

$1,2,3,4,5,6,7,8,9,10,11,12,13,14,15,16,17$

Choose row 16 and the result is 4\textvisiblespace\textvisiblespace\textvisiblespace\textvisiblespace 4 \textvisiblespace\textvisiblespace.

$\not 1,2,3,\not 4,5,\not 6,\not 7,8,\not 9,10,11,\not 12,\not 13,14,15,16,\not 17$

Choose row 14 and the result is 4\textvisiblespace 3\textvisiblespace\textvisiblespace 4{}3\textvisiblespace.

$\not 1,2,\not 3,\not 4,\not 5,\not 6,\not 7,8,\not 9,\not 10,11,\not 12,\not 13,14, \not 15,16,\not 17$

Choose 8 and the result is 4{}2{}3\textvisiblespace 2{}4{}3\textvisiblespace.

$\not 1,\not 2,\not 3,\not 4,\not 5,\not 6,\not 7,8,\not 9,\not 10,\not 11,\not 12,\not 13,14, \not 15,16,\not 17$

All of the choices for 1 have been eliminated.

Instead of 8 choose 11 and the result is 4\textvisiblespace 3\textvisiblespace 2{}4{}3{}2.

$\not 1,2,\not 3,\not 4,\not 5,\not 6,\not 7,\not 8,\not 9,\not 10,11,\not 12,\not 13,14, \not 15,16,\not 17$

Choose 2 and we have a solution 4{}1{}3{}1{}2{}4{}3{}2.

\bigskip


Instead of 14 choose 15 and the result is 4\textvisiblespace \textvisiblespace 3\textvisiblespace 4\textvisiblespace 3.

$\not 1,\not 2,3,\not 4,5,\not 6,\not 7,8,\not 9,\not 10,\not 11,\not 12,\not 13,\not 14,15,16,\not 17$

8 must be chosen and the result is 4{}2\textvisiblespace 3{}2{}4\textvisiblespace 3.

$\not 1,\not 2,\not 3,\not 4,\not 5,\not 6,\not 7,8,\not 9,\not 10,\not 11,\not 12,\not 13,\not 14,15,16,\not 17$

All of the choices for 1 have been eliminated.

\bigskip


Instead of 16 choose 17 and the result is \textvisiblespace 4\textvisiblespace \textvisiblespace \textvisiblespace\textvisiblespace 4\textvisiblespace.

$1,\not 2,3,4,\not 5,6,7,\not 8,9,\not 10,11,12,\not 13,\not 14,15,\not 16,17$

Choose 15 and the result is \textvisiblespace 4\textvisiblespace 3\textvisiblespace\textvisiblespace 4{}3.

$1,\not 2,3,\not 4,\not 5,\not 6,\not 7,\not 8,9,\not 10,\not 11,\not 12,\not 13,\not 14,15,\not 16,17$

Choose 9 and the result is \textvisiblespace 4{}2{}3\textvisiblespace 2{}4{}3.

$1,\not 2,\not 3,\not 4,\not 5,\not 6,\not 7,\not 8,9,\not 10,\not 11,\not 12,\not 13,\not 14,15,\not 16,17$

The only choice for the 1's is 1 and that is impossible.

\bigskip


$1,\not 2,3,4,\not 5,6,7,\not 8,9,\not 10,11,12,\not 13,\not 14,15,\not 16,17$

Instead of 15 choose 12 and the result is 3{}4\textvisiblespace \textvisiblespace 3\textvisiblespace 4.

$\not 1,\not 2,\not 3,\not 4,\not 5,\not 6,\not 7,\not 8,9,\not 10,\not 11,12,\not 13,\not 14,\not 15,\not 16,17$

All of the choices for 1 have been eliminated.
\end{comment}
