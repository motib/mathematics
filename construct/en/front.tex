% !TeX root = construct.tex

\thispagestyle{empty}

\begin{center}
\textbf{\LARGE Surprising Constructions with\\\bigskip Straightedge and Compass}

\bigskip
\bigskip
\bigskip
\bigskip

\textbf{\Large Moti Ben-Ari}

\bigskip
\bigskip

\url{http://www.weizmann.ac.il/sci-tea/benari/}
\end{center}

% !TeX root = construct.tex

\begin{center}
\begin{bfseries}
\bigskip
\bigskip
גרסה
$2.1$

\bigskip

\today

\end{bfseries}
\end{center}


\vfill

\begin{footnotesize}
\begin{center}
\copyright{}\ 2019 by Moti Ben-Ari.
\end{center}

This work is licensed under the Creative Commons Attribution-ShareAlike 3.0 Unported License. To view a copy of this license, visit \url{http://creativecommons.org/licenses/by-sa/3.0/} or send a letter to Creative Commons, 444 Castro Street, Suite 900, Mountain View, California, 94041, USA.
\end{footnotesize}

\bigskip

\begin{center}
\includegraphics[width=.15\textwidth]{../../by-sa.png}
\end{center}

\newpage
\thispagestyle{empty}
\mbox{}
\newpage
\thispagestyle{empty}

\tableofcontents
\newpage
\mbox{}
\newpage

\chapter*{Introduction}
\addcontentsline{toc}{chapter}{Introduction}

I don't remember when I first saw the article by Godfried Toussaint \cite{toussaint} on the ``collapsing compass,'' but it make a deep impression on me. It never occurred to me that the modern compass is not the one that Euclid wrote about. In this document, I present the collapsing compass and other surprising geometric constructions. The mathematics used is no more advanced than secondary-school mathematics, but some of the proofs are rather intricate and demand a willingness to deal with complex constructions and long proofs. The chapters are ordered in ascending levels of difficult (according to my evaluation).

\begin{description}
\item[The collapsing compass] Euclid showed that every construction that can be done using a compass with fixed legs can be done using a collapsing compass, which is a compass that cannot maintain the distance between its legs. The presentation is not difficult and uses only the geometry of circles and triangles. Over many years numerous incorrect proofs have been given based on incorrect diagrams. In order to emphasize that one should not trust diagrams, I have included the famous ``proof'' that every triangle is isoceles.

\item[Trisecting an angle] The Greeks sought a construction to trisect an angle into three equal parts. Only in the nineteenth century was this proved to be impossible. In fact, this result is of no practical importance because an angle can be trisected using tools only slightly more advanced than a straightedge and compass. The chapter presents three such constructions, two of which are very easy, while the third requires some knowledge of trigonometry and limits.

\item[Squaring a circle] Another problem posed by the Greeks was that of squaring a circle: given a circle, construct a square with the same area. This is equivalent to constructing a line segment of length $\pi$, which has been proved to be impossible. This chapter presents a construction by Ramanujan of the value $\disfrac{355}{113}$ that is very, very close to the precise value of $\pi$. The presentation is in stages with exercises given at each stage.

\item[Construction with only a compass] Who says that both a straightedge and a compass are needed? Hundreds of years ago, Lorenzo Mascheroni and Georg Mohr showed that it is possible to limit oneself to only a compass. The proof is not very difficult, but it is very long and requires patience to follow.

\item[Construction with only a straightedge] Is a straightedge sufficient? The answer is no because a straightedge can ``compute'' only linear functions, whereas a compass can ``compute'' quadratic functions. In 1833 Jacob Steiner proved that a straightedge if sufficient provided that somewhere in the plane a single circle exists. The proof uses only geometry but is very long.

\item[Triangles with the same area and perimeter] This chapter deals with a geometric problem that is not a construction, but it is fascinating. Do there exists two non-congruent triangles that have the same perimeter and the same area? The answer is yes but finding such pairs requires a long journey into trigonometry. I have added to this chapter an elegant proof of Heron's formula for the area of a triangle.

\end{description}
