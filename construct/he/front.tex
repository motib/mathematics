% !TeX root = construct.tex

\selectlanguage{hebrew}

\thispagestyle{empty}

\begin{center}
\textbf{\LARGE בניות מפתיעות עם סרגל ומחוגה}

\bigskip
\bigskip
\bigskip
\bigskip

\textbf{\Large מוטי בן-ארי}

\bigskip
\bigskip

\url{http://www.weizmann.ac.il/sci-tea/benari/}
\end{center}

% !TeX root = construct.tex

\begin{center}
\begin{bfseries}
\bigskip
\bigskip
גרסה
$2.1$

\bigskip

\today

\end{bfseries}
\end{center}


\vfill

\selectlanguage{english}

\begin{footnotesize}
\begin{center}
\copyright{}\ 2019 by Moti Ben-Ari.
\end{center}

This work is licensed under the Creative Commons Attribution-ShareAlike 3.0 Unported License. To view a copy of this license, visit \url{http://creativecommons.org/licenses/by-sa/3.0/} or send a letter to Creative Commons, 444 Castro Street, Suite 900, Mountain View, California, 94041, USA.
\end{footnotesize}

\bigskip

\begin{center}
\includegraphics[width=.15\textwidth]{../../by-sa.png}
\end{center}

\np
\thispagestyle{empty}
\mbox{}
\np
\thispagestyle{empty}

\selectlanguage{hebrew}
\tableofcontents
\np
\mbox{}
\np

\section*{הקדמה}
\addcontentsline{cot}{section}{\textbf{הקדמה}}

אינני זוכר מתי ראיתי את המאמר של 
\L{Godfried Toussaint}
\L{\cite{toussaint}}
על "מחוגה מתמוטטת", אבל הוא עשה עלי רושם חזק. לעולם לא עלה על דעתי שהמחוגה המודרנית איננה אותה מחוגה שאוקלידס התכוון אליה. במסמך זה אני מציג את המחוגה המתמוטטת ונושאים אחרים מפתיעים בבניות גיאומטריות. אין כאן מתמטיקה גבוהה יותר ממה שנלמד בבית הספר התיכון, אבל חלק מהחומר די מורכב ודורש נכונות להתמודד עם בניות מסובכות והוכחות ארוכות. הפרקים מסודרים לפי קושי עולה )לפי ההערכה שלי(.

\begin{description}
\item[המחוגה המתמוטטת]
אוקלידס הראה שעבור כל בנייה עם מחוגה קבועה, קיימת בנייה שקולה עם מחוגה מתמוטטת. ההצגה אינה קשה ומשתמשת רק בגואומטריה של מעגלים ומשולשים. לאורך השנים ניתנו הוכחות שגויות רבות, מבוססות על תרשימים שאינה נכונים בכל מצב. כדי להדגיש שאין לסמוך על תרשימים, הבאתי את "ההוכחה" המפורסמת שכל משולש שווה שוקיים.
\item[חלוקת זווית לשלושה חלקים]
היוונים חיפשו בנייה שתחלק זווית שרירותית לשלושה חלקים שווים. רק במאה ה-%
$19$
הוכח שהבנייה אינה אפשרית. למעשה, אין לבעייה שום משמעות מעשית כי ניתן לחלק זווית לשלושה חלקים עם כלים מעט יותר משוכללים ממחוגה וסרגל. אפילו סרגל עם שני סימנים עליו מספיק. פרק זה מביא שלוש בניות, כאשר שתי הבניות הראשונות פשוטות, והשלישי דורש ידע מסויים גם בטריגונמטריה וגבולות.

\item[ריבוע המעגל]
הבעייה השנייה שהיוונים העלו היא לרבע את המעגל: נתון מעגל, בנה ריבוע עם שטח זהה. הבנייה שקולה לבניית קטע קו באורך 
$\pi$,
וגם בעייה זו הוכחה כבלתי ניתנת לפתרון. פרק זה מביא שלוש בניות של קירובים ל-%
$\pi$,
אחת של
\L{Kochansky}
מ-%
$1685$,
ושתיים של
\L{Ramanujan}
מ-%
$1913$.

\item[בנייה עם מחוגה בלבד]
מי אומר שצריך גם מחוגה וגם סרגל? כבר לפני מאות שנים, הוכיחו
\L{Lorenzo Mascheroni}
ו-%
\L{Georg Mohr}
שניתן להסתפק במחוגה בלבד. אין קושי מיוחד בהוכחה אבל היא ארוכה מאוד ונדרשת מידה רבה של סבלנות ונחישות כי לעקוב אחריה.

\item[בנייה רק עם סרגל]
האם אפשר רק עם סרגל? התשובה היא לא, כי עם סרגל אפשר לחשב רק חישובים לינאריים לעומת מחוגה שמאפשר חישובים ריבועיים. ב-%
$1833$
\L{Jakob Steiner}
הוכיח שאפשר להסתפק בסרגל בלבד בתנאי שקיים אי-שם במישר מעגל אחת. ההוכחה משתמשת רק בגיאומטריה אבל גם היא ארוכה מאוד.

\item[משולשים עם אותו שטח ואותו היק]
פרק זה עוסק בנושא גיאומטרי שאינו בנייה אבל הוא מרתק ביותר. השאלה היא האם שני משולשים עם אותו שטח ואותו היקף חייבים להיות חופפים? התשובה היא כן, אבל מציאת זוגות לא חופפים מחייבת מסע דרך הרבה טריגונומטריה. לפרק הוספתי הוכחה אלגנטית לנוסחה של הרון לשטח של משולש.

\end{description}

\selectlanguage{english}
