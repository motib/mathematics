% !TeX root = construct.tex

\selectlanguage{hebrew}
\chapter[\R{האם משולשים עם אותו שטח ואותו היקף חופפים?}]{האם משולשים עם אותו שטח\\
ואותו היקף חופפים?}
%%%%%%%%%%%%%%%%%%%%%%%%%%%%%%%%%%%%%%%%%%%%%%%%%%%%%%%%%%%%%%%

האם משולשים עם אותו שטח ואותו היקף חופפים? לא בהכרח: לשני המשולשים הלא-חופפים עם הצלעות
$(17,25,28)$
ו-%
$(20,21,27)$
היקף
$70$
ושטח 
$210$.
ברבש
\L{\cite{marita}}
מראה שנתון משולש שווה-צלעות, קיימים משולשים לא חופפים עם אותו היקף ואותו שטח. אולם, ההוכחה שלה לא כוללת בנייה. פרק זה )המבוסס על 
\L{\cite{heron}}(
מראה שנתון משולש עם אורכי צלעות רציונליים, ניתן לבנות משולש לא-חופף עם אורכי צלעות רציונליים, ועם אותו היקף ושטח.

בסוף הפרק הבאתי הוכחה אלגנטית לנוסחה של הרון לשטח של משולש.


\section{%
ממשולשים לעקומות אליפטיות%
}

התרשים שלהלן מציג את 
$O$, 
מרכז המעגל החסום על ידי המשולש
$\triangle ABC$, 
שהוא החיתוך של חוצי הזווית בקודקודים. נוריד גבהים מ-%
$O$
לצלעות. לכל הגבהים אורך
$r$,
הרדיוס של המעגל החסום. הגבהים וחוצי הזוויות מייצרים שלושה משולשים
\textbf{ישר זווית}
חופפים:
\[
\triangle AOB'\cong \triangle AOC',\quad \triangle BOA'\cong \triangle BOC', \quad \triangle COA'\cong \triangle COB'\,.
\]

\vspace{-12ex}

\begin{center}
\selectlanguage{english}
\begin{tikzpicture}[baseline=-6mm,scale=1.8]
% Draw base and path two lines at known angles
\draw (0,0) coordinate (a) node[xshift=-6pt] {$A$} -- (0:6) coordinate (b) node[xshift=6pt] {$B$};
\path[name path=ac] (a) -- +(50:4);
\path[name path=bc] (b) -- +(150:5);
% Get their intersection and draw lines between vertices
\path[name intersections={of=ac and bc,by=c}];
\node[above] at (c) {$C$};
\draw (a) -- (c) -- (b) -- (a);
% Label angles with tick marks
\draw (a) ++(0:5mm) arc (0:50:5mm);
\draw (a) ++(10:4.5mm) -- +(10:1mm);
\draw (a) ++(15:4.5mm) -- +(15:1mm);
\draw (a) ++(35:4.5mm) -- +(35:1mm);
\draw (a) ++(40:4.5mm) -- +(45:1mm);
\draw (b) ++(150:7mm) arc (150:180:7mm);
\draw (b) ++(157.5:6.5mm) -- +(157.5:1mm);
\draw (b) ++(172.5:6.5mm) -- +(172.5:1mm);
\draw (c) ++(230:5mm) arc (230:330:5mm);
\draw (c) ++(250:4.5mm) -- +(250:1mm);
\draw (c) ++(255:4.5mm) -- +(255:1mm);
\draw (c) ++(260:4.5mm) -- +(260:1mm);
\draw (c) ++(300:4.5mm) -- +(300:1mm);
\draw (c) ++(305:4.5mm) -- +(305:1mm);
\draw (c) ++(310:4.5mm) -- +(310:1mm);
% Path bisectors of two lines
\path[name path=bia] (a) -- +(25:3.5);
\path[name path=bib] (b) -- +(165:5);
% Intersection of angle bisectors
\path [name intersections={of=bia and bib,by=center}];
% Draw angle bisectors to center
\draw (a) -- (center);
\draw (c) -- (center);
\draw (b) -- (center);
% Draw radii
\draw (center) -- node[left] {$r$} ($(a)!(center)!(b)$) node[below,yshift=-2pt] {$C'$} coordinate (ap);
\draw (center) -- node[left,yshift=-4pt] {$r$} ($(a)!(center)!(c)$) node[above left] {$B'$} coordinate (bp);
\draw (center) -- node[right] {$r$} ($(b)!(center)!(c)$) node[above right] {$A'$} coordinate (cp);
% Draw dots
\fill (center) circle (1pt) node[right,xshift=4pt,yshift=4pt] {$O$};
\fill (a) circle (1pt);
\fill (b) circle (1pt);
\fill (c) circle (1pt);
\fill (ap) circle (1pt);
\fill (bp) circle (1pt);
\fill (cp) circle (1pt);
% Draw right angle squares
\draw (ap) -- ++(90:4pt) -- ++(0:4pt) -- ++(-90:4pt);
\draw (bp) -- ++(-40:4pt) -- ++(-130:4pt) -- ++(-220:4pt);
\draw (cp) -- ++(-30:4pt) -- ++(-120:4pt) -- ++(-210:4pt);
\end{tikzpicture}
\end{center}


התרשים שלהלן מציג את הצלעות
$a,b,c$
המחולקות לקטעי קו
$u,v,w$,
והזוויות
$\alpha/2,\beta/2,\gamma/2$:

\vspace{-6ex}

\begin{center}
\selectlanguage{english}
\begin{tikzpicture}[baseline=-6mm,scale=1.8]
% Draw base and path two lines at known angles
\draw (0,0) coordinate (a) node[xshift=-8pt] {$A$} -- (0:6) coordinate (b) node[xshift=6pt] {$B$};
\path[name path=ac] (a) -- +(50:4);
\path[name path=bc] (b) -- +(150:5);
% Get their intersection and draw lines between vertices
\path[name intersections={of=ac and bc,by=c}];
\node[above] at (c) {$C$};
\draw (a) -- node[right] {$b$} (c) -- node[below] {$a$} (b) -- node[above] {$c$} (a);
% Path bisectors of two lines
\path[name path=bia] (a) -- +(25:3.5);
\path[name path=bib] (b) -- +(165:5);
% Intersection of angle bisectors
\path [name intersections={of=bia and bib,by=center}];
% Draw angle bisectors to center
\draw (a) -- (center);
\draw (c) -- (center);
\draw (b) -- (center);
% Labels of angles
\node[above,xshift=5pt,yshift=21pt] at (center) {$\gamma/2$};
\node[above left,xshift=-4pt,yshift=21pt] at (center) {$\gamma/2$};
\node[above right,xshift=4pt,yshift=-5pt] at (center) {$\beta/2$};
\node[below right,yshift=-6pt] at (center) {$\beta/2$};
\node[left,xshift=-8pt,yshift=3pt] at (center) {$\alpha/2$};
\node[below left,xshift=2pt,yshift=-6pt] at (center) {$\alpha/2$};
% Draw radii
\draw (center) -- node[near end,left] {$r$} ($(a)!(center)!(b)$) coordinate (cp) node[below,yshift=-2pt] {$C'$};
\draw (center) -- node[left,near end,yshift=-4pt] {$r$} ($(a)!(center)!(c)$) coordinate (bp) node[above left] {$B'$};
\draw (center) -- node[right,near end] {$r$} ($(b)!(center)!(c)$) coordinate (ap) node[above right] {$A'$};
\fill (ap) circle (1pt);
\fill (bp) circle (1pt);
\fill (cp) circle (1pt);
\fill (a) circle (1pt);
\fill (b) circle (1pt);
\fill (c) circle (1pt);
% Draw dot at center
\fill (center) circle (1pt);
% Labels of line segments
\path (a) -- node[below,yshift=-2pt] {$u$} (cp);
\path (a) -- node[left,xshift=-2pt]  {$u$} (bp);
\path (b) -- node[below,yshift=-2pt] {$v$} (cp);
\path (b) -- node[above,xshift=2pt] {$v$} (ap);
\path (c) -- node[above,xshift=-2pt] {$w$} (bp);
\path (c) -- node[above,xshift=2pt] {$w$} (ap);
\end{tikzpicture}
\end{center}

\np

השטח של 
$\triangle ABC$
הוא סכום השטחים של 
$\triangle AOC, \triangle BOC, \triangle AOB$:
\selectlanguage{english}
\begin{equation}
S_{\triangle ABC} = \frac{1}{2}(w+v)r + \frac{1}{2}(v+u)r + \frac{1}{2}(u+w)r = \frac{1}{2}\cdot 2(u+v+w)r = rs\,, \label{eq.area1}
\end{equation}
\selectlanguage{hebrew}
כאשר 
$s$
הוא מחצית ההיקף:
$s=u+v+w$.
נחשב את האורכים של 
$u,v,w$
מהזוויות ו-%
$r$:

\vspace{-6ex}

\erh{14pt}
\selectlanguage{english}
\begin{equationarray}{rcl}
\tan \frac{\alpha}{2} &=& \frac{u}{r}\label{eq.alpha}\\
\tan \frac{\beta}{2} &=& \frac{v}{r}\label{eq.beta}\\
\tan \frac{\gamma}{2} &=& \frac{w}{r}\label{eq.gamma}\,.
\end{equationarray}
\selectlanguage{hebrew}
כעת ניתן לבטא את
$s$
במונחים של טנגנסים:
\[
s = u+v+w = r\tan \frac{\alpha}{2}+r\tan \frac{\beta}{2}+r\tan \frac{\gamma}{2} = r\left(\tan \frac{\alpha}{2}+\tan \frac{\beta}{2}+\tan \frac{\gamma}{2}\right)\,,
\]
ולפי משוואה%
~\L{\ref{eq.area1}}
השטח הוא:
\selectlanguage{english}
\begin{equation}
A = rs = r^2\left(\tan \frac{\alpha}{2}+\tan \frac{\beta}{2}+\tan \frac{\gamma}{2}\right)\,.\label{eq.area2}
\end{equation}
\selectlanguage{hebrew}
מ-%
$A=rs$
אנו יודעים ש-%
$r=A/s$,
ולכן ניתן לבטא את משוואה%
~\L{\ref{eq.area2}}
כ:
\selectlanguage{english}
\begin{equation}
\tan \frac{\alpha}{2}+\tan \frac{\beta}{2}+\tan \frac{\gamma}{2} = \frac{A}{r^2} = \frac{A}{(A/s)^2} = \frac{s^2}{A}\,.\label{eq.area3}
\end{equation}
\selectlanguage{hebrew}
סכום הזוויות
$\alpha,\beta,\gamma$
הוא
$2\pi$,
ולכן:
\selectlanguage{english}
\erh{12pt}
\begin{equationarray}{rcl}
%\gamma &=& 2\pi - (\alpha + \beta)\\
\gamma/2 &=& \pi - (\alpha/2 + \beta/2)\\
\tan\gamma/2 &=& \tan(\pi - (\alpha/2 + \beta/2))\\
&=& -\tan (\alpha/2 + \beta/2)\\
&=& \frac{\tan\alpha/2 + \tan\beta/2}{\tan\alpha/2 \, \tan\beta/2-1}\,.\label{eq.tangent1}
\end{equationarray}
\selectlanguage{hebrew}
הנה הוכחה של הנוסחה לטנגנס של סכום של שתי זוויות:
\selectlanguage{english}
\erh{12pt}
\begin{equationarray}{rcl}
\tan (\theta+\phi) &=& \frac{\sin(\theta+\phi)}{\cos(\theta+\phi)}\\
&=&\frac{\sin\theta\cos\phi+\cos\theta\sin\phi}{\cos\theta\cos\phi-\sin\theta\sin\phi}\label{eq.tan2a}\\
%&=&\frac{\disfrac{\sin\theta}{\cos\theta}+\frac{\sin\phi}{\cos\phi}}{1-\disfrac{\sin\theta\sin\phi}{\cos\theta\cos\phi}}\label{eq.tangent2}\\
&=&\frac{\tan\theta + \tan \phi}{1-\tan\theta\tan\phi}\,,\label{eq.tangent3}
\end{equationarray}
\selectlanguage{hebrew}
כאשר חילקנו את
\L{\ref{eq.tan2a}}
ב-%
$\cos\theta\cos\phi$.

\np

נפשט את הסימון על ידי הגדרת נעלמים עבור הטנגנסים:

\vspace{-4ex}

\erh{12pt}
\selectlanguage{english}
\begin{equationarray*}{rcl}
x&=&\tan \frac{\alpha}{2}\\
y&=&\tan \frac{\beta}{2}\\
z&=&\tan \frac{\gamma}{2}\,.
\end{equationarray*}
\selectlanguage{hebrew}
לפי משוואה%
~\L{\ref{eq.tangent1}}
ניתן להחליף את
$z=\tan\gamma/2$
בביטוי עם 
$x,y$:
\selectlanguage{english}
\begin{equation}
z = \frac{x+y}{xy-1}\,.\label{eq.xy1}
\end{equation}
\selectlanguage{hebrew}
עם סימון זה משוואה
~\L{\ref{eq.area3}}
היא:
\selectlanguage{english}
\begin{equation}
x+y+\frac{x+y}{xy-1}=\frac{s^2}{A}\,.\label{eq.xy2}
\end{equation}
\selectlanguage{hebrew}

\vspace{-2ex}

האם קיימים פתרונות שונים למשוואה%
~\L{\ref{eq.xy2}}?


עבור משולש ישר-הזווית
$(3,4,5)$,
השטח ומחצית ההיקף שווים ל-%
$6$:
\selectlanguage{english}
\begin{equation}
\frac{\left(\frac{1}{2}(3+4+5)\right)^2}{\frac{1}{2}\cdot 3\cdot 4} = \frac{6^2}{6}=6\,.
\end{equation}
\selectlanguage{hebrew}
אם קיים פתרון נוסף למשוואה
\L{\ref{eq.xy2}}
עבור 
$s=6,A=6$,
ניתן לבטא אותה כ:
\selectlanguage{english}
\begin{equation}
x^2y + xy^2 -6xy + 6 = 0\,.\label{eq.elliptic}
\end{equation}
\selectlanguage{hebrew}

\vspace{-4ex}
זו משוואה עבור
\textbf{עקומה אליפטית}.
\L{Andrew Wiles}
השתמש בעקומות אליפטיות בהוכחה של המשפט האחרון של
\L{Fermat}.
משתמשים בעקומות  אליפטיותגם בהצפנה עם מפתח ציבורי.

\vspace{-4ex}

\section{פתרון המשוואה לעקומה האליפטית}

העקומה האפורה בגרף להלן מראה את%
~\L{\ref{eq.elliptic}}.
כל נקודה בעקומה ברביע הראשון היא פתרון, כי אורכי הצלעות חייבים להיות חיוביים. 
$A,B,D$
מתאימות למשולש
$(3,4,5)$
כפי שנראה בהמשך. כדי למצוא פתרונות 
\textbf{רציונליים}
נוספים, נשתמש ב-%
\textbf{שיטת שני סקנסים}
\L{(\textit{method of two secants})}.%
\footnote{\L{McCallum [2]}
כותב שיש מספר אינסופי של פתרונות רציונליים.}

\np

\selectlanguage{english}
\begin{center}
\includegraphics[width=.7\textwidth]{elliptic1}
\end{center}
\selectlanguage{hebrew}


ציירו סקנס דרך הנקודות
$A=(2,3)$
ו-%
$B=(1,2)$.
הוא חותך את העקומה ב-%
$C=(-1.5,-0.5)$.
נקודה זו אינה פתרון כי הקואורדינטות שליליים. אם נצייר סקנס שני מ-%
$C$
ל-%
$D=(3,2)$,
החיתוך שלו עם העקומה ב-%
$E$
כן מהווה פתרון נוסף.%
\footnote{$(1.5,1.2)$
הוא קירוב המוצג על ידי גיאוגברה. בהמשך נחשב את הקואורדינטות המדוייקות של
$E$.}

המשוואה של הקו )האדום( דרך 
$A,B$
היא
$y=x+1$. 
נציב עבור 
$y$
במשוואה%
~\L{\ref{eq.elliptic}}:
\[
x^2(x+1) + x(x+1)^2 -6x(x+1) +6 =0,\,
\]
ונפשט:
\[
2x^3 -3x^2 -5x +6 =0\,.
\]
מהנקודות
$A,B$
אנו יודעים שני שורשים
$x=2,x=1$,
כך שאפשר לפרק את הפולינום מדרגה שלוש כך:
\[
(x-2)(x-1)(ax+b)=0\,,
\]
כאשר רק השורש השלישי לא ידוע. נכפיל את הגורמים ונראה מיד ש-%
$a$,
המקדם של הגורם מדרגה שלוש
$x^3$,
חייב להיות
$2$,
ו-%
$2b$,
הקבוע, חייב להיות
$6$.
לכן, הגורם השלישי הוא
$2x+3$
ומכאן שהשורש השלישי הוא
$x=-\disfrac{3}{2}$.
נחשב
$y=x+1=-\disfrac{1}{2}$.
הקואורדינטות של הנקודה
$C$
הן
$(-\disfrac{3}{2},-\disfrac{1}{2})$.

המשוואה של הסקנס שני דרך
$D,E$
)בכחול( היא:
\selectlanguage{english}
\begin{equation}
y = \frac{5}{9}x + \frac{1}{3}\,.\label{eq.second-secant}
\end{equation}
\selectlanguage{hebrew}
נציב עבור 
$y$
במשוואה 
~\L{\ref{eq.elliptic}}:
\[
x^2\left(\frac{5}{9}x + \frac{1}{3}\right) + x\left(\frac{5}{9}x + \frac{1}{3}\right)^2 -6x\left(\frac{5}{9}x + \frac{1}{3}\right) +6 =0\,,
\]

\np

ונפשט:
\[
\frac{70}{81}x^3 - \frac{71}{27}x^2 - \frac{17}{9}x +6 =0\,.
\]
שוב יש לנו שני שורשים
$x=3,x=-\disfrac{3}{2}$,
וניתן לפרק את הפולינום מדרגה שלוש כ:
\[
(x-3)\left(x+\frac{3}{2}\right)(ax+b)=0\,.
\]
נשווה את המקדם של 
$x^3$
ונשווה את הקובע ונקבל:
\[
\frac{70}{81}x - \frac{4}{3}=0\,,
\]
ולכן:
\[
x=\frac{81}{70}\cdot \frac{4}{3}= \frac{27\cdot 4}{70} = \frac{54}{35}\,.
\]

נחשב את
$y$
ממשוואה
~\L{\ref{eq.second-secant}}
והקואורדינטות של
$E$
הן:
\[
\left(\frac{54}{35}, \frac{25}{21}\right)\,.
\]
לבסוף, נחשב את
$z$
ממשוואה
~\L{\ref{eq.xy1}}:
\[
z=\frac{x+y}{xy-1}=%
\left(\disfrac{54}{35} + \disfrac{25}{21}\right)%
 \, / \,%
\left(\disfrac{54}{35}\cdot\disfrac{25}{21}-1\right)=%
\frac{2009}{615} = \frac{49}{15}\,.
\]

\section{מפתורונות לעקומה האליפטית למשולשים}

מ-%
$x,y,z$, $a,b,c$, 
ניתן לחשב את אורכי הצלעות של המשולש
$\triangle ABC$:
\erh{1pt}
\begin{equationarray*}{rcl}
a&=&w+v = r(z+y)=(z+y)\\
b&=&u+w= r(x+z)=(x+z)\\
c&=&u+v=r(x+y)=(x+y)\,,
\end{equationarray*}
כי
$r=\disfrac{A}{s}=\disfrac{6}{6}=1$.

עבור הפתרון 
$A=(2,3)$
של העקומה האליפטית, ערכו של
$z$
הוא:
\[
z=\frac{x+y}{xy-1}=\frac{2+3}{2\cdot 3-1}=1\,,
\]
והצלעות של המשולש הם:
\erh{1pt}
\begin{equationarray*}{rcl}
a &=& z+y = 1+3 = 4\\
b &=& x+z = 2+1=3\\
c &=& x+y = 2+3=5\,.
\end{equationarray*}

\np

המשולש ישר-זווית עם
$s=A=6$.
חישוב הצלעות המתאימים ל-%
$B$
ן-%
$D$
נותן את אותו משולש.

עבור
$E$:
\erh{12pt}
\begin{equationarray*}{rcl}
a &=& z+y = \frac{49}{15} + \frac{25}{21} = \frac{156}{35}\\
b &=& x+z = \frac{54}{35} + \frac{49}{15} = \frac{101}{21}\\
c &=& x+y = \frac{54}{35} + \frac{25}{21}  = \frac{41}{15}\,,
\end{equationarray*}

נבדוק את התוצאה. מחצית ההיקף היא:
\[
s=\frac{1}{2}\left(\frac{156}{35} + \frac{101}{21}+\frac{41}{15}\right) = \frac{1}{2}\left(\frac{468+505+287}{105}\right) = \frac{1}{2}\left(\frac{1260}{105}\right)= 6\,,
\]
וניתן לחשב את השטח באמצעות הנוסחה של הרון:
\erh{16pt}
\begin{equationarray*}{rcl}
A &=& \sqrt{s(s-a)(s-b)(s-c)}\\
&=& \sqrt{6 \left(6-\frac{156}{35}\right) \left(6-\frac{101}{21}\right) \left(6-\frac{41}{15}\right)}\\
&=& \sqrt{6 \cdot \frac{54}{35}\cdot \frac{25}{21} \cdot \frac{49}{15}}\\
&=& \sqrt{\frac{396900}{11025}}\\
&=& \sqrt{36} = 6\,.
\end{equationarray*}

\vspace{-8ex}

\section{הוכחה של הנוסחה של הרון}

אם
$\phi+\theta+\psi=\pi$,
הנוסחה של סכום שלושת הזוויות היא:
\selectlanguage{english}
\begin{equation}
\tan\phi+\tan\theta+\tan\psi = \tan\phi\tan\theta\tan\psi\,. \label{eq.triple}
\end{equation}
\selectlanguage{hebrew}
ההוכחה היא מיידית ממשוואה
~\L{\ref{eq.tangent3}}:
\erh{12pt}
\begin{equationarray*}{rcl}
\tan\psi &=& \tan (\pi-(\phi+\theta))= -\tan (\phi+\theta)\\
&=& \frac{\tan\phi+\tan\theta}{\tan\phi\tan\theta-1}\\
\tan\phi\tan\theta\tan\psi-\tan\psi&=& \tan\phi+\tan\theta\\
\tan\phi\tan\theta\tan\psi &=&\tan\phi+\tan\theta+\tan\psi\,.
\end{equationarray*}

\np

ממשוואות 
~\L{\ref{eq.alpha}--\ref{eq.area2}}
ו-%
$r=A/s$:
\erh{16pt}
\begin{equationarray*}{rcl}
A &=& r^2\left(\tan \frac{\alpha}{2}+\tan \frac{\beta}{2}+\tan \frac{\gamma}{2}\right)\\
&=&r^2\left(\tan \frac{\alpha}{2}\tan \frac{\beta}{2}\tan \frac{\gamma}{2}\right)\\
&=&r^2\left(\frac{u}{r}\frac{v}{r}\frac{w}{r}\right)\\
&=&\frac{u\,v\,w}{r}\\
&=&\frac{s}{A}\,u\,v\,w\\
A^2&=&s\,u\,v\,w\,.
\end{equationarray*}

\vspace{-4ex}

$s=u+v+2$
ולכן:
\erh{2pt}
\begin{equationarray*}{rcl}
s - a &=& (u+v+w) - (w+v) = u\\
s - b &=& (u+v+w) - (u+w) = v\\
s - c &=& (u+v+w) - (u+v) = w\,,
\end{equationarray*}
ואנו מקבלים את הנוסחה של הרון:
\erh{12pt}
\begin{equationarray*}{rcl}
A^2 &=& s\,u\,v\,w\\
&=& s(s-a)(s-b)(s-c)\\
A &=& \sqrt{s(s-a)(s-b)(s-c)}\,.
\end{equationarray*}

\selectlanguage{english}