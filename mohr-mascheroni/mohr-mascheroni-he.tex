\documentclass[12pt,a4paper]{article}
\usepackage[utf8x]{inputenc}
\usepackage[english,hebrew]{babel}
\usepackage{graphicx}
\usepackage{verbatim}
\usepackage{url}
\usepackage{float}

\usepackage{tikz}
\usetikzlibrary{external,positioning,through,calc,intersections}
\tikzexternalize[prefix=tikz/]

\textwidth=15.5cm
\textheight=23cm
\topmargin=0pt
\headheight=0pt
\oddsidemargin=2em
\headsep=0pt
\renewcommand{\baselinestretch}{1.1}
\setlength{\parskip}{0.3\baselineskip plus 1pt minus 1pt}
\parindent=0pt

\begin{document}
\thispagestyle{empty}

\selectlanguage{hebrew}

\begin{center}
\textbf{\Huge%
אני מסתפק במחוגה%
}

\bigskip
\bigskip
\bigskip
\bigskip

\textbf{\Large מוטי בן-ארי}

\bigskip

\textbf{\Large המחלקה להוראת המדעים}

\bigskip

\textbf{\Large מכון ויצמן למדע}

\bigskip

\selectlanguage{english}
\url{http://www.weizmann.ac.il/sci-tea/benari/}
\end{center}

\bigskip
\bigskip
\bigskip
\bigskip

\begin{center}
\selectlanguage{english}
\copyright{}\  2018 by Moti Ben-Ari.

\end{center}

\selectlanguage{english}

{\small This work is licensed under the Creative Commons Attribution-ShareAlike 3.0 Unported License. To view a copy of this license, visit \url{http://creativecommons.org/licenses/by-sa/3.0/} or send a letter to Creative Commons, 444 Castro Street, Suite 900, Mountain View, California, 94041, USA.}

\bigskip

\begin{center}
\includegraphics[width=.2\textwidth]{../by-sa.png}
\end{center}

\selectlanguage{hebrew}

\newpage

%%%%%%%%%%%%%%%%%%%%%%%%%%%%%%%%%%%%%%%%%%%%%%%%%%%%%%%%%%%%%%%

\section{%
מבוא
}\label{s.intro}
בשנת
$1797$
המתימטיקאי האיטלקי
\L{Lorenzo Mascheroni}
הוכיח שכל בנייה גיאומטרית עם סרגל ומחוגה ניתנת לבנייה עם מחוגה בלבד! במאה העשרים התגלה שהמשפט הוכח בשנת
$1672$
על ידי המתימטיקאי הדני
\L{Georg Mohr}.
לכן, המשפט נקרא היום משפט
\L{Mohr-Mascheroni}.

במסמך זה אביא את הוכחת המשפט המבוססת על הוכחה שמופיעה כבעייה
$33$
בספר:

\selectlanguage{english}
Heinrich D\"{o}rrie: \textit{100 Problems of Elementary Mathematics: Their History and Solution} (Dover, 1965),
\selectlanguage{hebrew}

ועובדה על ידי
\L{Michael Woltermann}.\footnote{\url{http://www2.washjeff.edu/users/mwoltermann/Dorrie/DorrieContents.htm}.
ברצוני להודות לו על הרשות להתשמש בעבודתו.
},%
\footnote{%
הוכחה אחרת )מפורטת פחות( ניתן למצוא ב%
\textit{%
בניות גיאומטריות: בעיות קלאסיות, אתגריות וממוחשבות.
}%
משה סטופל, קלרה זיסקין )עורכים(, הוצאת שאנן,
$2015$.
הוכחה נוספת:
\L{Norbert Hungerb\"{u}hler. A short elementary proof of the Mohr-Mascheroni theorem. \textit{American Mathematical Monthly} 101(8), 1994, 784--787.}%
}


מה המשמעות של בנייה גיאומטרית עם מחוגה בלבד ללא סרגל? האיור הימני מראה את הבנייה של משולש שווה צלעות עם סרגל ומחוגה.
\begin{center}
\selectlanguage{english}
\begin{tikzpicture}[scale=0.6]
\coordinate (A) at (0,0);
\coordinate (B) at (4,0);
\path (A) node[below left] {$A$} -- (B) node[below right] {$B$};
\fill (A) circle[radius=3pt];
\fill (B) circle[radius=3pt];
\draw[name path=larc] (A) ++(-10:4cm) arc (-10:80:4cm);
\draw[name path=rarc] (B) ++(-170:4cm) arc (-170:-260:4cm);
\path [name intersections={of=larc and rarc,by={t}}];
\fill (t) node[above right,xshift=-2pt,yshift=3pt] {$C$} circle[radius=3pt];
\begin{scope}[xshift=10cm]
\coordinate (A) at (0,0);
\coordinate (B) at (4,0);
\draw (A) node[below left] {$A$} -- (B) node[below right] {$B$};
\fill (A) circle[radius=3pt];
\fill (B) circle[radius=3pt];
\draw[name path=larc] (A) ++(-10:4cm) arc (-10:80:4cm);
\draw[name path=rarc] (B) ++(-170:4cm) arc (-170:-260:4cm);
\path [name intersections={of=larc and rarc,by={t}}];
\fill (t) node[above right,xshift=-2pt,yshift=3pt] {$C$} circle[radius=3pt];
\draw (A) -- (t);
\draw (B) -- (t);
\end{scope}
\end{tikzpicture}
\selectlanguage{hebrew}
\end{center}
איך אפשר לבנות משולש ללא קטעי הקווים
$AB,AC,BC$?
למעשה, אין כל צורך
\textbf{לראות}
את הקווים. קו או קטע קו מוגדר על ידי שתי נקודות, ומספיק שנבנה את נקודות כדי לקבל בנייה שקולה בנייה עם סרגל. האיור השמאלי מראה בנייה של משולש שווה צלעות עם מחוגה ללא סרגל.

באיורים במסמך זה נצייר בכל זאת קווים, אולם הקווים משמשים אך ורק להבנת הבנייה ולהוכחת נכונותה. חשוב שתשתכנעו שבבנייה עצמה משתמשים רק במחוגה.

עיון בבנייה גיאומטרית יגלה שכל צעד הוא אחת משלוש פעולות:
\begin{itemize}
\item
מציאת נקודת החיתוך של שני קווים ישרים.
\item
מציאת נקודות החיתוך בין קו ישר ומעגל.
\item
מציאת נקודות החיתוך בין שני מעגלים.
\end{itemize}
ברור שניתן לבצע את הפעולה השלישית רק עם מחוגה. עלינו להראות שעבור שתי הפעולות הראשונות ניתן למוצא בנייה שקולה המשתמשת רק במחוגה.

\newpage

סימונים:
\begin{itemize}
\item $C(O,A)$: 
המעגל שמרכזו
$O$
העובר דרך הנקודה
$A$.
\item $C(O,r)$:
המעגל שמרכזו
$O$
עם רדיוס
$r$.
\item $C(O,AB)$:
המעגל שמרכזו
$O$
עם רדיוס שהוא אורך קטע קו נתון
$AB$.
\end{itemize}

תחילה נביא ארבע בניות עזר נחוצות )סעיפים
\L{\ref{s.reflection}--\ref{s.relative}}%
(,
ואחר כך נראה את הבעיות למציאת חיתוך של שני קווים )סעיף
\L{\ref{s.two-lines}}%
( ושל קו ומעגל )סעיף
\L{\ref{s.line-circle}}%
(.
%%%%%%%%%%%%%%%%%%%%%%%%%%%%%%%%%%%%%%%%%%%%%%%%%%%%%%%%%%%%%%%



\section{%
שיקוף נקודה%
}\label{s.reflection}
\textbf{%
נתון קטע קו
$AB$
ונקודה 
$C$
שלא נמצאת על
$AB$.
ניתן לבנות נקודה 
$C'$
שהיא השיקוף של
$C$
מסביב ל-%
$AB$.
}

\textbf{%
הגדרה:%
}
הנקודה
$C'$
היא
\textbf{%
שיקוף%
}
של הנקודה
$C$
מסביב לקטע קו
$AB$,
אם 
$AB$
)או הקו המכיל אותו( הוא האנך האמצעי של
$CC'$.

נבנה מעגל שמרכזו
$A$
העובר דרך
$C$
ומעגל שמרכזו
$B$
העובר דרך
$C$.
החיתוך של שני המעגלים הוא הנקודה
$C'$
שהיא השיקוף של
$C$.

\begin{center}
\selectlanguage{english}
\vspace*{-6pt}
\begin{tikzpicture}[scale=.65]
\coordinate (A) at (0,0);
\coordinate (B) at (4,0);
\coordinate (C) at (2.5,1.5);
\draw[thick,dashed,name path=ab] ($(B)!2!(A)$) -- ($(A)!2!(B)$);
\fill (A) node[above left] {$A$} circle[radius=2pt];
\fill (B) node[above right] {$B$} circle[radius=2pt];
\fill (C) node[above,yshift=4pt] {$C$} circle[radius=2pt];
\node[draw,circle through=(C),name path=ac] at (A) {};
\node[draw,circle through=(C),name path=bc] at (B) {};
\path [name intersections={of=ac and bc,by={x1,Cp}}];
\fill (Cp) node[below,yshift=-4pt] {$C'$} circle[radius=2pt];
\draw (C) -- (Cp);
\draw[thick,dashed] (A) -- (C);
\draw[thick,dashed] (B) -- (C);
\draw[thick,dashed] (A) -- (Cp);
\draw[thick,dashed] (B) -- (Cp);
\end{tikzpicture}
\selectlanguage{hebrew}
\vspace*{-8pt}
\end{center}
הוכחה:
$\triangle ABC$
ו-%
$\triangle ABC'$
חופפים לפי צלע-צלע-צלע, כי
$AC,AC'$
הם רדיוסים של אותו מעגל כמו גם
$BC,BC'$
ו-%
$AB$
הוא צלע משותף. מכאן ש-%
$\angle CAB = \angle C'AB$,
ולכן
$AB$
הוא חוצה הזווית של
$\angle CAC'$.
אבל
$\triangle CAC'$
הוא משולש שווה שוקיים, וחוצה הזווית
$AB$
הוא גם האנך האמצעי של בסיס המשולש
$CC'$.
לפי ההגדרה,
$C'$
היא השיקוף של
$C$
מסביב ל-%
$AB$.

%%%%%%%%%%%%%%%%%%%%%%%%%%%%%%%%%%%%%%%%%%%%%%%%%%%%%%%%%%%%%%%

\section{%
בניית מעגל עם רדיוס נתון
}\label{s.radius}

\textbf{%
נתונה שלוש נקודות
$A,B,C$.
ניתן לבנות מעגל
$c(A,BC)$
שמרכזו 
$A$
עם רדיוס שווה לאורך קטע הקו 
$BC$.
}

לכאורה, אין כאן שום בעייה. נציב את רגלי המחוגה כך שרגל אחת נמצאת על
$B$
והרגל השניה על
$C$,
ואז נציב את הרגל עם החוד על
$A$,
ולמעגל שיתקבל רדיוס שווה ל-%
$BC$.
הסיבה לבנייה המסובכת להלן היא שהמחוגה של אוקלידס היא מחוגה "מתמוטטת"
\L{(collapsing)},
שלא שומרת על המרחק בין הרגליים כאשר מרימים את המחוגה מהנייר. אוקלידיס הוכיח במשפט הנקרא היום
\L{Compass Equivalence Theorem}
שכל בנייה עם מחוגה קבועה ניתנת לבנייה עם מחוגה מתמוטטת. סעיף זה למעשה מוכיח המשפט כאשר משתמשים במחוגה בלבד. ראו באתר שלי את המאמר "אמאלה, המחוגה שלי התמוטטה!" עבור ההוכחה של אוקלידס.

נבנה את המעגלים 
$c(A,B)$, $c(B,A)$
ונסמן את נקודות החיתוך
$X,Y$.

\begin{center}
\vspace*{-4pt}
\selectlanguage{english}
\begin{tikzpicture}[scale=.55]
\coordinate (A) at (0,1.5);
\coordinate (B) at (0,-1.5);
\coordinate (C) at (1.5,-3);
\coordinate (Cp) at (1.5,3);
\fill (A) node[above] {$A$} circle[radius=3pt];
\fill (B) node[below] {$B$} circle[radius=3pt];
\fill (C) node[below] {$C$} circle[radius=3pt];
%\fill (Cp) node[above] {$C'$} circle[radius=3pt];
\node[draw,circle through=(B),name path=ab] at (A) {};
\node[draw,circle through=(A),name path=ba] at (B) {};
\path [name intersections={of=ab and ba,by={Y,X}}];
\fill (X) node[above right,xshift=4pt] {$X$} circle[radius=3pt];
\fill (Y) node[above left,xshift=-4pt] {$Y$} circle[radius=3pt];
\draw[thick,dashed] ($(X)!2.3!(Y)$) -- ($(Y)!2!(X)$);
%\draw[thick,dashed] (C) -- (Cp);
\end{tikzpicture}
\vspace*{-4pt}
\selectlanguage{hebrew}
\end{center}

נבנה את
$C'$,
השיקוף של
$C$
מסביב לקו
$XY$
לפי הבנייה בסעיף
\L{\ref{s.reflection}}.
\begin{center}
\selectlanguage{english}
\begin{tikzpicture}[scale=.45]
\coordinate (A) at (0,1.5);
\coordinate (B) at (0,-1.5);
\coordinate (C) at (1.5,-3);
\coordinate (Cp) at (1.5,3);
\fill (A) node[right] {$A$} circle[radius=3pt];
\fill (B) node[right] {$B$} circle[radius=3pt];
\fill (C) node[below,yshift=-2pt] {$C$} circle[radius=3pt];
\fill (Cp) node[above,xshift=2pt,yshift=2pt] {$C'$} circle[radius=3pt];
\node[circle through=(B),name path=ab] at (A) {};
\node[circle through=(A),name path=ba] at (B) {};
\path [name intersections={of=ab and ba,by={Y,X}}];
\fill (X) node[above right,xshift=4pt] {$X$} circle[radius=3pt];
\fill (Y) node[above left,xshift=-4pt] {$Y$} circle[radius=3pt];
\node[draw,circle through=(C)] at (X) {};
\node[draw,circle through=(C)] at (Y) {};
\draw[thick,dashed] ($(X)!2.3!(Y)$) -- ($(Y)!2!(X)$);
%\draw (X) -- (Y) -- (C) -- (X) -- (Cp) -- (Y);
\draw[thick,dashed] (C) -- (Cp);
\end{tikzpicture}
\vspace*{-6pt}
\selectlanguage{hebrew}
\end{center}

המעגל
$c(A,C')$
הוא המעגל המבוקש.

\begin{center}
\vspace*{-6pt}
\selectlanguage{english}
\begin{tikzpicture}[scale=.5]
\coordinate (A) at (0,1.5);
\coordinate (B) at (0,-1.5);
\coordinate (C) at (1.5,-3);
\coordinate (Cp) at (1.5,3);
\fill (A) node[above,yshift=2pt] {$A$} circle[radius=3pt];
\fill (B) node[below,yshift=-2pt] {$B$} circle[radius=3pt];
\fill (C) node[below,yshift=-2pt] {$C$} circle[radius=3pt];
\fill (Cp) node[above,yshift=2pt] {$C'$} circle[radius=3pt];
\node[circle through=(B),name path=ab] at (A) {};
\node[circle through=(A),name path=ba] at (B) {};
\path [name intersections={of=ab and ba,by={Y,X}}];
\fill (X) node[above right,xshift=4pt] {$X$} circle[radius=3pt];
\fill (Y) node[above left,xshift=-4pt] {$Y$} circle[radius=3pt];
\node[circle through=(C)] at (X) {};
\node[draw,circle through=(C)] at (Y) {};
\draw[thick,dashed] ($(X)!2.3!(Y)$) -- ($(Y)!2!(X)$);
\path[name path=xy] (X) -- (Y);
\node[draw,thick,circle through=(Cp)] at (A) {};
\draw[very thick] (A) -- (Cp);
\draw[very thick] (B) -- (C);
\draw[very thick,name path=abline] (A) -- (B);
\draw[very thick,name path=ccp] (C) -- (Cp);
\path [name intersections={of=xy and abline,by={D}}];
\path [name intersections={of=xy and ccp,by={E}}];
\fill (D) node[above left] {$D$} circle[radius=3pt];
\fill (E) node[below right] {$E$} circle[radius=3pt];
\draw[thick,dashed] (D) -- (Cp);
\draw[thick,dashed] (D) -- (C);
\end{tikzpicture}
\vspace*{-6pt}
\selectlanguage{hebrew}
\end{center}
הוכחה: הנקודה
$A$
היא השיקוף של 
$B$
סביב 
$XY$
)ניתן להוכיח בהסתמך על זה שהמשולשים
$\triangle YAX, \triangle YBX$
חופפים(, ו-%
$C'$
נבנה כשיקוף של 
$C$
סביב
$XY$.
לפי ההגדרה, 
$XY$
הוא האנך האמצעי לקטעי הקו 
$AB$, $CC'$,
ולכן
$C'E=EC$
)וגם
$AD=DB$(,
ו-%
$\angle DEC=\angle DEC'(=90^\circ)$.
$\triangle DEC$
חופף ל-%
$\triangle DEC'$
לפי צלע-זווית-צלע. מכאן, ש-%
$DC=DC'$
ו-%
$\angle ADC'=\angle BDC$
)כי הן זוויות משלימות ל-%
$\angle EDC, \angle EDC'$%
(.
המשולש
$\triangle ADC'$
חופף ל-%
$\triangle BDC$
לפי צלע-זווית-צלע,
כך ש-%
$AC'=BC$.

הוכחה זו מראה באופן כללי ששיקוף משמר מרחקים.

%%%%%%%%%%%%%%%%%%%%%%%%%%%%%%%%%%%%%%%%%%%%%%%%%%%%%%%%%%%%%%%

\section{%
בניית חיבור וחיסור של שני קטעי קווים%
}\label{s.add-subtract}

\textbf{%
נתון קטע קו
$PQ$
באורך
$a$
וקטע קו
$RS$
באורך
$b$.
ניתן לבנות קטעי קו
$QT,QU$
כך ש-%
$PUQT$
הוא קטע קו, כאשר האורך של
$PU$
הוא
$a-b$
והאורך של
$PT$
הוא
$a+b$.
}

\begin{center}
\selectlanguage{english}
\vspace*{-2ex}
\begin{tikzpicture}[scale=.8]
\draw (0,0) -- (5,0);
\fill (0,0) node[above] {$P$} circle[radius=2pt];
\fill (5,0) node[above left] {$Q$} circle[radius=2pt];
\fill (3,0) node[above left] {$U$} circle[radius=2pt];
\fill (7,0) node[above right] {$T$} circle[radius=2pt];
\draw[thick,dashed] (5,0) -- (8,0);
\draw (5,0) circle[radius=2cm];
\draw[thick,dashed] (5,0) -- node[left] {$b$} ++(60:2cm);
\draw (9,-1) node[above] {$R$} -- node[below right] {$b$} ++(20:2cm) node[above] {$S$};
\fill (9,-1) circle[radius=2pt];
\fill (9,-1) ++(20:2cm) circle[radius=2pt];
\draw[<->] (0,-.5) -- node[fill=white] {$a$} (5,-.5);
\draw[<->] (0,-1) -- node[fill=white] {$a-b$} (3,-1);
\draw[<->] (0,-1.5) -- node[fill=white] {$a+b$} (7,-1.5);
\end{tikzpicture}
\selectlanguage{hebrew}
\end{center}
\vspace*{-2ex}
שוב, לו היה לנו סרגל הבנייה היתה פשוטה ביותר: בנה מעגל שמרכזו
$Q$
עם רדיוס
$b$,
והנקודות
$U,T$
הן נקודות החיתוך של המעגל עם
\textbf{%
הקרן%
}
שממשיך את
$PQ$.

נבחר
$H$,
היא נקודה כלשהי על המעגל
$c(Q,b)$,
ונבנה את הנקודה
$H'$,
השיקוף שלה סביב
$PQ$.
$h$
הוא האורך של
$HH'$.
\begin{center}
\selectlanguage{english}
\begin{tikzpicture}[scale=.55]
\coordinate (Q) at (0,0);
\coordinate (P) at (-6.8,0);
\coordinate (B) at (-3,-2);
\draw[thick,dashed] ($(Q)!1.3!(P)$) -- node[above,near start] {$a$} ($(P)!2.3!(Q)$);
\fill (Q) node[above left] {$Q$} circle[radius=2pt];
\fill (P) node[above] {$P$} circle[radius=2pt];
\fill (B) circle[radius=2pt];
\node[draw,circle through=(B),name path=qb] at (Q) {};
\draw[thick,dashed] (Q) -- node[left,xshift=-1pt,yshift=2pt] {$b$} (B);
\path[name path=qh] (Q) -- (-40:5cm);
\path[name path=qhp] (Q) -- (40:5cm);
\path [name intersections={of=qb and qh,by={H}}];
\path [name intersections={of=qb and qhp,by={Hp}}];
\fill[below right] (H) node[right,xshift=2pt] {$H$} circle[radius=2pt];
\fill[above right] (Hp) node[right,xshift=2pt] {$H'$} circle[radius=2pt];
\draw[thick,dashed] (H) -- node[below left,yshift=-2pt] {$h$} (Hp);
\end{tikzpicture}
\selectlanguage{hebrew}
\end{center}
נבנה את המעגלים
$c(Q,h)$, $c(H,b)$.
$K$
היא נקודת החיתוך בין המעגלים,
ן-%
$K'$
היא השיקוף של
$K$
מסביב ל-%
$PQ$.
\begin{center}
\selectlanguage{english}
\vspace*{-4pt}
\begin{tikzpicture}[scale=.5]
\coordinate (Q) at (0,0);
\coordinate (P) at (-6.8,0);
\coordinate (B) at (-3,-2);
\draw[thick,dashed] ($(Q)!1.3!(P)$) -- ($(P)!2.3!(Q)$);
\fill (Q) node[above right] {$Q$} circle[radius=3pt];
\fill (P) node[above] {$P$} circle[radius=3pt];
\fill (B) circle[radius=3pt];
\node[draw,circle through=(B),name path=qb] at (Q) {};
\draw[thick,dashed] (Q) -- node[left,xshift=-1pt,yshift=2pt] {$b$} (B);
\path[name path=qh] (Q) -- (-40:5cm);
\path[name path=qhp] (Q) -- (40:5cm);
\path [name intersections={of=qb and qh,by={Hp}}];
\path [name intersections={of=qb and qhp,by={H}}];
\fill (H) node[right,xshift=2pt] {$H$} circle[radius=3pt];
\fill (Hp) node[right,xshift=2pt] {$H'$} circle[radius=3pt];
\draw (H) -- node[below left,yshift=-3pt] {$h$} (Hp);
\draw[thick,name path=circleqh] (Q) let
  \p1 = ($ (H) - (Hp) $),
  \n2 = {veclen(\x1,\y1)}
in
  circle (\n2)
  (Q) edge [dashed] node[below] {$h$} +(140:\n2) ++(140:\n2) coordinate (q);
\fill (q) circle[radius=3pt];
\draw[thick,name path=circlehb] (H) let
  \p1 = ($ (Q) - (B) $),
  \n2 = {veclen(\x1,\y1)}
in
  circle (\n2)
  (H) edge [dashed] node[below,near end] {$b$} +(50:\n2) ++(50:\n2)  coordinate (h);
\fill (h) circle[radius=3pt];
\path [name intersections={of=circleqh and circlehb,by={K}}];
\fill (K) node[above left] {$K$} circle[radius=3pt];
%\draw[thick] (H) -- (K);
\draw let
  \p1 = ($ (K) - (Q) $)
in
  coordinate (Kp) at (\x1,-\y1);
\fill (Kp) node[below left] {$K'$} circle[radius=3pt];
\draw (K) -- (Kp);
\end{tikzpicture}
\vspace*{-8pt}
\selectlanguage{hebrew}
\end{center}
$PQ$
הוא האנך האמצעי גם ל-%
$HH'$
וגם ל-%
$KK'$,
לכן שני קטעי הקו מקבילים. 
$KH = K'H' = b$
כי
$K$
נמצאת על המגעל שמרכזו 
$H$,
ו-%
$K',H'$
הן שיקופים של
$K,H$.
מכאן שהמרובע 
$KHH'K'$
הוא טרפז שווה שוקיים עם בסיסים
$KK' = 2h$, $HH'=h$.
נסמן ב-%
$d$
את האלכסונים
$K'H=KH'$.

\begin{center}
\selectlanguage{english}
\vspace*{-6pt}
\begin{tikzpicture}[scale=.5]
\coordinate (Q) at (0,0);
\coordinate (P) at (-6.8,0);
\coordinate (B) at (-3,-2);
\draw[thick,dashed] ($(Q)!1.3!(P)$) -- ($(P)!2.3!(Q)$);
\fill (Q) node[above left] {$Q$} circle[radius=3pt];
\fill (P) node[above] {$P$} circle[radius=3pt];
%\fill (B) circle[radius=3pt];
\node[draw,circle through=(B),name path=qb] at (Q) {};
%\draw[thick,dashed] (Q) -- node[left,xshift=-1pt,yshift=2pt] {$b$} (B);
\path[name path=qh] (Q) -- (-40:5cm);
\path[name path=qhp] (Q) -- (40:5cm);
\path [name intersections={of=qb and qh,by={Hp}}];
\path [name intersections={of=qb and qhp,by={H}}];
\fill (H) node[right,xshift=2pt] {$H$} circle[radius=3pt];
\fill (Hp) node[right,xshift=2pt] {$H'$} circle[radius=3pt];
\draw (H) -- node[below right,yshift=-2pt] {$h$} (Hp);
\path[name path=circleqh] (Q) let
  \p1 = ($ (H) - (Hp) $)
in
  circle ({veclen(\x1,\y1)});
\path[name path=circlehb] (H) let
  \p1 = ($ (Q) - (B) $)
in
  circle ({veclen(\x1,\y1)});
\path [name intersections={of=circleqh and circlehb,by={K,k2}}];
\fill (K) node[above left] {$K$} circle[radius=3pt];
\draw (Q) -- node[left] {$h$} (K);
\draw (H) -- node[right,xshift=4pt] {$b$} (K);
\draw let
  \p1 = ($ (K) - (Q) $)
in
  coordinate (Kp) at (\x1,-\y1);
\fill (Kp) node[below left] {$K'$} circle[radius=3pt];
\draw (Q) -- node[left] {$h$} (Kp) -- node[right,xshift=2pt,yshift=-2pt] {$b$} (Hp);
\draw (K) -- node[above right] {$d$} (Hp);
\draw (Kp) -- node[left] {$d$} (H);
\end{tikzpicture}\label{p.ptolemy}
\selectlanguage{hebrew}
\vspace*{-12pt}
\end{center}

אנחנו רוצים להוכיח ש-
$KHH'K'$
הוא
\textbf{%
מרובע ציקלי%
}
שניתן לבחסום במעגל. המשפט שאנחנו צריכים להוכחה הוא: אם  הזוויות הנגדיות של מרובע צמודות, אזי המרובע ציקלי. אם נוכיח שבטרפז שווה שוקיים הזוויות הנגדיות צמודות, נקבל שהטרפז הוא ציקלי. בספרי גיאומטריה ניתן למצוא הוכחה פשוטה לטיעון ההפוך: במרובע ציקלי, הזוויות הנגדיות הן צמודות, אבל קשה למצוא הוכחה של הטיעון עצמו. לכן, אביא כאן את שתי ההוכחות.

\textbf{%
הזוויות הנגדיות של מרובע ציקלי הן צמודות:%
}
ערכה של זווית היקפית הנשענת על קשת הוא מחצית ערכה של הקשת, לכן 
$\angle DAB$
היא מחצית מהקשת
$DCB$
והזווית
$\angle DCB$
היא מחצית מהקשת
$DAB$.
אבל שתי הקשתות נתמכות על כל היקף המעגל, ולכן הסכום שלהן הוא
$360^\circ$.
מכאן,
$\angle DAB + \angle DCB = \frac{1}{2} \cdot 360^\circ =  180^\circ$.
באופן דומה,
$\angle ADC + \angle ABC = 180^\circ$.


\begin{center}
\vspace*{-12pt}
\selectlanguage{english}
\begin{tikzpicture}[scale=.55]
\coordinate (origin) at (0,0);
\coordinate (A) at (1,3);
\node[draw,circle through=(A),name path=circle] at (origin) {};
\fill (A) node[above right] {$A$} circle[radius=3pt];
\path[name path=b] (A) -- (-50:4.5cm);
\path[name path=c] (A) -- (-120:4.5cm);
\path[name path=d] (A) -- (150:4.5cm);
\path [name intersections={of=circle and b,by={b1,B}}];
\fill (B) node[right] {$B$} circle[radius=3pt];
\path [name intersections={of=circle and c,by={c1,C}}];
\fill (C) node[below left] {$C$} circle[radius=3pt];
\path [name intersections={of=circle and d,by={d1,D}}];
\fill (D) node[above left] {$D$} circle[radius=3pt];
\draw (A) -- (B) -- (C) -- (D) -- cycle;
\end{tikzpicture}
\selectlanguage{hebrew}
\vspace*{-12pt}
\end{center}


\textbf{%
מרובע שהזוויות הנגדיות שלו צמודות הוא ציקלי:%
}
ניתן לחסום כל משולש במעגל. נבנה מעגל החוסם את 
$\triangle DAB$
ונניח ש-%
$C'$
היא נקודה כך ש-%
$\angle DAB + \angle DC'B = 180^\circ$,
אבל
$C'$
\textbf{אינה}
על היקף מעגל. ללא הגבלת הכללית, נניח ש-%
$C'$
נמצאת בתוך המעגל.

\begin{center}
\vspace*{-12pt}
\selectlanguage{english}
\begin{tikzpicture}[scale=.55]
\coordinate (origin) at (0,0);
\coordinate (A) at (1,3);
\node[draw,circle through=(A),name path=circle] at (origin) {};
\fill (A) node[above right] {$A$} circle[radius=3pt];
\path[name path=b] (A) -- (-50:4cm);
\path[name path=c] (A) -- (-120:4cm);
\path[name path=d] (A) -- (150:4cm);
\path [name intersections={of=circle and b,by={b1,B}}];
\fill (B) node[right] {$B$} circle[radius=3pt];
\path [name intersections={of=circle and c,by={c1,C}}];
\fill (C) node[below left] {$C$} circle[radius=3pt];
\path [name intersections={of=circle and d,by={d2,D}}];
\fill (D) node[above left] {$D$} circle[radius=3pt];
\coordinate (Cp) at ($(C)!.2!(D)$);
\draw (A) -- (B) -- (Cp) -- (D) -- cycle;
\fill (Cp) node[left,xshift=1pt,yshift=2pt] {$C'$} circle[radius=3pt];
\draw[thick,dashed] (D) -- (B) -- (C) -- (Cp);
\end{tikzpicture}
\selectlanguage{hebrew}
\vspace*{-12pt}
\end{center}

נבנה קרן היוצאת מ-%
$DC$
כאשר
$C$
היא נקודת החיתוך שלה עם המעגל. לפי הטיעון שהוכחנו לעיל, המרובע 
$ABCD$
חסום מעגל ולכן:
\begin{eqnarray*}
\angle DAB + \angle DCB &=& 180^\circ\\
\angle DAB + \angle DCB &=& \angle DAB + \angle DC'B\\
\angle DCB &=& \angle DC'B\,,
\end{eqnarray*}
מצב שאינו אפשרי אם 
$C$
נמצא על המעגל ו-%
$C'$
נמצאת בתוך המעגל.

כדי להשלים את ההוכחה, נראה שהזוויות נגדיות של טרפז שווה שוקיים צמודות, ולכן הטרפז הוא ציקלי וניתן לחוסם אותו במעגל.

\begin{center}
\vspace*{-12pt}
\selectlanguage{english}
\begin{tikzpicture}[scale=.55]
\coordinate (origin) at (0,0);
\coordinate (A) at (2.5,1.8);
\node[circle through=(A),name path=circle] at (origin) {};
\fill (A) node[above right] {$A$} circle[radius=3pt];
\path[name path=b] (A) -- ++(-80:4cm);
\path[name path=d] (A) -- ++(180:6cm);
\path [name intersections={of=circle and b,by={b1,B}}];
\fill (B) node[below right] {$B$} circle[radius=3pt];
\path [name intersections={of=circle and d,by={d1,D}}];
\fill (D) node[above left] {$D$} circle[radius=3pt];
\path[name path=c] (D) -- ++(-100:4cm);
\path [name intersections={of=circle and c,by={c1,C}}];
\fill (C) node[below left] {$C$} circle[radius=3pt];
\draw (A) -- node[right,xshift=8pt] {$x$} (B);
\draw[name path=bc] (B) -- node[below] {$y$} (C);
\draw (C) -- node[left,xshift=-8pt] {$x$} (D) -- node[above] {$y$} (A);
\path[name path=para] (A) -- ++(-100:4cm);
\path [name intersections={of=para and bc,by={Bp}}];
\fill (Bp) node[below left] {$B'$} circle[radius=3pt];
\draw[thick,dashed] (A) -- node[left,xshift=-2pt] {$x$} (Bp);
\end{tikzpicture}
\selectlanguage{hebrew}
\vspace*{-24pt}
\end{center}
נבנה קטע קו
$AB'$
מקביל ל-%
$CD$.
המרובע
$AB'CD$
הוא מקבילית והמשולש
$\triangle ABB'$
שווה שוקיים, כך ש-%
$\angle C= \angle AB'B = \angle B$.
באופן דומה,
$\angle A = \angle D$.
אבל הסכום של הזוויות הפנימיות של מרובע כלשהו שווה ל-%
$360^\circ$:
\begin{eqnarray*}
\angle A + \angle B + \angle C + \angle D &=& 360^\circ\\
2\angle A + 2 \angle C &=& 360^\circ\\
\angle A +  \angle C &=& 180^\circ\,,
\end{eqnarray*}
ובאופן דומה
$\angle B +  \angle D = 180^\circ$.

עכשיו נשתמש במשפט של תלמי
\L{(Ptolemy)}.
המשפט טוען שעבור כל מרובע החסום על ידי מעגל, מתקיים שוויון הקושר את אורכי האלכסונים ואורכי הצלעות: 
\[
ef = ac + bd\,.
\]
\begin{center}
\vspace*{-28pt}
\selectlanguage{english}
\begin{tikzpicture}[scale=.6]
\coordinate (origin) at (0,0);
\coordinate (A) at (1,3);
\node[draw,circle through=(A),name path=circle] at (origin) {};
\fill (A) node[above right] {$A$} circle[radius=3pt];
\path[name path=b] (A) -- (-50:4cm);
\path[name path=c] (A) -- (-120:4cm);
\path[name path=d] (A) -- (150:4cm);
\path [name intersections={of=circle and b,by={b1,B}}];
\fill (B) node[right] {$B$} circle[radius=3pt];
\path [name intersections={of=circle and c,by={C,c2}}];
\fill (C) node[below left] {$C$} circle[radius=3pt];
\path [name intersections={of=circle and d,by={D,d2}}];
\fill (D) node[above left] {$D$} circle[radius=3pt];
\draw (A) -- node[right] {$a$} (B) -- node[below,yshift=-10pt] {$b$} (C) -- node[left] {$c$} (D) -- node[above,xshift=2pt,yshift=16pt] {$d$}  cycle;
\draw (A) -- node[right,near start] {$e$} (C);
\draw (B) -- node[left,near end,yshift=-6pt] {$f$} (D);
\end{tikzpicture}
\selectlanguage{hebrew}
\vspace*{-12pt}
\end{center}
קיימת הוכחה גיאומטרית )ראו ויקיפדיה(, אבל אני אביא הוכחה טריגונומטרית פשוטה.

מחוק הקוסינוסים עבור המשולשים
$\triangle ABC$, $\triangle ADC$, $\triangle DAB$, $\triangle DCB$,
מקבלים את המשוואות:
\begin{eqnarray*}
e^2 &=& a^2 + b^2 - 2ab \cos \angle B\\
e^2 &=& c^2 + d^2 - 2cd \cos \angle D\\
f^2 &=& a^2 + d^2 - 2ad \cos \angle A\\
f^2 &=& b^2 + c^2 - 2bc \cos \angle C\,.
\end{eqnarray*}
הזוויות הנגדיות של מרובע חסום במעגל צמודות
$\angle C = 180^\circ - \angle A$
ו-%
$\angle D = 180^\circ - \angle B$,
ולכן:
\begin{eqnarray*}
\cos \angle D &=& - \cos \angle B\\
\cos \angle C &=& -\cos \angle A\,,
\end{eqnarray*}
וניתן להיפטר מהגורמים עם הקוסינוסים משתי המשוואות הראשונות ומשתי המשוואות האחרונות. לאחר חישובים מעיקים נקבל:
\begin{eqnarray*}
e^2 &=& \frac{(ac+bd)(ad+bc)}{(ab+cd)}\\
f^2 &=& \frac{(ab+cd)(ac+bd)}{(ad+bc)}\,.
\end{eqnarray*}
נכפיל את שתי המשוואות ונפשט כדי לקבל את המשפט של תלמי:
\begin{eqnarray*}
e^2\cdot f^2 &=& (ac+bd)^2\\
ef &=& (ac+bd)\,. 
\end{eqnarray*}
עבור הבנייה בעמוד~%
\L{\pageref{p.ptolemy}},
אורך האלכסונים הוא
$d$,
אורך השוקיים הוא
$b$,
ואורכי הבסיסים הם
$h$
ו-%
$2h$.
ממשפט תלמי:
$d\cdot d = b\cdot b + h\cdot 2h$
או
$d^2=b^2+2h^2$.

תהי
$X$
נקודה על הקו
$PQ$
המאריך את
$PQ$
ב-%
$b$.
בהמשך נבנה את 
$X$
ובינתיים נדמה לעצמנו שהיא קיימת. נגדיר 
$x = K'X$.
המשולש
$\triangle QK'X$
הוא משולש ישר זווית ולכן
$x^2 = b^2 + h^2$:
\begin{center}
\vspace*{-14pt}
\selectlanguage{english}
\begin{tikzpicture}[scale=.7]
\coordinate (Q) at (0,0);
\coordinate (P) at (-6.8,0);
\coordinate (B) at (-3,-2);
\draw[thick,dashed,name path=pq] ($(Q)!1.3!(P)$) -- ($(P)!2.3!(Q)$);
\fill (Q) node[above left] {$Q$} circle[radius=2pt];
\fill (P) node[above] {$P$} circle[radius=2pt];
%\fill (B) circle[radius=2pt];
\node[draw,circle through=(B),name path=qb] at (Q) {};
%\draw[thick,dashed] (Q) -- node[left,xshift=-1pt,yshift=2pt] {$b$} (B);
\path[name path=qh] (Q) -- (-40:5cm);
\path[name path=qhp] (Q) -- (40:5cm);
\path [name intersections={of=qb and qh,by={hp}}];
\path [name intersections={of=qb and qhp,by={H}}];
\fill (H) node[right,xshift=2pt] {$H$} circle[radius=2pt];
\fill (hp) node[right,xshift=2pt] {$H'$} circle[radius=2pt];
\draw[thick,dashed] (H) -- (hp);
\path[name path=circleqh] (Q) let
  \p1 = ($ (H) - (hp) $)
in
  circle ({veclen(\x1,\y1)});
\path[name path=circlehb] (H) let
  \p1 = ($ (Q) - (B) $)
in
  circle ({veclen(\x1,\y1)});
\path [name intersections={of=circleqh and circlehb,by={K,k2}}];
\fill (K) node[above left] {$K$} circle[radius=2pt];
\draw[thick,dashed] (Q) -- (K);
\draw[thick,dashed] (H) -- (K);
\draw[thick,dashed] let
  \p1 = ($ (K) - (Q) $)
in
  coordinate (kp) at (\x1,-\y1);
\fill (kp) node[below left] {$K'$} circle[radius=2pt];
\draw[thick,dashed] (Q) -- node[left] {$h$} (kp) -- (hp);
\draw[thick,dashed] (K) -- (hp);
\draw[thick,dashed] (kp) -- (H);
\path [name intersections={of=pq and qb,by={X,x2}}];
\fill (X) node[below right] {$X$} circle[radius=2pt];
\draw[thick,dashed] (kp) -- node[left] {$x$} (X);
\draw[very thick] (Q) -- (kp) -- (X) -- node[above,xshift=-8pt] {$b$} cycle;
\end{tikzpicture}
\selectlanguage{hebrew}
\vspace*{-12pt}
\end{center}
לפי המשפט של תלמי
$d^2 = b^2 + 2h^2$
ולכן:
\begin{eqnarray*}
d^2 &=& b^2 + 2h^2\\
&=&(x^2-h^2)+2h^2\\
&=&x^2+h^2\,.
\end{eqnarray*}
\vspace*{-6pt}
אל תחפשו משולש ישר זווית באיור. אנחנו רק קובעים 
\textbf{%
שניתן לבנות%
}
משולש ישר זווית עם צלעות
$x,h,d$.

נבנה את הנקודה
$S$
כנקודת החיתוך של המעגלים
$c(K,d)$
ו-%
$c(K',d)$:

\begin{center}
\vspace*{-16pt}
\selectlanguage{english}
\begin{tikzpicture}[scale=.55]
\coordinate (Q) at (0,0);
\coordinate (P) at (-6.8,0);
\coordinate (B) at (-3,-2);
\draw[dashed,name path=pq] ($(Q)!1.3!(P)$) -- ($(P)!2.3!(Q)$);
\fill (Q) node[above left] {$Q$} circle[radius=3pt];
\fill (P) node[above] {$P$} circle[radius=3pt];
\node[draw,circle through=(B),name path=qb] at (Q) {};
\path[name path=qh] (Q) -- (-40:5cm);
\path[name path=qhp] (Q) -- (40:5cm);
\path [name intersections={of=qb and qh,by={Hp}}];
\path [name intersections={of=qb and qhp,by={H}}];
%\fill (H) node[right,xshift=2pt] {$H$} circle[radius=3pt];
%\fill (Hp) node[right,xshift=2pt] {$H'$} circle[radius=3pt];
\path[name path=circleqh] (Q) let
  \p1 = ($ (H) - (Hp) $)
in
  circle ({veclen(\x1,\y1)});
\path[name path=circlehb] (H) let
  \p1 = ($ (Q) - (B) $)
in
  circle ({veclen(\x1,\y1)});
\path [name intersections={of=circleqh and circlehb,by={K,k2}}];
\fill (K) node[above left] {$K$} circle[radius=3pt];
\draw[thick,dashed] let
  \p1 = ($ (K) - (Q) $)
in
  coordinate (Kp) at (\x1,-\y1);
\fill (Kp) node[below left] {$K'$} circle[radius=3pt];
\draw[thick] (Q) -- node[left] {$h$} (Kp);
\draw[thick,name path=khp] (K) let
  \p1 = ($ (H) - (Kp) $),
  \n2 = {veclen(\x1,\y1)}
in
  (K) ++(-100:\n2) arc (-100:-30:\n2);
\draw[thick,name path=kph] (Kp) let
  \p1 = ($ (H) - (Kp) $),
  \n2 = {veclen(\x1,\y1)}
in
  (Kp) ++(100:\n2) arc (100:30:\n2);
\path [name intersections={of=kph and khp,by={S}}];
\fill (S) node[above right,xshift=6pt] {$S$} circle[radius=3pt];
\draw[thick] (Kp) -- node[right,near start,yshift=-6pt] {$d$} (S);
\draw[thick] (Q) -- (S);
\path [name intersections={of=pq and qb,by={X,Xp}}];
\fill (X) node[above right] {$X$} circle[radius=3pt];
%\fill (Xp) node[above left] {$X'$} circle[radius=3pt];
%\draw (Kp) -- node[left] {$x$} (X);
%\draw (K) -- node[left] {$x$} (X);
%\fill (B) circle[radius=3pt];
%\draw[thick,dashed] (Q) -- node[left,xshift=-1pt,yshift=2pt] {$b$} (B);
\end{tikzpicture}
\vspace*{-14pt}
\selectlanguage{hebrew}
\end{center}
מתקבל משולש ישר זווית
$\triangle QSK'$.
לפי משפט פיתגורס
$QS^2 + h^2 = d^2$,
ולכן:
\[
QS^2 = d^2 - h^2 = x^2\,,
\]
ו-%
$QS = x$.
ניתן לבנות את הנקודה
$X$
כנקודות החיתוך בין המעגלים
$c(K,x)$
ו-%
$c(K',x)$:

\begin{center}
\vspace*{-14pt}
\selectlanguage{english}
\begin{tikzpicture}[scale=.55]
\coordinate (Q) at (0,0);
\coordinate (P) at (-6.8,0);
\coordinate (B) at (-3,-2);
\draw[dashed,name path=pq] ($(Q)!1.3!(P)$) -- ($(P)!2.3!(Q)$);
\fill (Q) node[above left] {$Q$} circle[radius=3pt];
\fill (P) node[above] {$P$} circle[radius=3pt];
\node[draw,circle through=(B),name path=qb] at (Q) {};
\path[name path=qh] (Q) -- (-40:5cm);
\path[name path=qhp] (Q) -- (40:5cm);
\path [name intersections={of=qb and qh,by={Hp}}];
\path [name intersections={of=qb and qhp,by={H}}];
%\fill (H) node[right,xshift=2pt] {$H$} circle[radius=3pt];
%\fill (Hp) node[right,xshift=2pt] {$H'$} circle[radius=3pt];
\path[name path=circleqh] (Q) let
  \p1 = ($ (H) - (Hp) $)
in
  circle ({veclen(\x1,\y1)});
\path[name path=circlehb] (H) let
  \p1 = ($ (Q) - (B) $)
in
  circle ({veclen(\x1,\y1)});
\path [name intersections={of=circleqh and circlehb,by={K,k2}}];
\fill (K) node[above left] {$K$} circle[radius=3pt];
\path[thick,dashed] let
  \p1 = ($ (K) - (Q) $)
in
  coordinate (Kp) at (\x1,-\y1);
\fill (Kp) node[below left] {$K'$} circle[radius=3pt];
%\draw[thick] (Q) -- node[left] {$h$} (Kp);
\path[name path=khp] (K) let
  \p1 = ($ (H) - (Kp) $),
  \n2 = {veclen(\x1,\y1)}
in
  (K) ++(-100:\n2) arc (-100:-30:\n2);
\path[name path=kph] (Kp) let
  \p1 = ($ (H) - (Kp) $),
  \n2 = {veclen(\x1,\y1)}
in
  (Kp) ++(100:\n2) arc (100:30:\n2);
\path [name intersections={of=kph and khp,by={S}}];
\fill (S) node[above right,xshift=6pt] {$S$} circle[radius=3pt];
%\draw[thick] (Kp) -- node[right,near start,yshift=-6pt] {$d$} (S);
%\draw[thick] (Q) -- (S);
\path [name intersections={of=pq and qb,by={X,Xp}}];
\fill (X) node[above right,xshift=8pt] {$X$} circle[radius=3pt];
\fill (Xp) node[above left] {$X'$} circle[radius=3pt];
\draw (Kp) -- node[left] {$x$} (X);
\draw (K) -- node[left] {$x$} (X);
%\fill (B) circle[radius=3pt];
%\draw[thick,dashed] (Q) -- node[left,xshift=-1pt,yshift=2pt] {$b$} (B);
\draw[name path=kx] (K) let
  \p1 = ($ (X) - (Kp) $),
  \n2 = {veclen(\x1,\y1)}
in
  (K) ++(-100:\n2) arc (-100:-30:\n2);
\draw[name path=kpx] (Kp) let
  \p1 = ($ (X) - (Kp) $),
  \n2 = {veclen(\x1,\y1)}
in
  (Kp) ++(100:\n2) arc (100:30:\n2);
\path (Xp) -- node[below] {$b$} (Q);
\path (Q) -- node[below] {$b$} (X);
\node at (-5,2) {\mbox{\boldmath $PQ=a$}};
\draw[thick,dashed] (Q) -- node[left] {$h$} (Kp);
\draw[thick,dashed] (Q) -- (X);
\end{tikzpicture}
\vspace*{-10pt}
\selectlanguage{hebrew}
\end{center}
נזכור מה אנחנו רוצים: להאריך את אורכו של
$PQ$
ב-%
$b$
או לקצר אותו ב-%
$b$.
אורכו של
$QX$
הוא
$\sqrt{x^2-h^2}=b$,
ולכן אורכו של 
$PX$
הוא 
$a+b$
ואורכו של
$PX'$
הוא
$a-b$.


%%%%%%%%%%%%%%%%%%%%%%%%%%%%%%%%%%%%%%%%%%%%%%%%%%%%%%%%%%%%%%%

\section{%
בניית קטע קו שאורכו מוגדר יחסית לשלושה קטעי קו אחרים%
}\label{s.relative}

\textbf{%
נתון שלושה קטע קו באורכים 
$n,m,s$.
ניתן לבנות קטע קו שאורכו
$x = \frac{n}{m}s$.%
}

בנה שני מעגלים משותפי מרכז:
$c_1 = c(Z,m)$
,
$c_2 = c(Z,n)$.
נבחר נקודה
$A$
כלשהי על המעגל ונבנה את המיתר
$AB$
שאורכו
$s$
במעגל
$c_1$.
)בניית המיתר עם מחוגה בלבד לפי בסעיף
\L{\ref{s.radius}}.%
(
\begin{center}
\vspace*{-4pt}
\selectlanguage{english}
\begin{tikzpicture}[scale=.4]
\coordinate (Z) at (0,0);
\coordinate (A) at (-130:5cm);
\coordinate (B) at (-80:5cm);
\fill (Z) node[above left] {$Z$} circle[radius=4pt];
\fill (A) node[below left] {$A$} circle[radius=4pt];
\fill (B) node[below] {$B$} circle[radius=4pt];
\draw[name path=c1] (Z) circle[radius=5cm];
\draw[name path=c2] (Z) circle[radius=3cm];
\node at (2,5) {$c_1$};
\node at (2,3) {$c_2$};
\draw[thick] (A) -- node[below,yshift=-6pt] {$s$} (B);
\draw[thick,dashed] (Z) -- node[below] {$m$} ++(10:5cm);
\draw[thick,dashed] (Z) -- node[below] {$n$} ++(-40:3cm);
\fill (Z) ++ (10:5cm) circle[radius=4pt];
\fill (Z) ++ (-40:3cm) circle[radius=4pt];
\begin{scope}[xshift=-14cm,yshift=3cm]
\coordinate (m) at (0,0);
\coordinate (n) at (0,-1.5);
\coordinate (s) at (0,-3);
\coordinate (mp) at (5,0);
\coordinate (np) at (2,-1.5);
\coordinate (sp) at (4,-3);
\fill (m) circle[radius=4pt];
\fill (n) circle[radius=4pt];
\fill (s) circle[radius=4pt];
\fill (mp) circle[radius=4pt];
\fill (np) circle[radius=4pt];
\fill (sp) circle[radius=4pt];
\draw[thick,dashed] (m) -- node[above] {$m$} (mp);
\draw[thick,dashed] (n) -- node[above] {$n$} (np);
\draw[thick,dashed] (s) -- node[above] {$s$} (sp);
\end{scope}
\end{tikzpicture}
\vspace*{-16pt}
\selectlanguage{hebrew}
\end{center}
הנחות:
\vspace*{-10pt}
\begin{itemize}
\item
$m>n$:
אם לא, נחליף את הסימונים של
$m,n$.
\vspace*{-8pt}
\item
המיתר 
$s$
נמצא בתוך
$c_1$
ואינו חותך את
$c_2$.
אם לא, נשתמש בבנייה של סעיף
\L{\ref{s.add-subtract}}
כדי להכפיל את
$m,n$
במספר שלם
$k$
עד שהמיתר לא חותך. שימו לב שהכפלת הערכים אינה משנה את הערך שאנחנו בונים
$x = \frac{kn}{km}s = \frac{n}{m}s$.
\end{itemize}
\vspace*{-8pt}

נבחר נקודה כלשהי
$H$
על המעגל
$c_2$.
נסמן את אורך הקטע
$AH$
ב-%
$w$.
נבנה נקודה
$K$
על 
$c_2$
כך שאורך הקטע
$BK$
גם הוא
$w$.
\begin{center}
\selectlanguage{english}
\vspace*{-4pt}
\begin{tikzpicture}[scale=.4]
\coordinate (Z) at (0,0);
\coordinate (A) at (-130:5cm);
\coordinate (B) at (-90:5cm);
\fill (Z) node[above left] {$Z$} circle[radius=4pt];
\fill (A) node[below left] {$A$} circle[radius=4pt];
\fill (B) node[below] {$B$} circle[radius=4pt];
\draw[name path=c1] (Z) circle[radius=5cm];
\draw[name path=c2] (Z) circle[radius=2.5cm];
\node at (2,5) {$c_1$};
\node at (2,2.5) {$c_2$};
\draw[thick] (A) -- node[below,yshift=-6pt] {$s$} (B);
\draw[thick] (A) -- node[above,xshift=-4pt,yshift=-2pt] {$w$} +(20:120pt) coordinate (H);
\fill (H) node[above right,xshift=-2pt,yshift=4pt] {$H$} circle[radius=4pt];
\draw[thick] (B) -- node[right] {$w$} +(60:120pt) coordinate (K);
\fill (K) node[right] {$K$} circle[radius=4pt];
\begin{scope}[xshift=-14cm,yshift=3cm]
\coordinate (m) at (0,0);
\coordinate (n) at (0,-1.5);
\coordinate (s) at (0,-3);
\coordinate (w) at (0,-4.5);
\coordinate (mp) at (5,0);
\coordinate (np) at (2,-1.5);
\coordinate (sp) at (4,-3);
\coordinate (wp) at (4.5,-4.5);
\fill (m) circle[radius=4pt];
\fill (n) circle[radius=4pt];
\fill (s) circle[radius=4pt];
\fill (w) circle[radius=4pt];
\fill (mp) circle[radius=4pt];
\fill (np) circle[radius=4pt];
\fill (sp) circle[radius=4pt];
\fill (wp) circle[radius=4pt];
\draw[thick,dashed] (m) -- node[above] {$m$} (mp);
\draw[thick,dashed] (n) -- node[above] {$n$} (np);
\draw[thick,dashed] (s) -- node[above] {$s$} (sp);
\draw[thick,dashed] (w) -- node[above] {$w$} (wp);
\end{scope}
\end{tikzpicture}
\vspace*{-8pt}
\selectlanguage{hebrew}
\end{center}
המשולשים
$\triangle AHZ$, $\triangle BZK$
חופפים לפי צלע-צלע-צלע: 
$ZA=ZB=m$
הרדיוס של מעגל
$c_1$,
$ZH=ZK=n$
הרדיוס של המעגל
$c_2$,
ו-%
$AH=BK=w$
לפי הבנייה.
\begin{center}
\vspace*{-6pt}
\selectlanguage{english}
\begin{tikzpicture}[scale=.45]
\coordinate (Z) at (0,0);
\coordinate (A) at (-130:5cm);
\coordinate (B) at (-90:5cm);
\fill (Z) node[above left] {$Z$} circle[radius=4pt];
\fill (A) node[below left] {$A$} circle[radius=4pt];
\fill (B) node[below] {$B$} circle[radius=4pt];
\draw[name path=c1] (Z) circle[radius=5cm];
\draw[name path=c2] (Z) circle[radius=2.5cm];
\node at (2,5) {$c_1$};
\node at (2,2.5) {$c_2$};
\draw[thick,dashed] (A) -- node[below,yshift=-6pt] {$s$} (B);
\draw[thick] (A) -- node[above] {$w$} +(20:120pt) coordinate (H);
\fill (H) node[above right,xshift=-2pt,yshift=4pt] {$H$} circle[radius=4pt];
\draw[thick] (B) -- node[right] {$w$} +(60:120pt) coordinate (K);
\fill (K) node[right] {$K$} circle[radius=4pt];
\draw[thick] (Z) -- node[left,xshift=-2pt,yshift=-2pt] {$m$} (A);
\draw[thick] (Z) -- (B);
\draw[thick] (Z) -- (H);
\draw[thick] (Z) -- node[above] {$n$} (K);
\draw[thick,dashed] (H) -- (K);
\end{tikzpicture}
\vspace*{-12pt}
\selectlanguage{hebrew}
\end{center}
מהחפיפה של המשולשים
$\triangle AZH = \triangle BZK$,
אנו מקבלים
$\angle AZB = \angle HZK$.
קצת קשה לראות את השוויונות האלה באיור, אבל האיור שלהלן מבהיר את היחסים בין הזוויות. נגדיר
$\alpha = \angle AZH = \angle BZK$
ו-%
$\beta = \angle BZH$,
וקל לראות ש-%
$\angle AZB = \angle HZK = \alpha - \beta$.
\begin{center}
\vspace*{-8pt}
\selectlanguage{english}
\begin{tikzpicture}[scale=.6]
\coordinate (Z) at (0,0);
\coordinate (A) at (-150:5cm);
\coordinate (B) at (-100:5cm);
\coordinate (H) at (-60:4.5cm);
\coordinate (K) at (-20:4.5cm);
\fill (Z) circle[radius=2pt];
\fill (A) circle[radius=2pt];
\fill (B) circle[radius=2pt];
\fill (H) circle[radius=2pt];
\fill (K) circle[radius=2pt];
\draw[thick] (A) node[below left] {$A$} -- (Z) node[above] {$Z$} -- (B) node[below] {$B$};
\draw[thick] (H) node[below] {$H$} -- (Z) -- (K) node[below right] {$K$};
\draw (-150:1cm) arc (-150:-60:1);
\draw (-100:2cm) arc (-100:-20:2);
\draw (-100:3cm) arc (-100:-60:3);
\draw[thick,dashed] (-150:4cm) arc (-150:-100:4);
\draw[thick,dashed] (-60:4cm) arc (-60:-20:4);
\node at (-115:1.4) {$\alpha$};
\node at (-50:2.4) {$\alpha$};
\node at (-80:3.5) {$\beta$};
\node at (-40:5) {$\alpha - \beta$};
\node at (-125:5) {$\alpha - \beta$};
\end{tikzpicture}
\vspace*{-10pt}
\selectlanguage{hebrew}
\end{center}
זווית הקודקוד של שני משולשים שווי שוקיים שוות, ולכן
$\triangle AZB$
ו-%
$\triangle HZK$
דומים. 
\begin{center}
\selectlanguage{english}
\vspace*{-4pt}
\begin{tikzpicture}[scale=.45]
\coordinate (Z) at (0,0);
\coordinate (A) at (-130:5cm);
\coordinate (B) at (-90:5cm);
\fill (Z) node[above left] {$Z$} circle[radius=4pt];
\fill (A) node[below left] {$A$} circle[radius=3pt];
\fill (B) node[below] {$B$} circle[radius=3pt];
\draw[name path=c1] (Z) circle[radius=5cm];
\draw[name path=c2] (Z) circle[radius=2.5cm];
\node at (2,5) {$c_1$};
\node at (2,2.5) {$c_2$};
\draw[thick] (A) -- node[below,yshift=-6pt] {$s$} (B);
\path[thick,dashed] (A) -- +(20:120pt) coordinate (H);
\fill (H) node[below] {$H$} circle[radius=3pt];
\path[thick,dashed] (B) -- +(60:120pt) coordinate (K);
\fill (K) node[right] {$K$} circle[radius=3pt];
\draw[thick] (Z) -- node[left,xshift=-2pt,yshift=-2pt] {$m$} (A);
\draw[thick] (Z) -- (B);
\draw[thick] (Z) -- (H);
\draw[thick] (Z) -- node[above] {$n$} (K);
\draw[thick] (H) -- node[below right] {$x$} (K);
\draw[thick,dashed] (A) -- (H);
\draw[thick,dashed] (B) -- (K);
\end{tikzpicture}
\selectlanguage{hebrew}
\vspace*{-10pt}
\end{center}
נסמן את קטע הקו
$HK$
ב-%
$x$,
ונקבל:
\begin{eqnarray*}
\frac{m}{s} &=& \frac{n}{x}\\
x&=&\frac{n}{m}s\,.
\end{eqnarray*}
\vspace*{-20pt}
%%%%%%%%%%%%%%%%%%%%%%%%%%%%%%%%%%%%%%%%%%%%%%%%%%%%%%%%%%%%%%%

\section{%
מציאת נקודת החיתוך של שני קווים%
}\label{s.two-lines}

\textbf{%
נתון שני קווים המוגדרים על ידי קטעי הקו
$AB,CD$.
ניתן לבנות את נקודת החיתוך שלהם עם מחוגה בלבד.%
}

נבנה את הנקודה
$C'$
כשיקוף של
$C$
מסביב ל-%
$AB$,
ו-%
$D'$
כשיקוף של
$D$
מסביב לקו
$AB$.

נקודת החיתוך
$S$
נמצאת על הקו
$AB$,
כי
$\triangle CZS, \triangle C'SZ$
הם משולשים ישר זווית חופפים:
$CZ=C'Z$,
ו-%
$\angle CZS = \angle C'ZS = 90^\circ$
כי לפי ההגדרה של שיקוף
$AB$
הוא האנך האמצעי של 
$CC'$.
מכאן ש-%
$C'S=CS$,
ובאופן דומה
$D'S=DS$.
\begin{center}
\selectlanguage{english}
\begin{tikzpicture}[scale=.8]
\coordinate (A) at (-4,0);
\coordinate (B) at (2,0);
\coordinate (C) at (-3,2);
\coordinate (D) at (1,-1);
\coordinate (Cp) at (-3,-2);
\coordinate (Dp) at (1,1);
\fill (A) node[below] {$A$} circle[radius=2pt];
\fill (B) node[below] {$B$} circle[radius=2pt];
\fill (C) node[above] {$C$} circle[radius=2pt];
\fill (D) node[below] {$D$} circle[radius=2pt];
\fill (Cp) node[below] {$C'$} circle[radius=2pt];
\fill (Dp) node[above] {$D'$} circle[radius=2pt];
\draw[name path=ab] ($(A)!1.3!(B)$) -- ($(B)!1.3!(A)$);
\draw[name path=cd] ($(C)!1.2!(D)$) -- ($(D)!1.1!(C)$);
\path [name intersections={of=ab and cd,by={S}}];
\fill (S) node[above] {$S$} circle[radius=2pt];
\draw (Cp) -- (Dp);
\draw[thick,dashed] (C) -- node[above left] {$c$} (Cp);
\draw[thick,dashed] (D) -- node[above right] {$d$} (Dp);
\path (C) -- node[right,xshift=2pt] {$x$} (S);
\path (S) -- node[left,near end,xshift=-2pt] {$e-x$} (D);
\node at (1,-2.5) {\mbox{\boldmath $CD=C'D'=e$}};
\fill (-3,0) node[below right] {$Z$} circle[radius=2pt];
\end{tikzpicture}
\selectlanguage{hebrew}
\end{center}
המשולשים
$\triangle CSC'$
ו-%
$\triangle DSD'$
דומים, ולכן
$\frac{CS}{DS} = \frac{CC'}{DD'}$.

נסמן
$x = CS, c = CC', d = DD', e = CD$.
מהיחסים במשולשים בדומים מתקבל
$\frac{x}{e-x} = \frac{c}{d}$.
נפתור עבור
$x$
ונקבל
$x=\frac{c}{c+d}e$.

אם
$D$
נמצאת באותו צד של
$AB$
ש-%
$C$
נמצאת:
\begin{center}
\selectlanguage{english}
\begin{tikzpicture}[scale=.8]
\coordinate (A) at (-4,0);
\coordinate (B) at (2,0);
\coordinate (C) at (-3,2);
\coordinate (D) at (-1,1);
\coordinate (Cp) at (-3,-2);
\coordinate (Dp) at (-1,-1);
\fill (A) node[below] {$A$} circle[radius=2pt];
\fill (B) node[below] {$B$} circle[radius=2pt];
\fill (C) node[above] {$C$} circle[radius=2pt];
\fill (D) node[above] {$D$} circle[radius=2pt];
\fill (Cp) node[below] {$C'$} circle[radius=2pt];
\fill (Dp) node[below] {$D'$} circle[radius=2pt];
\draw[name path=ab] ($(A)!1.3!(B)$) -- ($(B)!1.3!(A)$);
\draw[name path=cd] ($(C)!2.2!(D)$) -- ($(D)!1.1!(C)$);
\path [name intersections={of=ab and cd,by={S}}];
\fill (S) node[above] {$S$} circle[radius=2pt];
\draw (Cp) -- (S);
\draw[thick,dashed] (C) -- node[above left] {$c$} (Cp);
\draw[thick,dashed] (D) -- node[above right] {$d$} (Dp);
\path (C) -- node[above] {$e$} (D);
\path (Cp) -- node[below] {$e$} (Dp);
\path (D) -- node[above right,xshift=-4pt] {$x-e$} (S);
\path (Dp) -- node[below right,xshift=-4pt] {$x-e$} (S);
\node at (1,-2.5) {\mbox{\boldmath $CS=C'S=x$}};
\end{tikzpicture}
\selectlanguage{hebrew}
\vspace*{-8pt}
\end{center}
 המשולשים
$\triangle CSC'$
ו-%
$\triangle DSD'$
דומים, ולכן
$\frac{x}{x-e}=\frac{c}{d}$.
נפתור עבור
$x$
ונקבל
$x=\frac{c}{c-d}e$.

נבנה את המעגלים
$c(C',d)$
,
$c(D,e)$,
ונסמן נקודת החיתוך שלהם ב-%
$H$.
סכום האורכים של שני הקטעים
$CC',C'H$
הוא
$c + d$.
יש להראות ש-%
$H$
נמצאת בהמשך הקו של
$CC'$
ואז אורך הקטע
$CH$
יהיה
$c+d$.
)במקרה ש-%
$D$
נמצאת על אותו צד של
$AB$
ש-%
$C$
נמצאת, 
$CH = c - d$.(
\begin{center}
\vspace*{-4pt}
\selectlanguage{english}
\begin{tikzpicture}[scale=.8]
\coordinate (A) at (-4,0);
\coordinate (B) at (2,0);
\coordinate (C) at (-3,2);
\coordinate (D) at (1,-1);
\coordinate (Cp) at (-3,-2);
\coordinate (Dp) at (1,1);
\fill (A) node[below left] {$A$} circle[radius=2pt];
\fill (B) node[below] {$B$} circle[radius=2pt];
\fill (C) node[above] {$C$} circle[radius=2pt];
\fill (D) node[below] {$D$} circle[radius=2pt];
\fill (Cp) node[left] {$C'$} circle[radius=2pt];
\fill (Dp) node[above] {$D'$} circle[radius=2pt];
\draw[name path=ab] ($(A)!1.3!(B)$) -- ($(B)!1.3!(A)$);
\draw[name path=cd] ($(C)!1.2!(D)$) -- ($(D)!1.1!(C)$);
\path [name intersections={of=ab and cd,by={S}}];
\fill (S) node[above,yshift=4pt] {$S$} circle[radius=2pt];
\draw (Cp) -- node[below right] {$e$} (Dp);
\path (C) -- node[above left] {$c$} (Cp);
\draw[thick,dashed] (D) -- node[above right] {$d$} (Dp);
\node at (3.5,-3) {\mbox{\boldmath $CD=C'D'=DH=e$}};
\draw[name path=circled] (D) let
  \p1 = ($ (D) - (C) $),
  \n2 = {veclen(\x1,\y1)}
in
  ++(130:\n2) arc (130:230:\n2);

\draw[name path=circlecp] (Cp) let
  \p1 = ($ (D) - (Dp) $),
  \n2 = {veclen(\x1,\y1)}
in
  ++(-180:\n2) arc (-180:0:\n2);
\path [name intersections={of=circled and circlecp,by={H}}];
\fill (H) node[below left] {$H$} circle[radius=2pt];
\draw[thick,dashed] ($(C)!1.2!(H)$) -- (C);
\draw (H) -- node[right] {$d$} (Cp);
\draw (D) -- node[right,xshift=14pt,yshift=8pt] {$e$} (H);
\end{tikzpicture}
\vspace*{-8pt}
\selectlanguage{hebrew}
\end{center}
מההגדרה של
$H$
כחיתוך של המעגלים
$c(C',d)$
,
$c(D,e)$
אנו מקבלים
$C'H=d$
,
$DH=e$.
אבל 
$C'D'=e, DD'=e$,
ולכן המרובע
$C'D'DH$
הוא מקבילית כי האורכים של זוגות הצלעות הנגדיות שוות. לפי הבנייה, קטע הקו
$DD'$
מקביל ל-%
$CC'$,
ולכן
$C'H$
שמקביל ל-%
$DD'$
מקביל גם ל-%
$CC'$.
אחת מנקודות הקצה של הקטע היא
$C'$,
והקטע חייב להיות על ההמשך של הקטע
$CC'$.

האורכים
$c,d,e$
נתונים והוכחנו בסעיף
\L{\ref{s.add-subtract}}
שניתן לבנות קטע באורך
$c+d$,
ובסעיף
\L{\ref{s.relative}}
הוכחנו שניתן לבנות קטע באורך
$x=\frac{c}{c+d}e$.
$S$
היא נקודת החיתוך של המעגלים
$c(C,x)$
ו-%
$c(C',x)$.
\begin{center}
\selectlanguage{english}
\begin{tikzpicture}[scale=.8]
\coordinate (A) at (-4,0);
\coordinate (B) at (2,0);
\coordinate (C) at (-3,2);
\coordinate (D) at (1,-1);
\coordinate (Cp) at (-3,-2);
\coordinate (Dp) at (1,1);
\fill (A) node[below left] {$A$} circle[radius=2pt];
\fill (B) node[below] {$B$} circle[radius=2pt];
\fill (C) node[above] {$C$} circle[radius=2pt];
\fill (D) node[below] {$D$} circle[radius=2pt];
\fill (Cp) node[left] {$C'$} circle[radius=2pt];
\fill (Dp) node[above] {$D'$} circle[radius=2pt];
\draw[name path=ab] ($(A)!1.3!(B)$) -- ($(B)!1.3!(A)$);
\draw[name path=cd] ($(C)!1.2!(D)$) -- ($(D)!1.1!(C)$);
\path [name intersections={of=ab and cd,by={S}}];
\fill (S) node[above,yshift=4pt] {$S$} circle[radius=2pt];
\draw (Cp) -- (Dp);
\path (C) -- node[above,yshift=4pt] {$x$} (S);
\path (Cp) -- node[below,yshift=-4pt] {$x$} (S);
\path (C) -- node[above left] {$c$} (Cp);
\draw[thick,dashed] (D) -- node[above right] {$d$} (Dp);
\node at (3,-3) {\mbox{\boldmath $CD=C'D'=DH=e$}};
\draw[name path=circled] (C) let
  \p1 = ($ (S) - (C) $),
  \n2 = {veclen(\x1,\y1)}
in
  ++(-10:\n2) arc (-10:-100:\n2);

\draw[name path=circlecp] (Cp) let
  \p1 = ($ (S) - (C) $),
  \n2 = {veclen(\x1,\y1)}
in
  ++(100:\n2) arc (100:0:\n2);
\draw[thick,dashed] (Cp) -- (C);
\end{tikzpicture}
\selectlanguage{hebrew}
\vspace*{-8pt}
\end{center}

%%%%%%%%%%%%%%%%%%%%%%%%%%%%%%%%%%%%%%%%%%%%%%%%%%%%%%%%%%%%%%%

\section{%
מציאת נקודת החיתוך של קו עם מעגל
}\label{s.line-circle}

\textbf{%
נתון מעגל
$k$
וקו
$AB$,
ניתן לבנות את נקודות החיתוך שלהם עם מחוגה בלבד.%
}

נסמן את מרכז המעגל
$k$
ב-%
$M$
והרדיוס שלו ב-%
$r$: $k = c(M,r)$,
ונבנה את 
$M'$,
השיקוף של
$M$
מסביב ל-%
$AB$.
נבנה את המעגל
$k'=c(M',r)$.
נקודות החיתוך של המעגלים
$k,k'$
הן נקודות החיתוך של הקו 
$AB$
והמעגל
$k$.
\begin{center}
\vspace*{-10pt}
\selectlanguage{english}
\begin{tikzpicture}[scale=.5]
\coordinate (A) at (-7,0);
\coordinate (B) at (8,0);
\coordinate (M) at (0,-2);
\coordinate (Mp) at (0,2);
\fill (A) node[below] {$A$} circle[radius=3pt];
\fill (B) node[below] {$B$} circle[radius=3pt];
\fill (M) node[below left] {$M$} circle[radius=3pt];
\fill (Mp) node[above left] {$M'$} circle[radius=2pt];
\draw[name path=c1] (M) circle[radius=3cm];
\draw[name path=c2] (Mp) circle[radius=3cm];
\draw[name path=ab] ($(A)!1.2!(B)$) -- ($(B)!1.2!(A)$);
\path [name intersections={of=c1 and c2,by={S1,S2}}];
\fill (S1) circle[radius=3pt];
\fill (S2) circle[radius=3pt];
\path[name path=radius1] (M) -- ++(15:4cm);
\path [name intersections={of=c1 and radius1,by={R1}}];
\draw[thick,dashed] (M) -- node[below] {$r$} (R1);
\path[name path=radius2] (Mp) -- ++(40:4cm);
\path [name intersections={of=c2 and radius2,by={R2}}];
\draw[thick,dashed] (Mp) -- node[above] {$r$} (R2);
\fill (R1) circle[radius=3pt];
\fill (R2) circle[radius=3pt];
\end{tikzpicture}
\selectlanguage{hebrew}
\vspace*{-10pt}
\end{center}
בנייה זו אינה אפשרית אם מרכז המעגל
$M$
נמצא על הקו
$AB$.
במקרה זה, יש להאריך ולקצר את הקטע
$AM$
באורך 
$r$
לפי הבנייה המתוארת בסעיף~
\L{\ref{s.add-subtract}}.
נקודות הקצה של הקטעים האלה הן נקודות החיתוך של
$k$
עם
$AB$.
\begin{center}
\vspace*{-8pt}
\selectlanguage{english}
\begin{tikzpicture}[scale=.5]
\coordinate (A) at (-7,0);
\coordinate (B) at (8,0);
\coordinate (M) at (0,0);
\fill (A) node[below] {$A$} circle[radius=3pt];
\fill (B) node[below] {$B$} circle[radius=3pt];
\fill (M) node[below left] {$M$} circle[radius=3pt];
\draw[name path=c1] (M) circle[radius=3cm];
\draw[name path=ab] ($(A)!1.2!(B)$) -- ($(B)!1.2!(A)$);
\path[name path=radius1] (M) -- ++(-30:4cm);
\path [name intersections={of=c1 and radius1,by={R1}}];
\draw[thick,dashed] (M) -- node[below] {$r$} (R1);
\path [name intersections={of=c1 and ab,by={S1,S2}}];
\fill (S1) node[above right] {$AM+r$} circle[radius=3pt];
\fill (S2) node[above left] {$AM-r$} circle[radius=3pt];
\fill (R1) circle[radius=3pt];
\end{tikzpicture}
\vspace*{-10pt}
\selectlanguage{hebrew}
\end{center}

\end{document}

