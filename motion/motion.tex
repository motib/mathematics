\documentclass[12pt,a4paper]{article}
\usepackage[utf8x]{inputenc}
\usepackage[english,hebrew]{babel}
\selectlanguage{hebrew}
\usepackage{graphicx}
\usepackage{verbatim}
\usepackage{url}
\newcommand*{\p}[1]{\textsf{\small #1}}

\graphicspath{{images/}}

\usepackage{tikz}
\usetikzlibrary{external,intersections}
\tikzexternalize[prefix=tikz/]

% Use stealth arrows
\tikzset {
  >=stealth
}

\textwidth=15cm
\textheight=23cm
\topmargin=0pt
\headheight=0pt
\oddsidemargin=2em
\headsep=0pt
\parindent=0pt

\begin{document}
\thispagestyle{empty}


\begin{center}
\textbf{\LARGE תרשימים דו-ממדיים עבור בעיות תנועה והספק}

\bigskip
\bigskip

\textbf{\Large מוטי בן-ארי}

\bigskip

\textbf{\large המחלקה להוראת המדעים}

\bigskip

\textbf{\large מכון ויצמן למדע}

\bigskip

\url{http://www.weizmann.ac.il/sci-tea/benari/}

\bigskip

\L{27/05/2018}

\end{center}

\selectlanguage{english}

\begin{footnotesize}
\copyright{}\  2017 by Moti Ben-Ari. This work is licensed under the Creative Commons Attribution-ShareAlike 3.0 Unported License. To view a copy of this license, visit \url{http://creativecommons.org/licenses/by-sa/3.0/} or send a letter to Creative Commons, 444 Castro Street, Suite 900, Mountain View, California, 94041, USA.
\end{footnotesize}

\bigskip

\begin{center}
\includegraphics[width=.15\textwidth]{../by-sa.png}
\end{center}

\selectlanguage{hebrew}

\section*{מבוא}

אביטל אלבוים-כהן הציעה להשתמש בתרשימים דו-ממדיים לפתרון בעיות תנועה. מצאתי שהתרשימים מאוד עוזרים בזיהוי הקשרים בין קטעי התנועה ובכתיבת הנוסחאות. ניתן להיעזר בתרשימים דו-ממדיים גם בבעיות הספק שיש להן מבנה דומה לבעיות תנועה.

\smallskip

מסמך זה מכיל תרשימים ופתרונות של בעיות התנועה וההספק משאלוני
$806$
מ-תשע"ד עד תשע"ח. תחילה מופיעות בעיות תנועה ואחר כך בעיות הספק.

\smallskip

הדגש הוא על התרשים וכתיבת הנוסחאות וקיצרתי בחישובים. לקורא שזמנו קצר, אני מציע לעיין בבחינה של תשע"ה קיץ מועד ב', שם לדעתי התרומה המרשימה ביותר של השיטה.

\smallskip

בתרשימים הציר האופקי הוא ציר הזמן והציר האנכי הוא ציר המרחק בבעיות תנועה וציר העבודה בבעיות הספק. היתרון של ייצוג זה הוא שמהיריות וההספקים מוצגים כשיפועים של הקווים. ככל שהמהירות או ההפסק גבוה יותר, הקו תלול יותר.

\smallskip

לכל דמות )מכונית, סירה, צבע, וכדומה( נצייר קו עבור כל קטע בתנועה או בעבודה. בבעיות תנועה, יש לשים לב שבניגוד לתרשימים חד-ממדיים בהם אורך קו הוא מרחק הנסיעה, כאן מרחק הנסיעה הוא ההפרש בציר האנכי בין הנקודה ההתחלתית לנקודה הסופית.

\smallskip

התרשימים קלים מאוד לציור ומועילים גם אם קני המידה בכלל לא מדוייקים, כך שניתן להשתמש בהם כאשר פותרים בחינות.

\bigskip


כהשלמה למסמך זה אני ממליץ על המאמר "פתרונות שונים לבעיות הספק באמצעים גרפיים" מאת אביטל אלבוים-כהן וג'ייסון קופר. על"ה גיליון
$51$,
מרץ
$2015$,
עמ'
$14$-$19$.
הם מביאים פתרונות גיאומטריים עבור בעיות הספק.

\newpage

%%%%%%%%%%%%%%%%%%%%%%%%%%%%%%%%%%%%%%%%%%%%%%%%%%%%%%%%%%%%%%%%

\begin{center}
\textbf{\huge
בעיות תנועה}
\end{center}


\section*{חורף תשע"ד}

\begin{center}
\selectlanguage{english}
\includegraphics[width=.8\textwidth]{winter-2014-1}
\end{center}

\begin{center}
\selectlanguage{english}
\begin{tikzpicture}
\draw (0,0) node[below left] {$A$} node[below,xshift=-14pt,yshift=-10pt] {\p{9:00}} -- (10,0);
\draw (0,0) -- (0,6) node[above left] {$B$};
\draw[dashed] (0,6) -- (10,6);
\draw[thick,name path=raft] (0,0) -- node[left,near start,xshift=-4pt] {
\R{רפסודה}
} (10,6);
\draw[thick,name path=boat] (0,6) -- node[right,near start,xshift=2pt] {
\R{סירה}
} (8,0) -- (10,6);
\path [name intersections={of=boat and raft,by=meeting}];
\draw[dashed] (meeting) |- coordinate (time) (0,0);
\draw[dashed] (meeting) -| coordinate (distance) (0,0);
\draw[dashed] (8,0) -- (8,6);
\draw[dashed] (10,0) -- (10,6);
\fill (meeting) circle [radius=2pt];
\fill (time) circle [radius=2pt];
\fill (0,0) circle [radius=2pt];
\fill (8,0) circle [radius=2pt];
\fill (8,6) circle [radius=2pt];
\fill (10,6) circle [radius=2pt];
\fill (10,0) circle [radius=2pt];
\fill (distance) circle [radius=2pt];
\draw[<->] (-.4,0) -- node[fill=white] {$d_1$} (distance -| -.4,0);
\draw[<->] (distance -| -.4,0) -- node[fill=white] {$d_2$} (-.4,6);
\draw[<->] (-.9,0) -- node[fill=white] {$d$} (-.9,6);
\draw[<->] (0,-.6) -- node[above] {
\R{שעות}
 $5$} (time |- 0,-.6);
\draw[<->] (0,-1.2) -- node[fill=white] {$t_1$} (8,-1.2);
\draw[<->] (8,-1.2) -- node[fill=white] {$t_2$} (10,-1.2);
\draw[<->] (0,-1.8) -- node[fill=white] {$t$} (10,-1.8);
\end{tikzpicture}
\end{center}

נסמן:
$=d$
מרחק בין שני הנמלים,
$=t$
זמן עד למפגש ב-
$B$,
$=v$
מהירות הזרם.

\smallskip

ניתן לחלק את הזמן
$t$
לשתי תקופות
$t=t_1+t_2$
לפי שני הקטעים בהם הפליגה הסירה במהירויות שונות. נכתוב משוואה המשווה את זמני ההפלגה עד למפגש ב-
$B$:
\[
\frac{d}{v} = \frac{d}{15-v} + \frac{d}{15+v}\,.
\]
מפישוט המשוואה מתקבלת משוואה ריבועית ב-
$v$:
\[
v^2+30v-225=0
\]
שהשורש החיובי שלה הוא
$v=6.21$.

\smallskip

את המרחק
$d$
בין הנמלים ניתן לחלק לשני קטעים, המרחק שהפליגה הסירה עד למפגש והמרחק שהפליגה הרפסודה עד למפגש
$d=d_1+d_2$.
נכתוב משוואה המשווה את מרחק שני הקעטים למרחק בין הנמלים:
\[
d = 5v + 5(15-v)\,.
\]
הפתרון הוא
$d=75$
)ללא תלות במהירות הזרם
$v$(.

\smallskip

את הזמן עד המפגש בנמל 
$B$
אפשר לחשב לפי ההפלגה של הסירה או לפי ההפלגה של הרפסודה. כמובן שפשוט יותר לחשב עבור הקטע היחיד של הרפסודה:
\[
t = \frac{d}{v} = \frac{75}{6.21} = 12.077\,.
\]
לכן המפגש השני מתקיים לאחר השעה
\L{\p{09:00}}
בערב.


\newpage

%%%%%%%%%%%%%%%%%%%%%%%%%%%%%%%%%%%%%%%%%%%%%%%%%%%%%%%%%%%%%%%%

\section*{קיץ תשע"ד מועד א}

\begin{center}
\selectlanguage{english}
\includegraphics[width=.8\textwidth]{summer-2014a-1}
\end{center}

\begin{center}
\selectlanguage{english}
\begin{tikzpicture}
\draw (0,0) node[left] {$A$} -- (10,0);
\draw (0,0) -- (0,6) node[left] {$B$};
\draw[dashed] (0,6) -- (10,6);
\draw[thick,name path=truck] (0,0) -- node[left,near start,xshift=-6pt] {
\R{משאית}
} (10,6);
\draw[thick,name path=car] (3,0) -- node[left,near end] {
\R{מכונית}
} (7,6);
\path [name intersections={of=truck and car,by=meeting}];
\draw[dashed] (meeting) |- coordinate (time) (0,0);
\draw[dashed] (10,0) -- (10,6);
\draw[dashed] (7,0) -- (7,6);
\fill (meeting) circle [radius=2pt];
\fill (time) circle [radius=2pt];
\fill (0,0) circle [radius=2pt];
\fill (3,0) circle [radius=2pt];
\fill (7,0) circle [radius=2pt];
\fill (10,0) circle [radius=2pt];
\draw[<->] (0,-.5) -- node[fill=white] {$t$} (3,-.5);
\draw[<->] (3,-.5) -- node[fill=white] {
\R{שעה}
 $1$} (time |- 0,-.5);
\draw[<->] (7,-.5) -- node[fill=white] {
\R{שעות}
 $2$} (10,-.5);
\draw[<->] (3.1,-1) -- node[fill=white] {$4-t$} (6.9,-1);
\draw[<->] (0,-1.5) -- node[fill=white] {
\R{שעות}
 $6$} (10,-1.5);
\end{tikzpicture}
\end{center}

נסמן:
$=t$
זמן יציאת המכונית,
$=v_c$
מהירות המכונית,
$=v_m$
מהירות המשאית.

\smallskip

נכתוב משוואות למרחקים שווים, מ-
$A$
עד למפגש ומ-
$A$
עד ל-
$B$:
\begin{eqnarray*}
v_m(t+1) &=& v_c\cdot 1\\
v_m \cdot 6 &=& v_c (4-t)\,.
\end{eqnarray*}
משתי המשוואות מתקבלת משוואה ריבועית ב-
$t$:
\[
t^2 - 3t + 2 = 0
\]
שיש לה שני פתרונות
$1=t$
שעה ו-
$2=t$
שעות.

\newpage

%%%%%%%%%%%%%%%%%%%%%%%%%%%%%%%%%%%%%%%%%%%%%%%%%%%%%%%%%%%%%%%%

\section*{קיץ תשע"ד מועד ב}

\begin{center}
\selectlanguage{english}
\includegraphics[width=.8\textwidth]{summer-2014b-1}
\end{center}

\begin{center}
\selectlanguage{english}
\begin{tikzpicture}
\draw (0,0) -- (10,0);
\draw (0,0) -- (0,6);
\draw[thick,name path=one] (0,0) -- node[below,near end,xshift=4pt] {I} (10,4);
\draw[thick,name path=two] (0,0) -- node[below,near end,xshift=16pt,yshift=6pt] {II} (10,6);
\draw[thick,name path=three] (2,0) -- node[above,near end,yshift=6pt] {III} (9,6);
\path [name intersections={of=one and three,by=one-three}];
\path [name intersections={of=two and three,by=two-three}];
\draw[dashed] (one-three) |- coordinate (time-one-three) (0,0);
\draw[dashed] (two-three) |- coordinate (time-two-three) (0,0);
\fill (0,0) circle [radius=2pt];
\fill (2,0) circle [radius=2pt];
\fill (one-three) circle [radius=2pt];
\fill (two-three) circle [radius=2pt];
\fill (time-one-three) circle [radius=2pt];
\fill (time-two-three) circle [radius=2pt];
\draw[<->] (0,-.6) -- node[fill=white] {
\R{שעה}
 $1/3$} (2,-.6);
\draw[<->] (2,-.6) -- node[fill=white] {$t$} (time-one-three |- 0,-.6);
\draw[<->] (time-one-three |- 0,-.6) -- node[fill=white] {
\R{שעה}
 $1$} (time-two-three |- 0,-.6);
\end{tikzpicture}
\end{center}

נסמן:
$=t$
הזמן בין היציאה של III ועד למפגש שלו עם I,
$=v$
המהירות של III.

נתון:
$=6$
מהירות של I ו-
$=7.5$
המהירות של II.

נכתוב את הנוסחאות למרחקים שווים, המרחק מהיציאה של I ועד המפגש שלו עם III, והמרחק מהיציאה של II ועד המפגש שלו עם III:
\begin{eqnarray*}
6(t+1/3) &=& vt\\
7.5(1/3+t+1) &=& v(t+1)\,.
\end{eqnarray*}
משתי המשוואות מתקבלת משוואה ריבועית ב-
$t$:
\[
1.5t^2 + 2t - 2 = 0
\]
שיש לה פתרון חיובי אחד
$t=2/3$.

\smallskip

הזמן מהיציאה של III ועד המפגש שלו עם II הוא
$5/3=t+1$
שעות.


\newpage

%%%%%%%%%%%%%%%%%%%%%%%%%%%%%%%%%%%%%%%%%%%%%%%%%%%%%%%%%%%%%%%%

\section*{קיץ תשע"ה מועד א}

\begin{center}
\selectlanguage{english}
\includegraphics[width=.8\textwidth]{summer-2015a-1}
\end{center}

\begin{center}
\selectlanguage{english}
\begin{tikzpicture}
\draw (0,0) -- (10,0);
\draw (0,0) -- (0,6);
\draw[thick,name path=one] (0,0) -- node[below,near end,xshift=50pt,yshift=28pt] {I} (10,6);
\draw[thick,name path=two] (0,0) -- node[below,near end,xshift=50pt,yshift=14pt] {II} (10,3);
\draw[thick,name path=three] (3,0) -- node[above,near end,xshift=2pt,yshift=12pt] {III} (8,6);
\path [name intersections={of=one and three,by=one-three}];
\path [name intersections={of=two and three,by=two-three}];
\draw[dashed] (two-three) |- coordinate (time-two-three) (0,0);
\fill (0,0) circle [radius=2pt];
\fill (3,0) circle [radius=2pt];
\fill (one-three) circle [radius=2pt];
\fill (two-three) circle [radius=2pt];
\fill (time-two-three) circle [radius=2pt];
\node[above left] at (one-three) {$B$};
\node[below right] at (two-three) {$A$};
\path [name path=distance] (two-three) -| +(0,2);
\path [name intersections={of=one and distance,by=fifteen}];
\draw (two-three) -- (fifteen) node[above left] {
\R{ק"מ}
$15$};
\draw[->] (3.3,2.5) -- (3.8,1.7);
\fill (fifteen) circle [radius=2pt];
\draw[<->] (0,-.6) -- node[fill=white] {
\R{שעה}
 $1/2$} (3,-.6);
\draw[<->] (3,-.6) -- node[fill=white] {$t$} (time-two-three |- 0,-.6);
\draw[->] (4.5,3.2) -- +(.45,-.55);
\draw[<->,thick,densely dotted] (5.3,3.1) -- +(0,-1.5);
\draw[<->,thick,densely dotted] (5,2.95) -- +(0,-.5);
\node at (5.6,2.4) {\p{I--II}};
\node at (4.5,3.4) {\p{I--III}};
\draw[<->,thick,densely dotted] (7.8,4.6) -- node[right] {\p{I--II}} +(0,-2.2);
\draw[<->,thick,densely dotted] (7.8,4.8) -- node[right,yshift=3pt] {\p{I--III}} +(0,.8);
\node at (5.6,2.4) {\p{I--II}};
\node at (4.5,3.4) {\p{I--III}};
\end{tikzpicture}
\end{center}

נסמן
$=t$
הזמן בין היציאה של III ועד למפגש שלו עם II,
$=v$
המהירות של III.

נתון
$=50$
מהירות של I,
$=40$
המהירות של II.

\paragraph{סעיף א}
נכתוב את המשוואות למרחקים שווים, המרחק מהיציאה של II ועד למפגש שלו עם III, והמרחק מהיציאה של I ועד למפגש של II עם III ועוד
$15$
ק"מ:
\begin{eqnarray*}
40(t+1/2) &=& vt\\
50(t+1/2) &=& vt + 15\,.
\end{eqnarray*}
מהמשוואות מתקבל
$1=t$
ואח"כ
$60=v$
קמ"ש.

\paragraph{סעיף ב: הוכחה מעיון בתרשים}
\mbox{}

נעיין בקווים מנוקדים בתרשים ונראה שהמרחקים לא יכולים שווים:

\smallskip

בנקודה
$A$
הרחקים שווים, אבל מנקודה זו ועד לנקודה
$B$,
המרחק 
\p{I--II}
גדל והמרחק
\p{I--III}
קטן.

\smallskip

בנקודה
$B$
המרחק 
\p{I--II}
חיובי והמרחק
\p{I--III}
שווה לאפס. מכאן והלאה, שני המרחקים גדלים באותו קצב כי הפרשי המהירויות שווים
$10$
קמ"ש.


\paragraph{סעיף ב: הוכחה בחישוב}
\mbox{}

נסמן
$=t_A$
זמן מנקודה
$A$,
$=t_B$
זמן מנקודה
$B$,
$=d_B$
המרחק בין
\p{I}
ל-
\p{II}
בנקודה
$B$.

\smallskip

משמאל לנקודה
$B$
המרחקים שווים אם:
\begin{eqnarray*}
15 + (v_1-v_2)t_A &\stackrel{?}{=}& 15 + (v_1-v_3)t_A\\
15+(50-40)t_A &\stackrel{?}{=}& 15+(50-60)t_A\,,
\end{eqnarray*}
המביא לתנאי השקרי
$10=-10$.

\smallskip

מימין לנקודה
$B$
המרחקים שווים אם:
\begin{eqnarray*}
(v_3-v_1)t_B &\stackrel{?}{=}& d_B + (v_1-v_2)t_B\\
(60-50)t_B &\stackrel{?}{=}& d_B + (50-40)t_B\,.
\end{eqnarray*}
המשוואה נכונה רק אם
$d_B=0$
אבל אנחנו יודעים ש-
$d_B > 15$.

\newpage

%%%%%%%%%%%%%%%%%%%%%%%%%%%%%%%%%%%%%%%%%%%%%%%%%%%%%%%%%%%%%%%%

\section*{קיץ תשע"ה מועד ב}

\begin{center}
\selectlanguage{english}
\includegraphics[width=.8\textwidth]{summer-2015b-1}
\end{center}


\begin{center}
\selectlanguage{english}
\begin{tikzpicture}[scale=.9]
\draw (0,0) -- (14,0);
\draw (0,-4) node[left] {
\R{פרידה}
} -- (0,-2) node[left] {
\R{מפגש 2}
} -- (0,0) node[left] {
\R{מפגש 1}
} -- (0,4) node[left] {
\R{תחנה}
};
\fill (0,0) circle [radius=2pt];
\fill (1,0) circle [radius=2pt] node[below right] {$M$};
\fill (4,0) circle [radius=2pt] node[above] {$N$};
\fill (8,0) circle [radius=2pt] node[above] {$P$};
\fill (11,0) circle [radius=2pt] node[below right] {$Q$};
\fill (14,0) circle [radius=2pt] node[below] {$R$};
\draw[thick] (0,0) -- node[left] {$a$} (1,4) -- node[right,near start] {$b$} (4,-2);
\draw[thick] (0,0) -- node[below,near start,xshift=-2mm] {$c$} node[right,near end,yshift=2mm] {$d$} (8,-4)  node[below] {$P'$} -- (14,4)  node[right] {$R'$};
\draw[dashed] (0,4) -- (14,4);
\draw[dashed] (0,-2) -- (14,-2);
\draw[dashed] (0,-4) -- (14,-4);
\draw[dashed] (1,4)  node[above] {$M'$} -- (1,0);
\draw[dashed] (4,0) -- (4,-2) node[below,yshift=-1mm] {$N'$};
\draw[dashed] (8,-4) -- (8,0);
\draw[dashed] (11,0) -- (11,4);
\draw[dashed] (14,0) -- (14,4);
\draw[<->] (0,.7) -- node[fill=white] {$10$} (1,.7);
\draw[<->] (1,.7) -- node[near start,fill=white] {$t$} (4,.7);
\draw[<->] (4,.7) -- node[fill=white] {$180$} (8,.7);
\draw[<->] (8,.7) -- node[fill=white] {$t_1$} (11,.7);
\draw[<->] (11,.7) -- node[fill=white] {$t_2$} (14,.7);
\path (8,-2) --  node[below right,xshift=5mm,yshift=-2mm] {$e_1$} (11,0);
\path (11,0) --  node[left,yshift=3mm] {$e_2$} (14,4);
\end{tikzpicture}
\end{center}

\newpage

בתרשים סימנו את הקטעים:

$=a$
יוסי נוסע באוטובוס

$=b$
יוסי רץ לפגישה עם אמא

$=c$
אמא הולכת עד למפגש עם יוסי

$=d$
יוסי ואמא הולכים ביחד

$=e_1+e_2$
יוסי רץ חזרה לתחנה

\smallskip

נסמן:
$=t$
הזמן שיוסי רץ מהתחנה כדי להשיג את אמא.

\smallskip
נסמן מהירויות:
$=v_y$
יוסי, 
$=v_a$
אמא, 
$=v_b$
אוטובוס.

\smallskip

נתון:
$v_y=2v_a$, $v_y=v_b/7$.

\paragraph{סעיף א}

$NN'$
הוא המרחק ממפגש
$1$
למפגש
$2$.
אמא הולכת לפי הקו
$c$
ולכן המרחק הוא
$v_a(t+10)$.
יוסי הולך לכיוון אחד לפי הקו
$a$
וחוזר לפי הקו
$b$.
\textbf{ההפרש}
בין מרחקים יהיה שווה גם הוא ל-
$NN'$.
נכתוב משוואה לשוויון המרחקים:

\[
v_a(t+10) = v_yt - v_b 10.
\]
לאחר הצבת יחסי המהירויות הנתונים:
\[
\frac{v_y}{2}(t+10) = v_yt - 7v_y 10
\]
נקבל 
$150=t$
שניות.

\paragraph{סעיף ב}

הקו
$e_1+e_2$
מתאר את הריצה של יוסי בחזרה לתחנה.
$PP'$
הוא המרחק שאמא עברה בין מפגש
$1$
לבין נקודת הפרידה, וגם המרחק של
$e_1$,
הקטע הראשון של הריצה של יוסי בחזרה מנקודת הפרידה לכיוון התחנה. פרק הזמן שאמא הלכה בין מפגש
$1$
ועד נקודת הפרידה הוא
$340=180+150+10$
שניות. יוסי רץ פי שניים מהר מאמא, ולכן את הדרך חזרה מנקודת הפרידה למפגש
$1$
הוא עובר ב
$170=t_1$
שניות.

\smallskip

$MM'$
הוא המרחק שהאוטובוס עבר ממפגש
$1$
ועד התחנה, והאוטובוס עובר מרחק זה ב
$10$
שניות.
$RR'=MM'$
הוא גם המרחק שיוסי רץ בין מפגש
$1$
בחזרה לתחנה, אבל הוא רץ פי שבע לאט מהאוטובוס, ולכן את הדרך הוא עובר ב
$t_2=70$
שניות.

\smallskip

נסכם ונקבל שיוסי רץ מנקודת הפרידה לתחנה ב
$240=t_1+t_2$
שניות.

\newpage

%%%%%%%%%%%%%%%%%%%%%%%%%%%%%%%%%%%%%%%%%%%%%%%%%%%%%%%%%%%%%%%%

\section*{חורף תשע"ו}

\begin{center}
\selectlanguage{english}
\includegraphics[width=.8\textwidth]{winter-2016-1}
\end{center}

\begin{center}
\selectlanguage{english}
\begin{tikzpicture}
\draw (0,0) node[left] {$A$} -- (10,0);
\draw (0,0) -- (0,6) node[left] {$B$};
\draw[dashed] (0,6) -- (10,6);
\draw[thick,name path=motor] (0,0) -- node[right,near start,xshift=1pt,yshift=-4pt] {
\R{אופנוע}
} (4,6);
\draw[thick,name path=bike] (0,6) -- node[above,near end,xshift=44pt,yshift=-10pt] {
\R{אופניים}
} (8,4);
\path [name intersections={of=motor and bike,by=meeting}];
\coordinate (fourth) at (0,4.5);
\coordinate (two-thirds) at (0,3.5);
\path [name path=path-fourth] (fourth) -- +(6,0);
\path [name path=path-two-thirds] (two-thirds) -- +(6,0);
\path [name intersections={of=path-fourth and bike,by=meeting-fourth}];
\path [name intersections={of=path-two-thirds and motor,by=meeting-two-thirds}];
\draw[dashed] (meeting) |- coordinate (time) (0,0);
\draw[dashed] (meeting-fourth) -| coordinate (fourth-y) (0,0);
\draw[dashed] (meeting-fourth) |- coordinate (fourth-x) (0,0);
\draw[dashed] (meeting-two-thirds) -| coordinate (two-thirds-y) (0,0);
\draw[dashed] (meeting-two-thirds) |- coordinate (two-thirds-x) (0,0);
\fill (meeting) circle [radius=2pt];
\fill (time) circle [radius=2pt];
\fill (0,0) circle [radius=2pt];
\fill (two-thirds) circle [radius=2pt];
\fill (fourth) circle [radius=2pt];
\fill (fourth-x) circle [radius=2pt];
\fill (fourth-y) circle [radius=2pt];
\fill (meeting-fourth) circle [radius=2pt];
\fill (two-thirds-x) circle [radius=2pt];
\fill (two-thirds-y) circle [radius=2pt];
\fill (meeting-two-thirds) circle [radius=2pt];
\path (0,0) -- node[left] {$2/3$} (two-thirds-y);
\path (0,6) -- node[left] {$1/4$} (fourth-y);
\draw[<->] (0,-.5) -- node[fill=white] {
\R{שעות}
 $3$} (time |- 0,-.5);
\draw[<->] (fourth-x |- 0,-1) -- node[fill=white] {
\R{שעות}
$5/4$} (two-thirds-x |- 0,-1);
\node[above left] at (two-thirds-x) {$T_m$};
\node[above right] at (fourth-x) {$T_b$};
\end{tikzpicture}
\end{center}

נסמן:
$=v_b$
מהירות אופניים,
$=v_m$
מהירות אופנוע,
$=x$
מרחק בין הערים.
\paragraph{סעיף א}

בשלוש שעות האופנוע והאופניים עוברים את כל המרחק בין הערים:
\[
x = 3v_b + 3 v_m\,.
\]
זמן הנסיעה של האופניים
$T_b$--$A$
שווה לזמן הנסיעה של האופנוע
$T_m$--$A$
ועוד
$5/4$
שעות:

\[
\frac{x/4}{v_b} = \frac{2x/3}{v_m} + \frac{5}{4}\,.
\]
נסמן את היחס בין המהירויות )התשובה הדרושה(
$r=\frac{v_m}{v_b}$
ונקבל משוואה הריבועית:
\[
3r^2 - 10r - 8 = 0\,.
\]
השורש החיובי הוא
$4=r$.

\newpage

\paragraph{סעיף ב}
ממשוואת המרחקים והיחס בין המהירויות נחשב:
\[
x = 3v_b + 3v_m = 3\cdot\frac{v_m}{r}+3v_m = 3v_m\left(\frac{1}{4}+1\right) =\frac{15}{4}v_m\,.
\]
הזמן שהאופנוע עובר את המרחק
$x$
הוא
$15/4$
שעות.

\newpage

%%%%%%%%%%%%%%%%%%%%%%%%%%%%%%%%%%%%%%%%%%%%%%%%%%%%%%%%%%%%%%%%

\section*{קיץ תשע"ו מועד א}

\begin{center}
\selectlanguage{english}
\includegraphics[width=.8\textwidth]{summer-2016a-1}
\end{center}

\begin{center}
\selectlanguage{english}
\begin{tikzpicture}
\draw (0,0) node[left] {
\R{א}
} -- (10,0);
\draw (0,0) -- node[left] {
\R{ק"מ}
$300$} (0,6) node[left] {
\R{ב}
};
\draw[dashed] (0,6) -- (10,6);
\draw[thick,name path=car1] (0,0) -- (3,4) coordinate (change) -- node[above left,near start] {$1$
\R{מכונית}
} (10,6);
\draw[thick,name path=car2] (0,0) -- node[right,xshift=1pt,yshift=-12pt] {$2$
\R{מכונית}
} (7,6);
\draw[dashed] (change) |- coordinate (time) (0,0);
\draw[dashed] (7,6) -- (7,0);
\draw[dashed] (10,6) -- (10,0);
\fill (0,0) circle [radius=2pt];
\fill (7,0) circle [radius=2pt];
\fill (10,0) circle [radius=2pt];
\fill (change) circle [radius=2pt];
\fill (time) circle [radius=2pt];
\draw[<->] (0,-5mm) -- node[fill=white] {
\R{שעות}
$3/2$} (time |- 0,-5mm);
\draw[<->] (7,-5mm) -- node[fill=white] {
\R{שעה}
$1/2$} (10,-5mm);
\draw[<->] (0,-10mm) -- node[fill=white] {$t$} (7,-10mm);
\end{tikzpicture}
\end{center}


נסמן:
$=v_1$
מהירות התחלתית של מכונית
$1$,
$=v_2$
מהירות מכונית
$2$,
$=t$
זמן נסיעה של מכונית
$2$
מעיר א' עד לעיר ב'.

נתון:
$v_1 = v_2+25$.

\paragraph{סעיף א}
נבתוב משוואות עבור המרחקים של שתי המכוניות:
\begin{eqnarray*}
v_1\cdot\frac{3}{2} + \frac{v_1}{2}\left(\left(t-\frac{3}{2}\right)+\frac{1}{2}\right) &=& 300\\
v_2t &=& 300\,.
\end{eqnarray*}
נציב ביטויים עבור
$v_1,t$
במשוואה הראשונה:
$v_1 = v_2+25$,
$t = 300/v_2$.
נקבל משוואה ריבועית ב-
$v_2$:
\[
v_2^2 - 125v_2 + 3750 = 0\,.
\]
השורשים הם
$50,75$
ולפי השאלה
$v_2>60$
כך שיש לבחור
$75=v_2$
קמ"ש.

\newpage

\paragraph{סעיף ב}

\begin{center}
\selectlanguage{english}
\begin{tikzpicture}
\draw (0,0) node[left] {
\R{א}
} -- (10,0);
\draw (0,0) -- node[left] {
\R{ק"מ}
$300$} (0,6) node[left] {
\R{ב}
};
\draw[dashed] (0,6) -- (10,6);
\draw[thick,name path=car1] (0,0) -- (3,4) coordinate (change) -- node[above left,near start] {$1$
\R{מכונית}
} (10,6);
\draw[thick,name path=car2] (0,0) -- node[right,xshift=32pt,yshift=20pt] {$2$
\R{מכונית}
} (8,6);
\path [name path=time1] (1.2,0) -- (1.2,6);
\path [name path=time2] (5,0) -- (5,6);
\path [name intersections={of=car1 and time1,by=meeting11}];
\path [name intersections={of=car1 and time2,by=meeting12}];
\path [name intersections={of=car2 and time1,by=meeting21}];
\path [name intersections={of=car2 and time2,by=meeting22}];
\draw[thick] (meeting11) -- (meeting21);
\draw[thick] (meeting12) -- (meeting22);
\draw[dashed] (meeting21) |- coordinate (t1) (0,0);
\draw[dashed] (meeting22) |- coordinate (t2) (0,0);
\draw[dashed] (change) |- coordinate (time) (0,0);
\fill (0,0) circle [radius=2pt];
\fill (t1) circle [radius=2pt];
\fill (t2) circle [radius=2pt];
\fill (change) circle [radius=2pt];
\fill (time) circle [radius=2pt];
\draw[<->] (0,-5mm) -- node[fill=white] {$t_1$} (t1 |- 0,-5mm);
\draw[<->] (0,-10mm) -- node[fill=white] {
\R{שעות}
$3/2$} (time |- 0,-10mm);
\draw[<->] (0,-15mm) -- node[fill=white] {$t_2$} (t2 |- 0,-15mm);
\end{tikzpicture}
\end{center}

הקווים האנכיים הכלואים בין הקווים של שתי המכוניות מסמנים מרחק של
$12.5$
ק"מ. קו אחד הוא לפני שינוי המהירות בזמן
$t_1$
מתחילת הנסיעה וקו שני לאחר שינוי המהירות בזמן
$t_2$
מתחילת הנסיעה.

\smallskip

בסעיף א' חישבנו
$v_2=75$
ולכן
$v_1=v_2+25=100$.

\smallskip

נכתוב את המשוואות עבור הפרשי המרחקים:
\begin{eqnarray*}
100t_1 - 75t_1 &=& 12.5\\
\left(100\cdot \frac{3}{2} + 50\left(t_2-\frac{3}{2}\right)\right) - 75t_2&=& 12.5\,.
\end{eqnarray*}
הפתרונות הם
$1/2, 5/2$
שעות.

\newpage

%%%%%%%%%%%%%%%%%%%%%%%%%%%%%%%%%%%%%%%%%%%%%%%%%%%%%%%%%%%%%%%%

\section*{קיץ תשע"ז מועד א}

\begin{center}
\selectlanguage{english}
\includegraphics[width=.8\textwidth]{summer-2017a-1}
\end{center}
\begin{center}
\selectlanguage{english}
\begin{tikzpicture}
\draw (0,0) node[below] {\p{08:00}} -- (11.25,0) node[below] {\p{11:45}};
\draw (0,0) -- (0,6);
\draw[thick,name path=noga] (0,0) -- (6,1.5) -- (9,3) -- (10.5,4.5) --  node[right] {
\R{נגה}
} (11.25,6);
\draw[dashed] (0,6) -- +(11.25,0);
\draw[dashed] (0,4.5) -- +(10.5,0);
\draw[dashed] (0,3) -- +(9,0);
\draw[dashed] (0,1.5) -- +(6,0);
\draw[dashed] (6,0) -- (6,1.5);
\draw[dashed] (9,0) -- (9,3);
\draw[dashed] (10.5,0) -- (10.5,4.5);
\draw[dashed] (11.25,0) -- (11.25,6);
\fill (0,0) circle [radius=2pt];
\fill (6,1.5) circle [radius=2pt];
\fill (9,3) circle [radius=2pt];
\fill (10.5,4.5) circle [radius=2pt];
\fill (11.25,6) circle [radius=2pt];
\path (0,0) -- node[left] {$x$} (0,1.5) -- node[left] {$x$} (0,3) -- node[left] {$x$} (0,4.5) -- node[left] {$x$} (0,6);
\draw[thick,name path=dan] (5.25,0) node[below] {\p{09:45}} --  node[left,xshift=16pt,yshift=20pt] {
\R{דניאל}
} (11.25,6);
\path [name intersections={of=noga and dan,by=meeting}];
\draw[dashed] (meeting) |- coordinate (time) (0,0);
\fill (5.25,0) circle [radius=2pt];
\fill (meeting) circle [radius=2pt];
\fill (time) circle [radius=2pt];
\draw[<->] (0,-7mm) -- node[below] {$t_1$} +(6,0);
\draw[<->] (6,-7mm) -- node[below] {$t_2$} +(3,0);
\draw[<->] (6,3mm) -- node[above] {$t$} (time |- 6,3mm);
\draw[<->] (9,-7mm) -- node[below] {$t_3$} +(1.5,0);
\draw[<->] (10.5,-7mm) -- node[below] {$t_4$} +(.75,0);
\end{tikzpicture}
\end{center}

נסמן:
$=x$
מרחק של מקטע,
$=t_1,t_2,t_3,t_4$
זמני רכיבה של נגה במקטעים.

נתון: 
$=40$
המהירות במקטע האחרון, לכן המהירויות של המקטעים האחרים הן
$5,10,20$.

\paragraph{סעיף א}
נסכם את הזמנים של המקטעים:
\[
\left(\frac{x}{5}+\frac{x}{10}+\frac{x}{20}+\frac{x}{40}\right) = \frac{15}{4}\,.
\]
הפתרון הוא
$x=10$
ולכן אורך המסלול הוא
$40=4\cdot 10$
ק"מ.
\newpage

\paragraph{סעיף ב}
המהירות של דניאל היא 
$40/2=20$
קמ"ש.

\smallskip

נגה עוברת
$10$
ק"מ בכל מקטע. מה המרחק שעובר דניאל עד סוף המקטע הראשון?

$t_1 = 10/5 = 2$
כך שסוף המקטע הוא ב- 
\L{\p{10:00}}.
דניאל רוכב רבע שעה מ-
\L{\p{09:45}}
ועד
\L{\p{10:00}}
ולכן המרחק שהוא עובר הוא רק
$20\cdot\frac{1}{4} = 5$
ק"מ והמפגש לא התקיים במקטע הראשון.

\smallskip

מתי נגה מגיעה לסוף המקטע השני?
$t_2=10/10 =1$
כך שסוף המקטע הוא ב
\L{\p{11:00}}.
בשעה ורבע בין 
\L{\p{09:45}}
ל
\L{\p{11:00}}
דניאל רוכב
$20\cdot \frac{5}{4}=25$
ק"מ, מרחק גדול מנגה, והמפגש מתקיים במקטע השני.

\smallskip

נסמן
$=t$
הזמן בתוך מקטע
$2$
עד למפגש, ונכתוב משוואה למרחקים השווים של נגה ודניאל:
\[
10 + 10t = 5 + 20t\,.
\]
הפתרון הוא
$t=1/2$
ושעת המפגש היא
\L{\p{10:30}}.

\newpage
%%%%%%%%%%%%%%%%%%%%%%%%%%%%%%%%%%%%%%%%%%%%%%%%%%%%%%%%%%%%%%%%

\section*{קיץ תשע"ז מועד ב}

\begin{center}
\selectlanguage{english}
\includegraphics[width=.9\textwidth]{summer-2017b-1}
\end{center}

\begin{center}
\selectlanguage{english}
\begin{tikzpicture}
\draw[name path=xaxis] (0,0) node[below left] {$A$} -- (10,0);
\draw (0,0) -- (0,6) node[above left] {$B$};
\draw[dashed] (0,6) -- (10,6);
\draw[thick,name path=raft] (0,0) -- node[left,near start,xshift=-4pt] {
\R{רפסודה}
} (10,4.5);
\draw[thick,name path=boat1] (0,6) -- node[right,near start,xshift=2pt] {
\R{סירה}
} (7,0);
\draw[thick,name path=boat2] (7,0) -- (10,6);
\path [name intersections={of=boat1 and raft,by=meeting1}];
\draw[dashed] (meeting1) |- coordinate (time) (0,0);
\draw[dashed] (meeting1) -| coordinate (distance) (0,0);
\draw[dashed] (10,0) -- (10,6);
\path[name path=t1] (7,0) -- (7,6);
\path [name intersections={of=raft and t1,by=a}];
\path [name intersections={of=raft and boat2,by=meeting2}];
\fill (meeting1) circle [radius=2pt];
\fill (meeting2) circle [radius=2pt];
\path [name path=t2] (meeting2) |- (0,0);
\path [name intersections={of=xaxis and t2,by=t2x}];
\fill (time) circle [radius=2pt];
\fill (0,0) circle [radius=2pt];
\fill (7,0) circle [radius=2pt];
\fill (10,6) circle [radius=2pt];
\fill (10,4.5) circle [radius=2pt];
\fill (0,6) circle [radius=2pt];
\fill (distance) circle [radius=2pt];
\draw[<->] (-.4,0) -- node[fill=white] {$d_r$} (distance -| -.4,0);
\draw[<->] (distance -| -.4,0) -- node[fill=white] {$d_s$} (-.4,6);
\draw[<->] (-.9,0) -- node[fill=white] {$d$} (-.9,6);
\draw[<->] (0,-.6) -- node[fill=white] { $15/4$} (time |- 0,-.6);
\draw[<->] (10.4,4.5) -- node[fill=white] { $35$} (10.4,6);
\end{tikzpicture}
\end{center}

נסמן:
$=d$
מרחק בין שני הנמלים,
$=d_r, d_s$
מרחקי ההפלגה של הסירה והרפסודה עד למפגש הראשון,
$=v_z$
מהירות הזרם,
$=v_s$
מהירות הסירה במים עומדים.

\newpage

א. נתון:
\begin{equation}
v_z = v_s/4\,.\label{eq.speeds}
\end{equation}
במפגש הראשון:
\[
d=d_s+d_r=\frac{15}{4}(v_s-v_z) + \frac{15}{4}v_z\,.
\]
מהירות הזרם מתאפסת ומתקבל:
\begin{equation}
d = \frac{15}{4}v_s\,.\label{eq.distance}
\end{equation}
בפרק הזמן שהסירה מפליגה ל-%
$A$
ובחזרה ל-%
$B$
)מרחק של
$d+d$(,
והרפסודה מפליגה מ-%
$A$
ומגיעה "כמעט" לנמל
$B$.
נשווה זמנים:
\[
\frac{d}{v_s-v_z} + \frac{d}{v_s+v_z} = \frac{d-35}{v_z}\,.
\]
מנוסחה~%
)\ref{eq.speeds}(
נציב עבור 
$v_z$,
מנוסחה~%
)\ref{eq.distance}(
נציב עבור
$d$,
ונקבל
$v_s=20$.
מאותן נוסחאות נקבל
$v_z=5$
ו-%
$d=75$.

\bigskip

\begin{center}
\selectlanguage{english}
\begin{tikzpicture}
\draw[name path=xaxis] (0,0) node[below left] {$A$} -- (10,0);
\draw (0,0) -- (0,6) node[above left] {$B$};
\draw[dashed] (0,6) -- (10,6);
\draw[thick,name path=raft] (0,0) -- node[left,near start,xshift=-4pt] {
\R{רפסודה}
} (10,4.5);
\draw[thick,name path=boat1] (0,6) -- node[right,near start,xshift=2pt] {
\R{סירה}
} (7,0);
\draw[thick,name path=boat2] (7,0) -- (10,6);
\path [name intersections={of=boat1 and raft,by=meeting1}];
\draw[dashed] (10,0) -- (10,6);
\path[name path=t1] (7,0) -- (7,6);
\path [name intersections={of=raft and t1,by=a}];
\path [name intersections={of=raft and boat2,by=meeting2}];
\fill (meeting2) circle [radius=2pt];
\fill (a) circle [radius=2pt];
\draw[<->] (7,.15) -- node[fill=white] {$d'$} (a);
\path [name path=t2] (meeting2) |- (0,0);
\fill (a -| meeting2) circle [radius=2pt];
\path [name intersections={of=xaxis and t2,by=t2x}];
\draw[<->] (a -| meeting2) -- node[fill=white,right,xshift=2pt] {$d''$} (meeting2);
\draw[dashed] (a -| meeting2) -- (t2x);
\draw[dashed] (a) -- (a -| meeting2) coordinate (one);
\fill (t2x) circle[radius=2pt];
\fill (0,0) circle [radius=2pt];
\fill (7,0) circle [radius=2pt];
\fill (10,6) circle [radius=2pt];
\fill (10,4.5) circle [radius=2pt];
\draw[<->] (0,-.4) -- node[fill=white] {$t_1$} (7,-.4);
\draw[<->] (7,-.4) -- node[fill=white] {$t_2$} (7,-.4 -| t2x);
\end{tikzpicture}
\end{center}

ב. נסמן:
$=t_1$
הזמן שהסירה מפליגה ל-%
$A$,
$=t_2$
הזמן שהסירה מפליגה מ-%
$A$
למפגש השני,
$=d'$
המרחק שהרפסודה מפליגה בזמן
$t_1$,
$=d''$
המרחק שהרפסודה מפליגה בזמן
$t_2$.

\smallskip
מהפלגה הסירה:
\[
t_1=\frac{d}{v_s-v_z}=\frac{75}{20-5}=5\,,
\]
ומהפלגה הרפסודה:
$d'=v_zt_1=5\cdot 5=25$.

בפרק הזמן 
$t_2$
הסירה מפליגה מרחק
$d'+d''$
והרפסודה מפליגה מרחק
$d''$.
נשווה זמנים:
\[
t_2=\frac{d'+d''}{v_s+v_z} = \frac{d''}{v_z},\;\;\; \frac{25+d''}{25}=\frac{d''}{5}\,.
\]
נקבל
$d''=25/4$
ו-%
$t_2=d''/v_z= 5/4$.

השאלה מבקשת את זמן ההפלגה של הרפסודה מנמל
$A$
ועד למפגש השני:
\[
t_1+t_2=5+\frac{5}{4}=\frac{25}{4}\,.
\]

\newpage

%%%%%%%%%%%%%%%%%%%%%%%%%%%%%%%%%%%%%%%%%%%%%%%%%%%%%%%%%%%%%%%%

\section*{קיץ תשע"ח מועד א}

\begin{center}
\selectlanguage{english}
\includegraphics[width=\textwidth]{summer-2018a-1}
\end{center}

\begin{center}
\selectlanguage{english}
\begin{tikzpicture}
\draw (0,0) node[left] {
\R{א}
} -- (10,0);
\draw (0,0) -- (0,6) node[left] {
\R{ב}
};
\draw[dashed] (0,6) -- (10,6);
\draw[thick,name path=amir] (0,0) -- node[above left,near start] {
\R{אמיר}
} (6,6);
\draw[thick,name path=moshe] (0,6) -- node[above right,near start] {
\R{משה}
} (10,0);
\fill (0,6) circle [radius=2pt];
\fill (0,0) circle [radius=2pt];
\fill (6,6) circle [radius=2pt];
\fill (10,0) circle [radius=2pt];
\path[name intersections={of=amir and moshe,by=meeting}];
\fill (meeting) circle [radius=2pt];
\draw[dashed] (meeting) |- coordinate (meeting-time) (0,0);
\fill (meeting-time) circle [radius=2pt];
\draw[thick] (meeting-time) -- node[right,yshift=20pt,xshift=20pt] {
\R{יסמין}
} (8,6);
\fill (8,6) circle [radius=2pt];
\draw[dashed] (6,6) -- (6,0);
\draw[dashed] (8,6) -- (8,0);
\draw[dashed] (meeting) -| coordinate (meeting-distance) (0,0);
\fill (8,0) circle [radius=2pt];
\fill (meeting-distance) circle [radius=2pt];
\draw[<->] (0,-5mm) -- node[fill=white] {$t$} (meeting-time |- 0,-5mm);
\draw[<->] (meeting-time |- 0,-5mm) -- node[fill=white] {$2$} (6,-5mm);
\draw[<->] (meeting-time |- 0,-10mm) -- node[fill=white] {$8$} (10,-10mm);
\draw[<->] (meeting-time |- 0,-15mm) -- node[fill=white] {$t_y$} (8,-15mm);
\end{tikzpicture}
\end{center}

נסמן:
$=t$
הזמן עד ממפגש בין אמיר למשה,
$=t_y$
זמן הנסיעה של יסמין מעיר א לעיר ב, 
$=v_y,v_m,y_a$
המהירויות של אמיר, משה ויסמין.

\paragraph{סעיף א}\mbox{}\\

מהתרשים רואים שיש שלושה ביטויים עבור המרחק בין הערים: סכום המרחקים שנסעו אמיר ומשה עד למפגש, והמרחקים שנסעו אמיר ומשה:
\[
tv_a + tv_m = (t+2) v_a = (t+8) v_m\,.
\]
משני הביטויים הראשונים אנו מקבלים 
$\displaystyle \frac{v_a}{v_m}=\frac{t}{2}$.
נציב בשני הביטויים האחרונים:
\[
(t+2)\cdot \frac{tv_m}{2} = (t+8) v_m\,.
\]
$v_m$
מצטמצם ונקבל משוואה ריבועית עם הפתרון חיובי 
$t=4$.
המפגש התקיים בשעה 
$10:00$.


\paragraph{סעיף ב}\mbox{}\\

המרחק בין הערים הוא 
$(t+2)v_a=6v_a$.


\paragraph{סעיף ג}\mbox{}\\

מהתרשים רואים ש:
\[
2 < t_y < 8\,.
\]
זמן הוא מרחק חלקי מהירות ואת המרחק חישבנו בסעיף ב:
\[
2 < \frac{6v_a}{v_j} < 8\,,
\]
או:
\[
\frac{1}{3v_a} < \frac{1}{v_j} < \frac{4}{3v_a}\,.
\]
כיווני האי-שוויון מתחלפים עם היפוך השבר:
\[
\frac{3}{4}v_a < v_j < 3v_a\,.
\]

\newpage

%%%%%%%%%%%%%%%%%%%%%%%%%%%%%%%%%%%%%%%%%%%%%%%%%%%%%%%%%%%%%%%%

\section*{קיץ תשע"ח מועד ב}

\begin{center}
\selectlanguage{english}
\includegraphics[width=\textwidth]{summer-2018b-1}
\end{center}

\begin{center}
\selectlanguage{english}
\begin{tikzpicture}
\draw (0,0) node[left] {$0$ } -- (10,0);
\node at (-.45,0.5) {
\R{בית}
};
\draw (0,0) -- (0,6) node[left] {$500$};
\node at (-.84,5.5) {
\R{בית ספר}
};
\draw[dashed] (0,6) -- (10,6);
\draw[thick] (0,0) -- node[above left] {
\R{רננה}
} (4.5,3);
\draw[thick] (2,0) -- node[above right,xshift=10pt,near start] {
\R{אבא}
} (4.5,3);
\draw[thick] (4.5,3) -- node[above] {
\R{רננה ואבא}
} (7,3);
\draw[thick] (7,3) -- node[above right] {
\R{אבא}
} (10,0);
\draw[thick] (7,3) -- node[above left] {
\R{רננה}
} (10,6);
\fill (4.5,3) circle [radius=2pt];
\fill (0,0) circle [radius=2pt];
\fill (2,0) circle [radius=2pt];
\fill (4.5,0) circle [radius=2pt];
\fill (7,0) circle [radius=2pt];
\fill (7,3) circle [radius=2pt];
\fill (10,0) circle [radius=2pt];
\fill (10,6) circle [radius=2pt];
\draw[dashed] (10,6) -- (10,0);
\draw[dashed] (4.5,3) -- (4.5,0);
\draw[dashed] (7,3) -- (7,0);
\draw[<->] (0,-5mm) -- node[fill=white] {$180$} (2,-5mm);
\draw[<->] (2,-5mm) -- node[fill=white] {$t$} (4.5,-5mm);
\draw[<->] (4.5,-5mm) -- node[fill=white] {$120$} (7,-5mm);
\draw[<->] (7,-5mm) -- node[fill=white] {$t'=\frac{5}{3}t$} (10,-5mm);
\end{tikzpicture}
\end{center}

נסמן:
$=v$
מהירות ההליכה של רננה,
$=t$
הזמן עד למפגש בין רננה לאביה,
$=t'$
הזמן מהפרידה בין רננה לאביה ועד ששניהם מגיעים ליעדם. המרחקים שהאב עובר מהבית עד למפגש ובחזרה שווים והמהירויות שלו ידועות, לכן ניתן לחשב את
$t'$:
\begin{eqnarray*}
\frac{5}{2}t &=& \frac{3}{2}t'\\
t' &=& \frac{2}{3}\cdot\frac{5}{2}t = \frac{5}{3}t\,.
\end{eqnarray*}
בהמשך לא נשתמש במשתנה
$t'$.

\paragraph{סעיף א}\mbox{}\\

עד למפגש המרחקים שעוברים רננה ואביה שווים:
\begin{equation}
v(t+180) = \frac{5}{2}t\,.\label{eq.rnna1}
\end{equation}
אנו זקוקים לשתי משוואות עם שני הנעלמים כדי למצוא את
$v$.
אי-אפשר למצוא משוואה שניה מהנתונים מהמפגש עד ליעדים, כי המרחקים והמהירויות לא בהכרח שווים. במקום זה נמצא דרך אחרת להשוות את המרחק שעוברים רננה ואבא מהבית עד למפגש. עבור אבא נשתמש באותו ביטוי 
$\frac{5}{2}t$
שהשתמשנו במשוואה%
~\ref{eq.rnna1}.
עבור רננה נשים לב שניתן לחשב את המרחק כהפרש בין המרחק מהבית לבית הספר לבין המרחק שרננה עוברת מהמפגש ועד לבית הספר:
\begin{equation}
\frac{5}{2}t = 500 - v\left(\frac{5}{3}t\right)\,.\label{eq.rnna2}
\end{equation}
ממשוואה%
~\ref{eq.rnna1}
ניתן למצוא משוואה עבור 
$t$:
\begin{equation}
t = \frac{360v}{5-2v}\,.\label{eq.rnna3}
\end{equation}
נציב את משוואה%
~\ref{eq.rnna3}
ב-%
~\ref{eq.rnna2}:
\[
500 - \frac{5}{3}v\left(\frac{360v}{5-2v}\right) = \frac{5}{2} \left(\frac{360v}{5-2v}\right)
\]
נחלק ב-%
$100$:
\[
5 - \frac{6v^2}{5-2v} = \frac{9v}{5-2v}\,
\]
ונקבל משוואה ריבועית עבור
$v$:
\[
6v^2 + 19v - 25 = 0\,.
\]
ניתן להשתמש בנוסחה כדי למצוא את השורשים, אבל קל לפרק לגורמים:
\[
(v-1)(6v+25)=0\,.
\]
המהירות חייבת להיות חיובית ולכן הפרתון היחיד הוא
$v=1$.

\paragraph{סעיף ב}\mbox{}\\

ממשוואה%
~\ref{eq.rnna1}
נקבל ש-%
$t=120$,
ונסכם את פרקי הזמן על הציר האופקי בתרשים:
\[
180 + 120 + 120 + \frac{5}{3}\cdot 120 = 620
\]
שניות.

\newpage

%%%%%%%%%%%%%%%%%%%%%%%%%%%%%%%%%%%%%%%%%%%%%%%%%%%%%%%%%%%%%%%%

\begin{center}
\textbf{\huge
בעיות הספק}
\end{center}


\section*{חורף תשע"ה}

\begin{center}
\selectlanguage{english}
\includegraphics[width=\textwidth]{winter-2015-1}
\end{center}

\paragraph{סעיף א}

\begin{center}
\selectlanguage{english}
\begin{tikzpicture}
\draw (0,0) -- (10,0);
\draw (0,0) -- (0,6);
\draw[dashed] (0,6) -- (10,6);
\draw[ultra thick] (0,0) -- node[left,near end,xshift=-2mm] {$2/t_v$} (8,4.5)  coordinate (two-one);
\draw[ultra thick] (0,4.5)  -- node[left,xshift=-2mm,yshift=2mm] {$1/t_m$} (8,6) coordinate (two-one-finish);
\draw[dashed,ultra thick] (0,4.5) -- (8,4.5);
\draw[thick] (0,0) -- node[left,yshift=2mm] {$1/t_v$} (10,2.25)  coordinate (one-two);
\draw[thick] (0,2.25)  -- node[left,near start,yshift=2mm] {$2/t_m$} (10,6) coordinate (one-two-finish);
\draw[dashed] (0,2.25) -- (10,2.25);
\draw[dashed] (two-one-finish) -- (two-one-finish |- 0,0);
\draw[dashed] (one-two-finish) -- (one-two-finish |- 0,0);
\draw[<->] (0,-.6) -- node[fill=white] {$T$} (two-one-finish |- 0,-.6);
\draw[<->] (0,-1.2) -- node[fill=white] {$1.25 T$} (one-two-finish |- 0,-1.2);
\end{tikzpicture}
\end{center}


נסמן את הזמנים לצביעת כל הדלתות. 
$=t_v$
הזמן שלוקח צבע ותיק,
$=t_m$
הזמן שלוקח צבע מתלמד,
$=T$
הזמן שלוקח שני צבעים ותיקים וצבע מתלמד אחד.

\smallskip

\noindent\textbf{הסבר על התרשים:}
אמנם הצבעים עובדים במקביל אבל מבחינת התרשים הם
\textbf{מחלקים}
את עבודה )הציר האנכי(. לכן בתרשים מסומן כאילו שצבע )או זוג צבעים( מסיים את חלקו בעבודה ואחר כך הצבע )או הזוג( השני מתחיל את חלקו. מה שחשוב לראות הוא ששני החלקים מסתכמים ליחידת העבודה המליאה. השתמשתי בקווים בעובי שונה ובצבע שונה כדי לבדל את שני ההרכבים: שני ותיקים ומתלמד לעומת ותיק ושני מתלמדים.

ההספקים מתקבלים מעבודה חלקי זמן ורשומים כשיפועים על התרשים.  אפשר להתייחס לזוג צבעים כצבע אחד עם הספק כפול.

\smallskip

מהתרשים רואים ששני ההרכבים סיימו את העבודה ומכאן המשוואה:
\[
\frac{2}{t_v}T + \frac{1}{t_m} T = \frac{1}{t_v} \cdot 1.25T + \frac{2}{t_m} \cdot 1.25 T\,.
\]
נחלק ב-
$T$
)שבעצם מיותר כי כל יחידת זמן, אפילו
$1$,
מתאימה(, נכפיל ב-
$t_m$,
ונקבל:
\[
\frac{t_m}{t_v}=2\,.
\]

\paragraph{סעיף ב}

\begin{center}
\selectlanguage{english}
\begin{tikzpicture}
\draw (0,0) -- (10,0);
\draw (0,0) -- (0,6);
\draw[dashed] (0,6) -- (10,6);
\draw[ultra thick] (0,0) -- node[left,near end,xshift=-2mm] {$2/t_v$} (8,4.5)  coordinate (two-one);
\draw[ultra thick] (0,4.5)  -- node[left,xshift=-2mm,yshift=2mm] {$1/t_m$} (8,6) coordinate (two-one-finish);
\draw[dashed,ultra thick] (0,4.5) -- (8,4.5);
\draw[thick] (0,0) -- node[left,yshift=2mm] {$1/t_v$} (8,2.25)  coordinate (one-two);
\draw[thick] (0,2.25)  -- node[left,near start,yshift=2mm] {$n/t_m$} (8,6) coordinate (one-two-finish);
\draw[dashed,blue] (0,2.25) -- (8,2.25);
\draw[dashed] (two-one-finish) -- (two-one-finish |- 0,0);
\draw[<->] (0,-.6) -- node[fill=white] {$1$} (two-one-finish |- 0,-.6);
\end{tikzpicture}
\end{center}

הפעם נשתמש ב-
$1$
כיחידת זמן. העבודה של שני ההרכבים שווה ולכן:
\[
\frac{2}{t_v} + \frac{1}{t_m} = \frac{1}{t_v} + \frac{n}{t_m}\,.
\]
נכפיל ב-
$t_m$,
נשתמש ביחס שחישבנו בסעיף א' ונקבל:
\[
2=\frac{t_m}{t_v}=n-1\,,
\]
והתשובה היא
$n=3$.

\newpage
%%%%%%%%%%%%%%%%%%%%%%%%%%%%%%%%%%%%%%%%%%%%%%%%%%%%%%%%%%%%%%%%

\section*{קיץ תשע"ו, מועד ב'}

\begin{center}
\selectlanguage{english}
\includegraphics[width=\textwidth]{summer-2016b-1}
\end{center}

\paragraph{סעיף א}

\begin{center}
\selectlanguage{english}
\begin{tikzpicture}
\draw (0,0) -- (10,0);
\draw (0,0) -- (0,6);
\draw[dashed] (0,6) -- (10,6);
\draw[thick,name path=gal] (0,0) -- node[right,near end,xshift=3mm] {
\R{גל}
} node[right,xshift=7mm] {$\displaystyle\frac{1}{7}$}(10,6);
\draw[thick,name path=danny] (1.2,0) -- node[left,near end,xshift=-3mm] {
\R{דני}
} node[left,yshift=3mm] {$\displaystyle\frac{1}{(1-t)+4}$} (7,6);
\draw[dashed] (7,6) -- (7,0);
\draw[dashed] (10,6) -- (10,0);
\path [name intersections={of=gal and danny,by=inter}];
\fill (inter) circle [radius=2pt];
\draw[dashed] (inter) -- (inter |- 0,0);
\draw[dashed] (inter) -- (inter -| 0,0);
\node[below] at (0,0) {\p{0800}};
\node[below] at (1.2,0) {$t$};
\node[below] at (inter |- 0,0) {\p{0900}};
\node[below] at (7,0) {\p{1300}};
\node[below] at (10,0) {\p{1500}};
\end{tikzpicture}
\end{center}

נסמן:
$=t$
הזמן שדני התחיל בהרכבה.

\smallskip

ההספקים הרשומים על התרשים מתקבלים מהנתונים על הזמן להשלמת ההרכבה.

\smallskip

נתון שבשעה 
\L{\p{0900}}
שניהם סיימו להרכיב אותו כמות של מחשבים:
\[
\frac{1}{7}\cdot 1 = \frac{1}{5-t} \cdot (1-t)\,.
\]
מתקבל שדני התחיל לעבוד
$\displaystyle\frac{1}{3}$
שעה לאחר
\L{\p{0800}}.

\paragraph{סעיף ב}

\begin{center}
\selectlanguage{english}
\begin{tikzpicture}
\draw (0,0) -- (10,0);
\draw (0,0) -- (0,6);
\draw[dashed] (0,6) -- (10,6);
\draw[thick] (0,0) -- node[right,near end,xshift=2mm,yshift=-2mm] {
\R{גל}
} node[right,xshift=6mm,yshift=-3mm] {$\displaystyle\frac{1}{7}$}(8,2.5);
\draw[thick] (0,0) -- node[left,near end,xshift=-3mm] {
\R{דני}
} node[left,xshift=-2mm,yshift=3mm] {$\displaystyle\frac{3}{14}$} (8,6);
\draw[dashed] (8,6) -- (8,0);
\draw[<->] (0,-.6) -- node[fill=white] {$T$} (8,-.6);
\draw[<->] (8.6,0) -- node[fill=white] {$w_g$} (8.6,2.5);
\draw[<->] (9.2,0) -- node[fill=white] {$w_d$} (9.2,6);
\end{tikzpicture}
\end{center}
נסמן: 
$=T$
הזמן ששניהם עבדו ביום השני. על התרשים סימנו גם את כמות העבודה שעשה כל אחד מהם:
$=w_g$
העבודה של גל,
$=w_d$
העבודה של דני.

\smallskip

מסעיף א' אנו יודעים מתי דני התחיל לעבוד ביום הראשון וניתן לחשב שההספק שלו:
\[
\frac{1}{\left(1-\frac{1}{3}\right)+4}=\frac{3}{14}\,.
\]

\smallskip

נתון שהם סיימו אותה כמות עבודה כמו היום הראשון כאשר כל אחד סיים יחידה שלמה של עבודה. מתקבלת המשוואה:
\[
1+1=w_g+w_d=\frac{1}{7}T + \frac{3}{14}T\,,
\]
והפתרון הוא 
$T=\displaystyle \frac{28}{5}$.

\newpage

%%%%%%%%%%%%%%%%%%%%%%%%%%%%%%%%%%%%%%%%%%%%%%%%%%%%%%%%%%%%%%%%

\section*{חורף תשע"ז}

\begin{center}
\selectlanguage{english}
\includegraphics[width=.9\textwidth]{winter-2017-1}
\end{center}

\paragraph{סעיף א}

\begin{center}
\selectlanguage{english}
\begin{tikzpicture}[scale=.8]
\draw (0,0) -- (10,0);
\draw (0,0) -- (0,6);
\draw[dashed] (0,6) -- (10,6);
\draw[thick] (0,0) -- node[right,near end,xshift=2mm,yshift=-2mm] {
\R{ב}
} node[right,xshift=12mm,yshift=4mm] {$\displaystyle\frac{1}{2m}$}(9,6);
\draw[thick] (0,0) -- node[left,near end,xshift=-3mm] {
\R{א}
} node[left,xshift=-2mm,yshift=3mm] {$\displaystyle\frac{1}{m}$} (4.5,6);
\draw[dashed] (9,6) -- (9,0);
\draw[dashed] (4.5,6) -- (4.5,0);
\draw[<->] (0,-.6) -- node[fill=white] {$m$} (4.5,-.6);
\draw[<->] (0,-1.2) -- node[fill=white] {$2m$} (9,-1.2);
\end{tikzpicture}
\end{center}
כאשר שני הצינורות פתוחים, ההספק הכולל הוא סכום ההספקים של הצינורות. לפי הנתונים:
\[
1/\left(\frac{1}{m}+\frac{1}{2m}\right) > 4\,,
\]
כך ש-
$m>6$.

\begin{center}
\selectlanguage{english}
\begin{tikzpicture}
\draw (0,0) -- (10,0);
\draw (0,0) -- (0,5);
%\draw[dashed] (0,6) -- (10,6);
\draw[thick] (4,0) -- node[above,near end,yshift=2mm] {
\R{ב}
} node[above,near start,yshift=2mm] {$\displaystyle\frac{1}{2m}$}(8,1);
\draw[thick] (0,0) -- node[left,near end,xshift=-3mm] {
\R{א}
} node[left,xshift=-2mm,yshift=3mm] {$\displaystyle\frac{1}{m}$} (8,4);
\draw[dashed] (8,4) -- (8,0);
\draw[dashed] (4,2) -- (4,0);
\draw[<->] (0,-.6) -- node[fill=white] {$2$} (4,-.6);
\draw[<->] (0,-1.2) -- node[fill=white] {$4$} (8,-1.2);
\draw[<->] (9,0) -- node[fill=white] {$w_a$} (9,4);
\draw[<->] (8.5,0) -- node[fill=white] {$w_b$} (8.5,1);
\end{tikzpicture}
\end{center}
נסמן:
$=w_a$
כמות המים שמילא צינור א',
$=w_b$
כמות המים שמילא צינור ב'.

\smallskip

כמות המים בבריכה לאחר ארבע שעות שווה לסכום הכמויות שכל צינור מילא והיא לפחות מחצית הבריכה:
\[
w_a + w_b = \frac{1}{m}\cdot 4 + \frac{1}{2m}\cdot 2 > \frac{1}{2}\,.
\]
מכאן,
$m<10$.

\paragraph{סעיף ב}

\begin{center}
\selectlanguage{english}
\begin{tikzpicture}[scale=.9]
\draw (0,0) -- (10,0);
\draw (0,0) -- (0,6);
\draw[dashed] (0,6) -- (10,6);
\draw[thick] (0,3) -- node[below right,xshift=10mm,yshift=-4mm] {$\displaystyle\frac{1}{m}-\frac{1}{2m}=\frac{1}{2m}$} (2,3.5);
\draw[->] (2.2,2.15) -- +(140:1.6cm);
\draw[thick] (2,3.5) -- node[left,xshift=-4mm,yshift=3mm] {$\displaystyle\frac{1}{m}+\frac{1}{2m}=\frac{3}{2m}$} (7,6);
\draw[dashed] (2,3.5) -- (2,0);
\draw[dashed] (2,3.5) -- (0,3.5);
\draw[dashed] (7,6) -- (7,0);
\draw[<->] (0,-.6) -- node[fill=white] {$1$} (2,-.6);
\draw[<->] (2,-1.2) -- node[fill=white] {$2.5$} (7,-1.2);
\node at (-.4,3) {$\displaystyle\frac{1}{2}$};
\end{tikzpicture}
\end{center}
כדי למלא את הבריכה, מתחילים ממחצית הכמות, מוסיפים )מחסירים כי שלילי( את הכמות של השעה הראשונה, ומוסיפים את הכמות מהתקופה השניה של שעתיים וחצי:
\[
\frac{1}{2} + \frac{1}{2m}\cdot 1 + \frac{3}{2m}\cdot 2.5 = 1\,.
\]
הפתרון הוא
$m=8.5$.

%%%%%%%%%%%%%%%%%%%%%%%%%%%%%%%%%%%%%%%%%%%%%%%%%%%%%%%%%%%%%%%%

\newpage

\section*{חורף תשע"ח}

\begin{center}
\selectlanguage{english}
\includegraphics[width=1.1\textwidth]{winter-2018-1}
\end{center}

\begin{center}
\selectlanguage{english}
\begin{tikzpicture}[scale=1.25]
\draw (0,0) -- (10,0);
\draw (0,0) -- (0,6);
\node at (-.5,1) {$\frac{1}{6}V_1$};
\node at (-.5,3) {$\frac{1}{2}V_1$};
\node at (-.5,6) {$V_1$};

\fill (0,0) circle [radius=2pt];
\draw[dashed] (0,1) -- (1,1);
\draw (0,0) -- node[below,xshift=4pt] {$4x$} (1,1);
\fill (1,1) circle [radius=2pt];

\draw[dashed] (0,3) -- (4,3);
\draw (1,1) -- node[below,xshift=4pt] {$3x$} (4,3);
\fill (4,3) circle [radius=2pt];

\draw[dashed] (0,6) -- (10,6);
\draw (4,3) -- node[below,xshift=4pt] {$x$} (10,6);
\fill (10,6) circle [radius=2pt];

\draw[dashed] (1,1) -- (1,0);
\draw[dashed] (4,3) -- (4,0);
\draw[dashed] (10,6) -- (10,0);

\draw[<->] (10.5,0) -- node[fill=white] {$V_2$} (10.5,5.5);

\fill (1,0) circle [radius=2pt];
\draw (1,0) -- node[below,xshift=4pt] {$x$} (4,1);
\fill (4,1) circle [radius=2pt];
\draw (4,1) -- node[below,xshift=4pt] {$3x$} (10,5.5);
\fill (10,5.5) circle [radius=2pt];

\draw[<->] (0,-.5) -- node[fill=white] {$t_1$} (1,-.5);
\draw[<->] (1,-.5) -- node[fill=white] {$t_2$} (4,-.5);
\draw[<->] (4,-.5) -- node[fill=white] {$t_3$} (10,-.5);


\end{tikzpicture}
\end{center}


נסמן:
$=x$
קצב המילוי של כל צינור בנפרד. 
$=t_1, t_2, t_3$
פרקי הזמן לכל חלוקה של הצינורות בין הבריכות. הקו העליון בתרשים מתאר את המילוי של בריכה א', והקו התחתון מתאר את מילוי של בריכה ב'.

נכתוב את המשוואות ההספק עבור בריכה א':
\[
4x t_1 = \frac{1}{6}V_1,\quad 3x t_2 = (\frac{1}{2}-\frac{1}{6})V_1,\quad xt_3=(1-\frac{1}{2})V_1\,,
\]
ונשתמש בהן כדי לחשב את פרקי הזמן כתלות בנפח בבריכה:
\[
t_1 = \frac{V_1}{24x},\quad t_2 = \frac{V_1}{9x},\quad t_3=\frac{V_1}{2x}\,.
\]
מהתרשים אנו רואים שאפשר לבטא את הנפח של
$V_2$
כסכום של שני חלקים נפרדים: הראשון בפרק הזמן
$t_2$
והשני בפרק הזמן
$t_3$.
כאשר נציב את המשוואות שקבלנו עבור בפרקי הזמן, נקבל את הנפח של
$V_2$
כתלות ב-%
$V_1$
בלבד, כי המשתנה 
$x$ מצטמצם:
\[
V_2 = xt_2 + 3xt_3 = \frac{x V_1}{9x} + \frac{3x V_1}{2x}\ = \frac{29}{18}V_1\,.
\]
מכאן קבלנו את התשובה הדרושה, היחס בין שני הנפחים:
\[
\frac{V_1}{V_2} = \frac{18}{29}\,.
\]
\begin{itemize}
\item 
לא היה צורך במידע על ההספק בפרק הזמן הראשון כאשר רק בריכה א' מתמלאת.
\item
מהתשובה אנו רואים שהנפח של בריכה ב' גדול מהנפח של בריכה א'. לא ידענו זאת לפני שפתרנו את השאלה, והתרשים מראה את המצב ההפוך. אין לזה חשיבות. מטרת התרשים היא להציג את התסריט כדי שנוכל לכתוב את המשוואות הנכונות. כאן, חשוב לשים לב שפרק הזמן הראשון לא נחוץ לפתרון, ושהמילוי של בריכה ב' מתבצע בשני שלבים שזמנם זהים לזמנם של שלבי המילוי של בריכה א'.

\end{itemize}

\end{document}
