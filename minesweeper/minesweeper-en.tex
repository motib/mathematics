\documentclass[11pt,a4paper]{article}

\usepackage{mathpazo}
\usepackage{microtype}
\usepackage{url}
\usepackage{verbatim}
\usepackage{graphicx}

\textwidth=15cm
\textheight=23cm
\topmargin=1cm
\headheight=0pt
\oddsidemargin=2em
\headsep=0pt
\renewcommand{\baselinestretch}{1.1}
\setlength{\parskip}{0.5\baselineskip plus 1pt minus 1pt}
\parindent=0pt

\newlength{\lng}
\setlength{\lng}{1pt}
\unitlength=\lng
\newcommand{\lrg}[0]{\Large\sf} 
\newcommand{\mdm}[0]{\normalsize\sf}

\newsavebox{\grid}
\sbox{\grid}{
\thicklines
  \put(0,0){\framebox(120,120){}}           % Outer box
  \multiput(20,0)(20,0){5}{\line(0,1){120}} % Vertical grid lines
  \multiput(0,20)(0,20){5}{\line(1,0){120}} % Horizontal grid lines
  \put(-20,0){\makebox(20,20){\sf 1}}
  \put(-20,20){\makebox(20,20){\sf 2}}
  \put(-20,40){\makebox(20,20){\sf 3}}
  \put(-20,60){\makebox(20,20){\sf 4}}
  \put(-20,80){\makebox(20,20){\sf 5}}
  \put(-20,100){\makebox(20,20){\sf 6}}
  \put(0,120){\makebox(20,20){\sf 1}}
  \put(20,120){\makebox(20,20){\sf 2}}
  \put(40,120){\makebox(20,20){\sf 3}}
  \put(60,120){\makebox(20,20){\sf 4}}
  \put(80,120){\makebox(20,20){\sf 5}}
  \put(100,120){\makebox(20,20){\sf 6}}
}

% A puzzle is a grid filled with zero and the parameter n
\newcommand{\puz}[1]{
  \thicklines
  \usebox{\grid}
  \multiput(20,20)(0,60){2}{
    \multiput(0,0)(20,0){4}{\makebox(20,20){\lrg #1}}
  }
  \multiput(20,40)(0,20){2}{
    \multiput(0,0)(60,0){2}{\makebox(20,20){\lrg #1}}
    \multiput(20,0)(20,0){2}{\makebox(20,20){\lrg 0}}
  }
}

% Save boxes for puzzles with 2's and 3's
\newsavebox{\puztwo}
\sbox{\puztwo}{\puz{2}}

\newsavebox{\puzthree}
\sbox{\puzthree}{\puz{3}}

% Commands for symbols
\newcommand{\mine}[0]{\makebox(20,20){\rule{9\lng}{9\lng}}}
\newcommand{\ques}[0]{\makebox(20,20){\lrg ?}}
\newcommand{\open}[0]{\put(10,10){\circle{10}}}
\newcommand{\incon}[0]{\put(5,5){\line(1,1){10}}\put(5,15){\line(1,-1){10}}}
\newcommand{\fulfil}[0]{\put(4,4){\framebox(12,12){}}}
\newcommand{\smallmine}[0]{\makebox(10,10){\rule{4\lng}{4\lng}}}
\newcommand{\smallopen}[0]{\put(5,5){\circle{5}}}
\newcommand{\smallincon}[0]{\put(1,1){\line(1,1){8}}\put(1,9){\line(1,-1){8}}}

% Commands for each configuration
\newcommand{\configa}[0]{
  \usebox{\puztwo}
  \multiput(80,100)(20,0){2}{\mine}
  \put(60,100){\open}
  \multiput(100,80)(0,-20){2}{\open}
  \put(80,80){\fulfil}
  \put(80,60){\incon}
}

\newcommand{\configb}[0]{
  \usebox{\puztwo}
  \multiput(60,100)(40,0){2}{\mine}
  \put(80,100){\open}
  \multiput(100,80)(0,-20){2}{\open}
  \put(80,80){\fulfil}
  \put(80,60){\incon}
}

\newcommand{\configc}[0]{
  \usebox{\puztwo}
  \multiput(40,100)(60,0){2}{\mine}
  \multiput(60,100)(20,0){2}{\ques}
  \put(60,80){\incon}
}

\newcommand{\configd}[0]{
  \usebox{\puztwo}
  \multiput(20,100)(80,0){2}{\mine}
  \multiput(40,100)(20,0){3}{\ques}
  \multiput(40,80)(20,0){2}{\incon}
}

\newcommand{\confige}[0]{
  \usebox{\puztwo}
  \multiput(0,100)(100,0){2}{\mine}
  \multiput(20,100)(20,0){4}{\ques}
  \multiput(40,80)(20,0){2}{\incon}
}

\newcommand{\configf}[0]{
  \usebox{\puztwo}
  \multiput(60,100)(20,0){2}{\mine}
  \put(40,100){\open}
  \multiput(100,100)(0,-20){3}{\open}
  \multiput(60,80)(20,0){2}{\fulfil}
  \put(80,60){\incon}
}

\newcommand{\configg}[0]{
  \usebox{\puztwo}
  \put(80,100){\mine}
  \put(100,80){\mine}
  \put(60,100){\open}
  \put(100,100){\open}
  \put(100,60){\open}
  \put(80,80){\fulfil}
}

\newcommand{\configh}[0]{
  \usebox{\puztwo}
  \put(80,100){\mine}
  \put(100,80){\mine}
  \put(40,100){\mine}
  \put(20,100){\mine}
  \put(60,100){\open}
  \put(100,100){\open}
  \put(100,60){\open}
  \put(80,80){\fulfil}
  \put(60,80){\fulfil}
  \put(40,80){\fulfil}
  \put(20,80){\fulfil}
  \put(0,100){\open}
  \put(0,80){\open}
  \put(0,60){\open}
  \put(20,60){\incon}
}

%\newcommand{\configh}[0]{
%  \usebox{\puztwo}
%  \multiput(20,100)(60,0){2}{\mine}
%  \multiput(40,100)(20,0){2}{\ques}
%  \multiput(40,80)(20,0){2}{\incon}
%}

\newcommand{\configi}[0]{
  \usebox{\puztwo}
  \multiput(0,0)(0,100){2}{
    \multiput(40,0)(20,0){2}{\mine}
  }
  \multiput(0,20)(0,60){2}{
    \multiput(20,0)(20,0){4}{\fulfil}
  }
  \multiput(0,0)(100,0){2}{
    \multiput(0,40)(0,20){2}{\mine}
  }
  \multiput(20,0)(60,0){2}{
    \multiput(0,40)(0,20){2}{\fulfil}
  }
}

\newcommand{\configj}[0]{
  \usebox{\puzthree}
  \multiput(40,100)(20,0){3}{\mine}
  \put(60,80){\fulfil}
}

\newcommand{\configk}[0]{
  \usebox{\puzthree}
  \multiput(0,0)(0,100){2}{
    \multiput(20,0)(20,0){4}{\mine}
  }
  \multiput(0,20)(0,20){4}{
    \multiput(0,0)(100,0){2}{\mine}
  }
  \multiput(0,20)(0,60){2}{
    \multiput(40,0)(20,0){2}{\fulfil}
    \multiput(20,0)(60,0){2}{\incon}
  }
  \multiput(0,40)(0,20){2}{
    \multiput(20,0)(60,0){2}{\fulfil}
  }
}


\begin{document}
\begin{center}

\textbf{\huge Minesweeper is NP-Complete}

\bigskip
\bigskip
\bigskip

\textbf{\LARGE Moti Ben-Ari}

\bigskip

\textbf{\Large Department of Science Teaching}

\bigskip

\textbf{\Large Weizmann Institute of Science}

\bigskip

\url{http://www.weizmann.ac.il/sci-tea/benari/}

\end{center}

\bigskip
\bigskip

\begin{center}
\copyright{}\  2018 by Moti Ben-Ari.
\end{center}

This work is licensed under the Creative Commons Attribution-ShareAlike 3.0 Unported License. To view a copy of this license, visit \url{http://creativecommons.org/licenses/by-sa/3.0/} or send a letter to Creative Commons, 444 Castro Street, Suite 900, Mountain View, California, 94041, USA.

\bigskip

\begin{center}
\includegraphics[width=.2\textwidth]{../by-sa.png}
\end{center}

\thispagestyle{empty}

\newpage

This document assumes a basic knowledge NP-completeness. An overview of this topic is given in the Appendix.


Richard Kaye showed that a puzzle based on the minesweeper game is NP-complete. This document presents Kaye's result with detailed explanations.

Kaye, R. (2000) Minesweeper is NP-complete. \textit{Mathematical Intelligencer} 22(2), 9—15. \url{http://web.mat.bham.ac.uk/R.W.Kaye/minesw/}.


\section{The Minesweeper puzzle}

I will assume that you are familiar with---if not addicted to---the \emph{minesweeper game}. There is a rectangular array of hidden cells partially filled with mines, and you have to uncover all non-mined cells without stepping on a mine. Cells adjacent to mines contain a number between 0 and 8 indicating the number of adjacent mines. 

The minesweeper \emph{puzzle} is the opposite of the game: given an array in which some cells contain mines and some adjacency numbers, is there an an arrangement of mines that is \emph{consistent} with the numbers?

\section{NP-completeness}

Kaye showed that the minesweeper puzzle is NP-complete by reducing instances of SAT (satisfiability of formulas of propositional logic) to instances of the puzzle. The SAT instances are given as circuits constructed of Boolean logic gates. Configurations of mines and adjacency numbers are constructed such that they are consistent iff they represent the logic gates \texttt{not}, \texttt{and} and \texttt{or}. In addition, configurations are needed to represent the wires in the circuit: straight wires, turns, crossovers, joiners and splitters. 

\section{Minesweeper puzzles}

In this document, the cells are numbered from the lower left-hand corner right and up.

An assignment of mines is consistent if it assigns mines to some of the open cells such that number of mines adjacent to each cell equals the number in the cell.

A cell is \emph{fulfilled} if the number in the cell is the same as the number of adjacent mines. If the number of mines is less than the number in the cell, the cell is \emph{not fulfilled}. If the number of mines is greater than the number in the cell, or if there are not sufficient open cells in order to place mines so that the cell is fulfilled, the cell is said to be \emph{inconsistent}.


The following notation is used:
\begin{center}
\begin{tabular}{|c|l|}
\hline
\begin{picture}(20,22)\mine{}  \end{picture}& mine\\\hline
\begin{picture}(20,22)\open{}  \end{picture}& no mine\\\hline
\begin{picture}(20,22)\ques{}  \end{picture}& possible mine\\\hline
\begin{picture}(20,22)
  \put(0,0){\makebox(20,20){\lrg 2}}
  \put(0,0){\incon}
\end{picture}& inconsistent cell\\\hline
\begin{picture}(20,22)
  \put(0,0){\makebox(20,20){\lrg 2}}
  \put(0,0){\fulfil}
\end{picture}& fulfilled cell\\\hline
\begin{picture}(20,22)
\put(0,0){\makebox(20,20){\lrg 2}} \end{picture} & unfulfilled cell\\\hline
\end{tabular}
\end{center}


\section{A simple puzzle---a square of twos}


\begin{center}
\begin{picture}(130,140)
\put(0,0){\usebox{\puztwo}}
\end{picture}
\end{center}
We first show that a mine cannot be placed in a corner cell. By symmetry, it is sufficient to show that a mine cannot be placed in cell $(6,6)$, the upper right-hand corner. If a mine is placed in $(6,6)$, the second mine needed to fulfill $(5,5)$ cannot be placed in $(6,5)$, because cells $(6,4)$, $(5,6)$, $(4,6)$ must be open, and there is no way to fulfill $(4,5)$.

\begin{center}
\begin{picture}(130,140)
\put(0,0){\configa}
\end{picture}
\end{center}

Suppose now that the second mine to fulfill $(5,5)$ is in cell $(6,4)$.
\begin{center}
\begin{picture}(130,140)
\put(0,0){\configb}
\end{picture}
\end{center}
Two mines cannot be placed to fulfill $(4,5)$, because $(4,6)$, $(5,6)$, $(6,5)$ must be open.

This proves that a mine cannot be placed in a corner cell.

\bigskip

Next we show that there cannot be a mine in a cell adjacent to a corner cell. By symmetry, we show this for cell $(6,5)$. We can fulfill cell $(5,5)$ by placing a mine in cell $(6,4)$, but this forces $(6,6), (5,6), (4,6)$ to be open and $(5,4)$ cannot be fulfilled.
\begin{center}
\begin{picture}(130,140)
\put(0,0){\configf}
\end{picture}
\end{center}
Instead, let us try to fulfill cell $(5,5)$ by placing the second mine in cell $(5,6)$ and leaving $(6,4)$ open.
\begin{center}
\begin{picture}(130,140)
\put( 10,10){\configg}
\end{picture}
\end{center}
To fulfill cell $(5,4)$ a mine must be placed in $(6,3)$ and to fulfill cell $(5,3)$ a mine must be placed in $(6,2)$, which also fulfills $(5,2)$. This forces cells in $(6,1)$, $(5,1)$, $(4,1)$ to be open, making $(4,2)$ inconsistent.
\begin{center}
\begin{picture}(130,140)
\put(0,10){\configh}
\end{picture}
\end{center}
Since mines cannot be placed in a corner cell or in a cell adjacent to a corner cell, the only possible configuration is:
\begin{center}
\begin{picture}(130,140)
\put(0,0){\configi}
\end{picture}
\end{center}
A quick check shows that this configuration is consistent.

\section{A simple puzzle---a square of threes}

\begin{center}
\begin{picture}(130,140)
\put(0,0){\usebox{\puzthree}}
\end{picture}
\end{center}
In order to fulfull a cell containing $3$ that is not at a corner, mines must be placed in all of its adjacent open neighbors. For example, to fulfill $(5,4)$ there must be mines in $(6,3)$, $(6,4)$ and $(6,5)$.
\begin{center}
\begin{picture}(130,140)
\put(0,0){\configj}
\end{picture}
\end{center}
When mines are placed to fulfill all non-corner cells, the four corner cells $(2,2)$, $(2,5)$, $(5,2)$ and $(5,5)$ will be inconsistent. Therefore, this puzzle has no solution.
\begin{center}
\begin{picture}(130,140)
\put(10,10){\configk}
\end{picture}
\end{center}


\section{Digital Logic Circuits}

Kaye showed that CNF formulas in propositional logic, expressed as digital logic circuits, can be reduced to minesweeper puzzles. The formulas are reduced to circuits using \texttt{not} and \texttt{or} gates. (\texttt{and} gates can be constructed from \texttt{or} and \texttt{not} gates.) The connecting wires also have to be represented. We show the representation for straight wires; see Kaye's paper for the reduction for other forms of wires.

\begin{center}
\unitlength=1.1pt
\begin{picture}(290,100)
\thicklines
\put(0,80){\makebox(20,30){\sf P}}
\put(0,60){\makebox(20,30){\sf Q}}
\put(0,40){\makebox(20,30){\sf R}}
\put(40,15){\line(1,0){10}}
\put(20,75){\line(1,0){30}}
\put(20,95){\line(1,0){30}}
\put(40,95){\line(0,-1){80}}
\put(38,93){\rule{4pt}{4pt}}
\put(20,55){\line(1,0){120}}
\put(120,55){\line(0,-1){45}}
\put(118,53){\rule{4pt}{4pt}}
\put(100,20){\line(1,0){40}}
\put(120,10){\line(1,0){20}}
\put(100,90){\line(1,0){20}}
\put(120,90){\line(0,-1){20}}
\put(120,70){\line(1,0){20}}
\put(190,65){\line(1,0){20}}
\put(210,65){\line(0,-1){10}}
\put(210,55){\line(1,0){20}}
\put(190,15){\line(1,0){20}}
\put(210,15){\line(0,1){20}}
\put(210,35){\line(1,0){20}}
\put(280,45){\line(1,0){25}}
\put(50,0){\framebox(50,30){\sf Not}}
\put(50,70){\framebox(50,30){\sf And}}
\put(140,0){\framebox(50,30){\sf Or}}
\put(140,50){\framebox(50,30){\sf Or}}
\put(230,30){\framebox(50,30){\sf And}}
\end{picture}
\end{center}


\newsavebox{\wire}
\sbox{\wire}{
\thicklines
\put(0,0){\framebox(220,100){}}
\multiput(20,0)(20,0){10}{\line(0,1){100}}
\multiput(0,20)(0,20){4}{\line(1,0){220}}
\multiput(0,0)(20,0){11}{
  \put(0,0){\makebox(20,20){\lrg 0}}
  \put(0,20){\makebox(20,20){\lrg 1}}
  \put(0,60){\makebox(20,20){\lrg 1}}
  \put(0,80){\makebox(20,20){\lrg 0}}
}
\multiput(40,40)(60,0){3}{\makebox(20,20){\lrg 1}}
  \put(-20,0){\makebox(20,20){\sf 1}}
  \put(-20,20){\makebox(20,20){\sf 2}}
  \put(-20,40){\makebox(20,20){\sf 3}}
  \put(-20,60){\makebox(20,20){\sf 4}}
  \put(-20,80){\makebox(20,20){\sf 5}}
  \put(0,100){\makebox(20,20){\sf 1}}
  \put(20,100){\makebox(20,20){\sf 2}}
  \put(40,100){\makebox(20,20){\sf 3}}
  \put(60,100){\makebox(20,20){\sf 4}}
  \put(80,100){\makebox(20,20){\sf 5}}
  \put(100,100){\makebox(20,20){\sf 6}}
  \put(120,100){\makebox(20,20){\sf 7}}
  \put(140,100){\makebox(20,20){\sf 8}}
  \put(160,100){\makebox(20,20){\sf 9}}
  \put(180,100){\makebox(20,20){\sf 10}}
  \put(200,100){\makebox(20,20){\sf 11}}
}

\newsavebox{\notgate}
\sbox{\notgate}{
\thicklines
\put(0,0){\framebox(260,60){}}
\put(100,-20){
  \put(0,0){\framebox(60,100){}}
%  \multiput(20,0)(0,80){2}{\line(0,1){20}}
%  \multiput(40,0)(0,80){2}{\line(0,1){20}}
  \multiput( 0,0)(0,80){2}{
    \multiput(0,0)(20,0){3}{\makebox(20,20){\lrg 1}}
  }
}
\multiput( 0,20)( 0,20){2}{\line(1,0){260}}
\multiput(20, 0)(20, 0){4}{\line(0,1){60}}
\multiput(120,-20)(20, 0){2}{\line(0,1){100}}
\multiput(180, 0)(20, 0){4}{\line(0,1){60}}
\multiput(0,0)(0,40){2}{
  \multiput(  0,0)(20,0){5}{\makebox(20,20){\lrg 1}}
  \multiput(160,0)(20,0){5}{\makebox(20,20){\lrg 1}}
  \multiput(100,0)(40,0){2}{\makebox(20,20){\lrg 2}}
  \put(120,0){\mine}
}
\multiput(40,20)(160,0){2}{\makebox(20,20){\lrg 1}}
\multiput(100,20)(40,0){2}{\makebox(20,20){\lrg 3}}
  \put(-20,0){\makebox(20,20){\sf 1}}
  \put(-20,20){\makebox(20,20){\sf 2}}
  \put(-20,40){\makebox(20,20){\sf 3}}
  \put(0,80){\makebox(20,20){\sf 1}}
  \put(20,80){\makebox(20,20){\sf 2}}
  \put(40,80){\makebox(20,20){\sf 3}}
  \put(60,80){\makebox(20,20){\sf 4}}
  \put(80,80){\makebox(20,20){\sf 5}}
  \put(100,80){\makebox(20,20){\sf 6}}
  \put(120,80){\makebox(20,20){\sf 7}}
  \put(140,80){\makebox(20,20){\sf 8}}
  \put(160,80){\makebox(20,20){\sf 9}}
  \put(180,80){\makebox(20,20){\sf 10}}
  \put(200,80){\makebox(20,20){\sf 11}}
  \put(220,80){\makebox(20,20){\sf 12}}
  \put(240,80){\makebox(20,20){\sf 13}}
}

Notation: an open cell followed by a cell with a mine (open/mine) represents $1$, while a cell with a mine followed by an open cell (mine/open) represents $0$. 


\section{Representation of a straight wire}

\begin{center}
\begin{picture}(240,120)
\usebox{\wire}
\end{picture}
\end{center}
An input of mine/open ($0$) at $(3,1)$--$(3,2)$, results in an output of mine/open ($0$) at $(3,10)$--$(3,11)$:
\begin{center}
\begin{picture}(240,120)
\usebox{\wire}
\multiput( 0,40)(60,0){4}{\mine}
\multiput(20,40)(60,0){4}{\open}
\end{picture}
\end{center}
An input of open/mine ($1$) at $(3,1)$--$(3,2)$, results in an output of open/mine ($1$) at $(3,10)$--$(3,11)$:
\begin{center}
\begin{picture}(240,120)
\usebox{\wire}
\multiput( 0,40)(60,0){4}{\open}
\multiput(20,40)(60,0){4}{\mine}
\end{picture}
\end{center}

\section{Representation of a \texttt{not} gate}
\begin{center}
\begin{picture}(280,100)
\usebox{\notgate}
\end{picture}
\end{center}
\bigskip
An input of open/mine ($1$) at $(2,1)$--$(2,2)$ results in an output of mine/open ($0$) at $(2,12)$--$(2,13)$:
\begin{center}
\begin{picture}(280,100)
\usebox{\notgate}
\multiput(0,  20)(60,0){3}{\open}
\multiput(180,20)(60,0){2}{\open}
\multiput(20, 20)(60,0){2}{\mine}
\multiput(160,20)(60,0){2}{\mine}
\end{picture}
\end{center}
\bigskip
There are mines at cells $(1,7)$ and $(3,7)$. Therefore, if there is an mine at $(2,5)$, the $3$ at $(2,6)$ forces $(2,7)$ to be open. It follows that there must be a mine at $(2,9)$ to fulfill $(2,8)$. It is now easy to see that $(2,10)$ is open, a mine is placed in $(2,12)$ and $(2,13)$ is open.

Similarly, an input of mine/open ($0$) at $(2,1)$--$(2,2)$ results in an output of open/mine ($1$) at $(2,12)$--$(2,13)$:
\begin{center}
\begin{picture}(280,100)
\usebox{\notgate}
\multiput(180,20)(60,0){2}{\mine}
\multiput(  0,20)(60,0){3}{\mine}
\multiput( 20,20)(60,0){2}{\open}
\multiput(160,20)(60,0){2}{\open}
\end{picture}
\end{center}
\bigskip

\setlength{\lng}{2pt}
\unitlength=\lng
\newcommand{\sml}[0]{\large\sf}


\newsavebox{\orgate}
\sbox{\orgate}{
\thicklines
\multiput(30,0)(50,0){2}{\line(1,0){40}}
\put(30,10){\line(1,0){90}}
\multiput(30,20)(0,10){2}{\line(1,0){100}}
\multiput( 0,40)(0,10){4}{\line(1,0){160}}
\put(30,80){\line(1,0){100}}
\multiput(40,90)(0,10){3}{\line(1,0){30}}
\put(80,90){\line(1,0){50}}

\multiput( 0,40)(10,0){3}{\line(0,1){30}}
\put(30,0){\line(0,1){80}}
\multiput(40,0)(10,0){4}{\line(0,1){110}}
\multiput(80,0)(10,0){5}{\line(0,1){90}}
\put(130,20){\line(0,1){70}}
\multiput(140,40)(10,0){3}{\line(0,1){30}}

\multiput(0,40)(10,0){3}{\makebox(10,10){\sml 1}}
\multiput(0,60)(10,0){3}{\makebox(10,10){\sml 1}}
\multiput(130,40)(10,0){3}{\makebox(10,10){\sml 1}}
\multiput(130,60)(10,0){3}{\makebox(10,10){\sml 1}}
\multiput(40,80)(0,10){3}{\makebox(10,10){\sml 1}}
\multiput(60,80)(0,10){3}{\makebox(10,10){\sml 1}}
\put(0,50){\makebox(10,10){\sml 1}}
\multiput(30,70)(40,0){2}{\makebox(10,10){\sml 1}}
\multiput(80,50)(40,0){2}{\makebox(10,10){\sml 1}}
\multiput(80,70)(40,0){2}{\makebox(10,10){\sml 1}}
\multiput(80,80)(40,0){2}{\makebox(10,10){\sml 1}}
\multiput(30,00)(30,0){2}{\makebox(10,10){\sml 1}}
\multiput(80,00)(30,0){2}{\makebox(10,10){\sml 1}}
\multiput(120,20)(0,10){2}{\makebox(10,10){\sml 1}}
\put(50,100){\makebox(10,10){\sml 1}}
\put(150,50){\makebox(10,10){\sml 1}}

\multiput(30,60)(40,0){2}{\makebox(10,10){\sml 2}}
\multiput(80,60)(40,0){2}{\makebox(10,10){\sml 2}}
\multiput(30,10)(40,0){2}{\makebox(10,10){\sml 2}}
\multiput(30,20)(0,10){3}{\makebox(10,10){\sml 2}}
\multiput(110,40)(10,0){2}{\makebox(10,10){\sml 2}}
\multiput(40,0)(10,0){2}{\makebox(10,10){\sml 2}}
\multiput(90,0)(10,0){2}{\makebox(10,10){\sml 2}}
\multiput(40,70)(20,0){2}{\makebox(10,10){\sml 2}}
\multiput(90,80)(20,0){2}{\makebox(10,10){\sml 2}}
\put(100,50){\makebox(10,10){\sml 2}}
\put(110,10){\makebox(10,10){\sml 2}}

\put(30,50){\makebox(10,10){\sml 3}}
\put(50,70){\makebox(10,10){\sml 3}}
\put(60,10){\makebox(10,10){\sml 3}}
\put(80,10){\makebox(10,10){\sml 3}}
\put(90,40){\makebox(10,10){\sml 3}}
\put(90,60){\makebox(10,10){\sml 3}}
\put(100,80){\makebox(10,10){\sml 3}}
\put(110,20){\makebox(10,10){\sml 3}}
\put(110,60){\makebox(10,10){\sml 3}}

\put(40,30){\makebox(10,10){\sml 4}}
\put(80,40){\makebox(10,10){\sml 4}}
\put(100,30){\makebox(10,10){\sml 4}}
\put(70,40){\makebox(10,10){\sml 5}}
\put(50,50){\makebox(10,10){\sml 6}}

\multiput(40,10)(10,0){2}{\makebox(10,10){\smallmine}}
\multiput(90,10)(10,0){2}{\makebox(10,10){\smallmine}}
\multiput(40,20)(60,0){2}{\makebox(10,10){\smallmine}}
\multiput(50,30)(10,0){5}{\makebox(10,10){\smallmine}}
\put(110,30){\makebox(10,10){\smallmine}}
\multiput(40,40)(20,0){2}{\makebox(10,10){\smallmine}}
\multiput(40,60)(20,0){2}{\makebox(10,10){\smallmine}}
\multiput(90,70)(10,0){3}{\makebox(10,10){\smallmine}}

  \put(-10,0){\makebox(10,10){\sf 1}}
  \put(-10,10){\makebox(10,10){\sf 2}}
  \put(-10,20){\makebox(10,10){\sf 3}}
  \put(-10,30){\makebox(10,10){\sf 4}}
  \put(-10,40){\makebox(10,10){\sf 5}}
  \put(-10,50){\makebox(10,10){\sf 6}}
  \put(-10,60){\makebox(10,10){\sf 7}}
  \put(-10,70){\makebox(10,10){\sf 8}}
  \put(-10,80){\makebox(10,10){\sf 9}}
  \put(-10,90){\makebox(10,10){\sf 10}}
  \put(-10,100){\makebox(10,10){\sf 11}}

  \put(0,110){\makebox(10,10){\sf 1}}
  \put(10,110){\makebox(10,10){\sf 2}}
  \put(20,110){\makebox(10,10){\sf 3}}
  \put(30,110){\makebox(10,10){\sf 4}}
  \put(40,110){\makebox(10,10){\sf 5}}
  \put(50,110){\makebox(10,10){\sf 6}}
  \put(60,110){\makebox(10,10){\sf 7}}
  \put(70,110){\makebox(10,10){\sf 8}}
  \put(80,110){\makebox(10,10){\sf 9}}
  \put(90,110){\makebox(10,10){\sf 10}}
  \put(100,110){\makebox(10,10){\sf 11}}
  \put(110,110){\makebox(10,10){\sf 12}}
  \put(120,110){\makebox(10,10){\sf 13}}
  \put(130,110){\makebox(10,10){\sf 14}}
  \put(140,110){\makebox(10,10){\sf 15}}
  \put(150,110){\makebox(10,10){\sf 16}}
}

\newpage

\section*{Representation of an \texttt{or} gate}

\begin{center}
\begin{picture}(170,120)
\usebox{\orgate}
\end{picture}
\end{center}
The inputs are placed at $(6,2)$--$(6,3)$ and $(10,6)$--$(9,6)$. The result appears at $(6,14)$--$(6,15)$.

We will show that if the input is open/mine ($1$) at the left and mine/open ($0$) at the top, then the result at the right is open/mine ($1$). The reader is invited to check the solutions to the puzzle for ($0$ \texttt{or} $0$), ($1$ \texttt{or} $0$) and ($1$ \texttt{or} $1$).


The initial configuration is:
\begin{center}
\begin{picture}(170,120)
\usebox{\orgate}
\put(10,50){\smallopen}
\put(20,50){\smallmine}
\put(50,90){\smallmine}
\put(50,80){\smallopen}
\end{picture}
\end{center}

\newpage

The $3$ at $(6,4)$ forces $(6,5)$ to be open, while the $3$ at $(8,6)$ forces a mine at $(7,6)$:
\begin{center}
\begin{picture}(170,120)
\usebox{\orgate}
\put(10,50){\smallopen}
\put(20,50){\smallmine}
\put(40,50){\smallopen}
\put(50,90){\smallmine}
\put(50,80){\smallopen}
\put(50,60){\smallmine}
\end{picture}
\end{center}
Next, we need to place mines so that the $6$ at $(6,6)$ is adjacent to six mines. There are two possibilities: (1), an open cell at $(6,7)$ and a mine at $(5,6)$, and (2), an open cell at $(5,6)$ and a mine at $(6,7)$.

\subsection*{Case 1}
There is an open cell at $(6,7)$ and a mine at $(5,6)$.
\begin{center}
\begin{picture}(170,120)
\usebox{\orgate}
\put(10,50){\smallopen}
\put(20,50){\smallmine}
\put(40,50){\smallopen}
\put(60,50){\smallopen}
\put(50,90){\smallmine}
\put(50,80){\smallopen}
\put(50,60){\smallmine}
\put(50,40){\smallmine}
\end{picture}
\end{center}
The $2$ at $(7,8)$ forces a mine $(6,8)$ and the $4$ at $(4,5)$ forces an open cell at $(3,6)$.
\begin{center}
\begin{picture}(170,120)
\usebox{\orgate}
\put(10,50){\smallopen}
\put(20,50){\smallmine}
\put(40,50){\smallopen}
\put(60,50){\smallopen}
\put(70,50){\smallmine}

\put(50,90){\smallmine}
\put(50,80){\smallopen}
\put(50,60){\smallmine}
\put(50,40){\smallmine}
\put(50,20){\smallopen}
\end{picture}
\end{center}
The $1$ at $(6,9)$ forces an open cell at $(6,10)$ and the $3$ at $(2,7)$ forces mines at $(3,7),(3,8)$.
\begin{center}
\begin{picture}(170,120)
\usebox{\orgate}
\put(10,50){\smallopen}
\put(20,50){\smallmine}
\put(40,50){\smallopen}
\put(60,50){\smallopen}
\put(70,50){\smallmine}
\put(90,50){\smallopen}

\put(50,90){\smallmine}
\put(50,80){\smallopen}
\put(50,60){\smallmine}
\put(50,40){\smallmine}
\put(50,20){\smallopen}
\put(60,20){\smallmine}
\put(70,20){\smallmine}
\end{picture}
\end{center}

\newpage
The $2$ at $(2,8)$ forces an open cell at $(3,9)$.\begin{center}
\begin{picture}(170,120)
\usebox{\orgate}
\put(10,50){\smallopen}
\put(20,50){\smallmine}
\put(40,50){\smallopen}
\put(60,50){\smallopen}
\put(70,50){\smallmine}
\put(90,50){\smallopen}

\put(50,90){\smallmine}
\put(50,80){\smallopen}
\put(50,60){\smallmine}
\put(50,40){\smallmine}
\put(50,20){\smallopen}
\put(60,20){\smallmine}
\put(70,20){\smallmine}
\put(80,20){\smallopen}
\end{picture}
\end{center}
The $3$ at $(2,9)$ forces a mine at $(3,10)$ and the $4$ at $(4,11)$ forces an open cell at $(5,11)$, but now the $3$ at $(5,10)$ is inconsistent, because it is adjacent to only two mines.
\begin{center}
\begin{picture}(170,120)
\usebox{\orgate}
\put(10,50){\smallopen}
\put(20,50){\smallmine}
\put(40,50){\smallopen}
\put(60,50){\smallopen}
\put(70,50){\smallmine}
\put(90,50){\smallopen}
\put(50,90){\smallmine}
\put(50,80){\smallopen}
\put(50,60){\smallmine}
\put(50,40){\smallmine}
\put(50,20){\smallopen}
\put(60,20){\smallmine}
\put(70,20){\smallmine}
\put(80,20){\smallopen}
\put(90,20){\smallmine}
\put(90,40){\smallincon}
\put(100,40){\smallopen}
\end{picture}
\end{center}

\newpage

\subsection*{Case 2}
An open cell at $(5,6)$ and a mine at $(6,7)$.
\begin{center}
\begin{picture}(170,120)
\usebox{\orgate}
\put(10,50){\smallopen}
\put(20,50){\smallmine}
\put(40,50){\smallopen}
\put(60,50){\smallmine}
\put(50,90){\smallmine}
\put(50,80){\smallopen}
\put(50,60){\smallmine}
\put(50,40){\smallopen}
\end{picture}
\end{center}
The $2$ at $(7,8)$ forces an open cell at $(6,8)$ and the $4$ at $(4,5)$ forces a mine at $(3,6)$. In addition, the $1$ at $(6,9)$ forces a mine at $(6,10)$.
\begin{center}
\begin{picture}(170,120)
\usebox{\orgate}
\put(10,50){\smallopen}
\put(20,50){\smallmine}
\put(40,50){\smallopen}
\put(60,50){\smallmine}
\put(70,50){\smallopen}
\put(50,90){\smallmine}
\put(50,80){\smallopen}
\put(50,60){\smallmine}
\put(50,40){\smallopen}
\put(50,20){\smallmine}
\put(90,50){\smallmine}
\end{picture}
\end{center}

\newpage

We now have two cases for the $3$ at $(2,7)$.

\subsection*{Case 2.1}

There is an open cell at $(3,7)$ and a mine at $(3,8)$.
\begin{center}
\begin{picture}(170,120)
\usebox{\orgate}
\put(10,50){\smallopen}
\put(20,50){\smallmine}
\put(40,50){\smallopen}
\put(60,50){\smallmine}
\put(70,50){\smallopen}
\put(90,50){\smallmine}
\put(50,90){\smallmine}
\put(50,80){\smallopen}
\put(50,60){\smallmine}
\put(50,40){\smallopen}
\put(50,20){\smallmine}
\put(60,20){\smallopen}
\put(70,20){\smallmine}
\end{picture}
\end{center}
The $2$ at $(2,8)$ forces a mine at $(3,9)$ and the $3$ at $(2,9)$ forces an open cell at $(3,10)$. This in turn causes the $4$ at $(4,3)$ to force a mine at $(5,11)$, but now the $3$ at $(5,10)$ is inconsistent because the cell has four adjacent cells with mines.
\begin{center}
\begin{picture}(170,120)
\usebox{\orgate}
\put(10,50){\smallopen}
\put(20,50){\smallmine}
\put(40,50){\smallopen}
\put(60,50){\smallmine}
\put(70,50){\smallopen}
\put(90,50){\smallmine}
\put(50,90){\smallmine}
\put(50,80){\smallopen}
\put(50,60){\smallmine}
\put(50,40){\smallopen}
\put(50,20){\smallmine}
\put(60,20){\smallopen}
\put(70,20){\smallmine}
\put(80,20){\smallmine}
\put(90,20){\smallopen}
\put(90,40){\smallincon}
\put(100,40){\smallmine}
%\put(100,30){\smallincon}
\end{picture}
\end{center}

\newpage

\subsection*{Case 2.2}

There is a mine at $(3,7)$ and an open cell at $(3,8)$.
\begin{center}
\begin{picture}(170,120)
\usebox{\orgate}
\put(10,50){\smallopen}
\put(20,50){\smallmine}
\put(40,50){\smallopen}
\put(60,50){\smallmine}
\put(70,50){\smallopen}
\put(90,50){\smallmine}
\put(50,90){\smallmine}
\put(50,80){\smallopen}
\put(50,60){\smallmine}
\put(50,40){\smallopen}
\put(50,20){\smallmine}
\put(60,20){\smallmine}
\put(70,20){\smallopen}
\end{picture}
\end{center}
The $3$ at $(2,9)$ forces mines at $(3,9)$ and $(3,10)$.
\begin{center}
\begin{picture}(170,120)
\usebox{\orgate}
\put(10,50){\smallopen}
\put(20,50){\smallmine}
\put(40,50){\smallopen}
\put(60,50){\smallmine}
\put(70,50){\smallopen}
\put(90,50){\smallmine}
\put(50,90){\smallmine}
\put(50,80){\smallopen}
\put(50,60){\smallmine}
\put(50,40){\smallopen}
\put(50,20){\smallmine}
\put(60,20){\smallmine}
\put(70,20){\smallopen}
\put(80,20){\smallmine}
\put(90,20){\smallmine}
\end{picture}
\end{center}

\newpage

The $4$ at $(4,11)$ forces an open cell at $(5,11)$, which is consistent with the $3$ at $(5,10)$. In addition, the $3$ at $(7,10)$ forces an open cell at $(7,11)$:
\begin{center}
\begin{picture}(170,120)
\usebox{\orgate}
\put(10,50){\smallopen}
\put(20,50){\smallmine}
\put(40,50){\smallopen}
\put(60,50){\smallmine}
\put(70,50){\smallopen}
\put(90,50){\smallmine}
\put(90,50){\smallmine}
\put(100,40){\smallopen}
\put(100,60){\smallopen}
\put(50,90){\smallmine}
\put(50,80){\smallopen}
\put(50,60){\smallmine}
\put(50,40){\smallopen}
\put(50,20){\smallmine}
\put(60,20){\smallmine}
\put(70,20){\smallmine}
\put(80,20){\smallopen}
\put(90,20){\smallmine}
\end{picture}
\end{center}
The puzzle is now quickly solved successfully. The $2$ at $(6,11)$ forces a mine at $(6,12)$, which forces an open cell at $(6,14)$, which in turn forces a mine at $(6,15)$:
\begin{center}
\begin{picture}(170,120)
\usebox{\orgate}
\put(10,50){\smallopen}
\put(20,50){\smallmine}
\put(40,50){\smallopen}
\put(60,50){\smallmine}
\put(70,50){\smallopen}
\put(90,50){\smallmine}
\put(90,50){\smallmine}
\put(100,40){\smallopen}
\put(100,60){\smallopen}
\put(110,50){\smallmine}
\put(130,50){\smallopen}
\put(140,50){\smallmine}
\put(50,90){\smallmine}
\put(50,80){\smallopen}
\put(50,60){\smallmine}
\put(50,40){\smallopen}
\put(50,20){\smallmine}
\put(60,20){\smallmine}
\put(70,20){\smallmine}
\put(80,20){\smallopen}
\put(90,20){\smallmine}
\end{picture}
\end{center}
The open cell at $(6,14)$ and the mine at $(6,15)$ form the output open/mine ($1$), which is the result of ($0$ \texttt{or} $1$). 



\newpage

\textbf{\Large Appendix}

\appendix

\section{NP-completeness}

A problem $Q$ is NP-complete if:
\begin{itemize}
\item A proposed answer to $Q$ can be checked in polynomial time.
\item All other problems in the class can be reduced to $Q$, meaning that if $Q$ has a polynomial-time algorithm, then so do all the other problems.
\end{itemize}
Given an assignment for a formula $A$ in CNF, it is easy to check if the truth value of $A$ is $T$, so SAT fulfills the first condition. Stephen Cook (1971) and Leonid Levin (1973) showed that the second condition also holds, so SAT is an NP-complete problem.

The first condition states that an NP-complete problem can be solved in polynomial time by a \emph{non-deterministic} algorithm. The question of whether NP-complete problems can be solved in polynomial time by a \emph{deterministic} algorithm is called the $\mathcal{P}=\mathcal{NP}?$ problem.

If there is an efficient (\emph{deterministic polynomial-time}) algorithm for one NP-complete problem, then there are efficient algorithms for all the problems in the class. Currently, no efficient algorithm is known for any problem in the class.

If there is \emph{no} efficient algorithm for one NP-complete problem, then there are no efficient algorithms for all the problems in the class. Currently, there is no proof that any of the problem in the class cannot be solved in deterministic polynomial time.

The Clay Mathematics Institute offers a $\$1$ million prize for a proof that answers $\mathcal{P}=\mathcal{NP}?$.\\ \url{http://claymath.org/millennium-problems/p-vs-np-problem}.

NP-completeness is presented in textbooks in theoretical computer science such as:

Hopcroft, J.E, Motwani, R., Ullman, J.D. \textit{Introduction to Automata Theory, Languages, and Computation}, Third edition,  Addison-Wesley, 2006.

Cormen, T.H., Leiserson, C.E., Rivest, R.L., Stein, C. \textit{Introduction to Algorithms}, Second edition, MIT Press, 2001.

Sipser, M. \textit{Introduction to the Theory of Computation}. PWS Publishing, 1997.

\end{document}
