\documentclass[11pt]{article}
\usepackage{url}
\usepackage{mathpazo}
\parindent=0pt
\renewcommand{\baselinestretch}{1.2}
\begin{document}
\newlength{\lng}               % Factor for sizing puzzle
\setlength{\lng}{1pt}          % Modify here
\unitlength=\lng               % See also \mine, below
\newcommand{\lrg}[0]{\Large}   % Size/style of numbers in cells
\newcommand{\mdm}[0]{\normalsize\sf}
                               % Size/style of numbers labelling cells
%  Slides for showing that Minesweeper is NP-Complete
%  as proved by Richard Kaye,
%  see: http://www.mat.bham.ac.uk/R.W.Kaye/minesw/
%
% Copyright \copyright{} 2009 by Mordechai (Moti) Ben-Ari.
% This work is licensed under the Creative Commons
% Attribution-Noncommercial-ShareAlike 3.0 License. To view a copy of
% this license, visit
% \url{http://creativecommons.org/licenses/by-nc-sa/3.0/}; or, (b) send a letter 
% to Creative Commons, 543 Howard Street, 5th Floor, San Francisco, California, 94105, USA.

% \puztwo draws the configuration for Kaye's square of 2's
% \puzthree draws the same configuration with 3's

\newsavebox{\grid}
\sbox{\grid}{
\thicklines
  \put(0,0){\framebox(120,120){}}           % Outer box
  \multiput(20,0)(20,0){5}{\line(0,1){120}} % Vertical grid lines
  \multiput(0,20)(0,20){5}{\line(1,0){120}} % Horizontal grid lines
}

% A puzzle is a grid filled with zero and the parameter n
\newcommand{\puz}[1]{
  \thicklines
  \usebox{\grid}
  \multiput(20,20)(0,60){2}{
    \multiput(0,0)(20,0){4}{\makebox(20,20){\lrg #1}}
  }
  \multiput(20,40)(0,20){2}{
    \multiput(0,0)(60,0){2}{\makebox(20,20){\lrg #1}}
    \multiput(20,0)(20,0){2}{\makebox(20,20){\lrg 0}}
  }
}

% Save boxes for puzzles with 2's and 3's
\newsavebox{\puztwo}
\sbox{\puztwo}{\puz{2}}

\newsavebox{\puzthree}
\sbox{\puzthree}{\puz{3}}

% Commands for symbols
\newcommand{\mine}[0]{\makebox(20,20){\rule{9\lng}{9\lng}}}
\newcommand{\ques}[0]{\makebox(20,20){\lrg\bf ?}}
\newcommand{\open}[0]{\put(10,10){\circle{10}}}
\newcommand{\incon}[0]{\put(5,5){\line(1,1){10}}\put(5,15){\line(1,-1){10}}}
\newcommand{\fulfil}[0]{\put(4,4){\framebox(12,12){}}}
\newcommand{\smallmine}[0]{\makebox(10,10){\rule{4\lng}{4\lng}}}
\newcommand{\smallopen}[0]{\put(5,5){\circle{5}}}
\newcommand{\smallincon}[0]{\put(1,1){\line(1,1){8}}\put(1,9){\line(1,-1){8}}}

% Commands for each configuration
\newcommand{\configa}[0]{
  \usebox{\puztwo}
  \multiput(80,100)(20,0){2}{\mine}
  \put(60,100){\open}
  \multiput(100,80)(0,-20){2}{\open}
  \put(80,80){\fulfil}
  \put(80,60){\incon}
}

\newcommand{\configb}[0]{
  \usebox{\puztwo}
  \multiput(60,100)(40,0){2}{\mine}
  \put(80,100){\open}
  \multiput(100,80)(0,-20){2}{\open}
  \put(80,80){\fulfil}
  \put(80,60){\incon}
}

\newcommand{\configc}[0]{
  \usebox{\puztwo}
  \multiput(40,100)(60,0){2}{\mine}
  \multiput(60,100)(20,0){2}{\ques}
  \put(60,80){\incon}
}

\newcommand{\configd}[0]{
  \usebox{\puztwo}
  \multiput(20,100)(80,0){2}{\mine}
  \multiput(40,100)(20,0){3}{\ques}
  \multiput(40,80)(20,0){2}{\incon}
}

\newcommand{\confige}[0]{
  \usebox{\puztwo}
  \multiput(0,100)(100,0){2}{\mine}
  \multiput(20,100)(20,0){4}{\ques}
  \multiput(40,80)(20,0){2}{\incon}
}

\newcommand{\configf}[0]{
  \usebox{\puztwo}
  \multiput(60,100)(20,0){2}{\mine}
  \put(40,100){\open}
  \multiput(100,100)(0,-20){3}{\open}
  \multiput(60,80)(20,0){2}{\fulfil}
  \put(80,60){\incon}
}

\newcommand{\configg}[0]{
  \usebox{\puztwo}
  \multiput(40,100)(40,0){2}{\mine}
  \put(60,100){\open}
  \put(20,100){\ques}
  \put(60,80){\fulfil}
  \put(40,80){\incon}
}

\newcommand{\configh}[0]{
  \usebox{\puztwo}
  \multiput(20,100)(60,0){2}{\mine}
  \multiput(40,100)(20,0){2}{\ques}
  \multiput(40,80)(20,0){2}{\incon}
}

\newcommand{\configi}[0]{
  \usebox{\puztwo}
  \multiput(0,0)(0,100){2}{
    \multiput(40,0)(20,0){2}{\mine}
  }
  \multiput(0,20)(0,60){2}{
    \multiput(20,0)(20,0){4}{\fulfil}
  }
  \multiput(0,0)(100,0){2}{
    \multiput(0,40)(0,20){2}{\mine}
  }
  \multiput(20,0)(60,0){2}{
    \multiput(0,40)(0,20){2}{\fulfil}
  }
}

\newcommand{\configj}[0]{
  \usebox{\puzthree}
  \multiput(40,100)(20,0){3}{\mine}
  \put(60,80){\fulfil}
}

\newcommand{\configk}[0]{
  \usebox{\puzthree}
  \multiput(0,0)(0,100){2}{
    \multiput(20,0)(20,0){4}{\mine}
  }
  \multiput(0,20)(0,20){4}{
    \multiput(0,0)(100,0){2}{\mine}
  }
  \multiput(0,20)(0,60){2}{
    \multiput(40,0)(20,0){2}{\fulfil}
    \multiput(20,0)(60,0){2}{\incon}
  }
  \multiput(0,40)(0,20){2}{
    \multiput(20,0)(60,0){2}{\fulfil}
  }
}


\begin{center}
\Large
\textbf{Examples for Minesweeper Consistency}\\
\bigskip

\large
Mordechai (Moti) Ben-Ari\\
Department of Science Teaching\\
Weizmann Institute of Science\\
Rehovot 76100 Israel\\
\textsf{http://stwww.weizmann.ac.il/g-cs/benari}
\end{center}

\bigskip

Copyright \copyright{} 2009 by Mordechai (Moti) Ben-Ari.
This work is licensed under the Creative Commons Attribution-Noncommercial-ShareAlike 3.0
License. To view a copy of this license, visit
\url{http://creativecommons.org/licenses/by-nc-sa/3.0/}; or, (b) send a letter
to Creative Commons, 543 Howard Street, 5th Floor, San Francisco,
California, 94105, USA.

\bigskip
\bigskip


\thispagestyle{empty}

Richard Kaye (\textsf{http://www.mat.bham.ac.uk/R.W.Kaye/minesw/})
posed the problem of finding a consistent assignment
of mines in the following Minesweeper configuration.

\begin{center}
\begin{picture}(130,130)
\put(10,10){\usebox{\puztwo}}
\end{picture}
\end{center}

An assignment of mines is consistent
if it assigns mines to (some of) the empty squares
such that the number in a cell is the number
of mines adjacent to the cell.
We give a detailed solution and further show that
if the two's are replaced by three's, there is no solution.

The cells are numbered from the lower left-hand corner right and up.
The following notation is used:
\begin{center}
\begin{tabular}{|c|l|}
\hline
\begin{picture}(20,22)\mine{}  \end{picture}& mine\\\hline
\begin{picture}(20,22)\open{}  \end{picture}& no mine\\\hline
\begin{picture}(20,22)\ques{}  \end{picture}& possible mine\\\hline
\begin{picture}(20,22)
  \put(0,0){\makebox(20,20){\Large\bf 2}}
  \put(0,0){\incon}
\end{picture}& inconsistent cell\\\hline
\begin{picture}(20,22)
  \put(0,0){\makebox(20,20){\Large\bf 2}}
  \put(0,0){\fulfil}
\end{picture}& fulfilled cell\\\hline
\end{tabular}
\end{center}
where a cell is \emph{fulfilled} iff the number
in the cell is the same as the number of adjacent mines,
and \emph{inconsistent} iff it cannot be be fulfilled
in a configuration.

We first show that a mine cannot be placed in a corner cell.
By symmetry, it is sufficient to show that a mine cannot
be placed in cell $(6,6)$, the upper right-hand corner.
The following figure shows that if a mine is placed in
$(6,6)$, then the second mine needed to fulfil $(5,5)$
cannot be placed in $(5,6)$ or in $(4,6)$.
Cell $(5,5)$ is fulfilled, which implies that certain
cells do not have a mine as indicated;
therefore, $(5,4)$ is inconsistent.

\begin{center}
\begin{picture}(310,130)
\put(10,10){\configa}
\put(180,10){\configb}
\end{picture}
\end{center}

Suppose now that there is a mine in some other
cell of the top row in addition to the mine in $(6,6)$.
If the mine is in $(3,6)$, there must be a mine
in $(4,6)$ or $(5,6)$ to fulfill $(4,4)$,
but we have just proved that such a configuation
is inconsistent.

\begin{center}
\begin{picture}(130,130)
\put(10,10){\configc}
\end{picture}
\end{center}

If there is a mine in $(2,6)$,
there must be another one to the right of it,
which we have just proved impossible,
and similarly for a mine in $(1,6)$.

\begin{center}
\begin{picture}(310,130)
\put( 10,10){\configd}
\put(180,10){\confige}
\end{picture}
\end{center}

Next we show that there cannot be a mine in a cell
adjacent to a corner cell.
By symmetry we show this for cell $(5,6)$.
If there is a mine in $(4,6)$,
cells $(4,5)$ and $(5,5)$ are fulfilled,
and clearly $(5,4)$ is inconsistent.

\begin{center}
\begin{picture}(130,130)
\put(10,10){\configf}
\end{picture}
\end{center}

If a mine is placed in cell $(3,6)$ to fulfil $(4,5)$,
then to fulfil $(3,5)$
a mine would have to be placed in $(2,6)$.
But we have just proved that it is impossible to put
a pair of mines adjacent to a corner cell (here $(1,6)$).
Similarly, placing a second mine in cell $(2,6)$
would require placing a third one in $(3,6)$ or $(4,6)$
to fulfill $(3,5)$ and $(4,5)$, again leading
to a pair of mines adjacent to a corner cell.

\begin{center}
\begin{picture}(310,130)
\put( 10,10){\configg}
\put(180,10){\configh}
\end{picture}
\end{center}

Therefore, configurations can only contain
mines which are not in a corner cell or adjacent to
a corner cell,
and the full solution is obtained by applying this
rule to all cells.

\begin{center}
\begin{picture}(130,130)
\put(10,10){\configi}
\end{picture}
\end{center}

Suppose now that the two's are replaced by three's.

\begin{center}
\begin{picture}(130,130)
\put(10,10){\usebox{\puzthree}}
\end{picture}
\end{center}

Any Minesweeper player worth his/her salt
immediately sees that cells containing $3$
like $(4,5)$
can only be fulfilled by assigning mines to
all three of its open neighbors.

\begin{center}
\begin{picture}(130,130)
\put(10,10){\configj}
\end{picture}
\end{center}

All non-corner cells must be fulfilled this way:

\begin{center}
\begin{picture}(130,130)
\put(10,10){\configk}
\end{picture}
\end{center}

and we see that the four corner cells $(2,2)$,
$(2,5)$, $(5,2)$ and $(5,5)$ are inconsistent.

\end{document}
