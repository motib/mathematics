\documentclass[12pt,a4paper]{article}
\usepackage[utf8x]{inputenc}
\usepackage[english,hebrew]{babel}
\usepackage{graphicx}
\usepackage{verbatim}
\usepackage{url}

\textwidth=15.5cm
\textheight=23cm
\topmargin=0pt
\headheight=0pt
\oddsidemargin=2em
\headsep=0pt
\parindent=0pt
\renewcommand{\baselinestretch}{1.1}
\setlength{\parskip}{0.3\baselineskip plus 1pt minus 1pt}

\newlength{\lng}
\setlength{\lng}{1pt}
\unitlength=\lng
\newcommand{\lrg}[0]{\Large\sf} 
\newcommand{\mdm}[0]{\normalsize\sf}

\newsavebox{\grid}
\sbox{\grid}{
\thicklines
  \put(0,0){\framebox(120,120){}}           % Outer box
  \multiput(20,0)(20,0){5}{\line(0,1){120}} % Vertical grid lines
  \multiput(0,20)(0,20){5}{\line(1,0){120}} % Horizontal grid lines
  \put(-20,0){\makebox(20,20){\sf 1}}
  \put(-20,20){\makebox(20,20){\sf 2}}
  \put(-20,40){\makebox(20,20){\sf 3}}
  \put(-20,60){\makebox(20,20){\sf 4}}
  \put(-20,80){\makebox(20,20){\sf 5}}
  \put(-20,100){\makebox(20,20){\sf 6}}
  \put(0,120){\makebox(20,20){\sf 1}}
  \put(20,120){\makebox(20,20){\sf 2}}
  \put(40,120){\makebox(20,20){\sf 3}}
  \put(60,120){\makebox(20,20){\sf 4}}
  \put(80,120){\makebox(20,20){\sf 5}}
  \put(100,120){\makebox(20,20){\sf 6}}
}

% A puzzle is a grid filled with zero and the parameter n
\newcommand{\puz}[1]{
  \thicklines
  \usebox{\grid}
  \multiput(20,20)(0,60){2}{
    \multiput(0,0)(20,0){4}{\makebox(20,20){\lrg #1}}
  }
  \multiput(20,40)(0,20){2}{
    \multiput(0,0)(60,0){2}{\makebox(20,20){\lrg #1}}
    \multiput(20,0)(20,0){2}{\makebox(20,20){\lrg 0}}
  }
}

% Save boxes for puzzles with 2's and 3's
\newsavebox{\puztwo}
\sbox{\puztwo}{\puz{2}}

\newsavebox{\puzthree}
\sbox{\puzthree}{\puz{3}}

% Commands for symbols
\newcommand{\mine}[0]{\makebox(20,20){\rule{9\lng}{9\lng}}}
\newcommand{\ques}[0]{\makebox(20,20){\lrg ?}}
\newcommand{\open}[0]{\put(10,10){\circle{10}}}
\newcommand{\incon}[0]{\put(5,5){\line(1,1){10}}\put(5,15){\line(1,-1){10}}}
\newcommand{\fulfil}[0]{\put(4,4){\framebox(12,12){}}}
\newcommand{\smallmine}[0]{\makebox(10,10){\rule{4\lng}{4\lng}}}
\newcommand{\smallopen}[0]{\put(5,5){\circle{5}}}
\newcommand{\smallincon}[0]{\put(1,1){\line(1,1){8}}\put(1,9){\line(1,-1){8}}}

% Commands for each configuration
\newcommand{\configa}[0]{
  \usebox{\puztwo}
  \multiput(80,100)(20,0){2}{\mine}
  \put(60,100){\open}
  \multiput(100,80)(0,-20){2}{\open}
  \put(80,80){\fulfil}
  \put(80,60){\incon}
}

\newcommand{\configb}[0]{
  \usebox{\puztwo}
  \multiput(60,100)(40,0){2}{\mine}
  \put(80,100){\open}
  \multiput(100,80)(0,-20){2}{\open}
  \put(80,80){\fulfil}
  \put(80,60){\incon}
}

\newcommand{\configc}[0]{
  \usebox{\puztwo}
  \multiput(40,100)(60,0){2}{\mine}
  \multiput(60,100)(20,0){2}{\ques}
  \put(60,80){\incon}
}

\newcommand{\configd}[0]{
  \usebox{\puztwo}
  \multiput(20,100)(80,0){2}{\mine}
  \multiput(40,100)(20,0){3}{\ques}
  \multiput(40,80)(20,0){2}{\incon}
}

\newcommand{\confige}[0]{
  \usebox{\puztwo}
  \multiput(0,100)(100,0){2}{\mine}
  \multiput(20,100)(20,0){4}{\ques}
  \multiput(40,80)(20,0){2}{\incon}
}

\newcommand{\configf}[0]{
  \usebox{\puztwo}
  \multiput(60,100)(20,0){2}{\mine}
  \put(40,100){\open}
  \multiput(100,100)(0,-20){3}{\open}
  \multiput(60,80)(20,0){2}{\fulfil}
  \put(80,60){\incon}
}

\newcommand{\configg}[0]{
  \usebox{\puztwo}
  \put(80,100){\mine}
  \put(100,80){\mine}
  \put(60,100){\open}
  \put(100,100){\open}
  \put(100,60){\open}
  \put(80,80){\fulfil}
}

\newcommand{\configh}[0]{
  \usebox{\puztwo}
  \put(80,100){\mine}
  \put(100,80){\mine}
  \put(40,100){\mine}
  \put(20,100){\mine}
  \put(60,100){\open}
  \put(100,100){\open}
  \put(100,60){\open}
  \put(80,80){\fulfil}
  \put(60,80){\fulfil}
  \put(40,80){\fulfil}
  \put(20,80){\fulfil}
  \put(0,100){\open}
  \put(0,80){\open}
  \put(0,60){\open}
  \put(20,60){\incon}
}

%\newcommand{\configh}[0]{
%  \usebox{\puztwo}
%  \multiput(20,100)(60,0){2}{\mine}
%  \multiput(40,100)(20,0){2}{\ques}
%  \multiput(40,80)(20,0){2}{\incon}
%}

\newcommand{\configi}[0]{
  \usebox{\puztwo}
  \multiput(0,0)(0,100){2}{
    \multiput(40,0)(20,0){2}{\mine}
  }
  \multiput(0,20)(0,60){2}{
    \multiput(20,0)(20,0){4}{\fulfil}
  }
  \multiput(0,0)(100,0){2}{
    \multiput(0,40)(0,20){2}{\mine}
  }
  \multiput(20,0)(60,0){2}{
    \multiput(0,40)(0,20){2}{\fulfil}
  }
}

\newcommand{\configj}[0]{
  \usebox{\puzthree}
  \multiput(40,100)(20,0){3}{\mine}
  \put(60,80){\fulfil}
}

\newcommand{\configk}[0]{
  \usebox{\puzthree}
  \multiput(0,0)(0,100){2}{
    \multiput(20,0)(20,0){4}{\mine}
  }
  \multiput(0,20)(0,20){4}{
    \multiput(0,0)(100,0){2}{\mine}
  }
  \multiput(0,20)(0,60){2}{
    \multiput(40,0)(20,0){2}{\fulfil}
    \multiput(20,0)(60,0){2}{\incon}
  }
  \multiput(0,40)(0,20){2}{
    \multiput(20,0)(60,0){2}{\fulfil}
  }
}


\begin{document}
\thispagestyle{empty}

\selectlanguage{hebrew}

\begin{center}
\textbf{\Huge שולת מוקשים היא
\L{NP-Complete}}

\bigskip
\bigskip

\textbf{\Large מוטי בן-ארי}

\bigskip

\textbf{\Large המחלקה להוראת המדעים}

\bigskip

\textbf{\Large מכון ויצמן למדע}

\bigskip

\url{http://www.weizmann.ac.il/sci-tea/benari/}

\bigskip

\end{center}

\selectlanguage{english}

\begin{center}
\copyright{}\  2018 by Moti Ben-Ari.
\end{center}

\begin{footnotesize}
This work is licensed under the Creative Commons Attribution-ShareAlike 3.0 Unported License. To view a copy of this license, visit \url{http://creativecommons.org/licenses/by-sa/3.0/} or send a letter to Creative Commons, 444 Castro Street, Suite 900, Mountain View, California, 94041, USA.
\end{footnotesize}

\bigskip

\begin{center}
\includegraphics[width=.2\textwidth]{../by-sa.png}
\end{center}

\bigskip

\selectlanguage{hebrew}
\newpage


מסמך זה מניח היכרות עם
\L{NP-completeness}.
סקירה של הנושא נמצאת בנספח.

\L{Richard Kaye}
הראה שחידה המבוססת על המשחק שולת מוקשים היא
\L{NP-complete}.
מסמך זה מביא את הבנייה של 
\L{Kaye}
ביחד עם הסברים מפורטים.

\selectlanguage{english}
Kaye, R. (2000) Minesweeper is NP-complete. \textit{Mathematical Intelligencer} 22(2), 9—15. \url{http://web.mat.bham.ac.uk/R.W.Kaye/minesw/}.

\selectlanguage{hebrew}

\section{%
חידה מבוססת שולת מוקשים%
}
אני מניח שאתם מכירים, ואולי אף מכורים, למשחק "שולת מוקשים". קיים סורג מרובע של תאים מוסתרים, חלקם עם מוקשים. יש למצוא את כל התאים שאין בהם מוקשים בלי לדרוך על מוקש. תאים פתוחים שהם שכנים לתאים עם מוקשים מכילים מספר בין 
$0$
ל-%
$8$
המציין את מספר התאים השכנים עם מוקשים.

חידת שולת המוקשים היא היפוכה של המשחק. נתון סורג עם תאים המכילים מוקשים ותאים המכילים מספרים, האם קיים דרך למלא חלק מהתאים הנותרים כך שמספרי השכנים עם מוקשים יתאים למספרים הרשומים בתאים.

\section{\L{NP-completeness}}

\L{Kaye}
הראה שחידת שולת המוקשים היא
\L{NP-complete}
על ידי תרגום לבעיית
\L{SAT}:
האם ניתן למצוא הצבה של
\L{T,F}
%
לטענות האטמיות של נוסח בתחשיב הפסוקים, כך שערך האמת של הנוסחה הוא
\L{T}.
הוא בנה תצורות של תאים עם מוקשים ומספרי שכנים, כך שאפשר להתאים להם מספרים אם ורק אם התצורות מהנהגות לפי השערים הלוגיים
\L{\texttt{not}, \texttt{and}, \texttt{or}}.
בנוסף יש לבנות תצורות המתנהגות כמו חוטים למיניהם: ישרים, פונים לכיוון אחר, עבורים אחד מעל לשני, מתלכדים ומתפצלים. 

\section{\R{חידות שלות מוקשים}}

נציג את הסורג כאשר סופרים שורות של תאים מלמטה למעלה וטורים של תאים משמאל לימין.

מונחים: תא שיש מספר מוקשים שכנים השווה למספר הרשום בתא נקרא תא מסופק. תא עם מספר שכנים קטן מהמספר בתא נרקאה תא לא מסופק. תא עם מספר שכנים גדול מהמספר בתא או תא שאין מספיק תאים פתוחים כדי להניח מוקשים ולספק אותו נקרא תא לא עיקבי.

נשתמש בסימונים שלהלן:
\begin{center}
\selectlanguage{english}
\begin{tabular}{|c|l|}
\hline
\begin{picture}(20,22)\mine{}  \end{picture}& \hfill\R{מוקש}\\\hline
\begin{picture}(20,22)\open{}  \end{picture}& \hfill\R{אין מוקש}\\\hline
\begin{picture}(20,22)\ques{}  \end{picture}& \hfill\R{מוקש אפשרי}\\\hline
\begin{picture}(20,22)
  \put(0,0){\makebox(20,20){\lrg 2}}
  \put(0,0){\incon}
\end{picture}& \hfill\R{תא לא עיקבי}\\\hline
\begin{picture}(20,22)
  \put(0,0){\makebox(20,20){\lrg 2}}
  \put(0,0){\fulfil}
\end{picture}& \hfill\R{תא מסופק}\\\hline
\begin{picture}(20,22)
\put(0,0){\makebox(20,20){\lrg 2}} \end{picture} & \hfill\R{תא לא מסופק}\\\hline
\end{tabular}
\selectlanguage{hebrew}
\end{center}




\section{\R{חידה פשוטה---ריבוע של תאים עם מספרי שכנים 2}}


\begin{center}
\selectlanguage{english}
\begin{picture}(130,140)
\put(0,0){\usebox{\puztwo}}
\end{picture}
\selectlanguage{hebrew}
\end{center}
תחילה נראה שלא ניתן להניח מוקש בתא פינתי. לפי סימטריה, מספיק לראות שאי אפשר להניח מוקש בפינה הימנית העליונה 
$(6,6)$.
אם נניח מוקש בתא זה, לא ניתן להניח את המוקש השני כדי לספק את תא 
$(5,5)$
בתא
$(6,5)$,
כי התאים
$(6,4)$, $(5,6)$, $(4,6)$
חייבים להיות ללא מוקשים, ואין דרך להניח מוקשים שיספקו את
$(4,5)$.

\begin{center}
\selectlanguage{english}
\begin{picture}(130,140)
\put(0,0){\configa}
\end{picture}
\selectlanguage{hebrew}
\end{center}

כעת נניח את המוקש השני לספק את תא
$(5,5)$
בתא
$(6,4)$.
\begin{center}
\selectlanguage{english}
\begin{picture}(130,140)
\put(0,0){\configb}
\end{picture}
\selectlanguage{hebrew}
\end{center}
אי אפשר להניח שני מוקשים שיספקו את תא
$(4,5)$,
כי
$(6,5), (5,6), (4,6)$
חייבים להיות פתוחים.

הוכחנו שאי אפשר להניח מוקש בתא פינתי.

\bigskip

כעת נראה שאי אפשר להניח מוקש בתא שהוא שכן של תא פינתי. לפי סימטריה, מספיק לבדוק עבור תא
$(6,5)$.
אפשר לספק את תא
$(5,5)$
על ידי הנחת מוקש בתא
$(6,4)$,
אבל זה מחייב שהתאים 
$(6,6), (5,6), (4,6)$
יהיו פתוחים, ואי אפשר לספק את תא
$(5,4)$.
\begin{center}
\selectlanguage{english}
\begin{picture}(130,140)
\put(0,0){\configf}
\end{picture}
\selectlanguage{hebrew}
\end{center}
אולי אפשר להניח את המוקש השני עבור 
$(5,5)$
בתא
$(5,6)$,
ולהשאיר את 
$(6,4)$
פתוח?
\begin{center}
\selectlanguage{english}
\begin{picture}(130,140)
\put( 10,10){\configg}
\end{picture}
\selectlanguage{hebrew}
\end{center}
כדי לספק תא
$(5,4)$
חייבים להניח מוקש בתא
$(6,3)$,
וכדי לספק תא
$(5,3)$
חייבים להניח מוקש בתא
$(6,2)$,
שגם מספק את תא
$(5,2)$.
זה מחייב שתאים
$(6,1), (5,1), (4,1)$
יהיו פתוחים, ולכן אי אפשר לספק את תא
$(4,2)$.
\begin{center}
\selectlanguage{english}
\begin{picture}(130,140)
\put(0,10){\configh}
\end{picture}
\selectlanguage{hebrew}
\end{center}

לאחר שהוכחנו שאי אפשר להניח מוקש בתא פינתי או בתא ליד תא פינתי, התצורה היחידה שנשארת היא:
\begin{center}
\selectlanguage{english}
\begin{picture}(130,140)
\put(0,0){\configi}
\end{picture}
\selectlanguage{hebrew}
\end{center}
בדיקה קצרה מראה שתצורה זו מספקת את כל התאים.


\section{\R{חידה פשוטה---ריבוע של תאים עם מספרי שכנים 3}}


\begin{center}
\selectlanguage{english}
\begin{picture}(130,140)
\put(0,0){\usebox{\puzthree}}
\end{picture}
\selectlanguage{hebrew}
\end{center}
כדי לספק תאים עם
$3$
שאינם בפינות הריבוע, חייבים להניח מוקשים בכל השכנים שלו. למשל, כדי לספק תא
$(5,4)$
חייבים להניח מוקשים ב-%
$(6,3)$, $(6,4)$
ו-%
$(6,5)$.
\begin{center}
\selectlanguage{english}
\begin{picture}(130,140)
\put(0,0){\configj}
\end{picture}
\selectlanguage{hebrew}
\end{center}
לאחר שנניח מוקשים כדי לספק את כל התאים שאינם בפינות, התאים
$(2,2)$, $(2,5)$, $(5,2)$, $(5,5)$
יהיו לא עקביים. מכאן שאין פתרון לחידה זו.
\begin{center}
\selectlanguage{english}
\begin{picture}(130,140)
\put(10,10){\configk}
\end{picture}
\selectlanguage{hebrew}
\end{center}


\section{\R{מעגלים ספרתיים}}

\L{Kaye}
הראה חידת שולת המוקשים היא
\L{NP-complete}
על ידי תרגום נוסחאות בתחשבים הפסוקים ב-%
\L{CNF}
לחידות שולת מוקשים. הנוסחאות מוצגות כמעגלים ספרתיים על שערים 
\L{\texttt{not}}
ו-%
\L{\texttt{or}}.
)ניתן לבנות שערי 
\L{\texttt{and}}
משערי
\L{\texttt{or},  \texttt{not}}.(
כמו כן, יש להציג את החוטים כחידות שולת מוקשים. נראה את הייצוג עבור חוטים ישרים. ניתן למצוא את הייצוג של חוטים אחרים במאמר של
\L{Kaye}.
\begin{center}
\selectlanguage{english}
\unitlength=1.1pt
\begin{picture}(290,100)
\thicklines
\put(0,80){\makebox(20,30){\sf P}}
\put(0,60){\makebox(20,30){\sf Q}}
\put(0,40){\makebox(20,30){\sf R}}
\put(40,15){\line(1,0){10}}
\put(20,75){\line(1,0){30}}
\put(20,95){\line(1,0){30}}
\put(40,95){\line(0,-1){80}}
\put(38,93){\rule{4pt}{4pt}}
\put(20,55){\line(1,0){120}}
\put(120,55){\line(0,-1){45}}
\put(118,53){\rule{4pt}{4pt}}
\put(100,20){\line(1,0){40}}
\put(120,10){\line(1,0){20}}
\put(100,90){\line(1,0){20}}
\put(120,90){\line(0,-1){20}}
\put(120,70){\line(1,0){20}}
\put(190,65){\line(1,0){20}}
\put(210,65){\line(0,-1){10}}
\put(210,55){\line(1,0){20}}
\put(190,15){\line(1,0){20}}
\put(210,15){\line(0,1){20}}
\put(210,35){\line(1,0){20}}
\put(280,45){\line(1,0){25}}
\put(50,0){\framebox(50,30){\sf Not}}
\put(50,70){\framebox(50,30){\sf And}}
\put(140,0){\framebox(50,30){\sf Or}}
\put(140,50){\framebox(50,30){\sf Or}}
\put(230,30){\framebox(50,30){\sf And}}
\end{picture}
\selectlanguage{hebrew}
\end{center}


\newsavebox{\wire}
\sbox{\wire}{
\thicklines
\put(0,0){\framebox(220,100){}}
\multiput(20,0)(20,0){10}{\line(0,1){100}}
\multiput(0,20)(0,20){4}{\line(1,0){220}}
\multiput(0,0)(20,0){11}{
  \put(0,0){\makebox(20,20){\lrg 0}}
  \put(0,20){\makebox(20,20){\lrg 1}}
  \put(0,60){\makebox(20,20){\lrg 1}}
  \put(0,80){\makebox(20,20){\lrg 0}}
}
\multiput(40,40)(60,0){3}{\makebox(20,20){\lrg 1}}
  \put(-20,0){\makebox(20,20){\sf 1}}
  \put(-20,20){\makebox(20,20){\sf 2}}
  \put(-20,40){\makebox(20,20){\sf 3}}
  \put(-20,60){\makebox(20,20){\sf 4}}
  \put(-20,80){\makebox(20,20){\sf 5}}
  \put(0,100){\makebox(20,20){\sf 1}}
  \put(20,100){\makebox(20,20){\sf 2}}
  \put(40,100){\makebox(20,20){\sf 3}}
  \put(60,100){\makebox(20,20){\sf 4}}
  \put(80,100){\makebox(20,20){\sf 5}}
  \put(100,100){\makebox(20,20){\sf 6}}
  \put(120,100){\makebox(20,20){\sf 7}}
  \put(140,100){\makebox(20,20){\sf 8}}
  \put(160,100){\makebox(20,20){\sf 9}}
  \put(180,100){\makebox(20,20){\sf 10}}
  \put(200,100){\makebox(20,20){\sf 11}}
}

\newsavebox{\notgate}
\sbox{\notgate}{
\thicklines
\put(0,0){\framebox(260,60){}}
\put(100,-20){
  \put(0,0){\framebox(60,100){}}
%  \multiput(20,0)(0,80){2}{\line(0,1){20}}
%  \multiput(40,0)(0,80){2}{\line(0,1){20}}
  \multiput( 0,0)(0,80){2}{
    \multiput(0,0)(20,0){3}{\makebox(20,20){\lrg 1}}
  }
}
\multiput( 0,20)( 0,20){2}{\line(1,0){260}}
\multiput(20, 0)(20, 0){4}{\line(0,1){60}}
\multiput(120,-20)(20, 0){2}{\line(0,1){100}}
\multiput(180, 0)(20, 0){4}{\line(0,1){60}}
\multiput(0,0)(0,40){2}{
  \multiput(  0,0)(20,0){5}{\makebox(20,20){\lrg 1}}
  \multiput(160,0)(20,0){5}{\makebox(20,20){\lrg 1}}
  \multiput(100,0)(40,0){2}{\makebox(20,20){\lrg 2}}
  \put(120,0){\mine}
}
\multiput(40,20)(160,0){2}{\makebox(20,20){\lrg 1}}
\multiput(100,20)(40,0){2}{\makebox(20,20){\lrg 3}}
  \put(-20,0){\makebox(20,20){\sf 1}}
  \put(-20,20){\makebox(20,20){\sf 2}}
  \put(-20,40){\makebox(20,20){\sf 3}}
  \put(0,80){\makebox(20,20){\sf 1}}
  \put(20,80){\makebox(20,20){\sf 2}}
  \put(40,80){\makebox(20,20){\sf 3}}
  \put(60,80){\makebox(20,20){\sf 4}}
  \put(80,80){\makebox(20,20){\sf 5}}
  \put(100,80){\makebox(20,20){\sf 6}}
  \put(120,80){\makebox(20,20){\sf 7}}
  \put(140,80){\makebox(20,20){\sf 8}}
  \put(160,80){\makebox(20,20){\sf 9}}
  \put(180,80){\makebox(20,20){\sf 10}}
  \put(200,80){\makebox(20,20){\sf 11}}
  \put(220,80){\makebox(20,20){\sf 12}}
  \put(240,80){\makebox(20,20){\sf 13}}
}

סימן: הערך 
$0$
יוצג על ידי שני תאים שכנים, כאשר בראשון מונח מוקש והשני הוא תא פתוח. הערך 
$1$
יוצג על ידי שני תאים שכנים, כאשר התא הראשון פתוח ובשני מונח מוקש. 

\section{\R{ייצוג חוט ישר}}

\begin{center}
\selectlanguage{english}
\begin{picture}(240,120)
\usebox{\wire}
\end{picture}
\selectlanguage{hebrew}
\end{center}
בתרשים שלהלן, אנו רואים שקלט של 
$0$
)מוקש/פתוח( בתאים
$(3,1)$--$(3,2)$
גורם לפלט של 
$0$
)מוקש/פתוח( בתאים
$(3,10)$--$(3,11)$.
\begin{center}
\selectlanguage{english}
\begin{picture}(240,120)
\usebox{\wire}
\multiput( 0,40)(60,0){4}{\mine}
\multiput(20,40)(60,0){4}{\open}
\end{picture}
\selectlanguage{hebrew}
\end{center}
קלט של 
$1$
)פתוח/מוקש( בתאים
$(3,1)$--$(3,2)$
גורם לפלט של 
$1$
)פתוח/מוקש( בתאים
$(3,10)$--$(3,11)$.
\begin{center}
\selectlanguage{english}
\begin{picture}(240,120)
\usebox{\wire}
\multiput( 0,40)(60,0){4}{\open}
\multiput(20,40)(60,0){4}{\mine}
\end{picture}
\selectlanguage{hebrew}
\end{center}

\section{יצוג של שער \L{\texttt{not}}}

\begin{center}
\selectlanguage{english}
\begin{picture}(280,100)
\usebox{\notgate}
\end{picture}
\selectlanguage{hebrew}
\end{center}
\bigskip
קלט של 
$1$
)פתוח/מוקש( בתאים
$(2,1)$--$(2,2)$
גורם לפלט של 
$0$
)מוקש/פתוח( בתאים
$(2,12)$--$(2,13)$.

\begin{center}
\selectlanguage{english}
\begin{picture}(280,100)
\usebox{\notgate}
\multiput(0,  20)(60,0){3}{\open}
\multiput(180,20)(60,0){2}{\open}
\multiput(20, 20)(60,0){2}{\mine}
\multiput(160,20)(60,0){2}{\mine}
\end{picture}
\selectlanguage{hebrew}
\end{center}
\bigskip

מוקשים מונחים בתאים 
$(1,7)$
ו-%
$(3,7)$.
לכן, אם יש מוקש ב-%
$(2,5)$,
תא
$(2,6)$
עם 
$3$
שכנים מסופק, ותא 
$(2,7)$
חייב להיות פתוח. מכאן חייב להיות מוקש ב-%
$(2,9)$
כדי לספק את תא
$(2,8)$.
קל לראות ש-%
$(2,10)$
פתוח, מוקש מונח ב-%
$(2,12)$,
ותא
$(2,13)$
פתוח.


\bigskip
באופן דומה, קלט של 
$0$
)מוקש/פתוח( בתאים
$(2,1)$--$(2,2)$
גורם לפלט של 
$1$
)פתוח/מוקש( בתאים
$(2,12)$--$(2,13)$.
\begin{center}
\selectlanguage{english}
\begin{picture}(280,100)
\usebox{\notgate}
\multiput(180,20)(60,0){2}{\mine}
\multiput(  0,20)(60,0){3}{\mine}
\multiput( 20,20)(60,0){2}{\open}
\multiput(160,20)(60,0){2}{\open}
\end{picture}
\selectlanguage{hebrew}
\end{center}
\bigskip

\setlength{\lng}{2pt}
\unitlength=\lng
\newcommand{\sml}[0]{\large\sf}


\newsavebox{\orgate}
\sbox{\orgate}{
\thicklines
\multiput(30,0)(50,0){2}{\line(1,0){40}}
\put(30,10){\line(1,0){90}}
\multiput(30,20)(0,10){2}{\line(1,0){100}}
\multiput( 0,40)(0,10){4}{\line(1,0){160}}
\put(30,80){\line(1,0){100}}
\multiput(40,90)(0,10){3}{\line(1,0){30}}
\put(80,90){\line(1,0){50}}

\multiput( 0,40)(10,0){3}{\line(0,1){30}}
\put(30,0){\line(0,1){80}}
\multiput(40,0)(10,0){4}{\line(0,1){110}}
\multiput(80,0)(10,0){5}{\line(0,1){90}}
\put(130,20){\line(0,1){70}}
\multiput(140,40)(10,0){3}{\line(0,1){30}}

\multiput(0,40)(10,0){3}{\makebox(10,10){\sml 1}}
\multiput(0,60)(10,0){3}{\makebox(10,10){\sml 1}}
\multiput(130,40)(10,0){3}{\makebox(10,10){\sml 1}}
\multiput(130,60)(10,0){3}{\makebox(10,10){\sml 1}}
\multiput(40,80)(0,10){3}{\makebox(10,10){\sml 1}}
\multiput(60,80)(0,10){3}{\makebox(10,10){\sml 1}}
\put(0,50){\makebox(10,10){\sml 1}}
\multiput(30,70)(40,0){2}{\makebox(10,10){\sml 1}}
\multiput(80,50)(40,0){2}{\makebox(10,10){\sml 1}}
\multiput(80,70)(40,0){2}{\makebox(10,10){\sml 1}}
\multiput(80,80)(40,0){2}{\makebox(10,10){\sml 1}}
\multiput(30,00)(30,0){2}{\makebox(10,10){\sml 1}}
\multiput(80,00)(30,0){2}{\makebox(10,10){\sml 1}}
\multiput(120,20)(0,10){2}{\makebox(10,10){\sml 1}}
\put(50,100){\makebox(10,10){\sml 1}}
\put(150,50){\makebox(10,10){\sml 1}}

\multiput(30,60)(40,0){2}{\makebox(10,10){\sml 2}}
\multiput(80,60)(40,0){2}{\makebox(10,10){\sml 2}}
\multiput(30,10)(40,0){2}{\makebox(10,10){\sml 2}}
\multiput(30,20)(0,10){3}{\makebox(10,10){\sml 2}}
\multiput(110,40)(10,0){2}{\makebox(10,10){\sml 2}}
\multiput(40,0)(10,0){2}{\makebox(10,10){\sml 2}}
\multiput(90,0)(10,0){2}{\makebox(10,10){\sml 2}}
\multiput(40,70)(20,0){2}{\makebox(10,10){\sml 2}}
\multiput(90,80)(20,0){2}{\makebox(10,10){\sml 2}}
\put(100,50){\makebox(10,10){\sml 2}}
\put(110,10){\makebox(10,10){\sml 2}}

\put(30,50){\makebox(10,10){\sml 3}}
\put(50,70){\makebox(10,10){\sml 3}}
\put(60,10){\makebox(10,10){\sml 3}}
\put(80,10){\makebox(10,10){\sml 3}}
\put(90,40){\makebox(10,10){\sml 3}}
\put(90,60){\makebox(10,10){\sml 3}}
\put(100,80){\makebox(10,10){\sml 3}}
\put(110,20){\makebox(10,10){\sml 3}}
\put(110,60){\makebox(10,10){\sml 3}}

\put(40,30){\makebox(10,10){\sml 4}}
\put(80,40){\makebox(10,10){\sml 4}}
\put(100,30){\makebox(10,10){\sml 4}}
\put(70,40){\makebox(10,10){\sml 5}}
\put(50,50){\makebox(10,10){\sml 6}}

\multiput(40,10)(10,0){2}{\makebox(10,10){\smallmine}}
\multiput(90,10)(10,0){2}{\makebox(10,10){\smallmine}}
\multiput(40,20)(60,0){2}{\makebox(10,10){\smallmine}}
\multiput(50,30)(10,0){5}{\makebox(10,10){\smallmine}}
\put(110,30){\makebox(10,10){\smallmine}}
\multiput(40,40)(20,0){2}{\makebox(10,10){\smallmine}}
\multiput(40,60)(20,0){2}{\makebox(10,10){\smallmine}}
\multiput(90,70)(10,0){3}{\makebox(10,10){\smallmine}}

  \put(-15,0){\makebox(10,10){\sf 1}}
  \put(-15,10){\makebox(10,10){\sf 2}}
  \put(-15,20){\makebox(10,10){\sf 3}}
  \put(-15,30){\makebox(10,10){\sf 4}}
  \put(-15,40){\makebox(10,10){\sf 5}}
  \put(-15,50){\makebox(10,10){\sf 6}}
  \put(-15,60){\makebox(10,10){\sf 7}}
  \put(-15,70){\makebox(10,10){\sf 8}}
  \put(-15,80){\makebox(10,10){\sf 9}}
  \put(-15,90){\makebox(10,10){\sf 10}}
  \put(-15,100){\makebox(10,10){\sf 11}}

  \put(0,110){\makebox(10,10){\sf 1}}
  \put(10,110){\makebox(10,10){\sf 2}}
  \put(20,110){\makebox(10,10){\sf 3}}
  \put(30,110){\makebox(10,10){\sf 4}}
  \put(40,110){\makebox(10,10){\sf 5}}
  \put(50,110){\makebox(10,10){\sf 6}}
  \put(60,110){\makebox(10,10){\sf 7}}
  \put(70,110){\makebox(10,10){\sf 8}}
  \put(80,110){\makebox(10,10){\sf 9}}
  \put(90,110){\makebox(10,10){\sf 10}}
  \put(100,110){\makebox(10,10){\sf 11}}
  \put(110,110){\makebox(10,10){\sf 12}}
  \put(120,110){\makebox(10,10){\sf 13}}
  \put(130,110){\makebox(10,10){\sf 14}}
  \put(140,110){\makebox(10,10){\sf 15}}
  \put(150,110){\makebox(10,10){\sf 16}}
}

\newpage

\section{יצוג של שער \L{\texttt{or}}}

\begin{center}
\selectlanguage{english}
\begin{picture}(170,120)
\usebox{\orgate}
\end{picture}
\end{center}
את שני ערכי הקלט מניחים ב-%
$(6,2)$--$(6,3)$
וב-%
$(10,6)$--$(9,6)$.
התוצאה מופיעה ב-%
$(6,14)$--$(6,15)$.

נראה שאם הקלט משמאל הוא )פתוח/מוקש( 
$1$,
והקלט מלמעלה הוא )מוקש/פתוח(
$0$,
אז התוצאה מימין הוא )פתוח/מוקש( 
$1$.
הקורא מוזמן לבדוק את הפתרונות לחידה עבור
\L{($0$ \texttt{or} $0$), ($1$ \texttt{or} $0$), ($1$ \texttt{or} $1$)}.

התצורה התחילתית היא:
\begin{center}
\selectlanguage{english}
\begin{picture}(170,120)
\usebox{\orgate}
\put(10,50){\smallopen}
\put(20,50){\smallmine}
\put(50,90){\smallmine}
\put(50,80){\smallopen}
\end{picture}
\end{center}

\newpage

ה-%
$3$
ב-%
$(6,4)$
מחייב שהתא ב-%
$(6,5)$
יהיה פתוח, וה-%
$3$
ב-%
$(8,6)$ 
מחייב הנחת מוקש ב-%
$(7,6)$.
\begin{center}
\selectlanguage{english}
\begin{picture}(170,120)
\usebox{\orgate}
\put(10,50){\smallopen}
\put(20,50){\smallmine}
\put(40,50){\smallopen}
\put(50,90){\smallmine}
\put(50,80){\smallopen}
\put(50,60){\smallmine}
\end{picture}
\selectlanguage{hebrew}
\end{center}
עלינו להניח מוקשים כך שלתא
$(6,6)$
עם המספר
$6$
יהיו ששה שכנים. יש שתי אפשרויות: )1( תא פתוח ב-%
$(6,7)$
ומוקש ב-%
$(5,6)$,
או )2( תא פתוח ב-%
$(5,6)$
ומוקש ב-%
$(6,7)$.

\subsection*{אפשרות
$1$}
תא פתוח ב-%
$(6,7)$
ומוקש ב-%
$(5,6)$.
\begin{center}
\selectlanguage{english}
\begin{picture}(170,120)
\usebox{\orgate}
\put(10,50){\smallopen}
\put(20,50){\smallmine}
\put(40,50){\smallopen}
\put(60,50){\smallopen}
\put(50,90){\smallmine}
\put(50,80){\smallopen}
\put(50,60){\smallmine}
\put(50,40){\smallmine}
\end{picture}
\selectlanguage{hebrew}
\end{center}
התא
$(7,8)$
עם
$2$
מחייב הנחת מוקש ב-%
$(6,8)$
והתא ב-%
$(4,5)$
עם
$4$
מחייב שהתא 
$(3,6)$
יהיה פתוח.
\begin{center}
\selectlanguage{english}
\begin{picture}(170,120)
\usebox{\orgate}
\put(10,50){\smallopen}
\put(20,50){\smallmine}
\put(40,50){\smallopen}
\put(60,50){\smallopen}
\put(70,50){\smallmine}

\put(50,90){\smallmine}
\put(50,80){\smallopen}
\put(50,60){\smallmine}
\put(50,40){\smallmine}
\put(50,20){\smallopen}
\end{picture}
\selectlanguage{hebrew}
\end{center}
תא 
$(6,9)$
עם
$1$
מחייב שתא 
$(6,10)$
יהיה פתוח, והתא
$(2,7)$
עם
$3$
מחייב הנחת מוקשים ב-%
$(3,7),(3,8)$:
\begin{center}
\selectlanguage{english}
\begin{picture}(170,120)
\usebox{\orgate}
\put(10,50){\smallopen}
\put(20,50){\smallmine}
\put(40,50){\smallopen}
\put(60,50){\smallopen}
\put(70,50){\smallmine}
\put(90,50){\smallopen}

\put(50,90){\smallmine}
\put(50,80){\smallopen}
\put(50,60){\smallmine}
\put(50,40){\smallmine}
\put(50,20){\smallopen}
\put(60,20){\smallmine}
\put(70,20){\smallmine}
\end{picture}
\selectlanguage{hebrew}
\end{center}

\newpage
התא
$(2,8)$
עם 
$2$
מחייב שתא
$(3,9)$
יהיה פתוח.
\begin{center}
\selectlanguage{english}
\begin{picture}(170,120)
\usebox{\orgate}
\put(10,50){\smallopen}
\put(20,50){\smallmine}
\put(40,50){\smallopen}
\put(60,50){\smallopen}
\put(70,50){\smallmine}
\put(90,50){\smallopen}

\put(50,90){\smallmine}
\put(50,80){\smallopen}
\put(50,60){\smallmine}
\put(50,40){\smallmine}
\put(50,20){\smallopen}
\put(60,20){\smallmine}
\put(70,20){\smallmine}
\put(80,20){\smallopen}
\end{picture}
\selectlanguage{hebrew}
\end{center}
תא 
$(2,9)$
עם
$3$
מחייב הנחת מוקש ב-%
$(3,10)$
והתא
$(4,11)$
עם
$4$
מחייב שהתא
$(5,11)$
יהיה פתוח. אבל במצב זה, תא
$(5,10)$
אינו עיקבי כי רשום בו
$3$
אבל יש לו רק שני תאים שכנים עם מוקשים.
\begin{center}
\selectlanguage{english}
\begin{picture}(170,120)
\usebox{\orgate}
\put(10,50){\smallopen}
\put(20,50){\smallmine}
\put(40,50){\smallopen}
\put(60,50){\smallopen}
\put(70,50){\smallmine}
\put(90,50){\smallopen}
\put(50,90){\smallmine}
\put(50,80){\smallopen}
\put(50,60){\smallmine}
\put(50,40){\smallmine}
\put(50,20){\smallopen}
\put(60,20){\smallmine}
\put(70,20){\smallmine}
\put(80,20){\smallopen}
\put(90,20){\smallmine}
\put(90,40){\smallincon}
\put(100,40){\smallopen}
\end{picture}
\selectlanguage{hebrew}
\end{center}

\newpage

\subsection*{אפשרות
$2$}
תא פתוח ב-%
$(5,6)$
ומוקש בתא
$(6,7)$.
\begin{center}
\selectlanguage{english}
\begin{picture}(170,120)
\usebox{\orgate}
\put(10,50){\smallopen}
\put(20,50){\smallmine}
\put(40,50){\smallopen}
\put(60,50){\smallmine}
\put(50,90){\smallmine}
\put(50,80){\smallopen}
\put(50,60){\smallmine}
\put(50,40){\smallopen}
\end{picture}
\selectlanguage{hebrew}
\end{center}
התא 
$(7,8)$
עם
$2$
מחייב שתא
$(6,8)$
יהיה פתוח, ותא
$(4,5)$
עם
$4$
מחייב הנחת מוקש ב-%
$(3,6)$.
בנוסף, תא
$(6,9)$
עם
$1$
מחייב מוקש ב-%
$(6,10)$.
\begin{center}
\selectlanguage{english}
\begin{picture}(170,120)
\usebox{\orgate}
\put(10,50){\smallopen}
\put(20,50){\smallmine}
\put(40,50){\smallopen}
\put(60,50){\smallmine}
\put(70,50){\smallopen}
\put(50,90){\smallmine}
\put(50,80){\smallopen}
\put(50,60){\smallmine}
\put(50,40){\smallopen}
\put(50,20){\smallmine}
\put(90,50){\smallmine}
\end{picture}
\selectlanguage{hebrew}
\end{center}

\newpage

יש שתי אפשרויות עבור תא
$(2,7)$
עם
$3$.

\subsection*{אפשרות
$2.1$}

תא
$(3,7)$
פתוח ויש מוקש ב-%
$(3,8)$.
\begin{center}
\selectlanguage{english}
\begin{picture}(170,120)
\usebox{\orgate}
\put(10,50){\smallopen}
\put(20,50){\smallmine}
\put(40,50){\smallopen}
\put(60,50){\smallmine}
\put(70,50){\smallopen}
\put(90,50){\smallmine}
\put(50,90){\smallmine}
\put(50,80){\smallopen}
\put(50,60){\smallmine}
\put(50,40){\smallopen}
\put(50,20){\smallmine}
\put(60,20){\smallopen}
\put(70,20){\smallmine}
\end{picture}
\selectlanguage{hebrew}
\end{center}
תא 
$(2,8)$ 
עם
$2$
מחייב הנחת מוקש ב-%
$(3,9)$,
ותא
$(2,9)$
עם
$3$
מחייב שתא 
$(3,10)$
יהיה פתוח. מכאן, ה-%
$4$
בתא
$(4,3)$
מחייב הנחת מוקש ב-%
$(5,11)$, 
אבל אז התא ב-%
$(5,10)$
לא עיקבי כי רשום שם
$3$
אבל יש לו ארבעה שכנים עם מוקשים.
\begin{center}
\selectlanguage{english}
\begin{picture}(170,120)
\usebox{\orgate}
\put(10,50){\smallopen}
\put(20,50){\smallmine}
\put(40,50){\smallopen}
\put(60,50){\smallmine}
\put(70,50){\smallopen}
\put(90,50){\smallmine}
\put(50,90){\smallmine}
\put(50,80){\smallopen}
\put(50,60){\smallmine}
\put(50,40){\smallopen}
\put(50,20){\smallmine}
\put(60,20){\smallopen}
\put(70,20){\smallmine}
\put(80,20){\smallmine}
\put(90,20){\smallopen}
\put(90,40){\smallincon}
\put(100,40){\smallmine}
%\put(100,30){\smallincon}
\end{picture}
\selectlanguage{hebrew}
\end{center}

\newpage

\subsection*{אפשרות
$2.2$}

מוקש מונח ב-%
$(3,7)$
ותא 
$(3,8)$
פתוח.
\begin{center}
\selectlanguage{english}
\begin{picture}(170,120)
\usebox{\orgate}
\put(10,50){\smallopen}
\put(20,50){\smallmine}
\put(40,50){\smallopen}
\put(60,50){\smallmine}
\put(70,50){\smallopen}
\put(90,50){\smallmine}
\put(50,90){\smallmine}
\put(50,80){\smallopen}
\put(50,60){\smallmine}
\put(50,40){\smallopen}
\put(50,20){\smallmine}
\put(60,20){\smallmine}
\put(70,20){\smallopen}
\end{picture}
\selectlanguage{hebrew}
\end{center}
תא 
$(2,9)$
עם
$3$
מחייב הנחת מוקשים ב-%
$(3,9)$
ו-%
$(3,10)$.
\begin{center}
\selectlanguage{english}
\begin{picture}(170,120)
\usebox{\orgate}
\put(10,50){\smallopen}
\put(20,50){\smallmine}
\put(40,50){\smallopen}
\put(60,50){\smallmine}
\put(70,50){\smallopen}
\put(90,50){\smallmine}
\put(50,90){\smallmine}
\put(50,80){\smallopen}
\put(50,60){\smallmine}
\put(50,40){\smallopen}
\put(50,20){\smallmine}
\put(60,20){\smallmine}
\put(70,20){\smallopen}
\put(80,20){\smallmine}
\put(90,20){\smallmine}
\end{picture}
\selectlanguage{hebrew}
\end{center}

\newpage

תא
$(4,11)$
עם
$4$
מחייב שתא
$(5,11)$
יהיה פתוח. תצורה זו עיקבית עם ה-%
$3$
בתא
$(5,10)$.
בנוסף,
$(7,10)$
עם
$3$
מחייב שתא
$(7,11)$
יהיה פתוח.
\begin{center}
\selectlanguage{english}
\begin{picture}(170,120)
\usebox{\orgate}
\put(10,50){\smallopen}
\put(20,50){\smallmine}
\put(40,50){\smallopen}
\put(60,50){\smallmine}
\put(70,50){\smallopen}
\put(90,50){\smallmine}
\put(90,50){\smallmine}
\put(100,40){\smallopen}
\put(100,60){\smallopen}
\put(50,90){\smallmine}
\put(50,80){\smallopen}
\put(50,60){\smallmine}
\put(50,40){\smallopen}
\put(50,20){\smallmine}
\put(60,20){\smallmine}
\put(70,20){\smallmine}
\put(80,20){\smallopen}
\put(90,20){\smallmine}
\end{picture}
\selectlanguage{hebrew}
\end{center}
מכאן אפשר להשלים את הפתרון לחידה במהירות. תא
$(6,11)$
עם 
$2$
מחייב הנחת מוקש ב-%
$(6,12)$,
וזה מחייב שתא
$(6,14)$ 
יהיה פתוח, המחייב שיהיה מוקש בתא
$(6,15)$.
\begin{center}
\selectlanguage{english}
\begin{picture}(170,120)
\usebox{\orgate}
\put(10,50){\smallopen}
\put(20,50){\smallmine}
\put(40,50){\smallopen}
\put(60,50){\smallmine}
\put(70,50){\smallopen}
\put(90,50){\smallmine}
\put(90,50){\smallmine}
\put(100,40){\smallopen}
\put(100,60){\smallopen}
\put(110,50){\smallmine}
\put(130,50){\smallopen}
\put(140,50){\smallmine}
\put(50,90){\smallmine}
\put(50,80){\smallopen}
\put(50,60){\smallmine}
\put(50,40){\smallopen}
\put(50,20){\smallmine}
\put(60,20){\smallmine}
\put(70,20){\smallmine}
\put(80,20){\smallopen}
\put(90,20){\smallmine}
\end{picture}
\selectlanguage{hebrew}
\end{center}
התא הפתוח ב-%
$(6,14)$
והמוקש ב-%
$(6,15)$
מהווים פלט )פתוח/מוקש(
$1$
הערך של
\L{($0$ \texttt{or} $1$)}.

\newpage

\textbf{\Large נספח}

\appendix

\section{\L{SAT} היא \L{NP-complete}}

בעיה 
$Q$
היא
\L{NP-complete}
אם:
\begin{itemize}
\item
ניתן לבדוק בזמן פולינומיאלי אם פתרון מוצע ל-%
$Q$
נכון.
\item
ניתן לתרגם את כל הבעיות במשפחה לבעיות מסוג 
$Q$,
כך שאם יש ל-%
$Q$
אלגוריתם פלינומיאלי, אזי לכל הבעיות במשפחה יש אלגוריתם פולינומיאלי.
\end{itemize}
נתונה הצבה עבור נוסחה
$A$
בצורה
\L{CNF},
קל לבדוק עם ערך האמת של
$A$
הוא
$T$,
ולכן בעיית
\L{SAT}
מקיימת את התנאי הראשון. בעיית 
\L{SAT}
היא 
\L{NP-complete}
כי היא מקיימת את התנאי השני, כפי שהוכח על ידי
\L{Stephen Cook (1971), Leonid Levin (1973)}.



התנאי הראשון שקול לטענה שבעיה במשפחת
\L{NP-complete}
ניתנת לפתרון בזמן פולינומיאלי על ידי אלגוריתם
\L{\emph{non-deterministic}}.
השאלה אם בעיות 
\L{NP-complete}
ניתנות לפתרון בזמן פולינומיאלי על ידי אלגוריתם
\L{deterministic}
ידועה בשם
$\mathcal{P}=\mathcal{NP}?$.

אם קיים אלגוריתם יעיל )המתבצע בזמן פולינומיאלי( בחישוב
\L{deterministic}
לבעיה במשפחת
\L{NP-complete},
אז קיים אלגוריתם יעיל לכל הבעיות במשפחה. נכון להיום, לא ידוע אם קיים אלגוריתם יעיל לאף אחת מבעיות.

אם ניתן להוכיח 
\textbf{שאין}
אלגוריתם יעיל לאחת מהבעיות במשפחת
\L{NP-complete},
אז אין אלגוריתם יעיל לכל הבעיות במשפחה. נכון להיום, אין הוכחה שאין אלגוריתם יעיל לאחת מהבעיות במשפחה.


ה-%
\L{Clay Mathematics Institute}
מציע פרס של מיליון דולר למי שיצליח למצוא תשובה ל-%
$\mathcal{P}=\mathcal{NP}?$\\
\L{\url{http://claymath.org/millennium-problems/p-vs-np-problem}}.

\L{NP-completeness}
מוצגת בספרי לימוד בתיאוריה של מדעי המחשב כגון:

\selectlanguage{english}
Hopcroft, J.E, Motwani, R., Ullman, J.D. \textit{Introduction to Automata Theory, Languages, and Computation}, Third edition,  Addison-Wesley, 2006.

Cormen, T.H., Leiserson, C.E., Rivest, R.L., Stein, C. \textit{Introduction to Algorithms}, Second edition, MIT Press, 2001.

Sipser, M. \textit{Introduction to the Theory of Computation}. PWS Publishing, 1997.

\end{document}
