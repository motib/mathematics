% !TeX root = bagrut-806.tex

\selectlanguage{hebrew}

\thispagestyle{empty}

\begin{center}
\textbf{\LARGE בחינות בגרות במתמטיקה: התהליך}
\end{center}

\bigskip
\bigskip

\begin{center}
\textbf{\Large מוטי בן-ארי}

\bigskip

\url{http://www.weizmann.ac.il/sci-tea/benari/}
\end{center}

\begin{center}	
\begin{bfseries}
\bigskip
\bigskip

\R{גרסה} \L{1.4.4} 

\bigskip

\today

\end{bfseries}
\end{center}

\vfill

\selectlanguage{english}

\begin{small}
\begin{center}
\copyright{}\ 2019 \R{מוטי בן-ארי}
\end{center}
This work is licensed under a Creative Commons Attribution-NonCommercial-ShareAlike 4.0 International License (\url{
https://creativecommons.org/licenses/by-nc-sa/4.0/}).
\end{small}

\bigskip

\begin{center}
\includegraphics[width=.3\textwidth]{../../by-nc-sa.png}
\end{center}

\np

\thispagestyle{empty}

\mbox{}

\np

\thispagestyle{empty}

\tableofcontents

\selectlanguage{english}
\cleardoublepage
\selectlanguage{hebrew}

\section*{הקדמה}
\addcontentsline{cot}{section}{הקדמה}

מתמטיקאים ידועים לשמצה כי הם מפרסמים הוכחות מסודרות וברורות, ומסתירים את העובדה שסל הניירות שלהם מלא עד אפס מקום בניסיונות שהובילו למבואות סתומים וטעויות. תלמידים לא נחשפים
\textbf{לתהליכים}
למציאת הפתרונות, וזה עלול לתסכל אותם. הם צריכים ללמוד לא להתייאש כאשר הם לא מצליחים לפתור בעיות בניסיון הראשון. לא חסרים פתרונות של בחינות הבגרות, אבל גם הם "נקיים" ללא ניסיונות שלא צלחו ודיונים על דרכי החשיבה שהובילו לפתרונות.

בחוברת זו פתרונות לבחינות הבגרות שאלון
$806$
מהשנים תשע"ד עד תשע"ח. אני משתדל לתאר את חוויותי בחיפוש פתרונות, כגון הבנה מוטעית של ניסוח השאלות, מלכודות שנפלתי בהם ופתרונות חלופיים שמצאתי. בסוף כל פרק רשמתי המלצות שגיבשתי לאורך העבודה.

השוואתי את הפתרונות שלי לפתרונות המופיעים ברשת, אבל הפתרונות הם שלי ובסגנון שלי. קיצרתי בהצגת חישובים ברורים ואינני משתמש תמיד בדרכים מקובלות להצגת פתרונות, כגון טבלאות בבעיות תנועה.

\subsection*{תנועה והספק}

הצעה של אביטל אלבוים-כהן כיוונה אותי לפתח את תהליך הפתרון של הבעיות הללו באמצעות תרשימים דו-ממדיים. מצאתי שהתרשימים מאוד עוזרים בזיהוי הקשרים בין קטעי התנועה ובכתיבת הנוסחאות. ניתן להיעזר בתרשימים דו-ממדיים גם בבעיות הספק שיש להן מבנה דומה לבעיות תנועה. התרשימים קלים מאוד לציור ומועילים גם אם קני המידה לא מדוייקים, כך שניתן להשתמש בהם כאשר פותרים בחינות.


הציר האופקי בתרשימים הוא ציר הזמן, והציר האנכי הוא ציר המרחק בבעיות תנועה וציר העבודה בבעיות הספק. היתרון של ייצוג זה הוא שמהיריות וההספקים מוצגים כשיפועים של הקווים. ככל שהמהירות או ההפסק גבוה יותר, הקו תלול יותר. לכל דמות )מכונית, סירה, צבע, וכדומה( ציירתי קו עבור כל קטע בתנועה או בעבודה.

המאמר "פתרונות שונים לבעיות הספק באמצעים גרפיים" מאת אביטל אלבוים-כהן וג'ייסון קופר. על"ה גיליון
$51$,
מרץ
$2015$,
עמ'
$14$-$19$,
מביא פתרונות גיאומטריים עבור בעיות הספק.

\subsection*{סדרות}

לדעתי, שאלות על סדרות הן הכי קלות כי בסופו של דבר יש יחסים ברורים בין איברים עוקבים בסדרה )חשבונית או הנדסית(, ובין האיברים לסכומם. עם זאת, מצאתי שקל מאוד לטעות, למשל, אם מבלבלים בין האינדקסים של איברי הסדרה לבין ערכיהם.


\subsection*{הסתברות}

החישובים בבעיות עם הסתברות פשוטים, אבל קשה לתרגם את העלילה המילולית למשוואות הנכונות. הדבר נכון במיוחד כאשר השאלה שואלת על הסתברות מותנית. מצאתי עושר רב של ביטויים המכוונים להסתברות מותנית )ראו בסעיף ההמלצות(, וזה לא מקל על הפתרון.

\np

קושי נוסף נובע מהעובדה שיש שתי דרכים לארגן את המידע הנתון ואת החישובים: בטבלה או בעץ. שאלה המנוסחת "גם א וגם ב" מכוונת לחיתוך של הסתברויות ולטבלה, לעומת שאלה המנוסחת "א ואחר כך ב" שמכוונת למכלפה של הסתברויות ולהצגה בעץ.

\subsection*{גיאומטריה וטריגונומריה}


הפתרונות מביאים ציטוטים של המשפטים המתקדמים מתוך רשימת המשפטים שהתלמידים רשאים לצטט ללא הוכחה. כל אחד זוכר ללא קושי שמשולשים חופפים לפי צ.צ.צ., אבל קשה יותר לזכור משפטים כגון שווין הזווית בין משיק למיתר.

יש חשיבות רבה לציורים גדולים שעליהם ניתן לרשום ערכים, נעלמים ובניות עזר בצורה ברורה. אני ממליץ להכין ציורים שונים לסעיפים שונים של אותה שאלה.

\subsection*{חשבון דיפרנציאלי ואינטגרלי}

בחדו"א שיטות ברורות לחישוב תחומי ההגדרה, נקודות הקיצון וה%
\asms{},
אבל לעתים החישובים ארוכים. חשוב לדייק כי שגיאה בסעיף אחד תגרום לשגיאות בהמשך.

הספר "ללמוד וללמד אנליזה" מציג את הנושא בצורה מקיפה ביותר, ומהווה משאב חשוב למורה.

\selectlanguage{english}
\begin{small}
\url{http://cms.education.gov.il/EducationCMS/Units/}\\\hspace*{3em}\url{Mazkirut_Pedagogit/Matematika/ChativaElyona/Analiza.htm}.
\end{small}

\vspace{-3ex}

\selectlanguage{hebrew}

\subsection*{נספחים}

בנספח א' "הוכחה" ידועה שכל משולש שווה שוקיים. ההוכחה מראה שתרשים אינו תחליף להוכחה.

נספח ב' מכיל ציורים צבעוניים של מספר משפטים מתקדמים בגיאומטריה. בנושא כל כך מוחשי קל יותר לזכור ציור ולא תיאור מילולי מסורבל. כדאי להדפיס עמודים אלה בצבע.

נספח ג' עוסק במעגל היחידה. כאשר אני פותר בעייה בבטריגונומטריה, אני מצייר בצד תרשים של מעגל היחידה כדי לראות את הקשרים של הפונקציות  הטריגונומטריות של זוויות שונות.  למשל, לא כדאי לזכור זהויות כגון
$\sin (180\!-\!\theta)=\sin \theta$
אלא לשחזר אותן מתרשים של מעגל יחידה.

\subsection*{הבעת תודה}

אני מודה לד"ר רונית בן-בסט לוי ולד"ר אביטל אלבוים-כהן שליוו אותי בצלילה למתמטיקה של בתי ספר תיכוניים, חמישים שנה לאחר שסיימתי את לימודי!

\npchap
