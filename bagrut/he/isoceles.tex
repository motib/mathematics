% !TeX root = bagrut-806.tex

\selectlanguage{hebrew}

\chapter{אין לסמוך על איורים}\label{a.isoceles}

הנה הוכחה "נכונה"
\textbf{\R{שכל}}
משולש שווה שוקיים!

נתון משולש שרירותי 
$ABC$,
תהי
$P$
נקודת החיתוך בין חוצה הזווית של
$\angle BAC$
לבין האנך האמצעי של 
$BC$.
סימנו ב-%
$D,E,F$
את נקודות החיתוך של האנחים מ-%
$P$
לצלעות
$BC,AB,AC$. $\triangle APE\cong \triangle APF$
כי הם משולשים ישר זווית עם זוויות שוות
$\alpha$
וצלע $AP$ משותף.

$\triangle DPB\cong \triangle DPC$
לפי צ.ז.צ. כי 
$PD$
הוא צלע משותף, ו-%
$BD=DC=a$
כי 
$PD$
הוא האנך האמצעי ל-%
$BC$.
$\triangle EPB\cong \triangle FPC$,
כי
$EP=PF$
לפי החפיפה הראשונה, ו-%
$PB=PC$
לפי החפיפה השנייה. נחבר את השוויונות ונקבל ש-%
$\triangle ABC$
שווה שוקיים:
\[
AB=AE+EB=AF+FC=AC\,.
\]
\vspace{-2ex}
\begin{center}
\selectlanguage{english}
\begin{tikzpicture}[scale=1.3]
\coordinate (P) at (0,0);
\node[xshift=4mm,yshift=1mm] at (P) {$P$};
\coordinate [label=left:$B$] (B)  at (-2,-2);
\coordinate [label=right:$C$] (C)  at (4,-2);
\coordinate [label=above:$A$] (A)  at (-1,2);
\node[below,yshift=-12pt,xshift=2pt] at (A) {$\alpha$};
\node[below,yshift=-12pt,xshift=15pt] at (A) {$\alpha$};
\draw[thick] (A) -- (B);
\draw[thick] (A) -- (C);
\draw[thick] (B) -- (C);
\draw[thick] (A) -- (P);
\draw[thick] (B) -- (P);
\draw[thick] (C) -- (P);
\coordinate[label=left:$E$] (E) at ($ (A) ! .44 ! (B) $);
\draw[rotate=-100] (E) rectangle +(4pt,4pt);
\draw[thick] (P) -- (E);
\coordinate[label=right:$F$] (F) at ($ (A) ! .33 ! (C) $);
\draw[rotate=-132] (F) rectangle +(4pt,4pt);
\draw[thick] (P) -- (F);
\coordinate[label=below:$D$] (D) at ($ (B) ! .33 ! (C) $);
\draw (D) rectangle +(4pt,4pt);
\draw[thick] (P) -- (D);
\node[left] at ($ (A) ! .5 ! (E) $) {};
\node[left] at ($ (B) ! .5 ! (E) $) {};
\node[below] at ($ (B) ! .5 ! (D) $) {$a$};
\node[below] at ($ (C) ! .5 ! (D) $) {$a$};
\node[right,xshift=2pt] at ($ (A) ! .5 ! (F) $) {};
\node[right,xshift=2pt] at ($ (C) ! .5 ! (F) $) {};
\fill (A) circle(1pt);
\fill (B) circle(1pt);
\fill (C) circle(1pt);
\fill (D) circle(1pt);
\fill (E) circle(1pt);
\fill (F) circle(1pt);
\fill (P) circle(1pt);
\end{tikzpicture}
\end{center}
הבעיה בהוכחה היא שהאיור אינו נכון כי הנקודה
$P$
נמצאות
\textbf{\R{מחוץ}}
למשולש:
\begin{center}
\selectlanguage{english}
\includegraphics[width=.5\textwidth]{isoceles}
\end{center}

\selectlanguage{english}
