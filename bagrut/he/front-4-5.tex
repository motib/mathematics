% !TeX root = chapter-4-5.tex

\thispagestyle{empty}

\begin{center}
\textbf{\LARGE בחינות בגרות במתמטיקה: החוויה}

\bigskip
\bigskip

\textbf{\LARGE גיאומטריה, טריגונומטריה}
\end{center}

\input{common-title-page}

\section*{הקדמה}
\addcontentsline{cot}{section}{הקדמה}

מתמטיקאים ידועים לשמצה כי הם מפרסמים והוכחות מסודרות וברורות, ומסתירים את העובדה שסל הניירות שלהם מלא עד אפס מקום בניסיונות שהובילו למבואות סתומים וטעויות. ההיעדר של 
\textbf{תהליכי}
הפתרון עלול לתסכל תלמידים שמתייאשים כאשר הם לא מצליחים לפתור בעיות בניסיון הראשון. לא חסרים פתרונות של בחינות הבגרות, אבל גם הם "נקיים" ללא ניסיונות שלא צלחו, פתרונות שונים לאותה בעייה, ודיונים על דרכי החשיבה שהובילו לפתרונות.

מסמך זה מכיל פתרונות לשאלות בפרק השני של הבחינות 
$806$
בשנים תשע"ד עד תשע"ח. פרק 
$4$
על גיאומטריה ופרק
$5$
על טריגונומטריה. השתדלתי לתאר את תהיליכי הפתרון, מלכודות אפשריים, וטעויות שנובעות מהיסח דעת או רשלנות.

הפתרונות מביאים ציטוטים של המשפטים המתקדמים מתוך רשימת המשפטים שהתלמידים רשאים לצטט ללא הוכחה. כל אחד זוכר ללא קושי שמשולשים חופפים לפי צ.צ.צ., אבל קשה יותר לזכור משפטים כגון שווין הזווית בין משיק למיתר.

מצאתי שיש חשיבות רבה לציורים גדולים שעליהם ניתן לרשום ערכים, נעלמים ובניות עזר בצורה ברורה. אני גם ממליץ להכין ציורים שונים לסעיפים שונים של אותה שאלה.

בסוף הפרקים רשמתי המלצות שגיבשתי לאורך העבודה.

נספח א' מכיל "הוכחה" ידועה שכל משולש שווה שוקיים. עם כל החשיבות לציורים בהבנת דמויות גיאומטריות, ציור אינו מהווה הוכחה.

נספח ב' מכיל ציורים צבעוניים של מספר משפטים מתקדמים. דווקא בנושא כל כך מוחשי כגון גיאומטריה קל יותר לזכור ציור ולא תיאור מילולי מסורבל.

נספח ג' עוסק בבטריגונומטריה של מעגל היחידה. לדעתי לא כדאי לזכור זהויות כגון
$\sin (180-\theta)=\sin \theta$,
אלא לצייר מעגל יחידה קטן במחברת "ולראות" את הזהויות.

\np
\mbox{}
\np
