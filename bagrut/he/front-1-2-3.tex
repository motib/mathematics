% !TeX root = chapter-1-2-3.tex

\selectlanguage{hebrew}

\thispagestyle{empty}

\begin{center}
\textbf{\LARGE בחינות בגרות במתמטיקה: החוויה}

\bigskip
\bigskip

\textbf{\LARGE אלגברה והסתברות}

\bigskip
\bigskip
\bigskip
\bigskip

\textbf{\Large מוטי בן-ארי}

\bigskip
\bigskip

\url{http://www.weizmann.ac.il/sci-tea/benari/}
\end{center}

% !TeX root = construct.tex

\begin{center}
\begin{bfseries}
\bigskip
\bigskip
גרסה
$2.1$

\bigskip

\today

\end{bfseries}
\end{center}


\vfill

\selectlanguage{english}

\begin{footnotesize}
\begin{center}
\copyright{}\ 2017--19 by Moti Ben-Ari.
\end{center}

This work is licensed under the Creative Commons Attribution-ShareAlike 3.0 Unported License. To view a copy of this license, visit \url{http://creativecommons.org/licenses/by-sa/3.0/} or send a letter to Creative Commons, 444 Castro Street, Suite 900, Mountain View, California, 94041, USA.
\end{footnotesize}

\bigskip

\begin{center}
\includegraphics[width=.15\textwidth]{../../by-sa.png}
\end{center}

\np
\thispagestyle{empty}
\mbox{}
\np
\thispagestyle{empty}

\tableofcontents
\np

\section*{הקדמה}
\addcontentsline{cot}{section}{הקדמה}

מתמטיקאים ידועים לשמצה כי הם מפרסמים והוכחות מסודרות וברורות, ומסתירים את העובדה שסל הניירות שלהם מלא עד אפס מקום בניסיונות שהובילו למבואות סתומים וטעויות. ההיעדר של 
\textbf{תהליכי}
הפתרון עלול לתסכל תלמידים שמתייאשים כאשר הם לא מצליחים לפתור בעיות בניסיון הראשון. לא חסרים פתרונות של בחינות הבגרות, אבל גם הם "נקיים" ללא ניסיונות שלא צלחו ודיונים על דרכי החשיבה שהובילו לפתרונות.

מסמך זה מכיל פתרונות לשאלות  
$1,2,3$
)תנועה והספק, סדרות, הסתברות( של הבחינות 
$806$
בשנים תשע"ד עד תשע"ח עם תיאורים של חוויותי בחיפוש פתרונות.

בסוף כל פרק רשמתי המלצות שגיבשתי לאורך העבודה.
\subsection*{תנועה והספק}

הצעה של אביטל אלבוים-כהן כיוון אותי לפתח את הפתרון של הבעיות ההלו באמצעות תרשימים דו-ממדיים. מצאתי שהתרשימים מאוד עוזרים בזיהוי הקשרים בין קטעי התנועה ובכתיבת הנוסחאות. ניתן להיעזר בתרשימים דו-ממדיים גם בבעיות הספק שיש להן מבנה דומה לבעיות תנועה. התרשימים קלים מאוד לציור ומועילים גם אם קני המידה בכלל לא מדוייקים, כך שניתן להשתמש בהם כאשר פותרים בחינות.


בתרשימים הציר האופקי הוא ציר הזמן והציר האנכי הוא ציר המרחק בבעיות תנועה וציר העבודה בבעיות הספק. היתרון של ייצוג זה הוא שמהיריות וההספקים מוצגים כשיפועים של הקווים. ככל שהמהירות או ההפסק גבוה יותר, הקו תלול יותר. לכל דמות )מכונית, סירה, צבע, וכדומה( ציירתי קו עבור כל קטע בתנועה או בעבודה. בבעיות תנועה, יש לשים לב שבניגוד לתרשימים חד-ממדיים בהם אורך קו הוא מרחק הנסיעה, כאן מרחק הנסיעה הוא ההפרש בציר האנכי בין הנקודה ההתחלתית לנקודה הסופית.

אני ממליץ על המאמר "פתרונות שונים לבעיות הספק באמצעים גרפיים" מאת אביטל אלבוים-כהן וג'ייסון קופר. על"ה גיליון
$51$,
מרץ
$2015$,
עמ'
$14$-$19$.
הם מביאים פתרונות גיאומטריים עבור בעיות הספק.

\subsection*{סדרות}

לדעתי, שאלות הסדרות הן הכי קלות לפתור, כדי בסופו של דבר יש יחסים ברורים בין איברים עוקבים בסדרה )חשבונית או הנדסית(, וכן בין האיברים לסכומם. עם זאת, מצאתי שקל מאוד לטעות, למשל, אם מבלבלים בין האינדקסים של איברי הסדרה לבין ערכיהם.

חשבו לשים לב שתת-סדרה של סדרה חשבונית / הנדסית לא "יורשת" את התכונה חשבונית / הנדסית, להיפך, סדרה הבנוייה מתת-סדרות חשבוניות / הנדסיות אינה בהכרח סדרה שהיא חשבונית / הנדסית.

לפני שניגשים לפתרון של שאלה עם סדרה באורך
$n$,
כדאי לרשום וחשב סדרה עם מספר קטן של איברים. כך אפשר לקבל תחושה של היחסים בין איברי הסדרה, ונוסף, זה עוזר כדי להבין מתי לסדרה יש מספר זוגי של איברים ומתי יש מספר אי-זוגי של איברים.

\np

\subsection*{הסתברות}

החישובים בבעיות עם הסתברות הם בדרך כלל פשוטים, אבל קשה יחסית לתרגם את העלילה המילולית למשוואות הנכונות. הדבר נכון במיוחד כאשר השאלה שואלת על הסתברות מותנית. מצאתי עושר רב של ביטויים המכוונים להסתברות מותנית )סיכמתי אותם בסעיף ההמלצות(, וזה לא מקל על הפתרון.

קושי נוסף נובע מהעובדה שיש שתי דרכים שונות לארגן את המידע הנתון והחישובים: בטבלה או בעץ. שאלה על משהו שהוא "גם א וגם ב" מכוון לחיתוך של הסתברויות, ומכוונת לטבלה, לעומת שאלה המנוסחת "א ואחר כך ב" מכוונות למכלפה של הסתברויות הכדאי להציג בעץ. ככל שפותרים יותר שאלות, 
קל יותר להבחין באפיונים של הבעייה ובחור את הדרך הנכונה למצוא פתרון.


\selectlanguage{english}
