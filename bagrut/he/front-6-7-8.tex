% !TeX root = chapter-6-7-8.tex

\selectlanguage{hebrew}

\thispagestyle{empty}

\begin{center}
\textbf{\LARGE בחינות בגרות במתמטיקה: החוויה}

\bigskip
\bigskip

\textbf{\LARGE חשבון דיפרנציאלי ואינטגרלי}
\end{center}

\input{common-title-page}

\section*{הקדמה}
\addcontentsline{cot}{section}{הקדמה}

בנוסחאון נתון:
\[
\textrm{(\R{ממשי} t)} \quad\quad (x^t)' = tx^{t-1}\,,
\]
אולם משתמשים בנוסחה רק עבור 
$t$
שלם וחיובי. אני מעדיף להשתמש בנוסחה עבור כל
$t$
כי קל לזכור את הנוסחה והחישובים פשוטים. למשל, הנוסחה הנתונה 
\[
(\sqrt{x})' = \disfrac{1}{2\sqrt{x}}
\]
מיותרת כי:
\[
(\sqrt{x})' = (x^\frac{1}{2})'= \frac{1}{2}(x^{-\frac{1}{2}}) = \frac{1}{2}\cdot\frac{1}{x^{\frac{1}{2}}}=\disfrac{1}{2\sqrt{x}}\,.
\]
החיסכון בולט יותר כאשר מנסים לחשב נגזרת רציונלית:
\[
\left(\frac{1}{x^t}\right)'\,.
\]
משתמשים בנוסחה המובכת עבור נגזרת של מנה כאשר
$f(x)=1, g(x)=x^t$.
לדעתי, פשוט יותר לחשב:
\[
\left(\frac{1}{x^t}\right)'=(x^{-t})'=-t(x^{-t-1})=\frac{-t}{x^{t+1}}\,.
\]
כאשר במנה יש קבוע ובמכנה יש פונקציה מורכבת החישוב לא מסובך במיוחד. למשל:
\erh{12pt}
\begin{equationarray*}{rcl}
\left(\frac{14}{x^2-3x+4}\right)'&=&14\left((x^2-3x+4)^{-1}\right)'\\
&=&14\cdot -1\cdot (x^2-3x+4)^{-2}(x^2-3x+4)'\\
&=&-14(x^2-3x+4)^{-2}(2x-3)\\
&=&\frac{-14(2x-3)}{(x^2-3x+4)^{-2}}\,.
\end{equationarray*}


כמובן שאתם חופשיים להשתמש בנוסחאות האחרות.

\erh{0pt}
\begin{equationarray*}{rcl}
\sin 2x &=& \sin (x+x) = \sin x \cos x + \sin x \cos x = 2\sin x \cos x\\
\cos 2x &=& \cos (x+x) = \cos x \cos x - \sin x \sin x = \cos^2 x - \sin^2 x\\
&=&(1-\sin^2 x) - \sin^2 x = 1 - 2\sin^2 x\,.
\end{equationarray*}

\selectlanguage{english}
