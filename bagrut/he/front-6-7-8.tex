% !TeX root = chapter-6-7-8.tex

\selectlanguage{hebrew}

\thispagestyle{empty}

\begin{center}
\textbf{\LARGE בחינות בגרות במתמטיקה: החוויה}

\bigskip
\bigskip

\textbf{\LARGE חשבון דיפרנציאלי ואינטגרלי}
\end{center}

\input{common-title-page}

\section*{הקדמה}
\addcontentsline{cot}{section}{הקדמה}

מתמטיקאים ידועים לשמצה כי הם מפרסמים והוכחות מסודרות וברורות, ומסתירים את העובדה שסל הניירות שלהם מלא עד אפס מקום בניסיונות שהובילו למבואות סתומים וטעויות. ההיעדר של 
\textbf{תהליכי}
הפתרון עלול לתסכל תלמידים שמתייאשים כאשר הם לא מצליחים לפתור בעיות בניסיון הראשון. לא חסרים פתרונות של בחינות הבגרות, אבל גם הם "נקיים" ללא ניסיונות שלא צלחו, פתרונות שונים לאותה בעייה, ודיונים על דרכי החשיבה שהובילו לפתרונות.

מסמך זה מכיל פתרונות לשאלות בפרק השלישי של הבחינות 
$806$
בשנים תשע"ד עד תשע"ח. 

\subsection*{שאלה 
$6$}

השאלות יחסית פשוטות כי יש שיטות מוגדרות לחישוב תחומי הגדרה, נקודות קיצון ו%
\asms{}.
עם זאת, לעתים החישובים הם ארוכים וחשוב לדייק כדי שגיאה בסעיף אחד יגרום לשגיאות בסעיפים הבאים.

\subsection*{שאלה 
$7$}

טרם נכתב.

\subsection*{שאלה 
$8$}

\subsection*{הערות על נוסחאות}

בנספח רשמתי כמה הערות על הדרך שלי לחשב נוסחאות הנחוצות הפתרון הבעיות בפרק זה. מומלוץ לעיין בנספח לפני קריאת הפתרונות.
טרם נכתב.
\selectlanguage{english}
