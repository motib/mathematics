\documentclass[12pt,a4paper]{article}
\usepackage[utf8x]{inputenc}
\usepackage[english,hebrew]{babel}
\usepackage{graphicx}
\usepackage{verbatim}
\usepackage{url}

\textwidth=15.5cm
\textheight=23cm
\topmargin=0pt
\headheight=0pt
\oddsidemargin=2em
\headsep=0pt
\parindent=0pt
\renewcommand{\baselinestretch}{1.1}
\setlength{\parskip}{0.3\baselineskip plus 1pt minus 1pt}

\begin{document}
\thispagestyle{empty}

\selectlanguage{hebrew}

\begin{center}
\textbf{\Huge מתמודדים עם סדרות}

\bigskip
\bigskip

\textbf{\Large מוטי בן-ארי}

\bigskip

\textbf{\Large המחלקה להוראת המדעים}

\bigskip

\textbf{\Large מכון ויצמן למדע}

\bigskip

\url{http://www.weizmann.ac.il/sci-tea/benari/}

\bigskip

\end{center}

\selectlanguage{english}

\begin{center}
\copyright{}\  2018 by Moti Ben-Ari.
\end{center}

\begin{footnotesize}
This work is licensed under the Creative Commons Attribution-ShareAlike 3.0 Unported License. To view a copy of this license, visit \url{http://creativecommons.org/licenses/by-sa/3.0/} or send a letter to Creative Commons, 444 Castro Street, Suite 900, Mountain View, California, 94041, USA.
\end{footnotesize}

\bigskip

\begin{center}
\includegraphics[width=.2\textwidth]{../by-sa.png}
\end{center}

\bigskip
\bigskip
\bigskip

\selectlanguage{hebrew}

אני מודה לרונית בן-בסט לוי על הערותיה המועילות.
\newpage

\begin{center}
\textbf{מבוא}
\end{center}

במסמך זה אנתח את השאלות על סדרות בבחינות האחרונות של שאלון
$806$.
אנסה לחפש תבניות המופיעות בשאלות ואצביע על דרכים לפתרונן. לא אביא כאן את השאלות והפתרונות המלאים, כי ניתן למצוא אותם בקלות באינטרנט. בסוף המסמך רשמתי מסקנות והמלצות למתמודד עם סדרות.


%%%%%%%%%%%%%%%%%%%%%%%%%%%%%%%%%%%%%%%%%%%%%%%%%%%%%%%%%%%%%%%%%%%

\begin{center}
\textbf{ניתוח השאלות על סדרות}
\end{center}

\textbf{\R{
חורף תשע"ד
}}
נתונה סדרה הנדסית אינסופית שממנה מייצרים שתי סדרות נוספות.

סעיף א מבקש להוכיח שסדרה השלישית היא הנדסית. 

כאשר סדרה מתקבלת מסדרה חשבונית / הנדסית נתונה, או כאשר לוקחים סדרה ומפצלים אותה לתת-סדרות, או כאשר בונים סדרה משתי סדרות קיימות, הסדרות החדשות הן
\textbf{לא בהכרח}
חשבוניות / הנדסיות. למשל:
\[
\begin{array}{rrrrrrrrrrr}
1,& 4,& 7,& 10,& 13,& \ldots\\
2,& 5,& 8,& 11,& 14, &\ldots\\
1,& 2,& 4,& 5,& 7,& 8,& 10,& 11,& 13,& 14, &\ldots
\end{array}
\]
שתי הסדרות הראשונות הן סדרות חשבוניות, אבל הסדרה השלישית המתקבלת מצירוף השתיים אינה חשבונית. מידע נתון נוסף הוא הסכומים של שתי הסדרות הראשונות ללא איבר מסויים.

בסעיף ב נתונה הסכום של
$n$
האיברים הראשונים של הסדרה השלישית ומבקשים לחשב את 
$n$.

בנוסחאות לסכום סדרה יש שלושה נעלמים: האיבר הראשון
$a_1$,
ההפרש
$d$
או המנה
$q$,
ומספר האיברים
$n$
)אלא אם מדובר בסדרה הנדסית אינסופית(. 
כאן, ברור שעלינו "להיפטר" מ-%
$a_1$
ו-
$q$.
נשתמש במידע הנתון על הסכום של
$n$
האיברים הראשונים, והחישוב פשוט.

%%%%%%%%%%%%%%%%%%%%%%%%%%%%%%%%%%%%%%%%%%%%%%%%%%%%%%%%%%%%%%%%%%%
\bigskip

\textbf{\R{
קיץ תשע"ד, מועד א
}}
נתונה סדרה חשבונית עם
$3n$
איברים, וקשר בין סכום האיברים בשליש האמצעי של הסדרה ובין שסכום של האיברים בשליש סופי של הסדרה:
$S_3 = 2S_2$.

סעיף א מבקש להוכיח שסכום השליש הראשון של הסדה הוא 0.

כדי לדייק עם האינדקסים, חובה לרשום את הסדרה עם סימון של הסדרות החלקיות:
\[
\underbrace{
\overbrace{\rule{0pt}{8pt}a_1, a_2, \ldots, a_n}^{S_1},
\overbrace{\rule{0pt}{8pt}a_{n+1}, a_{n+2}, \ldots, a_{2n}}^{S_2},
\overbrace{\rule{0pt}{8pt}a_{2n+1}, a_{2n+2}, \ldots, a_{3n}}^{S_3}
}_{S_{3n}}\,.
\]
קיימות שתי דרכים לפתור בעיות מסוג זה. ניתן להתייחס לכל תת-סדרה כסדרה עצמאית, כאשר האיבר הראשון שונה בכל תת-סדרה, ומתקבל על ידי הנוסחאה לאיבר ה-%
$n$
בסדרה:
\[
a_{n+1} = a_1 + nd, \quad a_{2n+1} = a_1 + 2nd\,.
\]

הדרך השניה, שהיא לעתים מסובכת יותר אבל כדאי לשקול להשתמש בה, היא להחסיר את הסכום של תת-סדרה מהסכום של הסדרה כולה. בשאלה זו:
\[
S_1 = S_n - (S_2+S_3)\,.
\]
השאלה על סדרות בבחינה של
\textbf{חורף תשע"ב}
קצת "נבזית". אורך הסדרה הוא
$2n-1$,
ונתונים הסכומים של
$n$
האיברים הראשונים ו-%
$n$
האיברים האחרונים. רק רישום זהיר של הסדרה יבהיר שיש חפיפה בין שתי התת-סדרות!
\[
\renewcommand{\arraystretch}{.4}
\begin{array}{ll}
\overbrace{\rule{0pt}{8pt}a_1, a_2, \ldots, a_n},&\hspace{-9pt}a_{n+1}, a_{n+2}, \ldots, a_{2n-1}\,.\\
&\hspace{-2em}\underbrace{\rule{10em}{0pt}}
\end{array}
\]
יש לנו נטייה לראות 
"$n$
איברים ראשונים ו-%
$n$
איברים אחרונים", ולחשוב שהחלוקה היא תת-סדרות זרות. אבל מספר האיברים
$2n-1$
הוא מספר אי-זוגי, ולכן קיימת חפיפה.

אין כאן שום בעיה בחישוב: האיבר הראשון בתת-סדרה הראשונה היא
$a_1$,
והאיבר הראשון בתת-סדרה השניה היא
$a_n=a_1+(n-1)d$.

נחזור לבחינה של תשע"ד. בסעיף ב נתון הסכום של הסדרה המקורית באורך
$3n$,
ומבקשים לחשב את ההפרש
$d$.
כדי להחשב את ההפרש מהסכום אנו זקוקים גם לערכו של
$a_1$.
במקום זה נתון ש-%
$a_5+a_7=0$,
ונשתמש במשוואה זו כדי לחשב את הקשר בין
$a_1$
ל-%
$d$.
לאחר הצבה של ביטוי עם
$d$
במקום
$a_1$,
מתקבלת משוואה המאפשרת לחשב את
$n$
ומכאן את
$d$.


%%%%%%%%%%%%%%%%%%%%%%%%%%%%%%%%%%%%%%%%%%%%%%%%%%%%%%%%%%%%%%%%%%%
\bigskip

\textbf{\R{
קיץ תשע"ד, מועד ב
}}
נתונה סדרה חשבונית ומשוואות הקושרות את הערכים
$a_n, a_{n+1}, a_{n+2}$:
\begin{eqnarray*}
a_{n+2}^2 - a_n^2 &=&216\\
a_n+a_{n+1}+a_{n+2}&=&54\,.
\end{eqnarray*}
סעיף א מבקש לחשב את
$a_n$.
בגלל שהסדרה חשבונית, ניתן להשתמש בהצבות:
\[
a_{n+1}=a_n+d, \quad a_{n+2}=a_n+2d
\]
כדי לקבל שתי משוואות עם שני נעלמים:
\begin{eqnarray*}
4a_nd+4d^2 &=& 216\\
3a_n + 3d &=& 54\,.
\end{eqnarray*}
הפתרון של המשוואת נותן את ערכים
$d=3,a_n=15$.

דרך אחרת לפתור את השאלה היא להשתמש בעובדה שבסדרה חשבונית:
\[
a_n+a_{n+2} = a_n+a_n+2d = (a_n+d)+(a_n+d) = 2a_{n+1}\,,
\]
ואז להציב 
$\frac{a_n+a_{n+2}}{2}$
עבור
$a_{n+1}$
במשוואה השניה. כעת יש לנו שתי משוואות עם שני נעלמים
$a_n,a_{n+2}$
וניתן לפתור אותן כדי לקבל את ערכו של
$a_n$.

בסעיף ב מוגדרת סדרה חדשה:
\[
a_5, a_9, a_{13},\ldots a_{4k+1}\,.
\]
ונתונים גם סכום סדרה זו והערך של
$a_1$
בסדרה המקורית. השאלה מבקשת לחשב את הערך של
$k$
שהוא מספר האיברים בסדרה החדשה. קל לראות שהסדרה החדשה היא סדרה חשבונית. למעשה, בסעיף א, ניתן גם לחשב את ההפרש של הסדרה המקורית. אם נרשום את הסדרה החדשה בתוך הסדרה המקורית נוכל למצוא את ההפרש החדש:
\[
a_5, \quad a_6=a_5+d,\quad  a_7=a_5+2d,\quad  a_8=a_5+3d,\quad  a_9=a_5+4d
\]
שהיא
$4d$.

%%%%%%%%%%%%%%%%%%%%%%%%%%%%%%%%%%%%%%%%%%%%%%%%%%%%%%%%%%%%%%%%%%%
\bigskip

\textbf{\R{
חורף תשע"ה
}}
נתונה סדדרה המוגדרת על ידי כלל נסיגה.

סעיף א מבקש את סכום שני האיברים האמצעיים של הסדרה אם לסדרה
$100$
איברים. מומלץ לרשום את הסדרה כדי לוודא מהם האיברים האמצעיים:
\[
a_1, a_2, \ldots \quad a_{50}, a_{51}, \quad \ldots, a_{100}\,.
\]
איתור האיברים האמצעיים יהיה קל יותר אם נרשום סדרה קטנה יותר כגון:
\[
a_1, a_2, a_3, \quad a_4,a_5,\quad a_6, a_7, a_8\,.
\]
הסדרה היא לא בהכרח חשבונית או הנדסית, אבל ניתן לחשב את הסכום בקלות מכלל הנסיגה.

סעיף ב מבקש להוכיח שהאיברים במקומות הזוגיים והאיברים במקומות האי-זוגיים מהווים סדרות חשבוניות. האיברים הזוגיים הם האיברים
$a_{2k}$
וצריך להוכיח שההפרש
$a_{2k+2} - a_{2k}$
קבוע, והאיברים האי-זוגיים הם האיברים
$a_{2k+1}$
וצריך להוכיח שההפרש
$a_{2k+1} - a_{2k-1}$
קבוע. הכלל הנסיגה אנו מקבלים ששני ההפרשים קבועים ושווים. שימו לב שהפרשים קבועים של תת-סדרות לא מחייב שהסדרה כולה חשבונית, כפי שראינו בדוגמה קודמת:
\[
\begin{array}{rrrrrrrrrrr}
1,& 4,& 7,& 10,& 13,& \ldots\\
2,& 5,& 8,& 11,& 14, &\ldots\\
1,& 2,& 4,& 5,& 7,& 8,& 10,& 11,& 13,& 14, &\ldots
\end{array}
\]
\vspace{-3ex}

סעיף ג מבקש את האיבר האמצעי כאשר יש 
$101$
איברים בסדרה. שוב, כדאי לרשום את הסדרה:
\[
a_1, a_2, \ldots a_{50}, \quad a_{51}, \quad a_{52}, \ldots, a_{101}\,,
\]
או סדרה קטנה יותר:
\[
a_1, a_2, a_3, a_4, \quad a_5, \quad a_6, a_7, a_8, a_9\,.
\]
מכלל הנסיגה יש לנו את
$a_1$
וההפרש
$d$
חושב בסעיף הקודם, ומהם אפשר לחשב את
$a_{51}$.

סעיף ד מבקש את סכום כל איברי הסדרה. ברור שלא ניתן לחשב אותו באמצעות נוסחה כי הסדרה היא לא בהכרח חשבונית, אבל ניתן לסכם את שתי תת-הסדרות בנפרד ולחבר אותם:
\[
S=S_{\mathit{odd}} + S_{\mathit{even}}\,.
\]
האיבר
$a_1$
נתון בכלל הנסיגה וניתן לחשב את
$a_2$
מכלל הנסיגה. ההפרשים חושבו בסעיף קודם. נשאר רק לדעת כמה איברים יש בכל תת-סדרה. כאן אפשר להתבלבל בקלות ולכן כדאי לבדוק את המספרים. נסמן את מספר האיברים האי-זוגיים ב-%
$k$,
ונבדוק:
\begin{eqnarray*}
k = 1 &\rightarrow& 2k-1=1\\
k = 2 &\rightarrow& 2k-1=3\\
&\cdots&\\
k = 26 &\rightarrow& 2k-1=51\\
&\cdots&\\
k = 51 &\rightarrow& 2k-1=101\,.
\end{eqnarray*}
מספר האיברים האי-זוגיים הוא
$51$
ומספר האיברים הזוגיים הוא
$101-51-50$.

%%%%%%%%%%%%%%%%%%%%%%%%%%%%%%%%%%%%%%%%%%%%%%%%%%%%%%%%%%%%%%%%%%%
\bigskip

\textbf{\R{
קיץ תשע"ה, מועד א
}}
נתונה יחס בין איברים עוקבים בסדרה הנדסית: כל איבר הוא
$2/5$
מסכום האיברים הסמוכים לו.

סעיף א מבקש למצוא את המנה, והחישוב נותן שהמנה היא
$2$
או
$1/2$.
קריאה זהירה של השאלה תגלה שנתונה שהסדרה היא סדרה הנדסית
\textbf{יורדת}, 
ולכן התשובה הנכונה היא 
$1/2$.

בסעיף ב נתונה סדרה חדשה המוגדרת מהסדרה המקורית:
\[
b_n = \frac{a_{n+1}}{a_n^2}\,.
\]
החלק הראשון של הסעיף מבקש להוכיח שהסדרה הנדסית. את החלוקה של שני איברים עוקבים אפשר לבטא באמצעות ההגדרה של הסדרה
$b_n$
לפי הסדרה המקורית
$a_n$
ולהוכיח שהמנה קבועה. חלק השני של הסעיף מבקש את הסכום של הסדרה הראשונה שהיא אינסופית. נתונה סכום חלקי של הסדרה החדשה, וממנו אפשר לחשב את
$a_1$.
המנה חושבה בסעיף הראשון ומאפשר חישוב סכום הסדרה האינסופית.


%%%%%%%%%%%%%%%%%%%%%%%%%%%%%%%%%%%%%%%%%%%%%%%%%%%%%%%%%%%%%%%%%%%
\bigskip

\textbf{\R{
קיץ תשע"ה, מועד ב
}}
נתונה סדרה המוגדרת על ידי כלל נסיגה.

סעיף א מבקש להוכיח שהאיברים הזוגיים מהווים סדרה הנדסית וגם האיברים האי-זוגיים מהווים סדרת הנדסית. כמו במבחן של חורף תשע"ה, צריך לבדוק את שני המקרים בנפרד תוך הקפדה על מיספור נכון של האיברים:
\[
\frac{b_{2k+2}}{b_{2k}},\quad \frac{b_{2k+1}}{b_{2k-1}}\,.
\]
בסעיף ב נתון הסכום של 
$8$
האיברים הראשונים. נוח כאן לחשב בנפרד את הסכום של ארבעת האיברים הזוגיים וארבעת האיברים האי-זוגיים. מתקבלת משוואה ריבועית ולכן יש שני פתרונות.


%%%%%%%%%%%%%%%%%%%%%%%%%%%%%%%%%%%%%%%%%%%%%%%%%%%%%%%%%%%%%%%%%%%
\newpage

\textbf{\R{
חורף תשע"ו
}}
נתונה ההפרשים בין איברים מסויים בסדרה הנדסית עולה:
\[
(1)\, a_4-a_3 = 4 (a_2-a_1),\quad (2)\, a_6 - a_1 = 31\,,
\]
סעיף א מבקש את המנה
$q$
והאיבר הראשון
$a_1$
של הסדרה. נציב במשוואה
$(1)$
 את הביטויים המתקבלים מ-%
$a_n=a_1q^{n-1}$
עבור
$a_2, a_3, a_4, a_6$.
באופן קצת מפתיעה מתקבלות ארבע תשובות:
\[
a_1 = 0\,\quad q=1,\quad q=2\, \quad q=-2\,.
\]
אם נקרא את השאלה בזהירות נראה שנתון שהסדרה היא סדרה הנדסית
\textbf{עולה},
זה מסתדר רק עם התשובה
$q=2$.
כעת ניתן לחשב את
$a_1$
מהמשוואה
$(2)$.

בסעיף ב יש נקודות רגישות
\ldots.
נתונות שתי סדרות I ו-II המוגדרות מהסדרה המקורית. החלק הראשון של הסעיף שואל אם הסדרות הנדסיות. מחלוקת איברים סמוכים מקבלת שני ערכים קבועים ולכן שתיהן הנדסיות. אבל, השאלה שואלת האם הסדרות הן הנדסיות 
\textbf{עולות},
והמנה של סדרה II היא
$1$
לכן התשובה שלילית.

כעת נתון הסכום של הסדרה I. החלק השני של הסעיף מבקש את מספר האיברים בסדרה I. הסדרה היא: 
\[
a_1\cdot a_2,\quad a_2\cdot a_3,\quad \ldots,\quad a_n\cdot a_{n+1},\quad a_{n+1} \cdot a_{n+2}\,.
\]
הוכחנו שהסדרה היא סדרה הנדסית ובחישוב הסכום נתקבל
$n=6$.
לכאורה, מספר האיברים בסדרה הוא
$7$,
כי האיבר האחרון הוא:
\[
a_{n+1} \cdot a_{n+2} = a_{7} \cdot a_{8}\,.
\]
\textbf{אבל זה לא אותו}
$n$.
בדיוק אמרנו שחישבנו
\textbf{שמספר האיברים בסדרה הוא}
$6$:
\[
(1)\, a_1\cdot a_2,\quad (2)\,a_2\cdot a_3,\quad(3)\, a_3\cdot a_4,\quad (4)\,a_4\cdot a_5,\quad (5)\,a_5\cdot a_6,\quad (6)\,a_6\cdot a_7 \;(= a_{n+1}\cdot a_{n+2})\,,
\]
ולכן, ה-%
$n$
של
\textbf{הסדרה המקורית}
הוא
$5$.

החלק השלישי של הסעיף מבקש את הסכום של האיברים בסדרה II:
\[
\frac{a_2}{a_1}+\frac{a_3}{a_2},\quad
\frac{a_3}{a_2}+\frac{a_4}{a_3},\quad
\ldots,
\frac{a_{n+1}}{a_n}+\frac{a_{n+2}}{a_{n+1}}\,.
\]
בסעיף הקודם חישבנו את
$a_1$
והמנה של סדרה II שהיא
$1$,
ולכן החישוב פשוט אם נדע את מספר האיברים בסדרה. נציב את
$n=5$
בהסדרה II נקבל
$5$
איברים:
\[
(1)\,\frac{a_2}{a_1}+\frac{a_3}{a_2},\quad
(2)\,\frac{a_3}{a_2}+\frac{a_4}{a_3},\quad
(3)\,\frac{a_4}{a_3}+\frac{a_5}{a_4},\quad
(4)\,\frac{a_5}{a_4}+\frac{a_6}{a_5},\quad
(5)\,\frac{a_6}{a_5}+\frac{a_7}{a_6} \left(= \frac{a_{n+1}}{a_n}+\frac{a_{n+2}}{a_{n+1}}\right)\,.
\]

%%%%%%%%%%%%%%%%%%%%%%%%%%%%%%%%%%%%%%%%%%%%%%%%%%%%%%%%%%%%%%%%%%%
\newpage

\textbf{\R{
קיץ תשע"ו, מועד א
}}
נתון הסכום של מספר איברים של סדרה חשבונית:
\[
a_4+a_8+a_{12}+a_{16}=224\,.
\]
סעיף א מבקש את סכום 
$19$
האיברים הראשונים. בתת-סדרה הנתונה, ניתן להחליף כל איבר ב-% 
$a_n=a_1+(n-1)d$
ומקבלים משוואה עם שני נעלמים
$a_1+9d=56$.
לכאורה אין לנו מספיק נתונים: כדי לחשב
$S_{19}$
צריכים את ערכם של
$n, a_1, d$.
נתון ש-%
$n=19$
אבל אין לנו ערכים עבור 
$a_1, d$,
אלא רק משוואה המקשרת את ערכם.
\textbf{לא להתייאש},
נחשב את הסכום
$S_{19}$
ונקווה לטוב. ואכן יש לנו מזל והסכום מכיל את הביטוי 
$a_1+9d$,
ובהצבת ערכו
$56$
מקבלים את התשובה.

השאלה ממשיכה עם סדרה של
\textbf{סכומים חלקיים}:
\[
S_1, S_2, \ldots, S_n\,.
\]
בסדרה חשבונית, ניתן לקבל את האיברים מהסכומים:
$a_1 = S_1,\,a_{n+1}=S_{n+1}-S_n$.

נתון מידע הנוסף ש-%
$S_n=n\cdot a_n$.

סעיף ב מבקש להראות שהפרש הסידרה הוא
$0$.
כאשר משווים 
$S_n=n\cdot a_n$
עם הנוסחה הרגילה לסכום של סדרה חשבונית, מקבלים משוואה מעט מוזרה
$d=d/2$,
אבל ברור שהמשוואה נכונה רק כאשר
$d$
הוא
$0$.

סעיף ג מבקש את ערכו של
$a_1$.
התשובה מתקבלת מייד הצבה של ערכו של
$d$
בהמשוואה
$a_1+9d=56$.

נתונה סדרה חדשה המוגדרת על ידי המשוואה:
$b_{n+1}-b_n=a_n+S_n$.

סעיף ד מבקש לחשב:
\[
(b_2-b_1) + (b_3-b_2) + \cdots + (b_{20}-b_{19}).
\]
במבט ראשון אנו רואים שהסדרה "מתקפלת" ל-%
$b_{20}-b_1$,
אבל זה מבוי סתום כי אין לנו דרך לחשב איברי הסדרה
$b_n$.
לכן יש לחשב את סכום
$19$
הביטויים
$(b_{i+1}-b_i)$,
תוך שימוש במשוואה הנתונה להפרשים. החישוב פשוט כי ההפרש של הסדרה המקורית הוא
$0$.

%%%%%%%%%%%%%%%%%%%%%%%%%%%%%%%%%%%%%%%%%%%%%%%%%%%%%%%%%%%%%%%%%%%
\bigskip

\textbf{\R{
קיץ תשע"ו, מועד ב
}}
נתונה סדרה חשבונית עם הפרש
$3$.

סעיף א מבקש להראות שהיחס בין הסכום של סדרה זו לבין הסכום של סדרה חשבונית חדשה הוא 
$\frac{2n-1}{n}$.
הסדרה החדשה מתקבלת על ידי הכנסת איבר נוסף בין כל שני איברים של הסדרה המקורית. כאן חשוב לשים לב שמספר האיברים החדשים הוא
$n-1$,
כפי שרואים אם רושמים את הסדרה:
\[
a_1, a'_1, a_2, a'_2, \ldots, a_{n-1}, a'_{n-1}, a_n\,.
\]
חשוב לשים לב לנתון שגם הסדרה החדשה היא חשבונית, לכן, יש מצב לא שגרתי שההפרש של הסדרה החדשה אינו מספר שלם אלא
$1.5$.
הוכחת היחס בין שני הסכומים היא מיידית אם חישבנו נכון את מספר האיברים של כל אחת מהסדרות.

החלק השני של הסעיף מבקש לחשב את האיבר הראשון בסדרה. נתון היחס 
$\frac{2n-1}{n}$
שממנו אפשר לחשב את מספר האיברים בסדרה כולה וכן בתת-סדרה. נתונה גם הסכום של 
\textbf{האיברים החדשים}
שהוכנסו לסדרה. כמו כן, ברור שההפרש של תת-סדרה זו הוא
$3$,
ומכאן נשאר רק נעלם אחד
$a_1$
לחשב.

סעיף ב מכליל את סעיף א כי הוא מבקש את ההפרש כאשר מכניסים
$k$
איברים חדשים בין כל שני איברים עוקבים של הסדרה המקורית, במקום להכניס רק איבר חדש אחד. נתון גם שהסדרה החדשה היא חשבונית, ולכן סכום ההפרשים החדשים חייב להיות
$3$.
ההפרש החדש מתקבל אם נרשום את הסדרה החדשה ונספור את מספר ההפרשים החדשים:
\[
a_i\rightarrow b_1\rightarrow b_2, \rightarrow \ldots, \rightarrow b_k\rightarrow a_{i+1}\,.
\]

בספירה:
\[
(1)\, a_i\rightarrow b_1,\quad (2)\, b_1 \rightarrow b_2, \quad \ldots, \quad (k)\, b_{k-1} \rightarrow b_k,\quad (k+1)\, b_k\rightarrow a_{i+1}\,,
\]
אנו רואים שיש
$k+1$
הפרשים, וערכו של ההפרש הוא
$\frac{3}{k+1}$.

%%%%%%%%%%%%%%%%%%%%%%%%%%%%%%%%%%%%%%%%%%%%%%%%%%%%%%%%%%%%%%%%%%%
\bigskip

\textbf{\R{
חורף תשע"ז
}}
נתונה סדרה
$a_n$
המוגדרת על ידי כלל נסיגה, וסדרה חדשה המוגדרת על ידי:
\[
b_n = \frac{1}{a_n} + 2\,.
\]
סעיף א מבקש להוכיח ש-%
$b_n$
היא סדרה הנדסית. חישוב היחס בין איברים עוקבים מראה שהמנה קבועה.

סעיף ב מבקש למצוא את הסכום של:
\[
\frac{1}{a_1} + \frac{1}{a_2} + \cdots + \frac{1}{a_n}\,.
\]
לא נתון האם הסדרה $a_n$ היא חשבונית או הנדסית, אבל, למזלנו, הוכחנו בסעיף הקודם שהסדרה 
$b_n$
היא הנדסית, והעברת הקבוע
$2$
לצד השני של המשוואה 
$\frac{1}{a_n} = b_n - 2$
מאפשרת לבטא את הסדרה
$\frac{1}{a_1}$
כסכום של 
$n$
פעמים הקבוע
$2$,
ועוד הסכום של הסדרה
$b_n$.
את המנה חישבנו בסעיף א והאיבר הראשון
$b_1$
ניתן לחשב מהערך של
$a_1$
הנתון בכלל הנסיגה.

סעיף ג מבקש לחשב את הסכום:
\[
\frac{1}{a_1} - \frac{1}{a_2} - \cdots + \frac{1}{a_{n-1}} - \frac{1}{a_n}\,.
\]
כאשר
$n$
זוגי. נשתמש במשוואה המקשרת את שתי הסדרות:
\[
(b_1 - 2) - (b_2 - 2) + \cdots + (b_{n-1} - 2) + (b_n - 2)\,.
\]
הסימנים מתחלפים ולכן הסכום של הקבועים מתאפס כי נתון
\textbf{שמספר האיברים זוגי}.
חישוב הסכום של הסדרה:
\[
b_1 - b_2 + \cdots + b_{n-1} - b_n
\]
הוא אותו חישוב מסעיף ג רק עם החלפת הסימן של המנה.

%%%%%%%%%%%%%%%%%%%%%%%%%%%%%%%%%%%%%%%%%%%%%%%%%%%%%%%%%%%%%%%%%%%
\bigskip

\textbf{\R{
קיץ תשע"ז, מועד א
}}
נתונה סדרה המוגדרת על ידי:
\[
a_n = \frac{(2^{n}+1)(2^{n}-1)}{2^{n}}\,.
\]
שימו לב שלא מדובר בכלל נסיגה כי איברים של הסדרה לא מופיעים בצד הימני של המשוואה. המשוואה פשוט מגדריה את
$a_n$
כפונקציה של
$n$.

נתונה שאיברי הסדרה הם ההפרשים בין האיברים של שתי סדרות
\textbf{הנדסיות}
$a_n = b_n - c_n$.
נתונה גם ערכם של שני האיברים
$b_6, c_3$.
אם נחשב את
$a_3, a_6$
לפי הפונקציה הנתונה, נוכל לקבל את הערכים של
$b_3, c_6$.

סעיף א מבקש למצוא את הערכים של
$b_1, c_1$,
ואת המנות של שתי הסדרות. בדרך כלל אנו מחשבים מנה על ידי
$q_x=x_{n+1}/x_n$,
אבל כאן אין לנו את הערכים של איברים עוקבים, ולכן החישוב הוא:
\[
\renewcommand{\arraystretch}{2}
\begin{array}{l@{\quad\quad\quad}l}
b_6 = b_3q_b^3 \rightarrow q_b^3 = \frac{b_6}{b_3}&
b_3 = b_1 q_b^2 \rightarrow b_1 = \frac{b_3}{q_b^2}\\
c_6 = c_3q_c^3 \rightarrow q_c^3 = \frac{c_6}{c_3} &
c_3 = c_1 q_c^2 \rightarrow c_1 =\frac{c_3}{q_c^2}\,.
\end{array}
\]
סעיף ב מבקש להוכיח ש-%
$C_n=B_n-A_n$.
ההוכחה פשוטה:
\begin{eqnarray*}
C_n &=& (b_1-a_1) + (b_2 - a_2) + \cdots + (b_n-a_n)\\
&=&(b_1 + b_2 + \cdots + b_n) + (a_1 + a_2 + \cdots + a_n)\\
&=& B_n - A_n\,.
\end{eqnarray*}
סעיף ג מבקש את הערכים עבורם
$0.9 < B_n-A_n < 1$.
בסעיף הקודם הוכחנו ש-%
$C_n=B_n-A_n$,
ונתונה שהסדרה 
$c_n$
היא סדרה הנדסית. מחישוב הסכום מקבלים
$C_n = 1-2^{-n}$.
בדיקה מראה ש-%
$0.9 \not< 1-2^{-3}=0.875$,
אבל
$0.9 < 1-2^{-4}=0.9375 < 1$.
שימו לב שהתשובה המליאה היא "כל מספר
\textbf{גדול או שווה}
ל-%
$4$",
כי גם עבור מספרים גדולים מ-%
$4$
הביטוי
$1-2^{-n}$
בתחום.


%%%%%%%%%%%%%%%%%%%%%%%%%%%%%%%%%%%%%%%%%%%%%%%%%%%%%%%%%%%%%%%%%%%
\bigskip

\textbf{\R{
קיץ תשע"ז, מועד ב
}}
שאלה זו שונה משאלות אחרות בבחינות, כי נתונה סדרה לא על ידי ביטוי עבור האיברים, אלא על ידי 
\textbf{ביטוי עבור הסכומים}.
בנוסף הביטוי כולל פרמטר:
\[
S_n = k-\frac{1}{3^{n+1}}\,.
\]
סעיף א מבקש ביטויים עבור
$a_1$
ו-%
$a_n$.
אלה מתקבלים מ-%
$a_1=S_1$
ו-%
$a_n=S_{n+1}-S_n$.

סעיף ב מבקש למצוא את הערכים של 
$k$
עבורם הסדרה הנדסית. חישוב של 
$a_{n+1}/a_n$
נותן מנה קבועה
\textbf{שאיננה תלויה}
ב-%
$k$.
במבט ראשון נראה שהתשובה היא שהסדרה הנדסית עבור כל ערך של
$k$,
אולם זו
\textbf{טעות}!
המנה המתקבלת מ-%
$a_2/a_1$
חייבת להיות
\textbf{שווה למנה}
המתקבלת במקרה הכללי
$a_{n+1}/a_n$.
בדיקה קצרה מראה שיש רק ערך אחד של
$k$
המקיים את השוויון.

סעיף ג מבקש לחשב את הסכום של סדרה חדשה:
\[
a_2^2 + a_5^2 + a_8^2 + \cdots\,.
\]
אנו יודעים שהסדרה
$a_n$
הנדסית, ולכן גם סדרה זו הנדסית. )יש כאן הנחה סמויה שסדרה זו מתקבלת מהסדרה עם הערך של
$k$
שהתקבל בסעיף ב.( המנה של סדרה זו מתקבלת על ידי העלאת המנה של הסדרה המקורית לחזקת
$3$
כי הסדרה החדשה מוגדרת על ידי של איבר שלישי, ובנוסף, העלאה בריבוע כי האיברים בסדרה החדשה מוגדרים על ידי העלאת האיברים מהסדרה המקורית בריבוע. בסך הכל
$q'=(q^3)^2=q^6$.
את הערך של האיבר הראשון
$a_2^2$
ניתן לחשב מהנוסחה לסכומים הנתונה בתחילת התרגיל.

%%%%%%%%%%%%%%%%%%%%%%%%%%%%%%%%%%%%%%%%%%%%%%%%%%%%%%%%%%%%%%%%%%%
\bigskip

\textbf{\R{
חורף תשע"ח
}}
נתונה סדרה חשבונית שההפרש שלה שונה מאפס. נתון גם
$a_7=-a_{17}$.

סעיף א מבקש את ערכו של
$a_{12}$.
לכאורה יש כאן בעיה כי יש שני נעלמים
$a_1, d$
אבל רק משוואה אחת נתונה, לכן לכל היותר נוכל לקבל משוואה המקשרת את ערכם של שני הנעלמים האלה. בכל זאת נצלול לחישוב ונגלה שהכל מצטמצם ו-%
$a_{12}=0$.

החלק הראשון של סעיף ב שואל אם יש איבר שערכו
$-a_1$.
נשווה את
$-a_1$
לנוסחה לאיבר כללי:
\[
-a_1 = a_n = a_1 + (n-1)d\,,
\]
ונשתמש במשוואה המקשרת בין
$a_1$
לבין 
$d$
שקיבלנו בסעיף א, ונקבל את התשובה
$n=23$.

החלק השני של סעיף ב ממשיך ומבקש ערך של 
$n$
שעבורו סכום 
$n$
האיברים הראשונים שווה לאפס. נשתמש בנוסחה לסכום של סדרה חשבונית, ובקשר בין 
$a_1$
לבין
$d$
מסעיף א:
\begin{eqnarray*}
0 &=& \frac{n}{2}(2a_1+(n-1)d)\\
&=& \frac{n}{2}(2\cdot -11d+(n-1)d)\\
&=& d n (n-23)\,.
\end{eqnarray*}
בתחילת השאלה נתון שערכו של ההפרש שונה מאפס, והשאלה אומרת ש-%
$n$
מספר טבעי ולכן שונה מאפס. מכאן, שהתשובה היחידה היא
$n=23$.

דרך אחרת, אינטואיטיבית יותר, מתחילה מ-%
$a_{23}=-a_1$.
נתבונן בסכום של
$23$
האיברים הראשונים בסדרה:
\[
a_1 + (a_1+d) + (a_1+2d) + \cdots + (a_1+21d) + (a_1+22d)\,.
\]
הסדרה חשבונית עם הפרש כלשהו
$d$,
לכן אם
$a_{23} = -a_1$,
אז
$a_{22} = -a_2$,
ו-%
$a_{21} = -a_3$,
וכך הלאה. ברור שהסכום הוא אפס. 

סעיף ג שואל האם יש זוג איברים עוקבים עם סימנים הפוכים
$a_n\cdot a_{n+1} < 0$.
בסדרה חשבונית שאחד מאיבריה הוא אפס אבל עם הפרש ללא אפס, לכל זוג איברים עוקבים
\textbf{אותו סימן},
פרט לשני הזוגות מסביב לאפס:
\[
a_k,\; a_{k+1}=0,\; a_{k+2}\,,
\]
עבורם
$a_k\cdot a_{k+1} = 0, a_{k+1}\cdot a_{k+2} = 0$,
כך שאין זוג של איברים עוקבים שמכפלתם שלילי.

סעיף ד שואל כמה איברים שליליים יש בסדרה, ונותן רמז שיש שתי אפשרויות. התשובה קופצת לעין אם רושמים את הסדרה:
\[
a_1, a_2, \ldots, a_{11}, a_{12}=0, -a_{11}, \ldots, -a_2, -a_1\,.
\]
או ש-%
$11$
האיברים הראשונים הם שליליים )אם ההפרש חיובי( או כל האיברים לאחר האיבר
$a_{12}=0$
שליליים )אם ההפרש שלילי(.

%%%%%%%%%%%%%%%%%%%%%%%%%%%%%%%%%%%%%%%%%%%%%%%%%%%%%%%%%%%%%%%%%%%
\newpage

\textbf{\R{
קיץ תשע"ח, מועד א
}}
השאלה בסעיף א יפה מאוד כי אין דרישה לחישובים, רק לחשיבה! נתונה סדרה הנדסית אין-סופית יורדת שסכומה שלילי, השאלה שואלת מה חייב להיות נכון עבור המנה
$q$
והאיבר הראשון 
$a_1$.
אני מעדיף להתחיל עם דוגמה, למרות שאין צורך בשאלה זו. הסדרה ההנדסית המתכנסת הפשוטה ביותר היא:
\[
1+ \frac{1}{2} + \frac{1}{4} + \frac{1}{8} + \cdots = 2\,.
\]
אם נהפוך את כל הסימנים למינוס, נקבל סדרה שסכומה שלילי:
\[
-1 - \frac{1}{2} - \frac{1}{4} - \frac{1}{8} - \cdots = -\left(1+ \frac{1}{2} + \frac{1}{4} + \frac{1}{8} + \cdots\right) = -2\,.
\]
ברור שאין שינוי למנה שהיא חיובית
$\frac{1}{2}$.
לכן, אפשר לפסול מייד תשובות 
\L{I, II, IV}
ונשאר רק תשובה
\L{III}
ש-% 
$a_1$
חייב להיות שלילי.

בכל זאת נבדוק לפי הנוסחה שסכום הסדרה. כדי שסדרה תתכנס, הערך המוחלט של המנה חייב להיות פחות מאחד. מהנוסחה עבור הסכום:
\[
S = \frac{a_1}{1-q}
\]
קל לראות )תנסו כמה ערכים כגון 
$\frac{7}{8}, \frac{1}{8}, -\frac{1}{8}, -\frac{7}{8}$
כדי להשתכנע( שחייב להתקיים:
\[
0 < 1-q < 2\,.
\]
לכן הסכום שלילי רק אם
$a_1$
שלילי.

בסעיף ב נתון יחס בין סכום האיברים הזוגיים וסכום האיברים האי-זוגיים. המנה היא
$q^2$
ולכן הסכומים הם:
\[
T = \frac{a_1}{1-q^2},\quad\quad R = \frac{a_1q}{1-q^2}\,.
\]
מהנתון
$T+pR=0$,
חישוב פשוט מראה ש-%
$1+pq=0$
ו-%
$p=-\displaystyle\frac{1}{q}$.

סעיף ג שואל על סדרה הנדסית עם מנה
$p$.
סדרה זו לא מתכנסת, כי 
$|q|<1$
גורר
$|p|>1$.

סעיף ד. 
\textbf{מלכודת!}
בגלל שסעיף ג התייחס לסדרה
$b_n$,
חשבתי בטעות שסעיף זה גם מתייחס לאותה סדרה. אבל השאלה שואלת על הסדרה המקורית
$a_n$.

מהנתון ש-%
$p$
שלילי, הנוסחה 
$p=-\displaystyle\frac{1}{q}$
והנתון שהסדרה
$a_n$
מתכנסת, קל לראות ש-%
$a_{n+1}>a_n$
ללא קשר לסימן של האיבר הראשון.

%%%%%%%%%%%%%%%%%%%%%%%%%%%%%%%%%%%%%%%%%%%%%%%%%%%%%%%%%%%%%%%%%%%
\newpage

\textbf{\R{
קיץ תשע"ח, מועד ב
}}
נתונה סדרה המוגדרת על ידי כלל נסיגה והאיבר הראשון שתלוי בפרמטר חיובי.

סעיף א מבקש להראות שסדרת האיברים הזוגיים וגם סדרת האיברים האי-זוגיים הן סדרות הנדסיות. בדרך כלל, הפתרון מתקבל על ידי חישוב החילוק
$a_{n+1}/a_{n-1}$,
אולם, אם מציבים את הנוסחה מכלל הנסיגה, מקבלים חילוק נוסף. הדרך הנכונה היא פשוט להציב את הנוסחה של כלל הנסיגה בתוך עצמו:
\[
a_{n+1} = -\frac{c^{n-2}}{a_n} = -\frac{c^{n-2}}{-\frac{c^{n-3}}{a_{n-1}}} = ca_{n-1}\,.
\]
מייד מקבלים ש-%
$a_{n+1}/a_{n-1}=c$,
והחישוב לא תלוי בזוגיות של האיברים.

בסעיף ב יש להיזהר כי הסדרות של הזוגיים והאי-זוגיים מהווים סדרות הנדסיות נפרדות, ולכן יש לחשב את האיברים
$(a_1,a_3,a_5,a_7), (a_2,a_4,a_6)$
בנפרד. האיברים בסדרה מצמצמים אחד את השני פרט לאיבר הראשון שערכו הוא ערך הסכום. זה נכון גם עבור תת-סעיף 
$(2)$
שמבקש את הסכום של שבעת האיברים הראשוניים, וגם עבור תת-סעיף
$(3)$
שמבקש את הסכום של תת-סדרה כללית. בסעיף 
$(3)$
צריך להשתמש בנוסחה לסכום פעמיים, בנפרד עבור הזוגיים ועבור האי-זוגיים.

בסעיף ג
$(1)$,
החישוב פשוט:
\[
\frac{b_{n+1}}{b_n} = \frac{a_n}{a_{n+2}} =  \frac{1}{\frac{a_{n+2}}{a_n}} = \frac{1}{c}\,.
\]
בתת-סעיף
$(2)$
הסדרה יורדת, ולכן
$0 < q=\frac{1}{c} < 1$.
ברור ש-%
$c$
חייב להיות חיובי, אבל זה נתון והתנאי הוא
$c>1$.

בתת-סעיף 
$(3)$
מחשבים את 
$b_1$
מהערכים הידועים של 
$a_1,a_2$,
ואז ניתן להשתמש בנוסחה עבור סכום סדרה אינסופית.

%%%%%%%%%%%%%%%%%%%%%%%%%%%%%%%%%%%%%%%%%%%%%%%%%%%%%%%%%%%%%%%%%%%

\newpage

\begin{center}
\textbf{מסקנות והמלצות}
\end{center}

\begin{itemize}
\item
\textbf{חובה לקרוא את השאלות בזהירות רבה.}
יש לשים לב אם סדרה היא חשבונית, הנדסית או לא זו ולא זו. מידע נוסף שנותנים: אם סדרה עולה או יורדת, ערכים קבועים או תחום של ערכים )למשל, ההפרש אינו אפס(.
\item 
לפעמים כתוב במפורש "היעזר בסעיפים הקודמים". גם ללא הנחיה מפורשת זו, בדרך כלל צריך להשתמש לא רק במה שנתון, אלא גם בתוצאות הקודמות. לעתים רחוקות, שאלה מורכבת משני סעיפים בלתי תלויים, והשאלה תציין זאת במפורש.
\item
חובה לרשום את הסדרות, במיוחד כאשר מופיע בשאלה פירוק לתת-סדרות. מקרה מעניין היה סדרה עם תת-סדרות חופפות:
\[
\renewcommand{\arraystretch}{.4}
\begin{array}{ll}
\overbrace{\rule{0pt}{8pt}a_1, a_2, \ldots, a_n},&\hspace{-9pt}a_{n+1}, a_{n+2}, \ldots, a_{2n-1}\\
&\hspace{-2em}\underbrace{\rule{10em}{0pt}}
\end{array}
\]
\vspace{-4ex}
\item
קיימות שתי דרכים לפתור שאלה המחלקת סדרה לתת-סדרות ומבקשת את הסכומים שלהן. דרך אחת היא לסכם את התת-סדרות עם
$a_1, d, n$
שלהן. דרך אחרת היא לחשב סכום תוך שימוש בסכומים שחישבנו קודם או שניתנו:
\[
S_1 = S_n - (S_2+S_3)\,.
\]
\vspace{-6ex}
\item
רצוי לרשום דוגמה מספרית ולא רק סדרה עם 
$n$
איברים, במיוחד כאשר שואלים על סדרות עם מספר זוגי או אי-זוגי של איברים:
\[
\begin{array}{l}
a_1, a_2, \ldots \quad\quad\quad a_{50}, a_{51}, \quad \ldots, a_{100}\\
a_1, a_2, \ldots a_{50}, \quad\quad a_{51}, \quad\quad a_{52}, \ldots, a_{101}\,.
\end{array}
\]
\vspace{-4ex}
\item 
מאוד שכיח ששואלים על סדרה
\textbf{שאיננה}
חשבוניות / הנדסיות אבל שיש לה תת-סדרות חשבוניות / הנדסיות:
\[
\begin{array}{rrrrrrrrrrr}
1,& 4,& 7,& 10,& 13,& \ldots\\
2,& 5,& 8,& 11,& 14, &\ldots\\
1,& 2,& 4,& 5,& 7,& 8,& 10,& 11,& 13,& 14, &\ldots
\end{array}
\]
מצב זה מופעה לעתים קרובות כאשר הסדרה המקורית מוגדרת על ידי כלל נסיגה או פונקציה, ואז מגדירים ממנה סדרה אחרת שהיא כן חשבונית / הנדסית. כדי לקבל את הסכום של הסדרה כולה, מחשבים את הסכומים של שתי תת-הסדרות ומחברים את התוצאות.
\item
בסדרה קיימים ארבעה נעלמים
$a_1, d / q, n, S_n$.
כדי לפתור בעיה המבקשת ערך מספרי של נעלם אחד, צריך לדעת ערכים של שלושת הנעלמים האחרים )או שניים אם לא מדובר בסכום(. לעתים מבקשים לבטא נעלם אחד, למשל הסכום, כתלות בנעלם אחר, למשל מספר האיברים, ואז כמובן אפשר להסתדר עם פחות ערכים. לפעמים, מספיק לדעת את הקשר בין שני נעלמים כדי לפתור בעיה:
$a_1+11d = 0$.
אם נראה שאין מספיק נתונים, נסו בכל זאת את החישוב, וסביר שתקבלו תשובה.

\end{itemize}

\end{document}
