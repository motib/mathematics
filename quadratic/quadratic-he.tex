\documentclass[12pt,a4paper]{article}

\usepackage[utf8x]{inputenc}
\usepackage[english,hebrew]{babel}

\usepackage{mathpazo}
\usepackage{verbatim}
\usepackage{url}

\usepackage{tikz}
\tikzset {>=stealth}

\textwidth=15.5cm
\textheight=24.5cm
\topmargin=0pt
\headheight=0pt
\oddsidemargin=2em
\headsep=0pt
\parindent=0pt
\renewcommand{\baselinestretch}{1.1}
\setlength{\parskip}{0.3\baselineskip plus 1pt minus 1pt}

\begin{document}
\thispagestyle{empty}

\selectlanguage{hebrew}

\begin{center}
\textbf{\LARGE שיטה חילופית לפתרון למשוואות ריבועיות}

\bigskip
\bigskip

\textbf{\Large מוטי בן-ארי}

\bigskip

\url{http://www.weizmann.ac.il/sci-tea/benari/}

\bigskip

\L{\copyright{}\  2020 by Moti Ben-Ari.}
\end{center}

\selectlanguage{english}
\begin{footnotesize}
This work is licensed under the Creative Commons Attribution-ShareAlike 3.0 Unported License. To view a copy of this license, visit \url{http://creativecommons.org/licenses/by-sa/3.0/} or send a letter to Creative Commons, 444 Castro Street, Suite 900, Mountain View, California, 94041, USA.
\end{footnotesize}

\selectlanguage{hebrew}

\section{מבוא}

מסמך זה מציג שיטה חלופית לפתרון משוואות ריבועיות של 
\L{[1, 2] Po-Shen Loh}.
הוספתי דוגמאות נוספות ופירוט של החישובים.

בסעיף~%
\ref{s.traditional}
חזרה על השיטות המסוריות לפתרון משוואות ריבועיות. השיטה של 
\L{Loh}
מבוסס על תובונה פשוטה מאוד שקשה לקלוט בצורה אינטואיטיבית. סעיף~%
\ref{s.computing}
מנסה לשכנע את הקורא שהתובנה מתקבלת על הדעת, ואז מסביר את השיטה לחישוב שורשים. פרטי החישוב עבור שתי דוגמאות נמצאים בסעיף~%
\ref{s.examples}.
סעיף~%
~\ref{s.general}
מראה איך הנוסחה במסורתית מתקבלת מהנוסחאות של
\L{Loh}.
סעיף~%
\ref{s.irreducible}
מראה שהשיטה עובדת גם עבור פולינומים שיש להם שורשים מרוכבים כי הם לא פריקים מעל למספרים הרציונליים. פרט לסעיף האחרן, תוכן המסמך אמור להיות נגיש לכל המורים והתלמידים בבתי ספר תיכוניים.

\section{השיטות המסורתיות לפתרון משוואות ריבועיות}\label{s.traditional}

כל תלמיד בבית ספר תיכון לומד בעל-פה את הנוסחה למצוא את השורשים של משוואה ריבועית
$ax^2+bx+c=0$:
\[
x_1, x_2 = \frac{-b\pm\sqrt{b^2-4ac}}{2a}\,.
\]
לעת עתה נעבוד רק עם משוואות שהמקדם הראשון הוא אחד. השואשים של
$x^2+bx+c=0$
הם:
\[
x_1, x_2 = \frac{-b\pm\sqrt{b^2-4c}}{2}\,.
\]

שיטה נוספת לפתרון משוואות ריבועיות היא לפרק את הפולינום הריבועי, פחות או יותר בניסוי וטעייה. לעיתם קל לפרק את הפולינום:
\begin{eqnarray*}
x^2-4x+3& =& (x-r_1)(x-r_2)=0\\
& =& (x-1)(x-3)=0\\
x_1,x_2&=&1, 3\,.
\end{eqnarray*}
קשה הרבה יותר לפרק:
\[
x^2-2x-24= (x-r_1)(x-r_2)=0\,.
\]
השורשים האפשריים
$(r_1,r_2)$
הם:
\[
(\pm 1,\mp 24)\,, (\pm 2,\mp 12)\,, (\pm 3,\mp 8)\,, (\pm 4,\mp 6)\,.
\]
ברור שהסימנים של
$r_1,r_2$
חייבים להיות שונים כי המכפלה שלילית
$-24$,
אבל עדיין עלינו לבדוק שמונה אפשרויות.



\section{חישוב השורשים}\label{s.computing}

\textbf{אם}
$r_1,r_2$
\textbf{הם השורשים של}
$x^2+bx+c$,
אזי:%
\footnote{%
\L{Loh}
מדגיש את ההבדל בין שיטתו לבין פירוק. בפירוק, אני 
\textbf{מניחים}
שקיימים שני שורשים. ההנחה נכונה לפי המשפט הבסיסי של אלגברה, אבל זה משפט "כבד" עבור המשימה הפשוטה של מציאת שורשים של משוואה ריבועית. לפי שיטתו, אנחנו רק אומרים:
\textbf{אם השורשים קיימים}.%
}
\[
(x-r_1)(x-r_2)=x^2 - (r_1+r_2)x + r_1r_2=x^2+bx+c\,.
\]
אפילו אם אין אנו יודעים את ערכי השורשים, אנו כן יודעים ש:
\[
r_1+r_2 = -b\,,\quad\quad r_1r_2=c\,.
\]
נסתכל על מספר ערכים עבור
$-b,r_1,r_2$
ונסמן ב-%
$m_{12}$
את הממוצע של
$r_1,r_2$:
\[
\renewcommand{\arraystretch}{1.3}
\begin{array}{|r|r|r|r|}
\hline
-b& r_1 & r_2 &m_{12}\\\hline
33 & 12 & 21 & 16\frac{1}{2}\\\hline
33 & 8 & 25 & 16\frac{1}{2}\\\hline
33 & 1 & 32 & 16\frac{1}{2}\\\hline
\hline
-4 & -16 & 12 & -2 \\\hline
-4 & -4 & 0 & -2 \\\hline
-4 & -3 & -1 & -2 \\\hline
\end{array}
\]
עבור כל משוואה ריבועית, הממוצע של שני השורשים קבוע:
\[
\frac{r_1+r_2}{2}=
\frac{(-b-r_2)+r_2}{2}=
\frac{-b}{2}+\frac{-r_2+r_2}{2}=
-\frac{b}{2}\,.
\]
יהי 
$s$ 
מספר כלשהו; אזי:
\[
-b=-b+s+(-s)=\left(\frac{-b}{2}+s\right) + \left(\frac{-b}{2}-s\right)=r_1+r_2\,.
\]
אם שורש אחד נמצא במרחק
$s$
מהממוצע, השורש השני נמצא במרחק
$-s$
מהממוצע:
\[
\renewcommand{\arraystretch}{1.3}
\begin{array}{|r|r|r|r|r|r|}
\hline
-b& r_1 & r_2 & m_{12}& m_{12}-r_1 & m_{12}-r_2\\\hline
33 & 12 & 21 & 16\frac{1}{2}&4\frac{1}{2} & -4\frac{1}{2}  \\\hline
33 & 8 & 25 & 16\frac{1}{2}&8\frac{1}{2}&-8\frac{1}{2}\\\hline
33 & 1 & 32 & 16\frac{1}{2}&15\frac{1}{2}&-15\frac{1}{2}\\\hline
\hline
-4 & -16 & 12 & -2 &14& -14\\\hline
-4 & -4 & 0 & -2&2&-2 \\\hline
-4 & -3 & -1 & -2&1&-1 \\\hline
\end{array}
\]

\bigskip

התרשים להלן מראה את היחסים הללו עבור
$r_1,r_2=2,6$,
כאשר
$m_{12}=4, s=2$:
\selectlanguage{english}
\begin{center}
\begin{tikzpicture}
\draw[->] (0,0) -- (10,0);
\foreach \x in {0,1,...,10}
  \draw (\x,-2mm) -- +(0,4mm) node[below,yshift=-4mm] {$\x$};
\draw[->,yshift=7mm] (0,0) -- node[above] {$r_1$} (20mm,0);
\draw[->,yshift=14mm] (0,0) -- node[above] {$r_2$} (60mm,0);
\draw[->,yshift=21mm] (0,0) -- (39.5mm,0);
\draw[->,yshift=21mm] (40mm,0) -- node[above] {$r_1+r_2$} (80mm,0);
\fill (40mm,21mm) circle(2pt) node[above] {$m_{12}$};
\draw[->,yshift=30mm] (20mm,0mm) -- node[above] {$m_{12}-r_1$} +(20mm,0);
\draw[->,yshift=30mm] (60mm,0mm) -- node[above] {$m_{12}-r_2$} +(-20mm,0);
\end{tikzpicture}
\end{center}
\selectlanguage{hebrew}
אם נבחר ערכים אחרים
$r_1,r_2=3,5$
עבורם
$r_1+r_2=8$, $m_{12}=4$ 
נשאר ללא שינוי, אבל
$s=1$
משתנה:
\selectlanguage{english}
\begin{center}
\begin{tikzpicture}
\draw[->] (0,0) -- (10,0);
\foreach \x in {0,1,...,10}
  \draw (\x,-2mm) -- +(0,4mm) node[below,yshift=-4mm] {$\x$};
\draw[->,yshift=7mm] (0,0) -- node[above] {$r_1$} (30mm,0);
\draw[->,yshift=14mm] (0,0) -- node[above] {$r_2$} (50mm,0);
\draw[->,yshift=21mm] (0,0) -- (39.5mm,0);
\draw[->,yshift=21mm] (40mm,0) -- node[above] {$r_1+r_2$} (80mm,0);
\fill (40mm,21mm) circle(2pt) node[above] {$m_{12}$};
\draw[->,yshift=30mm] (30mm,0mm) -- node[above left] {$m_{12}-r_1$} +(10mm,0);
\draw[->,yshift=30mm] (50mm,0mm) -- node[above right] {$m_{12}-r_2$} +(-10mm,0);
\end{tikzpicture}
\end{center}
\selectlanguage{hebrew}

לכאורה ההפרש
$s$
שרירותי ב:
\[
r_1=\left(\frac{-b}{2}+s\right)\,,\quad r_2=\left(\frac{-b}{2}-s\right)\,,
\]
אבל קיים אילוץ נוסף
$r_1r_2=c$
כאשר
$c$
הוא הקבוע בפולינום. אם נכפיל את שני ביוטיים במצאנו עבור
$r_1,r_2$,
נוכל לחשב את
$s$
אחר כך את
$r_1,r_2$.
\[
c=\left(-\frac{b}{2} +s\right)\left(-\frac{b}{2} -s\right)\,.
\]


\section{דוגמאות}\label{s.examples}

נשמתש בשיטה על הפולינום
$x^2-2x-24$,
כאשר
$b=-2,c=-24$:
\begin{eqnarray*}
c&=&\left(-\frac{b}{2} +s\right)\left(-\frac{b}{2} -s\right)\\
-24&=&(1 +s)(1 -s)\\
s^2&=&25\\
s&=&5\\
r_1&=&1+5=6\\
r_2&=&1-5=-4\,.
\end{eqnarray*}
נבדוק:
\[
(x-6)(x-(-4))=x^2-6x-(-4)x+(6\cdot -4)= x^2-2x-24\,.
\]

דוגמה נוספת: נמצא את השורשים של
$x^2-83x-2310$:
\begin{eqnarray*}
c&=&\left(-\frac{b}{2} +s\right)\left(-\frac{b}{2} -s\right)\\
-2310&=&\left(\frac{83}{2}+s\right)\left(\frac{83}{2} -s\right)\\
s^2&=&\frac{6889}{4}+2310=\frac{16129}{4}\\
s&=&\frac{127}{2}\\
r_1&=&\frac{83}{2}-\frac{127}{2}=-22\\
r_2&=&\frac{83}{2}+\frac{127}{2}=105\,.
\end{eqnarray*}
נבדוק:
\[
(x+22)(x-105)=x^2+22x-105x+(22\cdot -105)= x^2-83x-2310\,.
\]
נשווה את החישוב עם החישוב המשתמש בנוסחה המסורתית שכולנו למדנו בעל-פה:
\begin{eqnarray*}
\frac{-b\pm\sqrt{b^2-4c}}{2}&=&\frac{-(-83)\pm\sqrt{(-83)^2-4\cdot (-2310)}}{2}\\
&=& \frac{83\pm\sqrt{6889+9240}}{2} = \frac{83\pm\sqrt{16129}}{2}\\
&=& \frac{83\pm 127}{2}\\
r_1&=&\frac{83-127}{2}=-22\\
r_2&=&\frac{83+127}{2}=105\,.
\end{eqnarray*}
למרות שהחישוב בשיטה של
\L{Loh}
דומה לחישוב עם הנוסחה המסורתית, יש לו יתרון כי ניתן לקבל את החישוב מיידית מהממוצע והמכפלה של השורשים. בסעיף הבא נראה שקל לקבל את הנוסחה המסורתית משיטה זו.%
\footnote{%
ראו בהערת השוליים הקודמת על יתרון תיאורטית של השיטה.% 
}


\section{הנוסחה המסורתית}\label{s.general}

עם מקדמים שרירותיים, הנוסחאות הן:
\begin{eqnarray*}
c=r_1,r_2&=&\left(\frac{-b}{2}+s\right)  \left(\frac{-b}{2}-s\right)\\
&=&\left(\frac{b^2}{4}-s^2\right)\\
s&=&\sqrt{\left(\frac{b^2}{4}\right)-c}\\
r_1,r_2&=&\frac{-b}{2}\pm\sqrt{\left(\frac{b^2}{4}\right)-c}\,,
\end{eqnarray*}
שניתן לכתוב כ:
\[
r_1,r_2=\frac{-b\pm\sqrt{b^2-4c}}{2}\,,
\]
הנוסחה המסורתית לקבלת השורשים של פולינום עם מקדם אחד עבור
$x^2$.

עבור פולינום עם
$a\neq 1$, 
חלקו את המקדמים ב-%
$a$,
הצבו במשוואה ופשטו:
\begin{eqnarray*}
ax^2+bx+c&=&0\\
x^2+\frac{b}{a}x+\frac{c}{a}&=&0\\
r_1,r_2&=&\frac{-(b/a)\pm\sqrt{(b/a)^2-4(c/a)}}{2}\\
&=&\frac{-(b/a)\pm\sqrt{(b/a)^2-4(ac/a^2)}}{2}\\
&=&\frac{-b\pm\sqrt{b^2-4ac}}{2a}\,.
\end{eqnarray*}
קיבלנו את הנוסחאה המסורתית מהתובנה ש-%
$r_1+r_2=-b$
ו-%
$r_1r_2=c$.


\section{פולינומים שאי-אפשר לפרק}\label{s.irreducible}

קיימים פולינומים כגון
$x^2+1$
שלא ניתן לפרק מעל למספרים הרציונליים. לפי המשפט הבסיסי של אלגברה, ניתן למצוא שורשים מרוכבים לכל פולינום עם מקדמים מרוכבים. נראה איך עובדת השיטה עבור הפולינום
$x^2-2x+76$:

\begin{eqnarray*}
s^2&=&\frac{b^2}{4}-c=\frac{4}{4}-76=-75\\
s&=&\sqrt{-75}=\sqrt{-1\cdot 25\cdot 3}=i\,5\sqrt{3}\\
r_1,r_2&=&1\pm i\,5\sqrt{3}\,.
\end{eqnarray*}
נבדוק:
\[
\renewcommand*{\arraystretch}{1.3}
\begin{array}{ll}
(x-(1+i\,5\sqrt{3}))\;(x-(1-i\,5\sqrt{3}))&=\\
x^2 \,-\, (1+i\,5\sqrt{3})x\,-\,(1-i\,5\sqrt{3})x\,+\,(1^2-(i\,5\sqrt{3})^2)&=\\
x^2 -x -x + 1 - (-75)=\\
x^2-2x+76\,.
\end{array}
\]

\section*{מקורות}

\selectlanguage{english}
[1] Po-Shen Lo. \textit{A Different Way to Solve Quadratic Equations}, 2019,\\
\url{https://www.poshenloh.com/quadratic/}.

[2] Po-Shen Loh. \textit{A Simple Proof of the Quadratic Formula}, arXiv: 1910.06709, 2019, \url{https://arxiv.org/abs/1910.06709}.


\end{document}
